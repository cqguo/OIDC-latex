\documentclass[conference, 10pt]{IEEEtran}
%\setlength\columnsep{0.2in}
\usepackage{graphicx}
\usepackage{verbatim}
\usepackage{caption}
\usepackage{algorithm}
\usepackage{algorithmicx}
\usepackage{algpseudocode}
\usepackage{amsmath,amssymb,amsthm}
\newtheorem{theorem}{Theorem}
\newtheorem{corollary}{Corollary}
\usepackage{graphicx}
%\usepackage{geometry}
%\usepackage{subfigure}
\usepackage{url}
\usepackage{multirow}
\usepackage{listings}
\usepackage{cite}
\usepackage{array}
\usepackage{enumerate}
\usepackage{booktabs}
\usepackage{amsthm}
\usepackage{color}
\usepackage{soul}
%\geometry{left=0.7in,right=0.7in,top=1in,bottom=1in}

%IEEE S&P Submitted papers may include up to 13 pages of text and up to 5 pages for references and appendices, totalling no more than 18 pages.

\begin{document}
\pagenumbering{arabic}
\title{{UPRESSO}: An Unlinkable Privacy-REspecting Single Sign-On System}
%\title{{Recluse}: A Privacy-Respecting Single Sign-On System Achieving Unlinkable Users' Traces}
%\title{{Recluse}: A Privacy-Respecting Single Sign-On System Avoiding User Linkage and Tracing}
%\title{EOIDC: An Enhanced OpenID-Connect Protocol That Protect Users' Privacy}
%
% Single address.
% ---------------
%\name{Anonymous ICME submission}
%\address{}
\maketitle
\begin{abstract}
%(e.g., shifting the risks for managing the user's credentials) (e.g., reducing the number of credentials)
 The Single Sign-On (SSO) service, provided by identity provider (IdP),  is widely deployed and integrated to bring the convenience to both the relying party (RP) and the users.
% by shifting the users' credentials to the identity provider (IdP) and reducing the user's maintained  credentials.
 However, the privacy leakage is an obstacle to the users' adoption of SSO,
 as the curious IdP may track at which RPs users log in,
% as the curious IdP knows which RPs to be accessed by a user
while collusive RPs could link the user from a common or related identifer(s) issued by the IdP.
 %while the RPs may link the user with the non-independent identifiers in the identity proof provided by the IdP.
 Existing solutions preserve the user's privacy  from either the curious IdP or the collusive RPs, but never from the both entities.
In this paper, we provide an SSO system, named UPRESSO, to hide the user's accessed RPs from the curious IdP
and prevent the identity linkage from the collusive RPs.
In UPRESSO, IdP generates a different privacy-preserving ID ($PID_U$) for a user among RPs, and binds $PID_U$ with a transformation of the RP identifier ($PID_{RP}$), without obtaining the real RP identifier.
Each RP uses a trapdoor to derive the user's unique account from $PID_U$,
while a user's accounts are different among the RPs.
%based on the transformation of the RP identifiers and user's accounts on the discrete logarithm problem.
UPRESSO  is compatible with OpenID Connect, a widely deployed and well analyzed SSO system;
    where dynamic registration is utilized to make $PID_{RP}$ valid in the IdP.
    % finish xx and xx to xxx. % ���ö�̬ע����xx��xxx���㶨xxx��һ�仰�Ĵ��˵��
The analysis shows that user's privacy is preserved in UPRESSO without any degradation on the security of OpenID Connect.
%Recluse is proved to preserve the user's privacy without any degradation on the security of OpenID Connect.
We have implemented a prototype of UPRESSO. The evaluation demonstrates that UPRESSO is efficient, and only needs 208 ms for a user to login at an RP in our environment.
\end{abstract}
\begin{IEEEkeywords}
Single Sign-On, security, privacy, trace, linkage
\end{IEEEkeywords}

\section{Introduction}
\label{sec:intro}

%SSO的特点
%SSO的现状
Single sign-on (SSO) systems, such as OAuth~\cite{rfc6749}, OpenID Connect~\cite{OpenIDConnect} and SAML~\cite{SAML}, have been widely adopted nowadays as a convenient web authentication mechanism. SSO delegates user authentication from websites, so-called relying parties (RPs), to a third party, so-called identity providers (IdPs), so that users can access different services at cooperating sites via a single authentication attempt. Using SSO, a user no longer needs to maintain multiple credentials for different RPs, instead, she maintains only the credential for the IdP, who in turn will generate corresponding \emph{identity proofs} for those RPs. Moreover, SSO shifts the burden of user authentication from RPs to IdPs and reduces security risks and costs at RPs. As a result, SSO has been widely integrated with modern web systems.
Our study shows that 80\% of the Alexa top-100 websites~\cite{Alexa} support SSO services and study on the Alexa top 1 million websites has found that 6.30\% of websites support  SSO~\cite{GhasemisharifRC18}.
Meanwhile, many email and social networking providers such as Google, Facebook, Twitter, etc. have been actively serving as social identity providers to support social login.


%SSO系统的安全问题,需要保护identity proof的完整性,机密性,绑定性
%完整性:使用公开的为受保护的信息作为identity proof
%机密性:保证由IdP发送给对应的RP,并且传输过程中通道、user agent都是受到保护的
%绑定性:identity proof一定要与对应的RP实现绑定
A fundamental requirement of SSO systems is secure authentication~\cite{SPRESSO}, which should ensure that it is impossible (1) for an adversary to log in to an honest RP as an honest user (i.e., impersonation); and (2) for an honest user to log in to an honest RP as someone else such as an adversary (i.e., identity injection).
Extensive study have been performed on existing SSO systems exposing various vulnerabilities~\cite{ChenPCTKT14, FettKS16,WangCW12,ZhouE14,WangZLG16,YangLLZH16,SomorovskyMSKJ12,MohsenS16}. It is commonly recognized that SSO security highly depends on secure generation and transmission of identity proofs: (1) the identity proof should be generated in a way that it can never be forged or tampered. For example, when Google used user's attributes with a signature as the identity proof~\cite{WangCW12}, {\color{red}the adversary is able to bypass the verification of part of attributes in the identity proof. The root cause is that RP relies on the IdP signed user's attributes to identify the user, however, when the adversary acts as the user, it is able to modify the request and repose between RP and IdP transmitted by user, which allows the adversary add any honest user's attributes in the IdP's response and RP would accept all of them without verifying the signature. Finally the adversary is able to log in to the RP as the honest user}.
(2) The identity proof should be generated in a way that it is bound to the requesting RP. For example, if an identity proof is generated with nonbinding data such as access tokens in OAuth 2.0, an adversary such as a malicious RP is able to log in to an honest RP as the honest user~\cite{ChenPCTKT14, WangZLG16}.
%the parts of IdP developer choose the data (e.g., access token in OAuth 2.0) unbound with specific RP as the identity proof, which results in that the malicious RP is able to collect user's identity proof and use it to log in to other RPs.
(3) The identity proof should only be obtained by the requesting RP. For example, in some mobile SSO implementations using WebView~\cite{MohsenS16}, as the url check is not supported which the malicious RP app invokes the IdP's web site in its Webview, the adversary is able to steal the identify proof of an honest user when the app cheats her to consider it as another honest RP.

%第二段:
%SSO 引入新的隐私问题
%IdP知道用户登录哪个RP
%RP之间可以合谋知道同一个用户登录哪些RP
The wide adoption of SSO also raises new privacy concerns regarding online user tracking and profiling~\cite{maler2008venn,NIST2017draft}. In a typical SSO authentication session, for example the OpenID Connect (OIDC) authentication flow as shown in Fig.~\ref{fig:OpenID}, when a user attempts to log in to an RP, the authentication request is redirected from the RP to the IdP, which generates an identify proof containing information about the user (e.g., user identifier and other authorized user attributes) and the requesting RP (e.g., RP identifier, URL, etc.). If a common user identifier is issued by the IdP for a same user across different RPs or a user's identifiers can be derived from each other, which is the case even in several widely deployed SSO systems~\cite{BrowserID,SPRESSO}, collusive RPs could not only track her online traces but also correlate her attributes across the sites~\cite{maler2008venn}. We refer to this as {\em RP-based identity linkage}. Moreover, when a user leverages the identity issued by one IdP across multiple RPs, the IdP acquires a central role in web authentication, which enables it to collect information about the user's logins at different sites~\cite{maler2008venn}. Since the information of the RP is necessary in the construction of identity proof to make sure it is bound with specific RP~\cite{ChenPCTKT14, WangZLG16}, any interested IdP can easily discover all the RPs accessed by a user and reconstruct her access traces by the unique user identifier. We refer to this as {\em IdP-based access tracing}. Both RP-based identity linkage and IdP-based access tracing could lead to more severe privacy violations.




%第三段:
%google and facebook的负面新闻
%1. service provider(如DNS)知道你访问了,可以带来很多问题。但是还是不同的,DNS里profile需要评估的是two behavior vectors的similarities,而IdP中都不需要,因为IdP能够区分two behavior vectors 是否来自相同节点。

Meanwhile, large IdPs, especially social IdPs like Google and Facebook, are known to be interested in collecting users' online behavioral information for various purposes (e.g., Screenwise Meter~\cite{googlenews}, Onavo~\cite{Onavo}). By simply serving the IdP role, these companies can easily collect a large amount of continuous data to reconstruct users' online traces. %Unlike other privacy risks, such as user session re-identification that requires to compute the similarity between users' DNS queries~\cite{DNS},
Moreover, many service providers are also hosting a variety of web services, which make them easy to link the same user's multiple logins in each RP as the unique user identifier is contained in the identity proof. Through internal integration, they could obtain rich information from SSO data to profile their clients.



While the privacy problems in SSO have been widely recognized~\cite{maler2008venn,NIST2017draft}, only a few solutions have been proposed to protect user privacy~\cite{persona,SPRESSO}. Among them, Pairwise Pseudonymous Identifier (PPID)~\cite{OpenIDConnect, SAMLIdentifier} is a most straightforward and commonly accepted solution to defend against RP-based identity linkage, which requires the IdP to create different user identifiers for the user when she logs in to different RPs. In this way, even multiple malicious RPs collude with each other across the system, they cannot link the  pairwise pseudonymous identifiers of the user and track which RPs she has visited. As a recommended practice by NIST~\cite{NIST2017draft}, PPID has been specified in many widely adopted SSO standards including OIDC~\cite{OpenIDConnect} and SAML~\cite{SAMLIdentifier}.

However, PPID-based approaches cannot prevent the IdPs from tracking at which RPs users log in, since pseudonymous identifiers are generated by IdP which only hide users' identity from the RPs. To authenticate a user, the IdP has to know her identity. To the best of our knowledge, there are only two approaches (i.e., BrowserID~\cite{BrowserID} and SPRESSO~\cite{SPRESSO}) being proposed so far to prevent IdP-based access tracing. Both adopt the idea of hiding RPs' identities from the IdP during the construction and transmission of identity proofs. In particular, the identity proof in BrowserID (and its prototype system known as Mozilla Persona~\cite{persona} and Firefox Accounts~\cite{FirefoxAccount}) is combined with two parts, the binding of the public key generated by the user with her email (as the user identifier) signed by IdP as the user certificate (UC) and the origin of the RP signed by user's private key as the identity assertion (IA). The core idea of BrowserID is that the IdP-generating part of identity proof does not contains any RP's identity and the work of binding identity proof with RP is shift to the user. 
% uses email address as user identifier, and lets the IdP bind the public key generated by the user with her email. The identity proof , the signed with the corresponding private key of the user and transmitted to the RP via the user. 
With the user as the proxy, the IdP does not know RPs' identities throughout the authentication process. 
In SPRESSO, the RP generates a dynamic pseudonymous identifier by encrypting its domain and a nonce, and passes it to the IdP through the user to create the identity proof, which is returned to the RP through a trusted third party, called forwarder, chosen by the RP itself. While forward transmitting the identity proof, it decrypts the encrypted RP domain to make sure the identity proof only sent to the requesting RP, however, the identity proof is encrypted by another RP generated symmetric key before transmitting to avoid the forward knowing the user's identity from identity proof, which will result in the forward tracing the user. As it designed in SPRESSO, the user email is adopted as the user identifier for identity proof generating. 

%In BrowserID, the identity proof is signed with the private key generated by user, and transmitted to the RP through the user directly,  while the corresponding  public key is bound with users' email by IdP who need not obtain the information of accessed RP. In SPRESSO, RP encrypts its domain and a nonce as the identifier, so that the real identity of RP is never exposed to IdP, while the identity proof is transmitted to the RP through a trusted entity (named FWD) who doesn't know the user's identity.




%用户和每一个RP分别在IdP处有长期不变的ID,分别称为ID_U和ID_RP;IdP为用户签发的identity proof中存在着连接ID_U和ID_RP的ID,称为ID_U^Proof;用户在每一个RP处都有长期不变的Account。
%在每一次identity proof产生的过程中(也就是每一次log in rp),IdP会产生绑定了ID_U和ID_RP的identity proof
%原始的SSO方案中,会同时暴露ID_U和ID_RP
%为了保护用户隐私,应该保证ID_U与ID_U^RP不同,并且ID_U^RP在不同的RP不同,同时应该使用变换后的随机ID_RP代表RP身份
%在PPID方案中,ID_U与ID_U^RP不同,防止了RP-based linkage,但是向IdP暴露了ID_RP
%在SPRESSO中,使用了加密的ID_RP代表RP身份,但是ID_U与ID_U^RP相同
%in this paper, we propose xxxx。
%具体的,在我们的方案,,Account和ID_U不同;同时,在每一次identity proof 产生的过程之中,RP都会在IdP处动态地注册匿名ID,称为PID_RP;同时,IdP会根据PID_RP和IDU产生PID-U,并将PID_RP和PID-U绑定在identity proof中;当RP验证了identity proof之后,会利用PID_RP生成过程中存在的trapdoor得到不变的account。
%我们的方案满足(1)(2)(3)
Unfortunately, %none of the existing approaches provides sufficient and comprehensive protection for user security and privacy in SSO. For example, \cite{BrowserID} reported that BrowserID suffered from a severe privacy attack in which malicious IdPs could check users' login status at any RP. We also find that the security of SPRESSO highly relies on the third-party forwarder and therefore it is vulnerable to the impersonation attacks once the chosen forwarder is compromised. Moreover,
none of the existing SSO systems could address both RP-based identity linkage and IdP-based access tracing privacy problems at the same time.
The user's identity is represented by several forms in different environments. In detail, the IdP and RP store the unique user identifier (denoted as $ID_U$ in IdP and $Account$ in RP) for each users solely, which are linked by the identity proof containing the junctional user identifier (denoted as $ID_U^{Proof}$) issued by IdP. Moreover, there is also the unique RP identifier (denoted as $ID_{RP}$) stored in IdP. In each authentication, the identity proof issued by IdP should contain the $ID_{RP}$ (making sure the identity proof is issued for specific RP) and $ID_U^{Proof}$ (for RP to identify the user). In some traditional SSO systems~\cite{rfc6749}, the $ID_U$ and $ID_U^{Proof}$ are the same one, which results in the RP-based identity linkage, and the $ID_{RP}$ is always exposed to IdP, which results in the IdP-based access tracing. However, to avoid the privacy problems in SSO systems, the $ID_U^{Proof}$ should not be the same as the RP (or the $ID_U^{Proof}$ should not be the same in different RPs) and the $ID_{RP}$ should not be exposed to the IdP (using a transformed $ID_{RP}$ representing the RP) in each authentication. The widely adopted SSO standards, such as OIDC and SAML, use $PPID$ as the $ID_U^{Proof}$ transformed from $ID_U$ which is constant in one RP but different in RPs, but expose the $ID_{RP}$ straightforward to the IdP. Otherwise, the SPRESSO adopts the encrypted $ID_{RP}$ (denoted as $enc(ID_{RP})$) and only the $enc(ID_{RP})$ is exposed to the IdP, however, the $ID_U^{Proof}$ is the same one as the $ID_U$ in SPRESSO. As for the BrowserID, the identity proof is generated by the cooperation between IdP and user so that it could avoid exposing $ID_{RP}$ to IdP, but uses same identifier for $ID_U$ and $ID_U^{Proof}$. The worse thing is there is not a simple way to combine the existing schemes together to protect user's privacy comprehensively (the details are described in Section~\ref{sec:challenge}).

In this paper, we propose UPRESSO, an Unlinkable Privacy-REspecting Single Sign-On system, as a comprehensive solution to tackle the privacy problems in SSO. We propose novel identifier generation schemes to dynamically generate  privacy-preserving $ID_U^{Proof}$ and transformed $ID_{RP}$, denoted as $PID_U$ and $PID_{RP}$, to construct identity proofs for SSO. In our scheme, for each login, RP anonymously registers a random $ID_{RP}$ (denoted as $PID_{RP}$) with IdP, and exposes the $PID_{RP}$ for authentication. IdP generates the $PID_U$ with $PID_{RP}$ and $ID_U$ and issues the identity proof containing $PID_{RP}$ and $PID_U$. Finally, after RP validates the identity proof, it utilizes the trapdoor obtained when generating the $PID_{RP}$ to derive the $Account$ from $PID_U$. However the scheme must satisfy three properties: (1) when a same or differnt user(s) log in to a same RP, random $PID_{RP}$s are generated in different logins so that a curious IdP cannot infer the real identity of the RP or link multiple logins at that RP; (2) when a same user logs in to a same or different RPs, random $PID_U$s are generated so that collusive RPs cannot link the logins of that user; (3) when a same user logs in to a same RP, the RP can derive a unique user identifier from different PUIDs with a trapdoor so that it can provide a continuous service to the user during different logins.




%In BrowserID and SPRESSO, collusive RPs could link a user's multiple logins from the common user identifier.
%In this paper, we present UPRESSO, an Unlinkable Privacy-REspecting Single Sign-On system,
%UPRESSO, privacy-\(RE\)specting single sign-On system a\(C\)hieving un\(L\)inkable \(USE\)rs' traces
%as a comprehensive solution to tackle the privacy problems in SSO. We propose novel identifier generation schemes to dynamically generate  privacy-preserving user and RP identifiers, denoted as $PUID$ and $PRPID$, to construct identity proofs for SSO, which satisfy three properties: (1) when a same or differnt user(s) log in to a same RP, random $PRPID$s are generated in different logins so that a curious IdP cannot infer the real identity of the RP or link multiple logins at that RP; (2) when a same user logs in to a same or different RPs, random $PUID$s are generated so that
%neither a curious IdP nor
%collusive RPs cannot link the logins of that user; (3) when a same user logs in to a same RP, the RP can derive a unique user identifier from different PUIDs with a trapdoor so that it can provide a continuous service to the user during different logins.

%from both the IdP and RPs, named {UPRESSO}. To achieve this, we rely on the user to achieve the trusted transmit and correctness check of identity proof (same as in BrowserID~\cite{persona}), and propose two algorithms to achieve:



Unlike previous approaches that require non-trivial re-design of the existing SSO systems, UPRESSO can be implemented over a widely used OIDC system with small modifications with the support of its dynamic registration function~\cite{DynamicRegistration}.

%Reluse only requires the following modification on  OIDC implementations: (1) an additional web service at the IdP for providing a  set of public parameters; (2) the support for generating  the new RP identifier (at the user and RP), PPID (at the IdP) and user's account (at RP). The prototype demonstrates  that UPRESSO is incompatible with existing OIDC implementations.








%第六段
%我们的贡献
%提出协议
%考虑能否根据模型进行分析
%实现原型系统
The main contributions of UPRESSO are as follows:
\begin{itemize}
\item We systematically analyze the privacy issues in SSO systems and propose a comprehensive protection solution to hide users' traces from both curious IdPs and collusive RPs, for the first time. We also provide a systematic analysis to show that UPRESSO achieves the same level of security as existing SSO systems.
%deals with all the privacy issues introduced by SSO comprehensively. It has the ability to prevent IdP from tracking users' login trace, as well as multiple RPs are unable to link the users either.
%pratical extension for OIDC, which inherits the systematically and thoroughly analyzed  security and privacy mechanisms of OIDC, and achieves the full privacy for users by hiding the accessed RPs from IdP.
\item We develop a prototype of UPRESSO that is compatible with OIDC and demonstrate its effectiveness and efficiency with experiment evaluations.
\end{itemize}



%第七段
%文章结构
The rest of this paper is organized as follows. We introduce the background in Sections~\ref{sec:background}, and the challenges with solutions briefly~\ref{sec:challenge}. Section~\ref{sec:related} and Section~\ref{sec:UPRESSO} describe the threat model and the design of UPRESSO. A systematical analysis is presented in Section~\ref{sec:analysis}. We provide the implementation specifics and evaluation in Section~\ref{sec:implementation}, then introduce the related works in Section~\ref{sec:related}, and draw the conclusion finally.
% and Section~\ref{sec:evaluation}


\section{Background and Preliminary}
\label{sec:background}


\subsection{OpenID Connect}
\label{subsec:OIDC}

To be compatible with existing SSO systems, UPRESSO is designed on top of OpenID Connect (OIDC)~\cite{OpenIDConnect}, one of the most prominent SSO authentication protocols~\cite{SAMLIdentifier}. In this section, we introduce OIDC and its implicit flow as an example to describe the authentication flows in SSO. Moreover, our proposed framework can also work for other flows with only minor modifications.

OIDC is an extension of OAuth 2.0 to support user authentication. As a typical SSO authentication protocol, OIDC involves three entities, i.e., {\em users}, {\em identity provider (IdP)}, and {\em relying parties (RPs)}. Users register at the IdP to create credentials and identifiers (e.g. $ID_U$), which are securely maintained by the IdP. Moreover, an RP also has to register at the IdP with its endpoint information to create its unique identifier. During an SSO process, a user is also responsible for redirecting the identity proof request from the RP to the IdP, checking the scope of user attributes in the identity proof returned by the IdP, and forwarding it to the RP. Usually, the redirection and checking are handled by a user-controlled software, called {\em user agent} (e.g., browser). The IdP maintains user credentials and attributes. Once requested, it authenticates the user and generates the identity proof, which contains user identifier (e.g., $PPID$ in OIDC), RP identifier (e.g. $ID_{RP}$), and a set of user attributes that the user consents to share with the RP. The identity proof is then returned to the registered endpoint of the RP (e.g., URL). RP can be any web server that provides continuous and personalized services to its users. When a user attempts to log in, the RP sends an identity proof request to the IdP through the user, and parses the received identity proof to authenticate and authorize the user.



%****** Add discussion about why implicit flow is used in this work
\noindent\textbf{Implicit flow of user login.}
OIDC supports three processes for SSO, known as {\em authorization code flow}, {\em implicit flow} and {\em hybrid flow} (i.e., a mix-up of the previous two).
In the implicit flow of OIDC, a new token, known as {\em id token}, is introduced as the identity proof, which contains user identifier (i.e., PPID), RP identifier (i.e., $ID_{RP}$), the issuer (i.e., IdP), issuing time, the validity period, and other requested attributes. The IdP signs the id token by its private key to ensure integrity.
Moreover, in the authorization code flow, the identity proof is an authorization code bound to the RP. Only the RP with the corresponding secret obtained during the registration at the IdP can extract the user attributes from the identify proof.



\begin{figure}[t]
  \centering
  \includegraphics[width=\linewidth]{fig/OIDC1.pdf}
  %\subfigure[Authorization Code Flow]{\includegraphics[width=\linewidth]{fig/openidconnect2.pdf}\label{fig:OpenID_code}}
  %\subfigure[Hybrid Flow]{\includegraphics[width=\linewidth]{fig/openidconnect3.pdf}\label{fig:OpenID_hybrid}}
  \caption{The implicit protocol flow of OIDC.}
  \label{fig:OpenID}
\end{figure}

As shown in Figure~\ref{fig:OpenID}, the implicit flow of OIDC consists of 7 steps: when a user attempts to log in to an RP (step 1), the RP constructs a request for identity proof, which is redirected by the user to the corresponding IdP (step 2). The request contains $ID_{RP}$, RP's endpoint and a set of requested user attributes. If the user has not been authenticated yet, the IdP performs an authentication process (step 3). If the RP's endpoint in the request matches the one registered at the IdP, it generates an identity proof (step 4) and sends it back to the RP (step 5). Otherwise, IdP generates a warning to notify the user about potential identity proof leakage. The RP verifies the id token (step 6), extracts user identifier from the id token and returns the authentication result to the user (step 7).



\begin{figure}[t]
  \centering
  \includegraphics[width=\linewidth]{fig/overview1.pdf}
  %\subfigure[Authorization Code Flow]{\includegraphics[width=\linewidth]{fig/openidconnect2.pdf}\label{fig:OpenID_code}}
  %\subfigure[Hybrid Flow]{\includegraphics[width=\linewidth]{fig/openidconnect3.pdf}\label{fig:OpenID_hybrid}}
  \caption{The UPRESSO.}
  \label{fig:UPRESSO}
\end{figure}

\noindent\textbf{RP dynamic registration.} OIDC provides a dynamic registration mechanism~\cite{DynamicRegistration} for the RP to renew its $ID_{RP}$ dynamically. % shown in Figure~\ref{fig:OpenID}.
When an RP first registers at the IdP, it obtains a registration token, with which the  RP can invoke the dynamic registration process to
%After a successful initial registration, RP obtains a registration token from the IdP, and
update its information (e.g., the endpoint). % by a dynamic registration process with the  registration token. %which enables the RP to integrate the service provided by IdP,
After each successful dynamic registration, the RP obtains a new unique $ID_{RP}$ from the IdP.
In this work we adopt dynamic registration to support the dynamic generated privacy-preserving RP identifier (e.g. $PID_{RP}$).


\subsection{Discrete Logarithm Problem}
\label{sec:dlp}

Discrete logarithm problem is adopted in UPRESSO for privacy-preserving user identifier (e.g. $PID_U$) and RP identifier (e.g. $PID_{RP}$) generation.
Here, we provide a brief description of the discrete logarithm problem.
Given a large prime $p$, its primitive root $g$ and a number $y$, it is computationally infeasible to derive the discrete logarithm (here $x$ and $g^xmodp=y$) of $y$ (detailed in~\cite{WXWM}), which is called discrete logarithm problem.
The hardness of solving discrete logarithm has been a base of the security of several security primitives, including Diffie-Hellman key exchange and Digital Signature Algorithm (DSA).
The $PID_U$ and $PID_{RP}$ generation is based on the modular exponentiation, which requires the parameters, such as $ID_{RP}$ and $PID_{RP}$, should be the primitive root mod $p$ to prevent the user and RP' identity leakage.
To calculate the primitive root for a given large prime $p$, we retrieve a primitive root $g_m$ mod $p$, and then calculate the $g = g_{m}^{t} mod \ P$, so that $g$ is another primitive root mod $p$ where $t$ is an integer coprime to $p-1$.


\begin{comment}
%, to prevent IdP from inferring $ID_{RP}$ from $PID_{RP}$, and allow the RP to derive $Account$ from $PID_U$ without obtaining the $ID_U$.
Here, we provide a brief description of the discrete logarithm problem.
A number $g$ ($0<g<p$) is called a primitive root modular a prime $p$, if for ${\forall}y$ ($0<y<p$), there is a  number $x$ ($0\le x <p-1$) satisfying $y=g^x \pmod p$.
And, $x$ is called the discrete logarithm of $y$ modulo $p$. Given a large prime $p$ and a number $y$, it is computationally infeasible to derive the discrete logarithm (here $x$) of $y$ (detailed in~\cite{WXWM}), which is called discrete logarithm problem. The hardness of solving discrete logarithm has been a base of the security of several security primitives, including Diffie-Hellman key exchange and Digital Signature Algorithm (DSA).
To calculate the primitive root for a given large prime $p$,  we retrieve the least primitive root $g_m$  mod $p$,
and then calculate the primitive root $g = g_{m}^{t} mod \ P$, where $t$ is an integer coprime to $p-1$.
A lemma is proposed to check whether an integrity $\mu$ is the primitive root modulo $p$ where $p=2q+1$ ($q$ is a prime), that is,  an integer $\mu \in (1, p-1)$ is a primitive root if and only if $\mu^2\neq 1 \ mod \ p$ and $\mu^q\neq 1 \ mod \ p$~\cite{Shoup,Wang}.
%The details are provided in~\cite{Shoup,Wang}.
\end{comment}


\section{The Privacy Dilemma in Single Sign-On}
\label{sec:challenge}

In this section, we describe the challenges for developing privacy-preserving SSO systems and provide an overview of the solutions proposed in UPRESSO.



\subsection{Basic Security Requirements of SSO}
\label{subsec:basicrequirements}
Both the designs of SSO protocols and the implementations of SSO systems are challenging ~\cite{SPRESSO},
%SSO allows users to leverage their existing accounts in the IdP to access services at the RPs.
%However, building such a system is considered challenging~\cite{SPRESSO}, as several design and implementation 
and various vulnerabilities have been found in existing systems~\cite{SomorovskyMSKJ12,WangCW12,ArmandoCCCPS13,ZhouE14,WangZLLYLG15,WangZLG16,YangLLZH16,MainkaMS16,MohsenS16,MainkaMSW17,YangLCZ18,YangLS17,ShiWL19}.
With these vulnerabilities, the adversaries break the security of SSO systems and achieve the following two goals~\cite{SPRESSO}:
%In SSO systems, the adversaries' goals described in~\cite{SPRESSO} to break the secure authentication are:
\begin{itemize}
\item \textbf{Impersonation}: Adversary logs in to an honest RP as the victim user. %The adversary might achieve the goal by obtaining a user's identity proof in the ways, such as stealing the proof (from the unprotected HTTP transmission), forging the valid proof (if the integrity is not guaranteed), leading the user to upload a proof valid for other RPs (the proof is not bound with specific RP).
\item  \textbf{Identity injection}: A victim user logs in to an honest RP under the adversaries' identity. %The adversary might achieve this goal by replacing the identity transmitted from IdP to RP or lead the user uploads the malicious identity proof in various ways (e.g., CSRF).
\end{itemize}

We summarize basic requirements of SSO systems based on existing theoretical analysis~\cite{ArmandoCCCT08,FettKS16, FettKS17} and practical attacks~\cite{SomorovskyMSKJ12,WangCW12,ArmandoCCCPS13,ZhouE14,WangZLLYLG15,WangZLG16,YangLLZH16,MainkaMS16,MohsenS16,MainkaMSW17,YangLCZ18,YangLS17,ShiWL19} on SSO systems. These basic requirements focus on  functions and security of SSO systems, and are as follows: %not considering privacy protection. They are as follows:
%Based on the formal analysis~\cite{ArmandoCCCT08,FettKS16, FettKS17} of SSO protocls (e.g.,SAML, OAuth and OIDC) and the study of existing attacks, we summarize basic requirements that we believe a secure SSO platform which means, 1)RP is always able to identify the user; 2)the system should be protected from forementioned attacks, should meet.

\vspace{1mm}\noindent\textbf{User identification.} When a user logs in to a same RP multiple times, the RP should be able to associate these logins to provide a continuous and personalized service to that user. %The anonymous SSO schemes~\cite{ElmuftiWRR08,WangWS13,HanCSTW18} are also proposed to achieve that the user's identity is unknown the neither the IdP nor the RP. However, these schemes are not focally concerned in this paper as the completely anonymous SSO systems are not suitable to the currently web application where the RP need to know the user's identity to provide the personally service. The anonymous SSO schemes are discussed in Section~\ref{sec:related}.

\vspace{1mm}\noindent\textbf{Receiver designation.} The receiver designation requires that the identity proof should  be  transmitted only to the RP (and user) that the user visits, and be bound to this RP so that it will accepted only by this RP. %(through the user). Otherwise, the adversaries who is able to achieve honest user's identity proof for honest RP, have the ability to conduct impersonation attack~\cite{ChenPCTKT14, WangZLG16}


\begin{comment}
\vspace{1mm}\noindent\textbf{Identification.} %Identification is the main feature of SSO system, which enables the RP to identify the user.
After a successful SSO authentication, the RP should be able to uniquely identify the user. When a user logs in to a same RP multiple times, the RP should be able to associate these logins to provide a continuous and personalized service to that user. This is an essential requirement for any authentication mechanism. In SSO, this indicates that a same unique user identifier should be derived from multiple identity proofs of the user by the RP. In OIDC, for example, a user obtains different PPIDs from the IdP, each for a different RP.


\vspace{1mm}\noindent\textbf{Binding.} Each identity proof should be bound to one RP so that it will be accepted only by that RP. Otherwise, the identity proof may be abused to impersonate the victim user at other honest RPs~\cite{ChenPCTKT14, WangZLG16}. To achieve such binding in OIDC, the IdP embeds the identifier of the target RP in the identity proof and signs it with its private key. When receiving an identity proof, the RP needs to check if it is the intended receiver.

\vspace{1mm}\noindent\textbf{Confidentiality.} The identity proof should be securely returned to the requesting RP (and the user). Otherwise, an adversary who obtains an identity proof can impersonate the user at the RP specified in the identity proof~\cite{ChenPCTKT14,FettKS16,WangZLG16}. Currently, OIDC adopts TLS to ensure the confidentiality during the transmission. However, additional checking should be performed during the generation of the identity proof by the IdP and the transmission of the identity proof by the user and its trusted user agent.
\end{comment}

\vspace{1mm}\noindent\textbf{Integrity.} Only the IdP is able to generate a valid identity proof,
     no other entity should be able to modify or forge it~\cite{WangZLG16} without being found. And, the honest RP should only accept the valid identity proof.
    %Otherwise, an adversary could modify user identifier in the proof to launch impersonation or identity injection attacks. In OIDC, for example, the IdP signs identity proofs with its private key to provide integrity~\cite{WangCW12, SomorovskyMSKJ12}.

These basic requirements are the minimum properties that an SSO system has to provide.
User identification is necessary for all services except the anonymous systems which will be discussed in Section~\ref{sec:related},
 therefore RPs should be able to identify the user with the help from IdP.
While either the receiver designation and integrity not satisfied, the impersonation and identity injection will exist in the SSO systems.
For example, without the receiver designation,  the adversaries cloud use the leaked identity proof to impersonate the victim user at an honest RP~\cite{ChenPCTKT14,FettKS16,WangZLG16}, and  make the victim RP  incorrectly accept the identity proof intended for other RP, which allows the adversary to perform impersonation directly or inject the identity proof in the web session between the victim user and honest RP with other web attacks (e.g., CSRF);
while without integrity, the impersonation and identity injection attacks will be more easier as an adversary could directly modify user's identifier in the identity proof. % to launch impersonation or identity injection attacks
 
\begin{comment}
\vspace{1mm}Breaking the above requirements will lead to some attacks.
in SSO systems, the adversaries' goals described in~\cite{SPRESSO} to break the secure authentication are:
(1) \textbf{Impersonation}: Adversary logs in to the honest RP as an honest user. The adversary might achieve the goal by obtaining a user's identity proof in the ways, such as stealing the proof (from the unprotected HTTP transmission), forging the valid proof (if the integrity is not guaranteed), leading the user to upload a proof valid for other RPs (the proof is not bound with specific RP).
(2) \textbf{Identity injection}: Honest user logs in to the honest RP under adversaries' identity. The adversary might achieve this goal by replacing the identity transmitted from IdP to RP or lead the user uploads the malicious identity proof in various ways (e.g., CSRF).

Or break identification will lead to \textbf{anonymity}. the anonymous SSO schemes~\cite{ElmuftiWRR08,WangWS13,HanCSTW18} are also proposed to achieve that the user's identity is unknown the neither the IdP nor the RP. However, these schemes are not focally concerned in this paper as the completely anonymous SSO systems are not suitable to the currently web application where the RP need to know the user's identity to provide the personally service. The anonymous SSO schemes are discussed in Section~\ref{sec:related}.
\end{comment}

\subsection{The Privacy Dilemma and Existing Attempts}
\label{subsec:challenges}
\begin{comment}
In UPRESSO, the adversaries' goals to break the secure authentication are as follows:
\begin{itemize}
\item Impersonation attack: Adversary logs in to the honest RP as an honest user. The adversary might achieve the goal by obtaining a user's identity proof in the ways, such as stealing the proof (from the unprotected HTTP transmission), forging the valid proof (if the integrity is not guaranteed), leading the user to upload a proof valid for other RPs (the proof is not bound with specific RP).
\item Identity Injection: Honest user logs in to the honest RP under adversaries' identity. The adversary might achieve this goal by replacing the identity transmitted from IdP to RP or lead the user uploads the malicious identity proof in various ways (e.g., CSRF).
\end{itemize}
\end{comment}



%A privacy-preserving SSO needs to prevent both IdP-based access tracing and RP-based identity linkage.
In addition to the basic requirements, a privacy-preserving SSO system should prevent the IdP-based access tracing %login tracing 和 access tracing 没有已经广泛使用的说法
and RP-based identity linkage. 
In details, the privacy-preserving SSO system should prevent the curious IdP from obtaining any information that could identify the user's accessed RP (e.g., RP's identifier and URL),
and prevent  the collusive RPs from correlating the user's identifiers at different RPs. % in the identity proof.
%the user's identifiers provided by IdP for one user should never be same or derivable to each other.

The privacy-preserving requirements need to be integrated into the basic requirements, to ensure the correct functions and security of SSO systems.
The user identification requires IdP to provide a user's identifier ($PID_U$) to help the RP in identifying the user locally, 
while RP-based identity linkage prevention requires no correlation in $PID_U$s could be found by the collusive RPs to link the logins from a user.
The receiver designation requires IdP to bind an identity proof with an RP and send it only to this RP,
 while  IdP-based access tracing prevention requires that IdP can never identify which RP is visited by the user. 
  
%To conform the requirements of user identification and receiver designation, which is to be validated by the RP, the identity proof must contains (or is related with) specific RP identifier (e.g. $ID_{RP}$) and user identifier (e.g. $ID_U$). However, as the identity proof is generated by the IdP, the $ID_{RP}$ and $ID_U$ allows the IdP to know which RP the user accessed. Moreover, as the identity proof is verified by the RP, the colluded RPs are able to link the same user. Therefore, the key point of privacy-preserving SSO system is to hide the $ID_{RP}$ and $ID_U$ while generating the identity proof.



We define that the simplified identity proof model as the tuple $<ID_{RP}, ID_U>$. For each entity in the system, everyone knows the full tuple is able to trace the user (by IdP-based access tracing and RP-based identity linkage). It should be concerned that the RP always know the $ID_{RP}$. We can transform the problem into hiding the full tuple to IdP and RP.

In traditional SSO systems, the tuple is always exposed to both IdP and RP, so that they do not protect user from any kind of tracing. In OIDC, $PPID$ is adopted to hide the $ID_U$ from RP by using the RP-unique identifier. In SPRESSO, the identity proof is generated for a encrypted $ID_{RP}$ so that the IdP is unable to obtain the full tuple which avoids the IdP-based access tracing. In BrowserID, the generation of identity proof is split and the responsibility of binding it to RP is shifted to user, which also avoids the IdP-based access tracing. However OIDC allows the full tuple is exposed to IdP, while SPRESSO and BrowserID allows the RP obtains the full tuple, all of which cause the user to be tracked.

The key challenge is that there is no simple way to hide the full tuple to both IdP and RP. To tackle this problem, it requires the IdP to generate pairwise identifiers for a user and bind them to pseudo RP identifiers that seem random to IdP, in a way that user identifiers are unique at one RP but different at different RPs and user identifiers bound to pseudo identifiers of the same RP can be associated by that RP.


\begin{figure}
  \centering
  \includegraphics[width=\linewidth]{fig/mapping1.pdf}
  \caption{Traditional SSO.}
  \label{fig:TraditionalSSO}
\end{figure}
\begin{figure}
  \centering
  \includegraphics[width=\linewidth]{fig/mapping2.pdf}
  \caption{PPID.}
  \label{fig:PPID}
\end{figure}
\begin{figure}
  \centering
  \includegraphics[width=\linewidth]{fig/mapping3.pdf}
  \caption{SPRESSO.}
  \label{fig:SPRESSO}
\end{figure}
\begin{figure}
  \centering
  \includegraphics[width=\linewidth]{fig/mapping5.pdf}
  \caption{Our scheme.}
  \label{fig:Ourscheme}
\end{figure}
\begin{comment}

{\color{red}
The root reason that existing SSO systems suffer from privacy attacks is that they cannot meet secure authentication and privacy protection requirements
%why the current schemes for privacy-preserving SSO systems cannot deal with the IdP-based access tracing and RP-based identity linkage
at the same time. %is that no existing SSO systems satisfy all the requirements of secure authentication and privacy.
For secure authentication, the identity proof should be bound to a specific RP identifier and a specific user identifier. %The secure authentication requirements must be satisfied by all the SSO systems to avoid the potential attacks.
However, for privacy protection, the privacy-preserving SSO systems require (i) the RP identifier to be one-time and indistinguishable to IdP, and (ii) the user identifier to be globally indistinguishable to IdP but identifiable to the RP. %The ignoring of any privacy requirements is to eventually result in the user being tracked.

Several privacy-preserving schemes have been proposed to meet the privacy requirements while supporting secure authentication, however, they cannot satisfy both privacy requirements at the same time. For example, OIDC adopts PPID as an RP-specific user identifier, but for secure authentication, it has to use a unique RPID to generate the identity proof and thus suffers from IdP-based tracing. On the contrary, SPRESSO encrypts the RP domain to generate a one-time RPID that hides the RP identity against IdP, but it uses static user identifier in identity proof and thus is vulnerable to RP-based linkage. BrowserID takes a different approach by binding the globally unique user identifier with an asymmetric key pair and letting the user to sign RP's domain with the private key. However, the unique signature can be used to link the RPs accessed by the same user in RP-based linkage.
%However, is there a simple way to satisfy the privacy requirements by combining the exiting schemes together, for example using one-time encrypted RPID in OIDC or using PPID in SPRESSO?

The key challenge is that there is no simple way to bind the identity proof to a user's identity without exposing the requesting RP's identity to IdP. To tackle this problem, it requires the IdP to generate pairwise identifiers for a user and bind them to pseudo RP identifiers that seem random to IdP, in a way that user identifiers are unique at one RP but different at different RPs and user identifiers bound to pseudo identifiers of the same RP can be associated by that RP.
%without knowing the RP identifier. Such pairwise user identifiers should be uniquely bound to an .
%so that the pairwise user identifier changes and underivable even in the same RP.
}
\end{comment}

\begin{comment}
However, these two requirements for privacy preservation,  conflict with the basic requirements on SSO as follows: %, resulting in the following problems in SSO systems:
\begin{itemize}
	\item \textbf{No Identification.} In the privacy-preserving SSO, IdP doesn't know the RP's identifier, therefore fails to provide the  pairwise user's identifier which is unchanged in one RP but different in different RPs (e.g. PPID in OIDC and SAML~\cite{OpenIDConnect,SAMLIdentifier}). Also, the user's unique identifier (e.g., the email address in BrowserID~\cite{BrowserID} and SPRESSO~\cite{SPRESSO}) should not be sent to the RP, otherwise the collusive RPs may perform the identity linkage with the same identifier.
%User account is the unique identifier for the RP to provide the individual services for each user.
%In SSO systems, RP derives the user account from the identifier (i.e., PPID in OIDC) provided by the identity proof.
%However, in privacy-preserving SSO systems, IdP must provides the pairwise user identifier (never same for different RPs) without knowing the identity of RP.
%Therefore, IdP is only able to provide different user identifier for user's multiple logins at the same RP, which makes RP fail to provide the consecutive and individual services.

%Each RP provides the individual services for each user based on the unique account. In SSO systems, RP derives the user account from the identifier (i.e., PPID in OIDC) in the identity proof. With the correct RP identifier, IdP ensures the PPID is unique for the user's multiple logins at the same RP. However, when IdP fails to obtain the exact RP identifier, IdP is only able to provide different PPIDs for user's multiple logins at the same RP, which makes RP fail to provide the consecutive and individual services.
	\item \textbf{No Binding.} IdP, who doesn't know the RP's identifier, fails to bind the identity proof with a specified RP.
On receiving the identity proof not bound to it, the RP either (1) rejects the proof and halts its service; or (2) accepts the proof.
The second case will make  one identity proof  be accepted by multiple RPs, which results in the misuse of identity proof for impersonation attacks and identity injection~\cite{ChenPCTKT14,FettKS16,WangZLG16}.
    \item \textbf{No Confidentiality.}
    The potential leakage of the identity proof exists, as:
    (1) No reliable checks (from the IdP and user) during the generation of identity proof, as the IdP lacks the correct RP identifier to retrieve the exact information from the local storage,
     while the user fails to obtain the correct RP name (or URL) from IdP for the check.
     Therefore, the malicious RP may cheat user about its identity to request the identity proof for another RP without being found by the IdP and user.
     (2) Lack of correct URL for the transmission, as without the correct RP identifier, IdP fails to extract the correct (locally stored) URL.
      The trusted user agent may transmit the identity proof to the incorrect URL hold by the adversary.
    The leakage of identity proof may result in the impersonation attacks~\cite{ChenPCTKT14,FettKS16,WangZLG16}.
\end{itemize}
Due to these conflicts, no existing SSO systems satisfy the two privacy requirements.
IdP-based access tracing exists in the implementations based on SAML, OAuth, and OIDC, as IdP knows identifiers accessed by the user.
RP-based identity linkage is not prevented in BrowserID~\cite{BrowserID} and SPRESSO~\cite{SPRESSO}, as the email address is sent to all the RPs.
\end{comment}





\subsection{The Principles of UPRESSO}
\label{subsec:solutions}
In this work, we present UPRESSO to defend against both IdP-based access tracing and RP-based identity linkage privacy attacks with enhanced security for identification, binding and confidentiality. %to satisfy the basic requirements of SSO systems.
Since the integrity requirement for identity proof is orthogonal to the privacy requirement, UPRESSO adopts the same signature-based integrity mechanism of OIDC.
Next, we overview the new privacy-preserving user identification and receiver designation schemes in UPRESSO.


\vspace{1mm}\noindent \textbf{Trapdoor user identification.} The trapdoor identification make the RP possible to transform the user's identity proof into the constant Account in RP, even the $ID_U^{Proof}$ (user identifier in identity proof) is changing with the $PID_{RP}$, which is to prevent the RP-based identity linkage.

\vspace{1mm}\noindent \textbf{Transformed receiver designation.} The pseudonymous receiver uniqueness allows the IdP to generate the identity proof bound to specific RP without knowing any straightforward RP information but a transformed RP identifier, and the identity proof is bound to the transformed identifier. As the IdP does not the RP’s identity, responsibility of url checking should be shifted from IdP to RP, which guarantees the identity proof is only sent to the bound RP.

\begin{comment}
\vspace{1mm}\noindent \textbf{Trapdoor identification.}
The trapdoor identification make the RP possible to transform the user's identity proof into the constant $Account$ in RP, even the $ID_U^{Proof}$ (user identifier in identity proof) is changing with the $PID_RP$, which is to prevent the RP-based identity linkage.
The entity with the trapdoor can derive the unique user account associated with an RP from different identifiers in multiple identity proofs generated when the user logs in to that RP multiple times.
%The trapdoor-existing identification requires that the user identifier generated by IdP should be never same in each authentication as the RP's identity is unknown to IdP and the identifier should be derivable to the specific account unchanged in RP with the trapdoor.
%In UPRESSO, the user's account at an RP is a function of the RP's original identifier and the user's unique identifier.
%The generation of this unique user account consists of two steps executed by the IdP and RP independently to prevent not only the IdP from obtaining the RP identifier but also the RP from inferring the unique user identifier.
%In the first step, IdP generates a PPID with the user identifier and the RP identifier transformation. Since the RP identifier transformations are independent in multiple login flows, the PPIDs generated by the IdP in these flows are different.
%which results in different PPIDs for multiple logins at a RP as various transformations are adopted.
%In the second step, the requesting RP adopts the trapdoor of the transformation of the RP identifier to derive the unique user account from the PPIDs. Moreover, the accounts of a user at various RPs are different, as the original RP identifiers are different, which prevents the identity linkage.

\vspace{1mm}\noindent \textbf{Transformed binding.}
The transformed binding allows the IdP to generate the identity proof bound to specific RP without knowing any straightforward RP information.
The identity proof is bound to the transformation of a RP identifier, which allows the RP to check whether the identity proof is for it while preventing the IdP from inferring the original RP identifier. Only the identity proof bound to a fresh transformation of its identifier will be accepted by the RP.
%The transformed binding requires that the RP should provide the one-time transformed identifier, unlinkable to the RP's unique identifier without trapdoor and the identity proof should be bound with the one-time identifier.
%In UPRESSO, the transformation of the RP identifier is generated by the user and RP cooperatively, which prevents the adversary from constructing a same or related transformation for different RPs. The IdP obtains the transformation of an RP identifier and binds it to the identity proof.
%the identity proof is bound with a transformation of the RP's identifier by the IdP, who cannot infer the original identifier from the transformation.

%Moreover,
%Otherwise, the identity proof will be accepted by multiple RPs, which results in the misuse of the proof for the impersonate attack.

\vspace{1mm}\noindent \textbf{User-centric confidentiality.}
User-centric confidentiality shifts the responsibility of url checking from IdP to user agent, which guarantees the identity proof is only sent to the bound RP. The user-centric confidentiality should relies on the trustful user agent and IdP issued RP identity certificate.
%In UPRESSO, the user sends the authentication request to the IdP to generate an identity proof for the requested RP identifier transformation. To ensure the identity proof is securely returned to the requesting RP, UPRESSO adopts a user-centric confidentiality check, which requires the user to extract RP's identifying information (e.g., URL, name and RP identifier) from the RP certificate issued by a trusted entity (e.g., IdP). Therefore, it is impossible for an adversary to request the identity proof on behalf of others or intercept others' identify proofs with a forged return URL.
\end{comment}

To combine these three principles into one system, we propose novel identifier generation schemes to dynamically generate transformed RP identifier and trapdoor identity proof, which makes the RP's information never exposed to the IdP during the authentication and RP able to derive the unique user $Account$ from identity proof. Moreover, both the RP identifier transforming and identity proof transformation (including url check) require the trustful user agent. We do not explain how to finish each principle here, it will be explained in Section \ref{subsec:overview}.
%\vspace{1mm} \textbf{xxx xxxxx xxxxxxx
%(1) How to combine these three principles into one system?
%(2) Why these principles protect privacy while ensure security?
%(3) we do not explain how to finish each principle here, it will be explained in Section \ref{subsec:overview}.}










\section{Assumption and Threat Model}
\label{sec:assumptionandthreatmodel}
Recluse contains only three entities, i.e., the user, IdP and RP; and doesn't introduce any other (trusted) entity. In addition to the conventional processes (described in Section~\ref{subsec:OIDC}) as in the typical SSO systems~\cite{SAMLIdentifier,OpenIDConnect}, extra processes are required on the entities in Recluse for preserving the user's privacy:
\begin{itemize}
  \item User.
  \item IdP
  \item RP. 
\end{itemize}

\noindent\textbf{Assumption.}

\noindent\textbf{Threat model}
\begin{itemize}
  \item Security.
  \item Privacy
\end{itemize}

under which the adversary attempts to break the security and privacy of SSO systems
In SSO systems, IdP has the max authority in this system. Therefore, IdP should be considered honest but curious. Otherwise, an malicious IdP has the ability to log in to any RP as any honest user (impersonation attack) and enforce any honest user to log in honest RP under an adversary's identity (identity injection). Moreover, a user's login trace is never hidden from collusion between IdP and RP. It is considered that any RP could be corrupted and any user may be the adversary. User agent is considered completely honest but under control of the user. Therefore, the user agent is seemed as a part of user. Moreover, as network flows are protected by various ways, such as TLS, the network attacker is not considered. The ability of each entity acted by adversary are shown as follows:
\begin{itemize}
\item \textbf{Curious IdP} acts as an completely honest IdP.
\item \textbf{Malicious RP} has the ability to build any response, as well as the authentication request, for user's requestion.
\item \textbf{Malicious User} is able to intercept and tamper all the data transmitted through itself.
%\item \textbf{Network Attacker} has the ability to listen all the IP address on the Internet but unable to tamper any the network flows as they are protected by various ways, such as TLS.
\end{itemize}

To explicitly illustrate how an adversary works in the SSO system, the \verb+authentication flow+ is created to defined the authentication of specific IdP,  RP and user. For example, now there are $IdP$, $User_{A}$, $User_{B}$, $RP_{A}$ and $RP_{B}$, who are able to form 4 \verb+authentication flow+s, ($IdP$, $User_{A}$, $RP_{A}$), ($IdP$, $User_{A}$, $RP_{B}$), ($IdP$, $User_{B}$, $RP_{A}$) and ($IdP$, $User_{B}$, $RP_{B}$). An adversary has the ability to act one or more entities in single or multiple \verb+authentication flow+s. That is, an adversary is able to act as, i) the single entity in one \verb+authentication flow+, such as the curious IdP; ii) the same entity in multiple \verb+authentication flow+s, such as acting as different RPs for the same honest user ; iii) the different entities in multiple \verb+authentication flow+s, such as acting as the RP for the honest user and the user for the honest RP at the same time. However, it is considered an adversary should not act as both the IdP and RP in single \verb+authentication flow+.

%However, Identity Injection only occurs when 1) IdP is dishonest; 2) the transmission between RP and IdP is corrupted by either corrupted user agent or unprotected network flows. Therefore, Identity Injection is not considered.

\section{Recluse}
\label{sec:Recluse}
\begin{figure}
  \centering
  \includegraphics[width=\linewidth]{fig/Overview.pdf}
  \caption{Overview of Recluse.}
  \label{fig:overview}
\end{figure}


The overview of login flow is shown in Figure~\ref{fig:overview}, which contains RP identifier negotiation, dynamic registration and token obtaining.

The of each phase in login flow is shown as follows:
\begin{itemize}
\item[1.] RP Identifier Negotiation: The negotiation is required to generate the RP identifier. For each SSO procedure, user is going to start negotiation with user. RP identifier is a random number which does not represent any RP, generated by rp-id-generating algorithm. However, the identifier is bound with specific authentication which is able to be confirmed by user and RP. The details of identifier generation is shown in Section~\ref{sec:identifier-generation}.
\item[2.] Dynamic Registration: Dynamic registration is required to in Recluse to make sure the identifier generated by negotiation is valid in IdP. In OIDC system, RP should register its attributes (e.g., the endpoint for identity proof) with IdP and obtain the RP identifier generated by IdP so that IdP is able to authenticate the user for specific RP. Therefore, to make sure the newly generated RP identifier in RP Identifier Negotiation is valid in IdP, the dynamic registration is required before the authentication request is transmitted to IdP. IdP is to check whether the identifier is unique and require RP to restart identifier negotiation if the identifier has be used by another RP.
%To make the RP identifier generated by negotiation is valid in IdP, user is to register this identifier with IdP through the dynamic registration API provided by IdP. IdP is going to check whether the identifier is unique and require RP to restart identifier negotiation if the identifier has be used by another RP.
\item[3.] Authentication: After dynamic registration, RP builds the authentication request and redirects it to IdP through user agent. After receiving the request, IdP firstly authenticates user and then issues identity proof for RP, which contains the user id generated through the user-id-generating algorithm. Then IdP redirects the identity to RP through user agent, and RP identify the user through identity proof. The methods of user identifier generation and RP identifying the user are shown in Section~\ref{sec:identifier-generation}.
\end{itemize}

However, in addition to the login flow in each authentication, the initial registration is required before the RP and user integrates the Recluse service. The initial registration allows the RP to achieve the necessary parameters from the IdP and user to sign on in the IdP. In Recluse system, the initial registration is conducted by each RP and user only once, however, the login flow is conducted in each authentication integrally each time.



\subsection{Initial Registration}
The initial registration between RP and IdP is shown as Figure~\ref{fig:registration}.
\begin{figure}
  \centering
  \includegraphics[width=\linewidth]{fig/registration.pdf}
  \caption{Prior Registration}
  \label{fig:registration}
\end{figure}
The registration process is as follows:
\begin{itemize}
\item[1.] Uploading Attributes: Firstly, IdP generates its prime $P$, the primitive root $g$ of $P$, used for RP and user identifier generation and the key pair $pk$, $sk$.
\item[2.]The RP uploads it attributes, such as its name, endpoint, identity proof (e.g., business license) and so on.
\item[3.] Issuing RP Certification: IdP verifies the identity of RP and generates the RP certification including $basic\_rp\_id$, $rp\_name$, \verb+redirect_uri+ and \verb+IdP_origin+.  The RP certification is used for user agent to verify the basic attributes, such as $basic\_rp\_id$ and \verb+redirect_uri+.
\item[4.]IdP returns the RP certification, $P$, $g$ and $pk$ to RP.
\end{itemize}

The initial registration required for IdP to verify the basic attributes of RP, such as name, endpoints for identity proof, so that Idp is able to provide the RP certification to RP which includes the unique identifier for each RP and its attributes. With the RP certification, user agent has the ability to verify the RP's endpoint for identity proof and notify user with RP's identity. Additionally, the parameters, prime $P$ (used for user id generating) with its generator $g$, public key of IdP $pk$ is provided in registration as well.
Same as RP, user need to register with IdP and IdP generates unique user id for each user.


\subsection{Rp-id-generating and User-id-generating algorithm}
\label{sec:identifier-generation}
The rp-id-generating and user-id-generating algorithm are created based on Discrete Logarithm problem\cite{shiu2007cryptography:}.
IdP carefully chooses a big prime $P$\footnotetext[1]{$P$ is generated as $P=q\cdot 2+1$, while $q$ is prime as well.} and its primitive root \verb+g+ as generator for system. When the RP registers with IdP, IdP provides a unique primitive root as the RP's root identifier (called $basic_rp_id$).

\begin{comment}
The generation of $rp_id$ and $user_id$ is shown as Figure~\ref{fig:generating}, as well as the trapdoor for RP to derive $user_rp_id$ is as shown.

\begin{figure*}
  \centering
  \includegraphics[width=\linewidth]{fig/generating3.pdf}
  \caption{Generation of rp\_id and user\_id}
  \label{fig:generating}
\end{figure*}
\end{comment}

For each login process, the user and RP negotiate the temporary RP identifier bound with specific authentication.
\subsubsection{$rp\_id$ generating}
While starting a login procedure, there is Diffie-Hellman key Exchange\cite{DiffieH76} between RP and user, through which the random $r$ is generated. However, to make sure that there is $r^{-1}$, that $r\cdot r^{-1}=1 mod \phi(P)$, $r$ should be the relative prime of $\phi(P)$, so that if $r$ is even $r$ should be added by one. Although there is little possibility that $r$ is the multiple of $p$ or $q$, it is not considered in the illustration. However, the re-negotiation is required in the practical system if $r$ is the multiple of $p$ or $q$. The RP identifier is generated as:
$$rp\_id=basic\_rp\_id^r mod P\eqno(1)$$
such that $rp_id$ is another primitive element module $p$. And $r^{-1}$ is generated through Extended Euclidean algorithm.
\subsubsection{$user\_id$ generating}
IdP labels each user at IdP with the unique identifier called $baisc\_user\_id$. To generate the specific user identifier for each $rp\_id$, the algorithm is
$$user\_id=rp\_id^{basic\_user\_id} mod P\eqno(2)$$
so
$$user\_id=basic\_rp\_id^{r\cdot basic\_user\_id}modP\eqno(3)$$
\subsubsection{Trapdoor}
While receiving $user\_id$ from IdP, RP is able to derive the constant user identifier from if
$$user\_rp\_id=user\_id^{r^{-1}} mod P\eqno(4)$$
so
$$user\_rp\_id=basic\_rp\_id^{(1 mod \phi(P))\cdot basic\_user\_id} mod P\eqno(5)$$
so
$$user\_rp\_id=basic\_rp\_id^{basic\_user\_id} mod P\eqno(6)$$
For single user in a RP, $user\_rp\_id$ is unchanged. However, $user\_rp\_id$s are distinct in each RP because $basic\_rp\_id$s are different in each RP.



\subsection{Login Flow}
The login flow is shown as Figure~\ref{fig:process}, in which the $rp\_id$ is generated in step 6, the $user\_id$ is generated in step 15 and the RP derives the $user\_rp\_id$ in step 18.
\begin{figure*}
  \centering
  \includegraphics[width=\linewidth]{fig/process.pdf}
  \caption{Login Flow}
  \label{fig:process}
\end{figure*}

%抵抗phishing攻击:一定需要正确的RP参与,攻击者作为中间人
%1.使用IdP提供应用basic_rp_id与url的绑定,user agent保存映射
%缺点:占用空间,user agent需要缓存整个映射
%2.RP与用户通过加密的通道传输redirect_uri

%如果由RP选择basic_rp_id,那么多个rp之间的basic_rp_id有幂次关系,那么就可以关联用户

\subsubsection{RP Identifier Negotiation}
RP identifier negotiation starts form step 1 to step 7. The user accesses the service provided by RP in his/her browser. To log in this RP, user needs to click the login button offered by Recluse.
Firstly, the user agent sends the \verb+Start Negotiation+ request to RP, so that RP generates the random $sk\_rp$ and $pk\_rp=g^{sk\_rp}modP$ as the private key and public key for DH Key exchanging.
Secondly, RP builds the \verb+Negotiation Response+ with newly generated $pk\_rp$ as well as the \verb+RP_Cert+ issued by IdP. User agent similarly generates random $sk\_user$ and $pk\_user$, and $r=pk\_rp^{sk\_user}modP$. However, to make sure that $r$ is the relative prime of $\phi(P)$, it is required that $r$ should be odd and the greatest common divisor of $r$ and $\phi(P)$ is 1.
Then user agent continues the \verb+Negotiation+ sending $pk\_user$ and $r$ to RP. RP generates the local $r$ in the same way as user agent and compares the local $r$ and user agent generated $r$. If $r$s are equal, RP generates $rp\_id=basic\_rp\_id^rmodP$, as well as $r^{-1}$ through Extend Euclidean algorithm, which meets $r\cdot r^{-1}=1mod\phi(P)$.
Finally RP transmits the $rp\_id$ to user agent.
\subsubsection{Dynamic Registration}
Dynamic registration is from step 8 to step 13. While user agent receives teh $rp\_id$ from RP, it is required the $rp\_id$ from RP should be equal with it generated by user agent. Then user agent generates the \verb+fake_uri+ which contains the random string and keeps it for further identity proof transmission. User agent sends the \verb+Dynamic Registration+ request to IdP with newly generated $rp\_id$ and \verb+fake_uri+ and redirects the \verb+Dynamic Registration Response+ to RP.
\subsubsection{Authentication}
Authentication is from step 14 to step 19. After dynamic registration,  RP builds the \verb+Authentication Request+ including $rp\_id$ as well as the \verb+redirect_uri+ representing the endpoint, and redirects it to IdP through user agent. User agent tampers the authentication request, compares $rp\_id$ with the local one, verifies the validation of the \verb+redirect_uri+ and replaces it with the fake one. Then user agent transmits the \verb+Authentication Request+ to IdP. After receiving the request, IdP firstly authenticates user and then generates $user\_id=rp\_id^{basic\_user\_id}modP$. The identity proof signed with IdP's private key including the $user\_id$ is redirected to the \verb+fake_uri+ through user agent, who intercepts the transmission and transmit it to the endpoint \verb+redirect_uri+ in authentication request. Finally, RP derives the constant $user\_rp\_id$ from $user\_id$. If the $user\_rp\_id$ has already been registered, RP send \verb+Authentication Finished+ with the message \verb+success+ to user agent.

%描述user_id, rp_id and user_rp_id的生成


\section{Analysis}
\label{sec:analysis}
In this section, we firstly prove the privacy of Recluse, i.e., avoiding the identity linkage at the colluded malicious RPs, 
and preventing the curious IdP from inferring the user's accessed RPs.
Then, we prove that Recluse does not degrade the security of SSO systems by comparing it with OIDC, which has been formally analyzed in~\cite{FettKS17}.


\subsection{Privacy}
\label{subsec:privacy}
\noindent{\textbf{Curious IdP.}} The curious IdP might be interested in the user accessed RP or infer the correlation of RPs in two or more login flows by performing the analysis on the content and timing of received messages. However, it fails to obtain the user's accessed RPs directly, nor classifies the accessed RPs for RP's information indirectly.
%The curious IdP can only perform the analysis on the content and timing of received messages, however it fails to obtain the user's accessed RPs directly, nor infer classifies the accessed RPs for RP's information indirectly.
\begin{itemize}
  \item The curious IdP might be interested in the RP's identity but fails to derive RP's identifying information (i.e., $RPID$ and correct endpoint) through a single login flow. IdP only receives $PRPID$ and one-time endpoint, and fails to infer the discrete logarithm (i.e., $RPID$) from $PRPID$ without the  trapdoor $t$  due to hardness of solving discrete logarithm, or the RP's endpoint from the independent one-time endpoint.
  \item It also might try to infer the correlation of RPs in two or more login flows but fails achieve the relationship between the $PRPID$s. The secure random number generator ensures the random for generating $PRPID$ and the random string for one-time endpoint are independent in multiple login flows. Therefore, curious IdP fails to classify the RPs based on $PRPID$ and one-time endpoint.
  %It also might try to achieve the relationship between the $PRPID$s but fails to infer the correlation of RPs in two or more login flows from a single user or multiple users. The secure random number generator ensures the random for generating $PRPID$ and the random string for one-time endpoint are independent in multiple login flows. Therefore, curious IdP fails to classify the RPs based on $PRPID$ and one-time endpoint.
  \item It even fails to obtain the correlation of RPs through analyzing the timing of received messages. IdP fails to map user's accessed RP in the identity proof to the origin of dynamic registration based on timing, as both the dynamic registration and the identity proof request are sent by the user instead of the RP.
\end{itemize}

\noindent{\textbf{Malicious RPs.}} The malicious RPs may attempt to link the user passively by combining the $PUID$s received by the colluded RPs, or actively by tampering with the provided elements (i.e., $Cert_{RP}$, $Y_{RP}$ and $PRPID$). However, these RPs still fail to obtain the $UID$ directly, or trigger the IdP to generate a same or derivable $PUID$s.
%The malicious RPs may attempt to link the user passively by combining the PPIDs received by the colluded RPs, or actively by tampering with the provided elements (i.e., $Cert_{RP}$, $Y_{RP}$ and $PRPID$). However, these RPs still fail to obtain the user's unique identifier directly, or trigger the IdP to generate a same or derivable PPIDs.
\begin{itemize}
\item A single RP might try to find out the $UID$ presenting the unchanged user identity but fails to infer the user's unique information (e.g., $UID$ or other similar ones) in the passive way. The $PUID$ is the only element received by RP that contains the user's unique information. However, RP fails to infer (1) $UID$ (the discrete logarithm) from $PUID$, due to hardness of solving discrete logarithm; (2) or $g^{UID}$ as the $r$ in $RPID_O=g^r$ is only known by IdP and never leaked, which prevents the RP from calculating $r^{-1}$ to transfer $Account=RPID^{UID}$ into  $g^{UID}$.
\item A single RP fails to actively tamper with the messages to make $UID$ leaked. The modification of  $Cert_{RP}$ will make the signature invalid and be found by the user. The malicious RP fails to control the calculation of $PRPID$ by providing an incorrect $Y_{RP}$ as another element $n_u$ is  controlled by the user. Also, the malicious RP fails to make an incorrect $PRPID$ (e.g., 1)  be used for $PUID$, as the honest IdP only accepts a primitive root as the $PRPID$ in the dynamic registration. The RP also fails to change the accepted $PRPID$ in Step 11 in Figure~\ref{fig:process}, as the user checks it with the cached one.
\item Two or more RPs might try to find out whether the $Account$s in each RP are belong to one user or not but fail to link the user in the passive way. The analysis can only be performed based on $Account$ and $PUID$. The $Account$ is independent among RPs, as the $RPID$ chosen by honest IdP is random and unique. The $PUID$s are  also independent due to the unrelated $PRPID$.
\item Two or more RPs also might to lead IdP to generate the $PUID$ same or  derivable into same $Account$ in each RP. Since the $PUID$ is generated related with the $PRPID$, corrupted RPs might choose the related $n_{RP}$ to correlate their $PRPID$, however, the $PRPID$ is also generated with the participation of $n_{U}$, so that RP does not have the ability to control the generation of $PRPID$. Moreover, corrupted RPs might choose the same $PPID$ to lead the IdP to generate the $PUID$ derivable into same $Account$, however, $RPID$ is verified by the user with through the $Cert_{RP}$, where the tampered $RPID$ is not acceptable to the honest user.
%achieve it in the active way, by attempting to make $PRPID$ correlative by manipulating $n_{RP}$ or use the same $RPID$. However, the random $n_u$ chosen by the honest user will ensure the independence of $PRPID$ to protect its own privacy.

\end{itemize}

Colluded RPs even fail to correlate the users based on the timing of users' requests, when the provided services are unrelated. For the related services, (e.g., the online payment accessed right after an order generated on the online shopping), the user may break this linking by adding an unpredicted time delay between the two accesses. The anonymous network may be adopted to prevent colluded RPs to classify the users based on IP addresses.

\begin{comment}
\begin{itemize}
  \item A \textbf{curious} RP fails to infer the user's unique identifier (i.e., UID) through $PUID$.
   \item A \textbf{colluded curious} RP fails to link a user between RPs.
  \item A \textbf{malicious} RP fails to make the UID leaked.
  \item The \textbf{colluded malicious RPs} fail to (actively) make the UID leaked.
  \item The \textbf{colluded malicious RPs} fail to actively trigger the generation of a same PPID or derivable PPIDs.
  \item The \textbf{colluded malicious RPs} fail to passively link a user between RPs.
\end{itemize}
\end{comment}

\subsection{Security}
\label{subsec:security}
Recluse protects the user's privacy without breaking the security. That is, Recluse still prevents the malicious RPs and users from breaking the identification, integrity, confidentiality and binding of identity proof.

In Recluse, all mechanisms for integrity are inherited from OIDC. The IdP uses the un-leaked private key $SK_{ID}$ to prevent the forging and modification of identity proof. The honest RP (i.e., the target of the adversary) checks the signature using the public key $PK_{ID}$, and only accepts the elements protected by the signature.

For the confidentiality of identity proof, Recluse inherits the same idea from OIDC, i.e., TLS, a trusted user agent and the checks. TLS avoids the leakage and modification during the transmitting. The trusted agent ensures the identity proof to be sent to the correct RP based on the endpoint specified in the $Cert_{RP}$. The  $Cert_{RP}$ is protected by the signature with the un-leaked private key $SK_{Cert}$, ensuring it  will never be tampered with by the the adversary. For Recluse, the checks at the IdP is exactly the same as OIDC, that is, checking the RP identifier and endpoint in the identity proof with the registered ones, preventing the adversary from triggering the IdP to
generate an incorrect proof or transmit to the incorrect RP. However, the user in Recluse performs a two-step check instead of the direct check based on the RPID in OIDC. Firstly, the user checks the correctness of $Cert_{RP}$ and extracts  $RPID$ and the endpoint. In the second step, the user checks that the RPID in identity proof is a fresh $PRPID$ negotiated based on the $RPID$ and the endpoint is the one-time one corresponding to the one in $Cert_{RP}$. This two-step check also ensures the identity proof for the correct RP ($RPID$) is sent to correct endpoint (one specified in $Cert_{RP}$).

The mechanisms for binding are also inherited from OIDC. The IdP binds the identity proof with $RPID$ and $PUID$. The correct RP checks the binding by comparing the $PRPID$ with the cached one, and provides the service  to the $Account$ based on $PUID$.

Recluse binds the identity proof with $PRPID$, instead of a random string unique for each RP assigned by IdP in OIDC. However, the adversary (malicious users and RPs) still fails to  make one identity proof (or its transformation) accepted by the other correct RP. As the correct RP only accepts the valid identity proof for its fresh negotiated $PRPID$, we only need to ensure one $PRPID$ (or its transformation) never be accepted by the other correct RP.
\begin{itemize}
\item $PRPID$ is unique in one IdP. The honest IdP checks the uniqueness of $PRPID$ in its scope during the dynamic registration, to avoid one $PRPID$ (in its generated identity proof) corresponding to two or more RPs.
\item The mapping of $PRPID$ and \verb+issuer+ globally unique. The identity proof contains the identifier of IdP (i.e., \verb+issuer+), which is checked by the correct RPs. Therefore, the same $PRPID$ in different IdPs will be distinguished.
\item The $PRPID$ in the identity proof is protected by the signature generated with $SK_{ID}$. The adversary fails to replace it with a transformation without invaliding the signature.
\item The correct RP or user prevents the adversary from manipulating the $PRPID$. For extra benefits, the adversary can only know or control one entity in the login flow (if controlling the two ends, no victim exists). The other correct one provides a random nonce ($n_u$ or $n_{RP}$) for $PRPID$. The nonce is independent from the ones previously generated by itself  and the ones generated by others, which prevents the adversary controlling the $PRPID$.

\end{itemize}

Recluse ensures the identification by binding the identity proof with $PUID$  in the form of $RPID^{UID}$, instead of a random string generated by the IdP. However, the adversary still fails to login at the correct RP using a same $Account$ as the uncontrolled user. Firstly, the adversary fails to  modify the $PUID$ directly in the identity protected by $SK_{ID}$. Secondly, the malicious users and RPs fail to trigger the IdP generate a wanted $PUID$, as they cannot (1) obtain the uncontrolled user's $PUID$ at the correct RP; (2) infer the $UID$ of any user from all the received  information (e.g., $PUID$) and the calculated ones (e.g., $Account$); and (3) control the $PRPID$ with the participation of a correct user or RP.

The design of Recluse makes it immune to some existing know attacks (e.g., CSRF, 307 Redirect, IdP Mix-Up~\cite{FettKS16} and Man-in-middle attack) on the implementations. The Cross-Site Request Forgery (CSRF) attack is  usually exploited by the adversary to perform the identity injection. However, in Recluse, the correct user logs  $PRPID$ and one-time endpoint in the session,  and perform the checks before sending the identity proof to the RP's endpoint, which prevents the CSRF attack.  The 307 Redirect attacks~\cite{FettKS16} is due to the implementation error at the IdP, i.e. returning the incorrect status code (i.e., 307), which makes the IdP leak the user's credential to the RPs during the redirection. In Recluse, the redirection is intercepted by the trusted user agent which removes these sensitive information. In the IdP Mix-up attack, the adversary works as the IdP to collect the makes \verb+access token+ and \verb+authorization code+ (identity proof in OAuth 2.0) from the victim RP. Same as OIDC, Recluse includes the \verb+issuer+ in the identity proof (protected by the $SK_{ID}$), avoiding the victim RP to send the sensitive information to the IdP. The user established the TLS connection with RP and IdP, which avoids the Man-in-middle attack.

\begin{comment}
In order to prove the privacy and security properties of Recluse system, we firstly demonstrate that for any adversary in the system, a user's access to an specific  RP untraceable from it to another one. Besides, we illustrate the potential attacks discussed in the previous work about SSO security and the security issues introduced by Recluse.


\subsection{Privacy}
We now define the undistinguishability of SSO system. It is in the SSO system, for an adversary controlling curious IdP, it is impossible to inspect whether two \verb+authentication flow+s are to same RP or not, however, for an adversary controlling malicious RPs, it is impossible to inspect whether two \verb+authentication flow+s are from same user or not.

Assuming there are two \verb+authentication flow+s from the same user to the same RP. In Recluse system, for the curious IdP, the only parameter related with RP is $rp\_id$, as other parameters are related or generated by user, such as $fake\_uri$. To break the undistinguishability, IdP tries to derive the $basic\_rp\_id$ of RP from $rp\_id$ or inspect whether the $rp\_id$s are generated by the same $basic\_rp\_id$. However, as $rp\_id$ is generated by the formula (1), so
$$basic\_rp\_id=rp\_id^{r^{-1}}modP\eqno(7)$$
However, $r$ and $r^{-1}$ is unknown to the IdP, so that IdP is unable to derive the $basic\_rp\_id$ from the $rp\_id$. Moreover, even the IdP suspects that the $rp\_id$ is generated by the specific RP, it is impossible for IdP to verify it as the  $rp\_id$ can be generated based on any primitive root of $P$. Besides, assuming that the $rp\_id$s in different \verb+authentication flow+s are $rp\_id_1$ and $rp\_id_2$ generated by $r_1$ and $r_2$. There is
$$rp\_id_1=rp\_id_2^{r_1/r_2}modp\eqno(8)$$
So only the entity who carries the $r_1$ and $r_2$ is able to verify whether $rp\_id_1$ and $rp\_id_2$ generated by the same RP.

Inspecting whether the users in two \verb+authentication flow+s are the same user relies on the $baisc\_user\_id$ or the relation between $user\_id$s or $user\_rp\_ids$. Assuming the same user log in different RPs where the $user\_id$s are $user\_id_1$ and $user\_id_2$, $user\_rp\_id$s are $user\_rp\_id_1$ and $user\_rp\_id_2$, and $basic\_rp\_id$s are $basic\_rp\_id_1$ and $basic\_rp\_id_2$. We define that $\alpha=\log_{basic\_rp\_id_2}basic\_rp\_id_1$. There is
$$user\_rp\_id_1=user\_rp\_id_2^\alpha\eqno(9)$$
and
$$user\_id_1=user\_id_2^{\alpha\cdot r_1/r_2}\eqno(10)$$
The $\alpha$ is unknown to the malicious, so that the adversary is unable to  inspect whether the two \verb+authentication flow+s are from the same user. However, the malicious also tries to lead the same user in two \verb+authentication flow+s using the same $user\_rp\_id$ or $user\_id$. According to formula (2) and (4), the user should use the same $basic\_rp\_id$ or $rp\_id$ in two \verb+authentication flow+s. So the $user\_rp\_id$ is impossible to be same as $basic\_rp\_id$ is issued by IdP and verified by user agent. And $rp\_id$ is generated through the negotiation between user and RP, so that RP is unable to lead the user to use the same $rp\_id$ in different \verb+authentication flow+s.

\subsection{Security Consideration in OpenID Connect}
As it has been defined in ~\cite{OpenIDConnect} Section 16, the security consideration of the OIDC design contains authentication and authorization. Now we list the security consideration of authentication and prove that Recluse achieves this goals.
\begin{enumerate}
\item \textbf{Server Masquerading. } The malicious server might masquerade as the RP or IdP using various ways through which user might leak its identity proof for RP or credentials on IdP. The Recluse mitigate this threat in following ways. 1) The server visited through user agent is authenticated by HTTPS; 2) The user agent verifies the RP certification which makes sure that the token is only sent to the corresponding RP instead of the masqueraded one; 3) The user agent only visit the IdP's endpoint listed in the RP certification which avoid the access to the masqueraded IdP.
\item \textbf{Token Manufacture/Modification. } The adversary might generate a bogus token or tamper the contents in the existing token, which enables the adversary log in any RP as any honest user. The receiver of token must have the ability to verify whether the token is issued by IdP without any modification or not. The token used in Recluse is signed by IdP with its private key, so that the token is unable to be constructed or modified.
\item \textbf{Access/ID Token Disclosure. } The adversary might try to obtain an honest user's token to the honest RP through various ways, which enables the adversary log in this RP as the honest user. It is required the tokenisi never exposed to anyone except the corresponding user and RP.
\item \textbf{Access/ID Token Redirect. } The malicious RP might use the token from an honest user to access other RPs as this user only if the token is also valid in other RPs. It is required the token issued for specific RP should bound with this RP which means the RP has the ability to verify whether a token is valid in itself. The token used in Recluse containing the $rp\_id$ and $user\_id$ is solely bound with specific RP and user, which is checked by the RP.
\item \textbf{Issuer Identifier. } The issuer identifier contained in the token should be completely same as it provided by the IdP, so that the verifier of token has the ability to obtain the corresponding public key. It is also implemented in Recluse.
\item \textbf{TLS Requirements. } To avoid network attacker the transmission in OIDC system should be protected by TLS. It is implemented in Recluse.
\item \textbf{Implicit Flow Threats. } In OIDC implicit flow, token is transmitted through user agent and the TLS protections are only between user agent and IdP, and between user agent and RP. It is required the token should not be leaked to the adversary by the user agent. The user agent of Recluse is deployed based on the Chrome browser as the trust base.
\end{enumerate}

It is discovered that the security consideration of authentication in OIDC can be included into the security consideration in Section~\ref{sec:background} as table ~\ref{tab:security-consideration}. however, the threats introduced by Recluse which breaks the security consideration and the methods to mitigate these threats has also been discussed in Section~\ref{sec:overview}.


\begin{table}
\label{tab:security-consideration}
\caption{Security Consideration}
\begin{tabular}{|c|c|}
\hline  % 在表格最上方绘制横线
Security Consideration of SSO & Security Consideration of OIDC\\
\hline  %在第一行和第二行之间绘制横线
Content Checking& \multicolumn{1}{p{120pt}|}{Server Masquerading}\\
\hline
Confidentiality& \multicolumn{1}{p{120pt}|}{Server Masquerading, Access/ID Token Disclosure, TLS Requirements, Implicit Flow Threats}\\
\hline
Integrity& \multicolumn{1}{p{120pt}|}{Token Manufacture/Modification, Issuer Identifier}\\
\hline
Binding& \multicolumn{1}{p{120pt}|}{Access/ID Token Redirect}\\
\hline
\end{tabular}
\end{table}

\subsection{Related Security Analysis}
\noindent\textbf{307 Redirect. }It has been discussed in~\cite{FettKS16} that IdP might redirect the user to the RP immediately after the user inputs the credentials. For example, the HTTP response to the user's POST message with \verb+username+ and \verb+password+ might be the redirection to RP carrying user's identity proof. That is, as long as the 307 status code is used for this redirection, the user's credentials are also transmitted to the RP. However, in Recluse the redirections are intercepted by the user agent and rebuild the HTTP GET request to RP or IdP which is unable to leak the POST data of the user.

\noindent\textbf{IdP Mix-Up. } It has been illustrated in~\cite{FettKS16} that the adversary might intercept the the user's access to RP and modify the user's choice of IdP, assuming that RP integrates the SSO service provided by both the honest IdP (named \verb+HIdP+) and malicious IdP (named \verb+MIdP+). The user obtains the \verb+access token+ and \verb+authorization code+ from the \verb+HIdP+ and uploads them to the RP, however, the RP considers that the \verb+access token+ and \verb+authorization code+ are issued by the \verb+MIdP+. Therefore, the RP tries to exchange for the protected resources using the \verb+access token+ and \verb+authorization code+ with \verb+MIdP+. With this, the adversary carries the honest user's \verb+access token+ and \verb+authorization code+ which make the adversary able to log in this RP as the identity of the honest user. However, the threat is mitigated by the OIDC implicit flow, as the \verb+ID token+ issued by IdP contains the issuer identifier, so that RP is able to find out which IdP the token is from.

\noindent\textbf{Cross-Site Request Forgery (CSRF). } The CSRF attack on the RP makes the identity injection possible, through which the adversary might lead the honest user to upload the adversary's \verb+ID token+ to the RP. However, in Recluse the cross origin request should be repudiated by both RP and IdP excepted the request from the origin of the user agent. Therefore the CSRF attack on Recluse is impossible.

PRIOIDC tries to protect user's privacy by keeping RP anonymous to IdP. IdP is able to get client\_id and redirect\_uri. As redirect\_uri is generated by user, it will show nothing about RP. IdP can only undermine user's privacy by get RP's identity from client\_id. It's described in Client-id-generating algorithm: $client\_id=basic\_rp\_id^r mod p$. $p$ is a large prime and basic\_rp\_id is a primitive element module $p$. And $r$ is the random number generated by user and RP. IdP can only find out RP's real identity by finding out $r^{-1}$ and let $1=r\cdot r^{-1} mod (p-1)$, so that $$basic\_rp\_id=client\_id^{r^{-1}}modp$$
But $r$ is secret shared by user and RP, and according to \textbf{Discrete Logarithm} problem calculating $r$ from client\_id is difficult. So basic\_rp\_id is invisible to IdP. In other way if IdP gets a user's repeatedly login, it is going to find out whether they are about the same RP. If there are two client\_ids from the same RP marked as $client\_id_{1}=basic\_rp\_id^{r_1}modp$ and $client\_id_{2}=basic\_rp\_id^{r_2}modp$. Client\_id$_{1}$ and client\_id$_{2}$ meet the following formula
$$client\_id_1=client\_id_2^{r_2/r_1}modp$$
So that only when knowing $r_1$ and $r_2$ IdP can find out the relevance between Client\_id$_{1}$ and client\_id$_{2}$. But $r_1$ and $r_2$ are invisible to IdP. So IdP is never able to undermine user's privacy.

RPs try to find out user's login trace in three ways: 1) Getting the user's unique id in IdP. 2) Finding the relevance among user\_rp\_ids. 3) Deducing user's login trace from IP address. As user's id is used in generating user\_id in id\_token, RP is able to obtain $user\_id=client\_id^{id}modp$. Client\_id is primitive element module $p$. Although client\_id, user\_id and $p$ are known by RP, according to \textbf{Discrete Logarithm} problem calculating id from user\_id is difficult. For different RPs, they are able to get user's user\_rp\_id. User\_rp\_ids from different RPs can be marked as $user\_rp\_id_1=basic\_rp\_id_1^{id}modp$ and $user\_rp\_id_2=basic\_rp\_id_2^{id}modp$. As basic\_rp\_id$_1$ and basic\_rp\_id$_2$ are primitive element module $p$, there is $0<\alpha<p$ and $basic\_rp\_id_1=basic\_rp\_id_2^\alpha modp$.So user\_rp\_id$_1$ and user\_rp\_id$_2$ meet the following formula
$$user\_rp\_id_1=user\_rp\_id_2^\alpha modp$$
So RP is able to deduce the relevance between user\_rp\_id$_1$ and user\_rp\_id$_2$ only when knowing $\alpha$. As basic\_rp\_id is generated by IdP and calculating $\alpha$ from basic\_rp\_ids, RP is never able to find the relevance. If an RP does not use the basic\_rp\_id from IdP, user is able to find it dishonest through rp\_certificate and stop the login. Most of current users use dynamic IPs so that it is impossible to get user's login trace from user's IP.


\subsection{Impersonation attack}
RP conducts impersonation attack by getting user's id\_token which is valid in other RPs. OpenID Connect protocol protect id\_token from malicious RP by keep RP owns unique client\_id and check RP's redirect\_uri during login. Unique client\_id makes one RP's id\_token invalid in other RPs. And IdP only redirects id\_token to it's relevant RP's redirect\_uri registered in IdP so that attacker is never able get RP's id\_token. There are three conditions for a malicious to try getting a validate id\_token. 1) Malicious RP has already finished client\_id negotiation with an RP as a user. As client\_id is generated by both RP and user, malicious RP is unable to get the id\_token with the same client\_id. 2)Malicious RP has got a user's id\_token, same as condition 1 malicious RP is unable to negotiate the same client\_id with another RP. 3) Malicious RP acts as the man in the middle between RP and user. As RP sends its URL in rp\_certificate user only sends its id\_token to this URL so that attacker can never achieve id\_token. As a summary, malicious is unable to conduct impersonation attack.

Malicious user is only able to conduct impersonation attack by tempering id\_token. If attacker has already get victim's user\_rp\_id, attacker is able to calculate $user\_id=user\_rp\_id^rmodp$. $r$ is shared by RP and attacker. However id\_token is protected by the signature generated by IdP so that it is impossible for attacker to log in RP as victim.

%External attacker is going to steal user's id\_token from network flow to make the attack. As all the network flows are protected by https, external attacker is unable to conduct the attack.
\subsection{Abduction attack}
To lead user to login an RP as attacker, attacker needs to make sure that user receive a malicious token from IdP. As https is used to protect parameters transforming between user and IdP, it's impossible to temper user's token during transmission. The other way to conduct the attack is phishing attack on IdP.
In traditional SSO protocol such as OAuth 2.0 and OpenID Connect, it is possible for malicious to conduct phishing attack on IdP. As it is shown in ~\ref{fig:OpenID} step 2, the request from user to IdP is built by RP. If an malicious RP set the IdP'url as its phishing site, an unwary user may input its id and password on the phishing website so that attacker is able to get the full control of user's account.
In PriOIDC as RP\_Cert contains IdP's url, user agent is going to compare the IdP's url in request and RP\_Cert. If they are not matched, the request is deemed invalid.

Phishing attack on RP in SSO system is quite different from it in normal website. In SSO system even an unwary user has visited a phishing RP's website, IdP is going to ask user to make sure RP's identity in ~\ref{fig:OpenID} step 2. The identity is bound with RP's client id and client id is bound with its redirect uri. If malicious RP constructs the request in ~\ref{fig:OpenID} step 2 to IdP with its personal client id, user is able to find out the true identity of RP and protect itself from phishing attack. In traditional SSO system if malicious uses a client id of another RP, IdP is going to redirect user to the corresponding redirect uri. In PriOIDC user agent is going to compare redirect uri from RP with the redirect uri in RP\_Cert.If uris are not matched, the request is regarded invalid. A phishing RP can never achieve another RP's token and never lead user to log in its website.

%In phishing attack, adversary forges an RP's web application and induces user to log in it. In the disguise of the real RP, phishing site must send the real RP's rp\_token to user. In obtaining token phase if RP redirects user to real IdP, user is going to send its id\_token to the address from rp\_certificate. RP is not able to get user's id\_token and User does not log in adversary's RP finally. If RP redirects user to a fake IdP, the URL in redirection request is not mapped with it in rp\_certificate. User is going to stop the login. So phishing attack is not possible.

%External attacker is willing to conduct abduction attack. Attacker is going to make the attack by temper user's id\_token into attacker's id\_token when id\_token is transformed on the network. But all the network flows are protected by https so that user's id\_token is secure.
%如果不同RP之间Basic_Client_ID 找到幂次的关系,那么就可以对应到同一个用户的身份


\subsection{Discussion}
An external attacker is also taken into account in SSO system. External attacker is able to capture and temper all the network flow through user, RP and IdP. External attacker's targets include impersonation attack, abduction attack and privacy undermining attack.
%As SSO login requires IdP and RP to adopt https, external attacker is unable to get user's login trace through network flows. And user's dynamic IP makes external attacker impossible to get user's login trace from user's IP.
If an attacker keeps its eye on a specific user, it is able to find that the user's login on different RPs. So it is easy for an external attacker to draw a user's login trace. Privacy protection is not effective for external attacker.
To protect user from privacy leaking a proxy is probably a appropriate scheme. Proxy is able to mix multi-user's request and keep user's login trace invisible to attacker. User's dynamic IP makes proxy impossible to get user's login trace from user's IP
External attacker is going to steal user's id\_token from network flow to make the attack and it is also going to make the attack by temper user's id\_token into attacker's id\_token when id\_token is transformed on the network. As all the network flows are protected by https, external attacker is unable to conduct the attacks.
\end{comment}

\section{Implementation And Evaluation}
\label{sec:implementation}
We have implemented the prototype of Recluse, and compared its performance with the original OIDC implementation and SPRESSO.

\subsection{Implementation}
We adopt SHA-256 to generate the digest, and  RSA-2048 for the signature in  the $Cert_{RP}$, identity proof and the dynamic registration response. We  choose a random 2048-bit strong prime as $P$, and the smallest primitive root (3 in the prototype)  of $P$ as $g$. The  $n_u$, $n_{RP}$ and $UID$  are 256-bit odd numbers, which provides no less security strength than RSA-2048~\cite{barkerecommendation}.

The IdP is implemented based on MITREid Connect~\cite{OIDF}. an open-source OIDC Java implementation certificated by the OpenID Foundation~\cite{OIDF}. Alghough, OIDC standard specifies that RP's identifier should be generated by IdP in the dynamic registration, MITREid Connect allows the user to provide a candidate RP identifier to the IdP who checks the uniqueness, which simplifies the implementation of Recluse. In Recluse, we add 3 lines Java code for generation of $PPID$, remove 1 line for checking the registration token in dynamic registration, while the calculation of $RPID_O$, $Cert_{RP}$,  $PPID$, and the RSA signature is implemented using the Java built-in cryptographic libraries (e.g., BigInteger)

The user-side processing is implemented as a Chrome extension with about 330 lines JavaScript code and 30 lines  Chrome extension configuration files (specifying the required permissions). The cryptographic calculation in $Cert_{RP}$ verification, $RPID_T$ negotiation, dynamic registration, is based on an efficient JavaScript cryptographic library  jsrsasign~\cite{jsrsasign}. The Chrome extension clears the \verb+referer+ in the HTTP header, to avoid the RPs' URL leaked to the IdP.


We provide the SDK for RP to integrate Recluse easily. The SDK provides 3 functions:  RP initial registration, processing of the user's login request and  identity proof parsing. The Java SDK is implemented based on the Spring Boot framework  with about 1100 lines JAVA code. The cryptographic computation is completed through Spring Security library. The user's login request contains Step 2, 3, 6, 7 and 12 in Figure~\ref{fig:process}; while  identity proof parsing contains Step 18 in Figure~\ref{fig:process}.


\noindent\textbf{Cross-Origin Resource Sharing (CORS).} The chrome extension needs to construct cross-origin requests to communicate with the RP and IdP, which is forbidden by default by the same-origin security policy. Recluse adopts CORS to achieve this cross-origin communication. In details, we requires the RP and IdP to specify \verb+chrome-extension://chrome-id+ in the \verb+Access-Control-Allow-Origin+ field of its response header, which makes the request pass the permission checks at the browser. As \verb+chrome-id+ is unique assigned by the Google, no other (malicious) entity can perform the cross-origin communication.



% While the response of \verb+http://www.B.com+ is transmitted to browser, the browser is to check whether the response carries the \verb+Access-Control-Allow-Origin:+ \verb+http://www.A.com+ in the http header. If the \verb+Access-Control-Allow-Origin+ is missed, the response is intercepted by the browser. The request initiated by the Chrome extension belongs to the origin \verb+chrome-extension://chrome-id+, while the chrome-id is provided by Google when the extension is uploaded to chrome web store. Therefore, it is required that the RP and IdP's web interfaces accessed by the user agent should add the  \verb+Access-Control-Allow-Origin:+ \verb+chrome-extension://chrome-id+ in its http header to make CORS available.


\begin{comment}

In this section, we firstly illustrate the parameters requirement in Recluse. After, we describe how the prototype is built.

\subsection{Parameters Requirement}
The $SK_Cert$, $PK_CERT$, $SK_ID$ and $PK_ID$ are two pairs of 2048-bit RSA key for different modulus. The $P$ is the 2048-bit strong prime number, as $(P-1)/2$ is also a prime number. And the $g$ is the smallest primitive root of $P$. The $RPID_O$ is another primitive root of $P$ generated from $g$. The $n_u$, $n_RP$ and $UID$ are required 256-bit odd numbers, which satisfies the security consideration about Discrete Logarithm problem.

\subsection{Prototype Implementation}
The Recluse prototype is consisted of the IdP, the RP and the user agent.

The IdP is built base on the MITREid Connect, which is one of the open-source OpenID Connect implementations certificated by  the OpenID Foundation. The MITREid Connect is based on the Spring framework, one of the most popular MVC frame work implemented with JAVA. Compared with the MITREid Connect, the Recluse IdP introduces the novel $PPID$ generating algorithm and enables the RP to register the new $RPID_T$ through dynamic registration. To change the $PPID$ generating method, we add 3 lines of JAVA code. However, although the specification of OpenID Connect defines that the RP's identifier in dynamic registration should be generated by IdP, MITREid allows the RP to decide the identifier instead of the IdP. Therefore, we only need to delete 1 line of code about dynamic registration to avoid the IdP to verify the \verb+registration token+ related with the specific RP which is required in dynamic registration.

In Recluse, the user agent takes the responsibilities to negotiate the $PPID_T$ with RP, register the newly generated $PPID_T$ with IdP, tamper the authentication request with one-time endpoint and transmit the identity proof from IdP to RP. The implementation of user agent is based on the Chrome extension, the function provided by Google for developers to create the plug-in for Chrome browser. The extension is built by about 330 lines of JavaScript code and 30 lines of Chrome extension configuration files. The cryptographic operations of user agent is provided by the jsrsasign, one of the most popular libraries providing Modular Exponentiation computing on the Github.


We provide the Recluse SDK for RP developers, based on which the RP is easily to be built or transformed from the traditional OpenID Connect system. The SDK takes the responsibility to negotiate the RP's identifier, provide the web interface to user agent, verify the identity proof from the IdP and derive the $Account$ from the $PPID$. The SDK is implemented base on the Spring Boot framework, including about 1100 lines of JAVA code. The cryptographic operations is provided by Spring Security library. With the SDK, we simply build the prototype RP with about only 30 lines of JAVA code, which provides the two web interfaces for user initial login and user identity  proof uploading. It is convenient for developers to transform the traditional OpenID Connect system to Recluse.

Besides, there are also some security considerations about the implementation of SSO systems, it is to be discussed as follows.

\noindent\textbf{307 Redirect. }It has been discussed in~\cite{FettKS16} that IdP might redirect the user to the RP immediately after the user inputs the credentials. For example, the HTTP response to the user's POST message with \verb+username+ and \verb+password+ might be the redirection to RP carrying user's identity proof. That is, as long as the 307 status code is used for this redirection, the user's credentials are also transmitted to the RP. However, in Recluse the redirections are intercepted by the user agent and rebuild the HTTP GET request to RP or IdP which is unable to leak the POST data of the user.


\noindent\textbf{Cross-Origin Resource Sharing (CORS).} Same-origin policy enables the browser restrict the request from one origin to another, for example, the request from the the web page \verb+http://www.A.com+ to \verb+http://www.B.com+ should be intercepted by the browser as default. However, to , the CORS is provided in the http protocols which allows the cross-origin transmission. In Recluse, the CORS is required for identity proof request and transmission. However, we strictly define the allowed cross-origin request, only the request from user agent is permitted, by setting the header of the response from RP and IdP with \verb+Access-Control-Allow-Origin:+ \verb+chrome-extension://chrome-id+. It is able to prevent the possible Xss attacks.
% While the response of \verb+http://www.B.com+ is transmitted to browser, the browser is to check whether the response carries the \verb+Access-Control-Allow-Origin:+ \verb+http://www.A.com+ in the http header. If the \verb+Access-Control-Allow-Origin+ is missed, the response is intercepted by the browser. The request initiated by the Chrome extension belongs to the origin \verb+chrome-extension://chrome-id+, while the chrome-id is provided by Google when the extension is uploaded to chrome web store. Therefore, it is required that the RP and IdP's web interfaces accessed by the user agent should add the  \verb+Access-Control-Allow-Origin:+ \verb+chrome-extension://chrome-id+ in its http header to make CORS available.


\noindent\textbf{Cross-Site Request Forgery (CSRF). } The CSRF attack might lead the user to access the malicious url provided by the adversary's web page, through which the adversary might lead the honest user to upload the adversary's \verb+ID token+ to the RP. However, in Recluse the cross origin request should be repudiated by both RP and IdP excepted the request from the origin of the user agent, which is able to prevent the CSRF attack.


The implementation of Recluse IdP is based on the MITREid Connect, which is the open-source OpenID Connect implementation in Java on the Spring framework, one of the most popular MVC frame work. Until July 30, 2019, the project of MITREid Connect on github owns 233 uses and 994 stars. It has already been certified by the OpenID Foundation as well.

The implementation of user agent is based on the Chrome extension, the function provided by Google for developers to create the plug-in for Chrome browser. The main programming language of Chrome extension is JavaScript. The cryptographic computing of user agent is provided by the jsrsasign, which has 6878 uses and 1986 stars on github.

The implementation of RP SDK is also based on the Spring framework and the cryptographic computing is provided by the Java Platform, Standard Edition. An RP is able to use the service provided by the Recluse conveniently with this SDK. However, to make it convenient to evaluate the time cost of prototype system, we build the RP based on the Spring frame work instead of modifying the open-source RP implementation.

The modification of IdP introduces about 5 lines of code changing, including deleting 1 line about verifying the authority of dynamic registration, deleting 1 line about getting user identifier from database which is replaced by 3 lines about generating it through the user-id-generating algorithm. Additionally, to make the CORS (Cross-Origin Resource Sharing) available, we add 6 lines of configuration code. The implementation of user agent contains about 330 lines of code which imports 3 libraries and about 30 lines of configuration. The implementation of RP SDK is about 1100 lines. However, we easily build the simple RP with only 30 lines of code based on the RP SDK ignoring the auto generated code by Spring framework.

\noindent\textbf{CORS.} Same-origin policy enables the browser restrict the request from one origin to another, for example, the JavaScript code on the web page created by \verb+http://www.A.com+ defines the request to \verb+http://www.B.com+, which carries the \verb+Origin:+ \verb+http://www.A.com+ in its http header. While the response of \verb+http://www.B.com+ is transmitted to browser, the browser is to check whether the response carries the \verb+Access-Control-Allow-Origin:+ \verb+http://www.A.com+ in the http header. If the \verb+Access-Control-Allow-Origin+ is missed, the response is intercepted by the browser. The request initiated by the Chrome extension belongs to the origin \verb+chrome-extension://chrome-id+, while the chrome-id is provided by Google when the extension is uploaded to chrome web store. Therefore, it is required that the RP and IdP's web interfaces accessed by the user agent should add the  \verb+Access-Control-Allow-Origin:+ \verb+chrome-extension://chrome-id+ in its http header to make CORS available.
\end{comment}



\subsection{Performance Evaluation}
\label{sec:evaluation}
We have compared the processing time of each user login in UPRESSO, with the original OIDC implementation (MITREid Connect) and SPRESSO which only hides the user's accessed RPs from IdP.

%(with 4 cores, 8 threads)\
%\noindent{\textbf{Configuration.}} 
We run the evaluation on 3 physical machines connected in a separated 1Gbps network. A DELL OptiPlex 9020 PC (Intel Core i7-4770 CPU, 3.4GHz, 500GB SSD and 8GB RAM) with Window 10 prox64 works as the IdP. A ThinkCentre M9350z-D109 PC (Intel Core i7-4770s CPU, 3.1GHz, 128GB SSD and 8GB RAM) with  Window 10 prox64 servers as RP. The user adopts Chrome v75.0.3770.100 as the user agent on the Acer VN7-591G-51SS Laptop (Intel Core i5-4210H CPU, 2.9GHz, 128GB SSD and 8GB RAM) with  Windows 10 prox64. For SPRESSO, the extra trusted entity FWD is deployed on the same machine as IdP. The monitor demonstrates that  the calculation and network processing of the IdP does not become a bottleneck.

%\noindent{\textbf{Performance.}} 
We have measured the processing time for $1000$ login flows, and the the results is demonstrated in Figure~\ref{fig:evaluation}. The average time is 208 ms, 113 ms and 308 ms for UPRESSO, MITREid Connect and SPRESSO respectively.


For better comparison, we further divide a SSO login flow into 4 phases, which : 1. \textbf{Authentication request initiation} (Steps 1-14 in Figure~\ref{fig:process}), the period which starts before the user sends the login request and ends after the user receive the identity proof request transmitted from itself.
%IdP has received the identity proof request;
2. \textbf{Identity proof generation} (Step 15 in Figure~\ref{fig:process}), denoting the construction of identity proof at the IdP (excluding the user authentication); 3. \textbf{Identity proof transmition} (Steps 16-17 in Figure~\ref{fig:process}), for transmitting the proof from the IdP to the RP with the user's help; and 4. \textbf{Identity proof verification} (Steps 18 in Figure~\ref{fig:process}), for the RP  verifying and parsing the proof for the user's $Account$. 
%However, the HTTP transmission is consisted of the pairwise request and response, so that in the implementation of timer, the step 14 and 19 are counted as the identity proof transmitting. To avoid the time difference in each computer, we consistently anchor the time point at user agent where the time is always achieved from the user's PC. The detailed comparison is shown in Figure~\ref{fig:evaluation}.



\begin{figure}
  \centering
  \includegraphics[width=\linewidth]{fig/evaluation2.pdf}
  \caption{The Evaluation.}
  \label{fig:evaluation}
\end{figure}
In the authentication request initiation, MITREid Connect requires the shortest time (10 ms); SPRESSO needs 19 ms for RP to obtain the IdP's public information and encrypt its domain; UPRESSO needs 98 ms, for the $PRPID$ calculation (1 modular exponentiation at the user and 2 at the RP) and dynamic registration.

For identity proof generation, MITREid Connect needs 32 ms (information constructing and signing); UPRESSO needs an extra 6 ms for the generation of $PPID$;  SPRESSO requires 71 ms for a different format of identity proof, which is longer than others as the processing in SPRESSO is implemented with JavaScript while the others are using Java and it costs more time for signature generation by JavaScript because of the programming language feature.

For identity proof transmition, IdP in  MITREid Connect provides the proof as a fragment component (i.e., proof is preceded by \#) to RP to avoid the reload of RP document; and RP uses the JavaScript code to send the proof to the background server; the total transmitting requires 57 ms. In UPRESSO, a chrome extension relays the identity proof from the IdP to RP, which needs 14 ms. The transmitting in SPRESSO is much complicated: The user's browser creates an iframe of the trusted entity (FWD), downloads the JavaScript from FWD, who obtains the RP's correct URL through a systematic decryption and communicates with the parent opener (also RP's document, but avoiding leaking RP to IdP) and RP's document through 3 post messages, which need about 193 ms.
%The time for transmitting identity proof in SPRESSO, relies on the performance of the user's host, 193 ms in our original user, and 127 ms in a stronger user.

In identity proof verification, the RP in MITREid Connect needs 14 ms for verifying the signature, SPRESSO requires 17 ms for a systematic decryption and signature verification, while UPRESSO needs 58 ms for calculation of $Account$ and signature verification.



\begin{comment}
The consideration of usability about UPRESSO is time cost in each authentication. However, the UPRESSO also introduces the extra storage as IdP and RP has to remember the longer identifier of user and RP. But the storage cost is within the range of TBs, which is to be ignored.

In the authentication request initiated phase, MITREid Connect uses the shortest time, 10 ms, as it only builds the request with the stored parameters, such as RP identifier and endpoint. SPRESSO introduces the additional request from RP to IdP for IdP's public parameters and the encryption operation to generate \verb+tag+ (the encrypted RP identifier), so that the time cost is 19 ms. UPRESSO use the longest time, 98 ms , which introduces the extra negotiation and dynamic registration in this phase including 1 time Modular Exponentiation computations at user agent and 2 Modular Exponentiation computations at RP.

In the phase of identity proof generated, all of the systems offer the signed identity proof. Compared with MITREid Connect, UPRESSO only introduces the extra Modular Exponentiation computation for $PPID$. However, the SPRESSO provides the totally different format of identity proof. The time cost in this phase of MITREid Connect, SPRESSO and UPRESSO are 32 ms, 78 ms and 38 ms.

In the phase of identity proof transmitted, UPRESSO transmits the identity proof through the extension and the MITREid Connect uses JavaScript code in RP's web page , the time cost of which are 14 ms and 57 ms. The SPRESSO requires the additional opened iframe, which downloads the script from FWD for extra encryption and decryption operation for identity proof transmitted. However, the time cost of these operations are significant different while running user agent in different devices. We get the 127 ms in average at best, however, in another low-performance device (), the number is 453 ms. To verify whether the performance of user agent has the same influence on UPRESSO, we use several PCs to log in UPRESSO and record the time. The result is the time costs in these PCs are quite similar which means the performance of user agent is not apparently influential with UPRESSO.

The identity proof verified at RP only costs about 1 ms for signature verifying, which is 58 ms at RP requiring additional Modular Exponentiation and Extended Euclidean computation. The RP of SPRESSO has to decrypt the identity proof and verify the signature, which costs 13 ms.

In conclusion, compared with MITREid Connect, UPRESSO introduces about 135 ms extra time cost in authentication request initiated and identity proof verified, which is acceptable. Compared with the best result of SPRESSO, UPRESSO still saves 85 ms.
\end{comment}




\section{Discussion}
\label{sec:discussion}
%SPRESSO跨平台:我们目前的方案只在browser实现
%兼容:目前只提供隐式模式,对其他模式与协议的兼容方法
%Dos attack to RP:
%side channel attack: for example the RPID negotation

\section{Related Works}%各个方向全都加入,例如安全分析
\label{sec:related}
In 2014, Chen et al.\cite{ChenPCTKT14} concludes the problems developers may face to in using sso protocol. It describes the requirements for authentication and authorization and different between them. They illustrate what kind of protocol is appropriate to authentication. And in this work the importance of secure base for token transmission is also pointed. 

In 2016, Daniel et al.\cite{FettKS16} conduct comprehensive formal security Analysis of OAuth 2.0. In this work, they illustrate attacks on OAuth 2.0 and OpenID Connect. Besides they also presents the snalysis of OAuth 2.0 about authorization and authentication properties and so on.

Besides of OAuth 2.0 and OpenID Connect 1.0, Juraj et al.\cite{SomorovskyMSKJ12} find XSW vulnerabilities which allows attackers insert malicious elements in 11 SAML frameworks. It allows adversaries to compromise the integrity of SAML and causes different types of attack in each frameworks.

Other security analysis\cite{WangCW12}\cite{ZhouE14}\cite{WangZLG16}\cite{YangLLZH16}\cite{WangZLLYLG15} on SSO system concludes the rules SSO protocol must obey with different manners.  

In 2010, Han et al.\cite{HanMSY10} proposed a dynamic SSO system with digital signature to guarantee unforgeability. To protect user's privacy, it uses broadcast encryption to make sure only the designated service providers is able to check the validity of user's credential. User uses zero-knowledge proofs to show it is the owner of the valid credential. But in this system verifier is unable to find out the relevance of same user's different requests so that it cannot provide customization service to a user. So this system is not appropriate for current web applications.

In 2013, Wang et al. proposed anonymous single sign-on schemes transformed from group signatures. In an ASSO scheme, a user gets credential from a trusted third party (same as IdP) once. Then user is able to authenticate itself to different service providers (same as RP) by generating a user proof via using the same credential. SPs can confirm the validity of each user but should not be able to trace the user’s identity.


Anonymous SSO schemes prevents the IdP from obtaining the user's identity for RPs who do not require the user's identity nor PII, and just need to check whether the user is authorized or not. These anonymous schemes, such as the anonymous scheme proposed by Han et al.~\cite{HanCSTW18}, allow user to obtain a token from IdP by proving that he/she is someone who has registered in the Central Authority based on  Zero-Knowledge Proof. RP is only able to check the validation of the token but unable to identify the user.
In 2018, Han et al.\cite{HanCSTW18} proposed a novel SSO system which uses zero knowledge to keep user anonymous in the system. A user is able to obtain a ticket for a verifier (RP) from a ticket issuer (IdP) anonymously without informing ticket issuer anything about its identity. Ticket issuer is unable to find out whether two ticket is required by same user or not. The ticket is only validate in the designated verifier. Verifier cannot collude with other verifiers to link a user's service requests. Same as the last work, system verifier is unable to find out the relevance of same user's different requests so that it cannot provide customization service to a user. So this system is not appropriate for current web applications.





BrowserID\cite{BrowserID}\cite{FettKS14} is a user privacy respecting SSO system proposed by Molliza. BrowserID allows user to generates asymmetric key pair and upload its public to IdP. IdP put user's email and public key together and generates its signature as user certificate (UC). User signs origin of the RP with its private key as identity assertion (IA). A pair containing a UC and a matching IA is called a certificate assertion pair (CAP) and RP authenticates a user by its CAP. But UC contains user's email so that RPs are able to link a user's logins in different RPs. 

SPRESSO\cite{SPRESSO} allows RP to encrypt its identity and a random number with symmetric algorithm as a tag to present itself in each login. And token containing user's email and tag signed by IdP is also encrypted by a symmetric key provided by RP. During parameters transmission a third party credible website is required to forward important data. As token contains user's email, RPs are able to link a user's logins in different RPs.



All the SSO system protocols above are quite different from current popular SSO protocol. So it is difficult for IdPs and RPs to remould their system into new protocols. 
\section{conclusion}
\label{sec:conclusion}
In this paper, we, for the first time, propose Recluse to preserve the users' privacy from the curious IdP and colluded RPs,  without breaking the security of SSO systems. The identity proof is bound with a transformation of the original identifier, hiding the users' accessed RPs from the curious IdP. The user's account is independent for each RP, and unchanged to the destination RP who has the trapdoor, which prevents the colluded RPs from linking the users and allows the RP to provide the consecutive and individual services. The trusted user ensures the correct content and transmit of the identity proof with a self-verifying RP certificate. The evaluation demonstrates the efficiency of Recluse, about 200 ms for one user's login at a RP in our environment.


\bibliographystyle{IEEEbib}
\bibliography{ref}

\end{document}
