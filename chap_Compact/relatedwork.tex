\subsection{Anonymous Tokens and OPRF Applications}
\label{sec:related}

\newc
PrivacyPass and TrustToken \cite{privacypass,trusttoken}
adopted the oblivious pseudo-random function (OPRF) protocol \cite{oprf-proved,voprf-proved,oprf-bitcoin-wallet} to generate anonymous tokens with which authorized users could access the server's resources anonymously.
To do so, a user generates a random number $e_i$ to blind an unsigned token $T_i$ into $[e_i]T_i$. After authenticating the user, the token server signs $[e_i]T_i$ with its private key $k$ and returns $[k e_i]T_i$ to the user, who converts it into an anonymous token  ($T_i, [k]T_i$) using $e_i$. In this process, the server obliviously calculates the pseudo-random output (i.e., anonymous token). However, as the tokens are completely indistinguishable. If they are used to realize SSO, the RPs cannot identify each user.

\usso\ applies a similar cryptographic skill to design identity transformations. A user selects a random number $t_i$ to transform $ID_{RP}$ to $PID_{RP} = [t_i]ID_{RP}$, which blinds the IdP from knowing $ID_{RP}$. Then, the IdP uses the user's permanent identity $u$ to compute $PID_U = [u]PID_{RP}$. %= [ut]ID_{RP}$
This is similar to the ``token signing'' process taken by the token server in PrivacyPass/TrustToken. Hence, \usso~provides the \emph{IdP-untraceability} property, i.e., the IdP cannot link $[t_i]ID_{RP}$ and $ID_{RP}$, which roughly corresponds to the {\em token signing and redemption unlinkability} in PrivacyPass/TrustToken, i.e., the token server cannot link ($[e_i]T_i, [ke_i]T_i$) and  ($T_i, [k]T_i$).

%Finally, the visited RP calculates $Acct = [t_i^{-1}]PID_{U}$ using the same $t_i$ shared by the user.

%PrivacyPass/TrustToken adopted the OPRF protocol for token generation. Given a user input, the server obliviously calculates the pseudo-random output (i.e., anonymous token). The tokens are completely indistinguishable. So, if they are used to realize SSO, the RPs cannot identify each user.

On the other hand, \usso\ presents a new design that leverages this cryptographic skill to provide a non-anonymous but privacy-preserving SSO service, unlike the completely anonymous service in PrivacyPass/TrustToken. This also requires extending the two-party OPRF protocol to work for three parties in SSO services. To achieve this goal, the random number $t_i$ selected by the user is shared with the target RP to enable it to independently derive the user's permanent account at itself, i.e., $Acct = [t_i^{-1}]PID_{U}$. Furthermore, this account is ``obliviously'' determined by the IdP when it generates the user's pseudo-identity 

This enables the IdP to ``obliviously'' determine the user's account at an RP based on the user's identity and the RP's pseudo-identity (i.e., the ``blinded'' identity).

Moreover, the blind parameter (i.e., input to the pseudo-random functions) is utilized as the \emph{user identity} (i.e., $u$ in identity transformations) to provide sufficient flexibility required by the design. This is different from PrivacyPass/TrustToken and other OPRF-based applications \cite{privacypass,trusttoken,strong-oprf,oprf-bitcoin-wallet,pesto,oprf-ot-si,pp-ss,Private-Contact-Discovery,o-kms,oprf-deduplication} that always use it as the \emph{secret key} of the token server.%In this process, we apply pseudo-random functions but use the \emph{secret key} of pseudo-random functions as a \emph{user identity} (i.e., $u$ in identity transformations), while it always is used as \emph{secret keys} in PrivacyPass/TrustToken and other OPRF-based applications \cite{privacypass,trusttoken,strong-oprf,oprf-bitcoin-wallet,pesto,oprf-ot-si,pp-ss,Private-Contact-Discovery,o-kms,oprf-deduplication}.
This unusual application requires an in-depth understanding of the variables in OPRFs and the (pseudo-)identities in SSO services.
%This extension is unlikely to occur in OPFRs that consider $k$ as the private key of the server, but it is reasonable in \usso\ as it uses random numbers as identities and trapdoors to design oblivious pseudo-random functions for identity transformations.
%Compared with the original OPRFs, we use the $k$ key (they are always private keys) as a user identity.
% 第一段:说明2个点:
% 1. 2 party -> 3 party: user把随机数分享给RP。
% 2. 把key用作uid,在SSO中。把prf的输入输出,用作PID和account。
% 他们的token,与uid/account无关
Moreover, \usso\ leverages the randomness property of OPRFs
 to provide \emph{RP unlinkability}, while this property is not leveraged in PrivacyPass/TrustToken.
It ensures that colluding RPs cannot link any two logins across RPs, i.e., ($ID_{RP}, t, [u]ID_{RP}$) and ($ID_{RP'}, t', [u']ID_{RP'}$),
  even if they share the knowledge about the pseudo-identities and permanent accounts of the users initiating these logins.
This property results in the \emph{indistinguishability} of different users for RPs in \usso,
    therefore corresponding to different keys for signing anonymous tokens,
    which is not considered in PrivacyPass/TrustToken.
% but they intentionally distinguish different keys by configuring a unique public key at each user to verify the signed tokens.

%In addition to the extension to three parties, \usso\ leverages the OPRF cryptographic technique in greater depth than existing anonymous tokens \cite{privacypass,trusttoken}. \usso\ leverages more properties of the technique than PrivacyPass/TrustToken. The unlinkability between token signing and redemption, or ($[e_i]T_i, [ke_i]T_i$) and  ($T_i, [k]T_i$), roughly corresponds to only the IdP-untraceability in UPPRESSO: an IdP cannot link any pair of $[t_i]ID_{RP}$ and $ID_{RP}$.  % $i = 1, 2, \cdots$.
%Moreover, we prove the additional unlinkability across colluding RPs by showing that their knowledge about any two users $u$ and $u'$, i.e., ($ID_{RP}, t, [u]ID_{RP}$) and ($ID_{RP'}, t', [u']ID_{RP'}$), obtained from different logins, is indistinguishable. This property is not considered or supported in anonymous tokens \cite{privacypass,trusttoken}.
% 相比PP/TT,我们把OPRF用得更彻底,有一个新的属性。



\usso\ leverages even more properties of the OPRF protocol \cite{oprf-proved,voprf-proved}, %which are also utilized by other OPRF-based designs \cite{oprf-proved, voprf-proved, oprf-bitcoin-wallet, privacypass, trusttoken},
 to ensure its security and privacy. % in Theorems 4, 6, and 7.
It uses the \emph{obliviousness} property to prevent the IdP from learning any information about $ID_{RP}$ when receiving a token request for $[t_i]ID_{RP}$, the \emph{deterministicness} property of \emph{pseudo-random} functions to enable the RP to derive the permanent account for any $t$,
and the \emph{randomness} property to provide RP unlinkability.
 %when user identities (i.e., the key secret $k$ of pseudo-random functions) are unknown.
Last but not least, \usso\ requires an additional property for SSO services, called \emph{RP designation}.
It is satisfied only if no collision exists in RPs' pseudo-identities (see Lemma 1), which are actually the blinded inputs of the evaluated pseudo-random function.
This property   %, i.e., \emph{collision-freeness of blinded inputs},
is not explicitly required in other OPRF-based solutions and therefore may not be supported by all OPRF protocols.
Thus, an OPRF protocol is not always ready to implement identity transformations in \usso, unless no collision exists in the blinded inputs of the pseudo-random function.
%For example, we plan to analyze this property of quantum-secure ORPF protocols \cite{ideal-lattice-oprf,isogency-oprf} in the future, to find out whether they work effectively with the identity transformations in \usso.

%More importantly, \usso\ even leverages more properties of this cryptographic technique than the OPRF protocols \cite{oprf-proved,voprf-proved,oprf-bitcoin-wallet}. We prove that \usso\ satisfies three requirements of privacy-preserving SSO (i.e., IdP ) in Theorems 6, 4, and 7. The 1st feature of the cryptographic technique used in OPRFs, provides the \emph{obliviousness} property of OPRFs, also prevents the IdP from learning anything about $ID_{RP}$ when receiving a token request for $[t]ID_{RP}$, the \emph{deterministicness} property of \emph{pseudo-random functions} used in OPRFs, enables the derivation of permanent accounts at the RP for a random $t$, while the \emph{randomness} property used in OPRFs in the case of unknown $u$ or $k$, prevents colluding RPs from linking $Acct = [u]ID_{RP}$ across different RPs.

%Last but not least, \usso\ requires \emph{RP designation}, which is ensured by another property of this cryptographic technique: there is no $PID_{RP}$ collision in the \usso\ protocol. This property is proved in Lemma 1. However, this property is \emph{not} considered even in OPRF protocols; i.e., the OPRF protocol \cite{oprf-proved,voprf-proved,oprf-bitcoin-wallet} does not require no collision of the blinded inputs to the OPRF server. An OPRF protocol does not always result in a variation of the \usso\ protocol, unless the additional property of input-collisionlessness is proved.

%    \footnote{Due to the extension of three parties and the requirement of input-collisionlessness, we did not realize this cryptographic technique has been published in OPRFs
%        until anonymous reviewers pointed it out.}

%It shares a random number $t$ between the user and the RP to transform $ID_{RP}$ to $PID_{RP} = [t]ID_{RP}$ so that $PID_{RP}$ is related to $ID_{RP}$ but ``oblivious'' to the IdP to achieve IdP-untraceability. Then, the IdP transforms $PID_{RP}$ further to construct $PID_U=[ut]ID_{RP}$, which is ``pseudo-random'' to the RP and achieves RP-unlinkability. Finally, $PID_U$s of the same user at the same RP are associated by $[u]ID_{RP}$, which can be derived using the trapdoor $t$.
%Besides, the SSO service also requires $PID_{RP}$ to be unique (i.e., RP designation), which is not required in the original OPRF \cite{oprf-proved,voprf-proved,oprf-bitcoin-wallet}. We support this property in \usso\ and prove it in Section~\ref{analysis-security}.


\begin{comment}
%The random numbers (or blinding factors) prevent the token server from linking token signing and redemption.

Anonymous tokens may be used to build anonymous authentication services, where an RP does not identify each user.
% they can be used to implement anonymous SSO services that are secure against IdP-based tracing and RP-based linkage.}
%\textcolor{blue}{They provide SSO services that are fundamentally different from the ones offered by UPPRESSO. As shown in Table~\ref{tbl:comparison-protocol}, they lack support for all three SSO features. First, the anonymous tokens used by different users or by the same user in different logins cannot be distinguished (i.e., no user identification at RPs), since they are signed blindly using the same server secret. Besides, the users are required to maintain anonymous tokens for asynchronous authentication in addition to the credentials for the IdP, similar to other privacy-preserving solutions based on anonymous credentials. Finally, PrivacyPass and TrustToken do not support selective user attribution provisioning. In contrast, UPPRESSO formalizes the privacy-preserving SSO process as two ID-transformation problems and generates unlinkable pseudoidentities and proofs to support the desired SSO features. Therefore,  UPPRESSO is  compatible with widely-adopted SSO protocols such as OIDC and SAML.}
%\textcolor{blue}{PrivacyPass and TrustToken allow a user to receive tokens \cite{privacypass,trusttoken}, each of which is denoted as ($T, T^{k}$), where $k$ is the token server's private key. These tokens are used to access resources anonymously in the future. To unlink token signing and redemption, a user generates a random number $e$ for each token, blinds $T$ into $T^{e}$, and sends it to request ($T^e, T^{ek}$) from the token server. The user then utilizes $e$ to obtain $T^k$ from $T^{ek}$, and then only ($T, T^{k}$) is redeemed to access resources. This cryptographic skill \cite{oprf-proved} is used in UPPRESSO similarly: a user transforms $ID_{RP}$ to $PID_{RP} = [t]ID_{RP}$ by a random number $t$, and $PID_{RP}$ is transformed again by an IdP to $[tu]ID_{RP}$. The visited RP calculates $Acct = [u]ID_{RP}$ from $[tu]ID_{RP}$ by using $t$ (see Table \ref{tbl:notations-protocol} for detailed descriptions of these notations).}

In \usso, the identity transformations are implemented using a similar cryptographic technique. In particular, a user transforms $ID_{RP}$ to $PID_{RP} = [t]ID_{RP}$ using a random number $t$, and then the IdP transforms $PID_{RP}$ further to $[ut]ID_{RP}$. The visited RP finally calculates $Acct = [u]ID_{RP}$ from $[ut]ID_{RP}$ by using $t$.
\usso\ supports the IdP-untraceability property that prevents the IdP from linking $ID_{RP}$ and a corresponding $[t]ID_{RP}$.
This is similar to the unlinkability between token signing and redemption in PrivacyPass/TrustToken, which prevents the token server from linking ($T_i^{e_i}, T_i^{e_ik}$) and  ($T_i, T_i^k$).

However, \usso\ explores this cryptographic technique in greater depth, resulting in an additional privacy property that is not considered or supported in existing anonymous tokens \cite{privacypass,trusttoken}. %or OPRFs \cite{oprf-proved,voprf-proved}
The exponent $k$ in PrivacyPass/TrustToken is held only by the server as a consistent private key, and the random number $e$ is only known to a user. In contrast, the scalar $u$ in \usso\ denotes a user identity known to the user and the IdP, and the random number $t$ is an ephemeral trapdoor shared between the user and the RP.
%%The application of the cryptographic technique differs from PrivacyPass/TrustToken
%%This different design allows \usso\ to explore and offer additional privacy properties. %Furthermore, more privacy properties are explored in UPPRESSO.
The sharing of random numbers enables the derivation of permanent accounts in \usso,
 while we prove the additional unlinkability across colluding RPs by showing that their knowledge about any two users $u$ and $u'$, i.e., ($ID_{RP}, t, [u]ID_{RP}$) and ($ID_{RP'}, t', [u']ID_{RP'}$), obtained from different logins, is indistinguishable.

%given multiple users, e.g., identified as $u$ and $u'$, ($ID_{RP}, t, [u]ID_{RP}$) and ($ID_{RP'}, t', [u']ID_{RP'}$) are indistinguishable to colluding RPs.
%This property requires designing the identity transformations under not only ECDLP but also ECDDH assumptions (see Section~\ref{sec-:analysis}).

% \noindent\textbf{Anonymous SSO.}
% Such schemes allow authenticated users to access a service protected by an IdP,
%     without revealing their identities.
% Anonymous SSO was proposed for GSM communications \cite{ElmuftiWRR08},
%     and formalized \cite{WangWS13}.
% Privacy-preserving primitives, such as group signature, zero-knowledge proof, Chebyshev Chaotic Maps, and proxy re-verification,
%      were adopted to design anonymous SSO \cite{WangWS13,HanCSTW18,Lee18,HanCSTWW20}.
% Anonymous SSO schemes work for some special applications,
%     but are inapplicable in most systems that require user identification for customized services.

\end{comment}

\oldc
\subsection{Extended Related Work}
%Cryptographic primitives are used for protecting user privacy.
\noindent\textbf{Cryptographic primitives for sign-on privacy.}
Distributed key generation and zero-knowledge proofs are used to generate ring-signature private keys as untraceable pseudonyms \cite{crypto-book} or prove users' statements about their credentials without revealing the credential content \cite{zklaim}.
One-time key-share tokens \cite{tandem} are generated in two-party threshold cryptography systems to protect users' key-usage patterns.
\newc
PESTO \cite{pesto} combines distributed partially-oblivious PRFs, distribution signatures,
    and proactive security to build SSO services,
    and it proposes to sign a commitment of an identity token (but not the token itself) for better user privacy.
%Other cryptographic primitives such as zero-knowledge proofs are used in ZKlaims \cite{zklaim} to allow users to prove statements on the credentials issued by a trusted server without revealing the credentials' content. Crypto-Book \cite{crypto-book} adopts distributed key generation to generate linkable-ring-signature private keys for users, and each key pair is used as an untraceable pseudonym. Tandem \cite{tandem} generates one-time-use key-share tokens for a two-party threshold cryptographic scheme implemented with a central server, to protect the privacy of key-usage patterns.


%two-party threshold cryptographic scheme implemented with a central server, to protect user private keys \cite{mRSA,ss-rsa}: to sign/decrypt a message, a user needs a token from the server.
%    Tandem \cite{tandem} decouples the obtaining and using of such tokens, for the privacy of key usage.

%\vspace{0.5mm}
 \oldc\noindent\textbf{Formal analysis of SSO protocols.}
A formal analysis on SAML-based SSO \cite{ArmandoCCCT08} found that a Google-implemented variant does not bind RP identities correctly in the identity tokens.
Fett et al. \cite{FettKS16, FettKS17} formally analyzed OIDC and OAuth 2.0 using a Dolev-Yao style model \cite{FettKS14} and reported the 307 redirection and IdP mix-up attacks.
In this paper, we also developed a Dolev-Yao style model for \usso, helping to prove its security and privacy (see Section~\ref{dy-model}).


%\vspace{0.5mm}
\noindent\textbf{Implementation vulnerabilities in SSO.}
Various vulnerabilities have been found in several SSO systems for web applications, resulting in attacks %of impersonation and identity injection
that break the confidentiality \cite{WangCW12,ccsSunB12,ArmandoCCCPS13,DiscoveringJCS,dimvaLiM16}, integrity \cite{WangCW12,SomorovskyMSKJ12,WangZLG16,MainkaMS16, MainkaMSW17,dimvaLiM16} or RP designation \cite{WangZLG16,MainkaMS16,MainkaMSW17,YangLCZ18,dimvaLiM16} of identity tokens.
%In the SSO services of Google and Facebook, %from the view of browser-relayed traffics
%    logic flaws of the IdPs and RPs were detected \cite{WangCW12}.  % to break the confidentiality and integrity of identity tokens.
The integrity of identity tokens was violated %\cite{SomorovskyMSKJ12,WangCW12,WangZLG16,MainkaMS16, MainkaMSW17}
due to software flaws such as defective verification by RPs \cite{WangCW12,WangZLG16,MainkaMSW17}, XML signature wrapping \cite{SomorovskyMSKJ12}, and IdP spoofing \cite{MainkaMS16,MainkaMSW17}.
Meanwhile, the RP designation was broken because of incorrect binding by an IdP \cite{YangLCZ18,WangZLG16} or insufficient verification by RPs \cite{MainkaMS16,MainkaMSW17,YangLCZ18}.
A defective IdP does not always enclose an identifiable Email address in signed tokens \cite{WangCW12},
 which breaks the user identification.
Similar vulnerabilities have been found in Android Apps that break the confidentiality \cite{ChenPCTKT14,WangZLLYLG15,YangLS17,ShiWL19}, integrity \cite{ChenPCTKT14,YangLS17}, and RP designation \cite{ChenPCTKT14,ShiWL19,WangZLLYLG15} of SSO identity tokens.

%Navas et al. \cite{NavasB19} discussed the possible attack patterns against OIDC services.
%Automatic tools such as SSOScan \cite{ZhouE14}, OAuthTester \cite{YangLLZH16} and S3KVetter \cite{YangLCZ18},
% detect the violations of confidentiality, integrity, or RP designation of SSO identity tokens.
% Wang et al. \cite{ExplicatingSDK} detect the vulnerable applications
%     built with authentication/authorization SDKs,
%      due to the implicit but unsuitable assumptions of these SDKs.


%Furthermore, if a user is compromised, the attacker can log in to RPs on his behalf. So, we consider malicious users, malicious RPs, and colluding users and RPs in our threat model (see Section~\ref{subsec:threatmodel}).





% In a mobile system,
% browsers, IdP Apps,
%     or IdP-provided SDKs %(e.g., an encapsulated WebView)
%          are responsible for forwarding identity tokens, %from the IdP App to RP Apps.
% but none of them ensures an identity token is sent to the designated RP only \cite{ChenPCTKT14,WangZLLYLG15}.
% %    because a WebView or the system browser cannot authenticate the RP Apps and the IdP App may be repackaged.
% %SSO protocols are modified for mobile Apps, but the modifications are not well understood by developers \cite{ChenPCTKT14,YangLS17}.
% Vulnerabilities were found in Android Apps,
%     to break confidentiality \cite{ChenPCTKT14,WangZLLYLG15,YangLS17,ShiWL19}, integrity \cite{ChenPCTKT14,YangLS17}, and RP designation \cite{ChenPCTKT14,ShiWL19} of identity tokens.
% A flaw was found in Google Apps \cite{ArmandoCCCPS13}, allowing a malicious RP to hijack a user's authentication attempt and inject a payload to steal the cookie or identity token belonging to another RP.

% If a user is compromised,
%     attackers will log in to RPs on his behalf.
% Single sign-off helps the victim user
%  to revoke all his tokens accepted and log out from the RPs  \cite{GhasemisharifRC18}.
% FedCM \cite{FedCM} attempts to disable iframe and third-party cookies in SSO, which could be exploited to track users.
% %UPRRSSO protects privacy in SSO through ID transformations and our prototype does NOT use either iframe or third-party cookies.

