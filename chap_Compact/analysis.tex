\section{Security and Privacy Analysis}
\label{sec:analysis}
We define three adversarial scenarios under the threat model in Section \ref{subsec:threatmodel}, develop a Dolev-Yao-style model \cite{BrowserID} to depict the SSO login flow,
 and formally prove the security and privacy guarantees provided by \usso~using the conditions analyzed in the Dolev-Yao-style model.

\newc
\subsection{Adversarial Scenarios}

Based on our design goals (i.e., the desired security and privacy guarantees) and the potential adversaries discussed in Section \ref{subsec:threatmodel}, we consider three adversarial scenarios as below.

\noindent\textbf{Security.} Malicious users could collude with each other or with malicious RPs, attempting to (\emph{a}) impersonate an honest user to log into an honest RP or (\emph{b}) entice an honest user to log into an honest RP under a malicious user's account.

\noindent\textbf{Privacy against the IdP.}
The honest-but-curious IdP tries to infer the identities of the RPs that a user requests to access. %or link multiple logins to any RP initiated by a user.

\noindent\textbf{Privacy against RPs.}
Malicious RPs could collude with each other or with malicious users, attempting to link logins across these RPs that are initiated by a user.


% \textcolor{blue}{Based on the threat model and assumptions proposed in Section \ref{sec:UPPRESSO},
%     different types of adversaries are considered in the analysis of security and privacy.
% First of all, in the proofs of security,
%     malicious RPs collude with malicious users,
%         attempting
%         to break any of the four security properties of SSO identity tokens for an honest user to visit an honest RP.
% Then, in the analysis of privacy against the IdP-based login tracing,
%    an honest-but-curious IdP is the only adversary.
% Finally,
%     in the privacy analysis against the RP-based identity linkage,
%     a number of malicious RPs collude, attempting to link an honest user's accounts across these RPs.}

% We first analyzed UPPRESSO %especially confidentiality and integrity,
%      based on a Dolev-Yao-style model \cite{SPRESSO}.
% % which has been used in the formal analysis of SSO protocols such as OAuth 2.0 \cite{FettKS16} and OIDC \cite{FettKS17}.
% The model abstracts the entities in a web system,
%     such as web servers and browsers,
%     as \emph{atomic processes}. %which communicate with each other through events. % such as HTTPS requests and responses.
% It defines \emph{script processes} to formulate client-side scripts.
% %The script is dependently invoked by the browser to process the server-defined logic.
%   %such as verifying $Certificate_{RP}$.
% %
% %postmessage events;
% %
% %atomic process <-> script process, communication.
% %
% %Other events change self-trigger.
% %
% UPPRESSO contains atomic processes including:
% an IdP process,
%     a finite set of web servers for honest RPs, a finite set of honest browsers, and a finite set of attacker processes.
% The processes communicate with each other through events such as HTTPS requests and responses.
% %We consider all RP and browser processes are honest,
% An RP or a browser controlled by adversaries is modeled as an attacker process.
% Within a browser,
%  an honest IdP script, an honest RP script, and also attacker scripts which are downloaded from attacker processes,
%   are invoked.
% %Although the scripts coexist in the same browser, they are strictly separated.
% Script processes communicate with each other through \verb+postMessage+,
%     modelled as transmitted-to-itself events of a browser process.
% %To clearly indicate the action of postMessage communication, we define it as the transmitting-to-itself event of the browser (which is not defined in SPRESSO).


% \textcolor{blue}{After formulating the system by this model,
%     we analyze the following data for the proofs in Sections \ref{analysis-security} and \ref{sec-:analysis},
%      when there are corresponding adversaries.
% We (\emph{a}) trace the lifecycle of an identity token for an honest user to visit an honest RP,
%         starting when it is generated and ending when accepted by the RP,
%     to ensure it is not leaked to adversaries,
% (\emph{b})
%     locate all places
%         where $PID_U$, $PID_{RP}$ and other parameters enclosed in the token are processed,
%      to ensure no adversary able to manipulate them,
% and (\emph{c})
%     locate the places where $PK$ is transmitted and used in the IdP script,
%         to ensure no adversary tampering with it.
% These conclusions are used to prove security of the UPPRESSO protocols.}
% %
% % to ensure it is not leaked to attackers or tampered with by any adversary without checking.
% \textcolor{blue}{In the meantime,
%         this model ensures that (\emph{a}) $t$ is unaccessible to the honest-but-curious IdP,
%  which is necessary to prevent the IdP-based login tracing,
%  and (\emph{b}) $u$ and $r$ are not leaked to RPs in the protocols,
%     necessary to prevent the RP-based identity linkage.}


%Next, we prove that \usso~is secure in the first adversarial scenario in Section~\ref{analysis-security} and it can prevent privacy threats in the other two adversarial scenarios in Section~\ref{sec-:analysis}.


%The RP cannot derive $ID_U$ from either $PID_U$ or $Acct$ due to the elliptic curve discrete logarithm problem (ECDLP). Since $t$ is random in $\mathbb{Z}_n$ and unknown to the IdP, from the IdP's view, $PID_{RP}$ is indistinguishable from a random variable on $\mathbb{E}$. So, the IdP cannot learn anything about $ID_{RP}$ from $PID_{RP}$.
%Section \ref{sec:analysis} presents more detailed analyses.

\subsection{The Dolev-Yao-Style Model for \usso}
\label{dy-model}

We develop a Dolev-Yao style model \cite{BrowserID, SPRESSO, FettKS16, FettKS17} for \usso, referred to as the \dyu~model, to formalize the login flow of \usso.
% Dolev-Yao style models abstract cryptographic concepts into an algebra of symbolic messages to discover structural flaws using simple formal logic. % which has been used in the formal analysis of SSO protocols such as OAuth 2.0 \cite{FettKS16} and OIDC \cite{FettKS17}.
The model abstracts the entities in a web system, such as web servers and browsers, as \emph{atomic processes}, %which communicate with each other through events. % such as HTTPS requests and responses.
and defines \emph{script processes} to formulate client-side scripts.
%The script is dependently invoked by the browser to process the server-defined logic.%such as verifying $Certificate_{RP}$. %postmessage events; %atomic process <-> script process, communication. %Other events change self-trigger.
The atomic processes of \usso~include an {\em IdP process}, a finite set of {\em web servers} for honest RPs, a finite set of honest {\em browsers}, and a finite set of {\em attacker processes} that model malicious RPs and malicious users.
A browser may invoke an honest IdP script and multiple RP scripts that could be honest or malicious.
The processes communicate with each other through events such as HTTPS requests and responses,
%Although the scripts coexist in the same browser, they are strictly separated.
except that the script processes communicate with each other through \verb+postMessage+ which are modeled as transmitted-to-itself events of a browser process.
%To clearly indicate the action of postMessage communication, we define it as the transmitting-to-itself event of the browser (which is not defined in SPRESSO).

Applying the DYU model, we trace the lifecycle of an identity token from its generation at the IdP to its acceptance at an RP, locate the places where $PID_U$, $PID_{RP}$, and other elements related to the identity token such as $t$ and $u$ are processed, and locate the places where $PK$ is transmitted and used in the IdP script.
We confirm the following conclusions in the DYU model:
 (\emph{a}) an identity token binding pseudo-identities of honest entities, cannot be leaked to any malicious process;
 (\emph{b}) pseudo-identities and other elements in verified identity tokens cannot be manipulated by any malicious process;
    (\emph{c}) the IdP's public key set in the IdP script cannot be replaced or tampered with by any malicious process, within an honest browser;
 (\emph{d}) the IdP receives nothing about $t$ shared between two honest processes;
 (\emph{e}) $r$ is not leaked to any malicious process as it never leaves the IdP;
and (\emph{f}) the RPs cannot receive anything about $u$ shared between two honest processes.

%With the DYU model, we (\emph{a}) trace the lifecycle of an identity token from its generation at the IdP to its acceptance at an RP to prove that it cannot be leaked to adversaries; (\emph{b}) locate the places where $PID_U$, $PID_{RP}$, and other elements in the identity token are processed to prove that they cannot be manipulated by any adversary; and (\emph{c}) locate the places where $PK$ is transmitted and used in the IdP script to prove that it cannot be replaced by any adversary.


\subsection{Security}
\label{analysis-security}

%A secure SSO system allows a \emph{legitimate} user to login to an \emph{honest} RP with her account at this RP,
% by presenting \emph{identity tokens} issued by a \emph{trusted} IdP.

We prove that identity tokens in \usso~and the enclosed pseudo-identities satisfy four properties, namely, \emph{RP designation}, \emph{user identification}, \emph{token confidentiality}, and \emph{token integrity}, which together ensure the security of \usso~in the first adversarial scenario.
Let us consider an identity token $TK$ binding $PID_{RP}$ and $PID_U$, which is generated by the IdP upon a request from an authenticated user with $ID_U$.

%These conclusions are used to prove the security of the UPPRESSO protocols.
%We consider an arbitrary login in which an arbitrary RP with $ID_{RP}$ receives an integer $t$ and an identity token $TK$ issued by the IdP, binding a $PID_U$ and a $PID_{RP}$, from the RP script in a user browser. $TK$ is considered a valid identity token if the RP could verify its signature using the IdP's public key $PK$.

\vspace{2mm}
\noindent\textsc{\textbf{Theorem 1.} (RP Designation of $TK$)} { $PID_{RP}$ in $TK$ uniquely designates an RP with $ID_{RP}$, where $PID_{RP}= [t]ID_{RP}$, $t \in [1,n)$, $ID_{RP} = [r]G$, $r$ is a random number known only to the IdP, and $G$ is a generator on $\mathbb{E}$ of order $n$.}
%{\em Provided that $r$ is known only to the IdP, $PID_{RP}$ in the identity token uniquely designates the RP with $ID_{RP} = [r]G$.}}
%{\em If $TK$ is a valid identity token and $PID_{RP} = [t]ID_{RP}$ is satisfied, $PID_{RP}$ uniquely designates the RP that receives $TK$.}

\vspace{0.75mm}
\noindent\textsc{Proof.} In \usso, $PID_{RP}=[t]ID_{RP}$ is generated by a user based on the target RP's identity $ID_{RP}$ and a user-selected random number $t \in [1,n)$.
The target RP with $ID_{RP}$ receives $t$,
     and it will also calculate $PID_{RP}=[t]ID_{RP}$ to match $PID_{RP}$ extracted from a token received.
%It is computationally easy for any party who knows $ID_{RP}$ and $t$ to validate the $PID_{RP}$ in an identity token. A valid
Thus, $PID_{RP}$ always specifies an RP, i.e., %$PID_{RP}$ sent by a user in her identity-token request is calculated as $PID_{RP} = [t]ID_{RP}$, where $ID_{RP}$ is the target RP's identity and $t$ is a random number selected by the user and shared with this RP.
designates the target RP that knows $t$.

Next, according to Lemma 1, given $PID_{RP} = [t]ID_{RP}$, the probability that $PID_{RP}$ designates another RP with $ID_{RP'}$ is negligible. %This means that $PID_{RP}$ cannot be associated with any other RPs in the system.
Therefore, $PID_{RP}$ designates only the target RP with $ID_{RP}$ in the system.  \hfill $\square$

\vspace{2mm}
\noindent\textsc{\textbf{Lemma 1.}} { For any two RPs in a finite set of RPs, the probability of finding different numbers $t$ and $t' \in [1,n)$ that satisfy $[t]ID_{RP_j} = [t']ID_{RP_{j'}}$ is negligible, where $ID_{RP_j}=[r]G$, $ID_{RP_{j'}}=[r']G$, $r$ and $r'$ are different numbers unknown to the RPs, and $G$ is a generator on $\mathbb{E}$ of order $n$.}

%Based on the ECDLP we prove that, for adversaries, the probability of finding $t$ and $t'$ satisfying $[t]ID_{RP_j} = [t']ID_{RP_{j'}}$ is negligible, where $RP_j$ and $RP_{j'}$ are any two RPs in the finite set of RPs (i.e., $ID_{RP_j} = [r_j]G$ and $ID_{RP_{j'}} = [r_{j'}]G$, while $r_j$ and $r_{j'}$ are kept secret to the adversaries). This negligible probability means $PID_{RP_j} = [t]ID_{RP_j}$ designates \emph{only} the target RP with $ID_{RP_j}$.

\oldc
\vspace{0.75mm}
\noindent\textsc{Proof.}
Finding $t$ and $t'$ that satisfy $[t]ID_{RP_j} = [t']ID_{RP_{j'}}$, i.e., a $PID_{RP}$ collision, can be described as a game $\mathcal{G}_c$ between an adversary and a challenger: the adversary receives from the challenger a finite set of RP identities, i.e., $ID_{RP_1}$, ..., $ID_{RP_m}$, where $m$ is the number of RPs in the system, and outputs $(a, b, t, t')$ where $a \neq b$. If $[t]ID_{RP_a}=[t']ID_{RP_b}$, which occurs with a probability ${\rm Pr}_s$, the adversary succeeds in this game.
%The attack success probability is defined as ${\rm Pr}_s$.

Figure \ref{fig:ecdlp_algorithm} depicts a probabilistic polynomial time (PPT) algorithm $\mathcal{D}^*_c$, which is constructed based on this game $\mathcal{G}_c$, to solve the elliptic curve discrete logarithm problem (ECDLP): find a number $x \in \mathbb{Z}_n$ satisfying $Q = [x]G$,
where $Q$ is a point on $\mathbb{E}$ and $G$ is a generator on $\mathbb{E}$ of order $n$.
The probability of solving the ECDLP using $\mathcal{D}^*_c$ is denoted as ${\rm Pr}\{\mathcal{D}^*_c(G, [x]G)=x\}$.
%For any PPT algorithm $\mathcal{D}$ is used to calculate $x$, we define the probability of finding $x$ as: %Therefore, the probability of finding $x$ using a probabilistic polynomial time (PPT) algorithm is negligible.
Due to the ECDLP assumption,
    ${\rm Pr}\{\mathcal{D}^*_c(G, [x]G)=x\} = \epsilon_{c}(\lambda)$ becomes negligible when the security parameter $\lambda$ is sufficiently large.

%where ${\rm Pr}\{\}$ denotes the probability.
%where $k$ denotes the security parameter and $\epsilon_{c}(k)$ becomes negligible when $k$ is sufficiently large.
%For any sufficiently large $k$, $m \ll 2^k$ since $m$ is a finite integer.


The input of $\mathcal{D}^*_c$ is in the form of ($G, Q$). Upon receiving an input ($G$, $Q$), the challenger first randomly chooses $r_1, \cdots, r_m$ in $\mathbb{Z}_n$ to calculate $[r_1]G, \cdots, [r_m]G$, randomly replaces $[r_j]G$ with $Q$, and sends $m$ RP identities to the adversary, which then returns the result ($a$, $b$, $t$, $t'$). Finally, the challenger calculates $s = t^{-1}t'r_b \bmod n$ and returns $s$ as the output of $\mathcal{D}^*_c$.

\begin{figure}[tb]
  \centering
  \includegraphics[width=0.97\linewidth]{fig/ecdlp_algorithm.pdf}
  \caption{The PPT algorithm $\mathcal{D}^*_c$ constructed based on the $PID_{RP}$ collision game, to solve the ECDLP problem.}
  \label{fig:ecdlp_algorithm}
\end{figure}

If the adversary succeeds in $\mathcal{G}_c$ and $[r_a]G$ happens to be replaced with $Q$,
 $\mathcal{D}^*_c$ outputs $s=x$ because $[tr_a]G = [t]Q = [t'r_b]G$. For the adversary, $Q$ is indistinguishable from any other RP identities in the input set, as $[r_j]G$ is randomly replaced by the challenger.
Hence, the probability of solving the ECDLP by $\mathcal{D}^*_c$ is formulated as:
\begin{equation*}
{\rm Pr}\{\mathcal{D}^*_c(G, [x]G)=x\} = {\rm Pr}\{s = x\}={\rm Pr}\{a=j\}{\rm Pr}_s=\frac{1}{m}{\rm Pr}_s
\end{equation*}

\newc
If the probability of finding $t$ and $t'$ satisfying $[t]ID_{RP_j} = [t']ID_{RP_{j'}}$ is non-negligible,
 then the adversary would have advantages in $\mathcal{G}_c$ and ${\rm Pr}_s$ is non-negligible regardless of $\lambda$.
Therefore, we would find that ${\rm Pr}\{\mathcal{D}^*_c(G, [x]G)=x\}$ also becomes non-negligible  when $\lambda$ is sufficiently large, because $m$ is a finite integer and $m \ll 2^\lambda$.
\oldc
This violates the ECDLP assumption. Thus, the probability of finding $t$ and $t'$ that satisfy $[t]ID_{RP_j} = [t']ID_{RP_{j'}}$ is negligible. \hfill $\square$


\newc
\vspace{2mm}
\noindent\textsc{\textbf{Theorem 2.} (User Identification of $TK$)} { $PID_U= [ID_U]PID_{RP}$ in $TK$ uniquely identifies an account at the RP designated by $PID_{RP}$, and this account is uniquely mapped to a user with $ID_U$.}

%In the identity token binding $PID_U$ and $PID_{RP}$, the user pseudo-identity $PID_U$ identifies the authenticated user with $ID_U$, % as $Acct = [ID_U][ID_{RP}]$,
%and only this user,  at the target RP with $ID_{RP} = [r]G$.}
%That is, in UPPRESSO, $Acct$ identifies the mapping $[ID_u][ID_{RP}]$.

\vspace{0.75mm}
\noindent\textsc{Proof.}
To issue an identity token requested for $PID_{RP}$, the honest IdP calculates $PID_U = [ID_U]PID_{RP}$ following Equation \ref{equ:PIDU} after authenticating the user with $ID_U$. The designated RP then calculates $Acct = [t^{-1}]PID_{U} = [ID_U]ID_{RP}$ following Equation \ref{equ:AccountNotChanged}.
$Acct = [ID_U]ID_{RP}$ is a \emph{permanent} identifier that is determined once the RP and the user have registered at the IdP. Therefore, $PID_U$ in $TK$ always identifies an $Acct$ at the designated RP, and $Acct$ is mapped to a user with $ID_U$.

Next, we prove that it \emph{uniquely} identifies one user in the system and one account at the RP. Since $\mathbb{E}$ is a finite cyclic group, $ID_{RP} = [r]G$ is also a generator on $\mathbb{E}$ of order $n$. Given a user with $ID_U$, $Acct = [ID_U]ID_{RP}$ is a unique point on $\mathbb{E}$ for $1 \leq u < n$, which is uniquely associated to $ID_U=u$. \hfill $\square$

%According to the \dy~model, the RP may receive a $TK$ issued for another RP from a malicious RP script. However, if $PID_{RP} = [t]ID_{RP}$ holds, the RP can verify its designation by $PID_{RP}$ based on Theorem 1 and associates both $t$ and $TK$ with the current login.
%Then, the RP can calculate $Acct$ using $t$ and $PID_U$ following Eq.~\ref{equ:Account}. $Acct$ is determined only by $ID_U$ and $ID_{RP}$ based on Eq.~\ref{equ:AccountNotChanged}, which are permanent identifiers issued by the IdP during registration. Hence, $Acct$ cannot be manipulated by adversaries and is independent of logins. Therefore, in a user's multiple logins to the same RP, $PID_U$s are always mapped to one and only one $Acct$ at that RP.

%The detailed process of proof is shown in Appendix.

\newc
\vspace{2mm}
\noindent\textsc{\textbf{Theorem 3.} (Token Integrity)} { An identity token issued by the IdP cannot be forged or manipulated.}

%{\em Consider an arbitrary identity token $TK$ binding $PID_{RP}$ and $PID_U$. An honest RP accepts $TK$ if and only if $TK$ is valid, $PID_{RP}$ designates this RP with $ID_{RP}$, and $PID_U$ uniquely identifies a user account $Acct=[ID_U]ID_{RP}$ at this RP, indicating that $TK$ binds $ID_{RP}$ and $Acct$.}

%An honest RP accepts only identity tokens binding its pseudo-identity $PID_{RP}$ and the authenticated user's pseudo-identity $PID_U$, and actually binding $ID_{RP}$ and $Acct=[ID_U]ID_{RP}$, when $SK$ is held by only the IdP.

\vspace{0.75mm}
\noindent \textsc{Proof.} Identity tokens are generated and signed by the honest IdP using its private key, which is sufficiently protected at the IdP against adversaries.
%Meanwhile, the IdP's public key $PK$ is sent from the IdP to the RPs.
%and $PK$ which is pre-installed by an RP cannot be manipulated by adversaries.
With the pre-installed public key, an RP verifies the identity tokens it receives and rejects any forged or manipulated identity tokens. \hfill $\square$

%Due to the one-to-one mapping between (\emph{a}) the pair of $Acct$ and $PID_{RP}$ and (\emph{b}) the triple ($PID_U$, $PID_{RP}$, $t$), $TK$ binds $ID_{RP}$ and $Acct$ implicitly. \hfill $\square$

% A signed identity token binds $PID_{RP} = [t]ID_{RP}$ and $PID_U = [ID_U]PID_{RP}$, % $Acct$ and $ID_{RP}$ implicitly,
% and any breaking results in some failed checking or verification in the login flow as below.
%First of all, the identity token is signed by the honest IdP using $SK$ and verified by the RP using $PK$, so any modification will be rejected by the RP.
% According to the proof of RP designation, % there is no $t' \neq t$ but satisfying that $PID_{RP} = [t]ID_{RP_j} = [t']ID_{RP_{j'}}$.
% $PID_{RP}$ identifies only the RP with $ID_{RP}$; according to the proof of user identification, $PID_U$ identifies only the user with $Acct = [ID_U]ID_{RP}$ at the RP.
%Therefore, the identity token explicitly binding $PID_U$ and $PID_{RP}$, matches \emph{only} one $ID_{RP}$ and \emph{only} one $Acct = [t^{-1}]PID_{U}$.
%Therefore, $Acct$ and $ID_{RP}$ are actually bound in the token by the IdP's signatures,


\vspace{2mm}
\noindent{\textsc{\textbf{Theorem 4.} (Token Confidentiality)}} { An identity token is accessible only to its designated RP, besides the requesting user and the IdP.}
%An identity token is accessible to only the authenticated user and the target RP, in addition to the IdP signing this token.

\vspace{0.75mm}
\noindent\textsc{Proof.}
An identity token is generated by the IdP and then sent to the requesting user (i.e., its IdP script).
The IdP script verifies if $ID_{RP}$ specified in a verified RP certificate is designated by $PID_{RP}$ in the identity token and forward the token to the correct RP script, which is downloaded from the origin of $Enpt_{RP}$ specified also in the RP certificate. %Because the IdP script calculates $PID_{RP} =[t]ID_{RP}$ based on $ID_{RP}$ in this verified RP certificate, this RP is designated in the identity token and will receive the token from the RP script.
As all the communications between the IdP, RPs, and users are protected by HTTPS and two scripts communicate with each other through restricted \verb+postMessage+ HTML5 channels, according to the conclusions of the DYU model, the identity tokens cannot be leaked to any other entities. \hfill $\square$


\vspace{2mm}
\noindent\textsc{\textbf{Theorem 5.} (Security)} {\usso~is secure.}

\vspace{0.75mm}
\noindent\textsc{Proof.}
According to the formal analysis on SSO security \cite{SPRESSO, FettKS14},
    an SSO system satisfying the following two requirements is considered security in the first adversarial scenario: (\emph{a}) an adversary never obtains a valid identity token of an honest user and presents it to an honest RP, and (\emph{b}) an honest user never presents a valid identity token that is not issued for herself to an honest RP.
%(\emph{a}) an adversary never presents an identity token that is accepted by an honest RP to derive an honest user's account at this RP and (\emph{b}) an honest user never presents an identity token that is accepted by an honest RP to derive another user's account at this RP.


%In a secure SSO that is robust against impersonation attacks, an honest user should be able to prove her identity to the target RP with an identity token issued by the IdP. As pseudo-identities are used in \usso, an RP needs to verify that ({\em a}) $PID_{RP}$ in the received identity token is a transformation from its own $ID_{RP}$ and tied to the current login, which requires {\em RP designation}; and ({\em b}) $PID_U$ included in the identity token can be associated with a unique, long-term user account it maintains, which requires {\em user identification}. Besides, the identity token for an honest RP should not be intercepted by malicious users or RPs, requiring the {\em confidentiality of identity tokens}, nor be forged or tampered with to include a fake $PID_U$ or $PID_{RP}$, requiring the {\em integrity of identity tokens}.

Following the login flow of \usso, since the confidentiality and integrity of identity tokens are satisfied, no adversary could obtain a valid identity token that is issued to an honest user for accessing an honest RP. %and accepted by an honest RP.
Meanwhile, if an adversary presents an identity token to an honest RP, which may be issued to the adversary for accessing another RP or to an honest user for accessing a colluding RP, RP designation and user identification ensure that from this identity token, the honest RP does not derive an honest user's account.
%Meanwhile, $PID_U$ can only be calculated by the IdP and the user, since no one else knows or could intercept $u$ according to the DYU model. \oldc

According to the login flow and the \dyu~model, an honest user always obtains identity tokens issued to herself, because $PID_U$ is calculated based on her own identity.
  %% which is protected against adversaries. 这一句,对于security没有用,protecting u是用于privacy。
The confidentiality and integrity of identity tokens ensure that an honest user never presents any token issued to someone else to an honest RP. Finally, with user identification,\footnote{RP designation is a precondition of user identification in \usso, but this may not always be necessary for other SSO systems.} her identity tokens are always associated to her account at the target RP.
%According to the conclusions of the DYU model, $PID_U$ is calculated based on the user when the IdP issues an identity token for an authenticated user.
%Moreover, because the confidentiality and integrity of identity tokens are satisfied, an honest user never presents a valid token that is issued to other users.
%When the user identification is satisfied, this token always derives this honest user's account.
%So, an honest user never presents an identity token that is accepted by an honest RP to derive another user’s account at this RP.

Thus, \usso~is a secure SSO system.
\hfill $\square$

%Finally, according to Theorems 1, 2, 3, 4, and 5, UPPRESSO is secure.

\subsection{Privacy}
\label{sec-:analysis}
Next, we prove that \usso~effectively prevents privacy threats introduced by IdP-based login tracing and RP-based identity linkage in the respective adversarial scenarios.

\newc
We show that an honest-but-curious IdP cannot trace a user's login activities. In \usso, a user sends an identity-token request in each login to the IdP (Step 3.1 in Section~\ref{implementations}), which reveals only the target RP's \emph{ephemeral} pseudo-identity $PID_{RP}$ along with the user's identity $ID_U$.
The IdP cannot (\emph{a}) associate multiple logins with different $PID_{RP}$s to a given RP,
 or (\emph{b}) distinguish a login to one RP from other logins to another RP based on their $PID_{RP}$s,
 because $PID_{RP}$ is \emph{indistinguishable} from random variables to the IdP, as proved in Theorem 6.

\vspace{2mm}
\noindent{\textsc{Theorem 6. (Privacy against the IdP)}} { In \usso, the IdP cannot distinguish $PID_{RP} = [t]ID_{RP}$ from a random variable on $\mathbb{E}$, where $t$ is random in $\mathbb{Z}_n$.} %and unknown to the IdP.}

\vspace{0.75mm}
\noindent \textsc{Proof.}
Consider a finite cyclic group $\mathbb{E}$ where the number of points on $\mathbb{E}$ is $n$.
Because $G$ is a generator of order $n$, $ID_{RP} = [r]G$ is also a generator on $\mathbb{E}$ of order $n$. According to the DYU model, $t$ is randomly chosen in $\mathbb{Z}_n$ and kept unknown to the IdP. Therefore, $PID_{RP} = [t]ID_{RP}$ is \emph{indistinguishable} from a point $Q$ that is randomly chosen on $\mathbb{E}$ \cite{oprf-proved,strong-oprf,voprf-proved}.\footnote{This property has been actually proved in OPRFs \cite{oprf-proved,strong-oprf,voprf-proved},
    where the OPRF server works similarly to the IdP and learns \emph{nothing}.} \hfill $\square$

%In order to receive $[k]Q$ for any given $Q$ on $\mathbb{E}$, 
%    an OPRF client first generates a random number $e$, converts $Q$ to $[e]Q$,
%        and sends $[e]Q$ to the server.
%Then, the OPRF server returns $[ek]Q$, and the client obtains $[k]Q$ by using $e$.


%Consider a user's two arbitrary logins to an RP where $PID^{i}_{RP} = [t]ID_{RP}$ and $PID^{i'}_{RP} = [t']ID_{RP}$, and her another login to another RP where $PID^{i''}_{RP'} = [t'']ID_{RP'}$. According to the \dyu~model, $t$, $t'$, and $t''$ are randomly selected by the IdP script and shared only with the corresponding RP through the RP script, so they are unknown to the IdP. As stated in Theorem 2, $ID_{RP} = [r]G = Q$ and $ID_{RP'} = [r']G = P$, where $P$ and $Q$ are two different points on the curve $\mathbb{E}$. Therefore, in the IdP's view, $[t]Q$, $[t']Q$, and $[t'']P$ are three random points on $\mathbb{E}$ when $t$, $t'$, and $t''$ are random and they are indistinguishable \cite{oprf-proved}. \hfill $\square$

%as well as from a $PID_{RP}$ generated in an arbitrary login for an arbitrary user to log in to an

%\textcolor{blue}{The information accessible to the IdP and derived from the RP's identity, is only $PID_{RP}$ in identity-token requests, where $PID_{RP} = [t]ID_{RP}$ is calculated by a user. % and $t$ is kept secret to the IdP.
%So the prevention against the IdP-based login tracing in UPPRESSO is expressed formally as below.}

% \vspace{1mm}
% \noindent\textcolor{blue}{\textbf{Privacy against the IdP.}~~If $t$ is random in $\mathbb{Z}_n$ and unknown to the IdP, the IdP cannot infer any information about $ID_{RP_j}$ or link any pair of $PID_{RP_j}^i$ and $PID_{RP_{j'}}^{i'}$ ($i \neq i'$ but $j = j'$), from a user's identity-token requests for $PID_{RP_j}^i$ ($i,j = 1, 2, \cdots$).}

% \vspace{0.75mm}
% \noindent\textbf{Proof.}
% Because (\emph{a}) $ID_{RP} = [r]G$ is also a generator of order $n$, where $G$ is a generator of finite cyclic group $\mathbb{E}$ \textcolor{blue}{and (\emph{b}) $t$ is a random number in $\mathbb{Z}_n$ and kept unknown to the IdP,} from the IdP's view,
%  $PID_{RP}$ is \emph{indistinguishable} from a random variable on $\mathbb{E}$.
% Thus, the IdP cannot infer any information about $ID_{RP}$ from $PID_{RP} = [t]ID_{RP}$, or distinguish $[t]ID_{RP_j} = [tr]G$ from any other $[t']ID_{RP_{j'}} = [t'r']G$. So the IdP-based login tracing is impossible. $\square$


\vspace{2mm}

In the third adversarial scenario, malicious RPs collude with malicious users, attempting to link the logins of honest users across these colluding RPs.
That is, when $c+1$ malicious RPs and users collude in \usso,
    given any login $\mathcal{L'}$ initiated by an honest user $U'$ to one of the malicious RPs,
    this RP with all the other colluding entities aims to link $\mathcal{L'}$ to the logins at other $c$ colluding RPs
     which are also initiated by $U'$.
To show that \usso~prevents such linkage, we prove in Theorem 7 that the colluding adversaries cannot distinguish a login initiated by $U'$ to some of the $c$ malicious RPs from the ones initiated by any other honest users to the $c$ malicious RPs.
As a result, given $\mathcal{L'}$ and a shared view of all logins to the colluding RPs, the adversaries have no clue to link $\mathcal{L'}$ to one login instead of another (or even to a |$w$| subset of logins instead of others), because any login is indistinguishable.
%Malicious RPs colluding with malicious users, attempt to link logins of honest users across these malicious RPs. We prove that UPPRESSO prevents RP-based identity linkage, by showing that,  the colluding adversaries cannot distinguish a login by $U'$ to any of other $c$ malicious RP from the logins by any honest users to the $c$ malicious RPs.

Let us investigate the knowledge that an RP obtains from each login. In \usso, an RP receives a random number $t$ and an identity token enclosing $PID_{RP}$ and $PID_U$ in each login. Meanwhile, it knows its own identity $ID_{RP}$ and derives the user's account $Acct$.
For $PID_{RP}$ and $PID_U$ can be deduced from $ID_{RP}$ and $Acct$ with $t$, respectively, we ignore duplicated elements and define an RP's knowledge about a login $\mathcal{L}$ as $L=(ID_{RP}, t, Acct)=(ID_{RP}, t, [ID_{U}]ID_{RP})$. %Note that a login $L$ includes the user's account at the RP.
%When $v$ malicious users collude with malicious RPs, they could notify the RPs about their own logins. As a result, the colluding RPs could further associate their knowledge to form a shared view: %, denoted by

%Consider the system with $c$ malicious RPs and $v$ malicious users. Without loss of generality, assume the first $c$ RPs are malicious and share their views about previously authenticated users (denoted as $\mathbb{U}_c$). Furthermore, assume the first $v$ users seen by each malicious RP are malicious. They provide additional information for linking their $PID_U$s and $L_{i, j}$s across colluding RPs. Then, a malicious $RP_j$ constructs a list $\mathbb{L}_j=\{L_{i,j}\}$ for $1 \le i \le v$ and $1 \le j \le c$, where $\mathbb{L}_j \subseteq \mathbb{V}_j$.
%%Among them, assume the first $w$ users, without loss of generality, have logged in to all $c$ RPs. %$c \in [2,m]$.
%and share it with other colluding RPs.

%The colluding RPs share $\mathbb{V}_j$s and $\mathbb{L}_j$s in RP-based linkage attacks, where a malicious RP aims to link an honest user logged in to itself (e.g., $PID_a$ and $L_{a,j}$) to another user authenticated to a colluding RP (e.g., $PID_b$ and $L_{b,k}$ for any $k \neq j$) by finding a relationship between $L_{a,j}$ and $L_{b,k}$.

\vspace{2mm}
\noindent \textsc{Theorem 7. (Privacy against Colluding RPs)} { When $c+1$ malicious RPs and $v$ users collude in \usso, given any login $\mathcal{L'}$ initiated by an honest user $U'$ to a malicious RP, the adversaries cannot distinguish a login initiated by $U'$ to any of the other $c$ malicious RP from the logins initiated by any honest users to these $c$ colluding RPs.}


%the shared knowledge $\mathbb{V}=\bigcup_{1 \le j \le c}\mathbb{V}_j$ and $\mathbb{L}=\bigcup_{1 \le j \le c}\mathbb{L}_j$ from a group of $c$ colluding RPs and $v$ colluding users, any malicious RP cannot differentiate if an honest user who has logged in to itself is among the users who have logged in to any of the colluding RPs.}

%Next, we show that the colluding RPs have limited knowledge about the users who have authenticated with them (using $PID_U$s). Even if they share this knowledge, it is insufficient to distinguish the pseudo-identities of different users nor link the ones belonging to the same user.

\vspace{0.75mm}
\noindent\textsc{Proof.}
When a malicious RP, denoted as $RP'$, receives and verifies an identity token from an honest user $U'$,
 it obtains $L' = (ID_{RP'}, t', [ID_{U'}]ID_{RP'})$. Suppose another malicious RP, denoted as $RP_j$, colludes with $RP'$ and shares the knowledge about all its $w$ logins, denoted as $\mathbb{L} = \{{L_{i}}\}$ and $1 \leq i \leq w$. These logins are initiated by a set of users, denoted as $\mathcal{U}_j$, including a set of honest users $\mathcal{U}_{j,H} \subseteq \mathcal{U}_{j}$ and
 a subset of malicious users $\mathcal{U}_{j,M} \subseteq \mathcal{U}_j$.
$\mathcal{U}_j = \mathcal{U}_{j,H} \cup \mathcal{U}_{j,M}$.

Let us first assume malicious users do not collude with malicious RPs, so they do not reveal their accounts.
As a result, even if $RP_j$ and $RP'$ collude, they are not told if some of the logins are initiated by the same malicious user. According to Lemma 2, given $L'$ and ``any subset'' of $\mathbb{L}$, $RP'$ and $RP_j$ cannot decide if $U' \in$ ``the corresponding subset'' of $\mathcal{U}$ or not.
%Therefore, they cannot determine if a login in $\mathbb{L}$ is initiated by $U'$ or by any other users.
     %which means they cannot link the $PID_U$ of $L'$ to the $PID_U$ of any login in $\mathbb{L}$.
     % 不需要专门谈PID_U, colluding user合作之后,就是account就连起来了、不必专门谈PID_U。
That is, given $\mathcal{L}'$,
any $\mathcal{L}$ in $\mathbb{L}$ is \emph{indistinguishable} from the others in $\mathbb{L}$,
 because the above conclusion is about \emph{any subset}  of $\mathbb{L}$ (i.e., $\mathbb{L}$ $\setminus$ some $\mathcal{L}_k$s).


Since $RP_j$ can be any of the $c$ malicious RPs in the system, we can easily extend this proof to include all of the $c$ malicious RPs, by defining $\mathbb{L}^{\Sigma}$ as the union of the logins to all of the $c$ colluding RPs and $\mathcal{U}^{\Sigma}$ as the set of users who initiate these logins.
That is, given $\mathcal{L}'$,
    any $\mathcal{L}$ in $\mathbb{L}^{\Sigma}$ is indistinguishable.

%If we denote the set of users who initiate logins to all malicious RPs as $\mathcal{U}$ and the set of honest users in $\mathcal{U}$ as $\mathcal{U}_H$, the RP-based linkage attack is to determine if $U' \in \mathcal{U}_H$.


%Let us denote $c$ malicious RPs as $RP_j$ ($1 \leq j \leq c$) and the $(c+1)$th RP as $RP'$.

%We first consider a malicious RP denoted as $RP'$ that receives a login $L'$ by an honest user $U'$, the other $c$ malicious RPs denoted as $RP_j$ ($1 \leq j \leq c$), and no malicious user. According to Lemma 2, a pair of $RP_j$ and $RP'$ cannot decide whether a login initiated by $U'$ to $RP'$ is initiated by one of the users visiting $RP_j$ or not.

%After $RP'$ colludes with each $RP_j$ ($1 \leq j \leq c$),  we find that, given $L'$ initiated by $U'$ to $RP'$, the adversaries cannot distinguish a login by $U'$ to any $RP_j$, from the logins by any users to $RP_j$ ($1 \leq j \leq c$). Therefore, given a login $L'$ by $U'$ to a malicious RP, the adversaries cannot distinguish a login by $U'$ to any of the other $c$ malicious RPs, from the logins by any users to the $c$ malicious RPs.

Next, consider $v$ malicious users colluding with malicious RPs.
 %and sharing their $PID_U$s.
 % 不需要专门谈PID_U, colluding user合作之后,就是account就连起来了、不必专门谈PID_U。
Now, $c$ colluding RPs could link logins initiated by the same malicious user $U_i$ as $(L_{i,1},..., L_{i,c})$. Denote their shared knowledge about all $w$ logins as $\mathbb{L}$ and assume the first $v$ users are malicious. We have:

{\centering $\mathfrak{L}=\left \{ \begin{matrix} L_{1,1},&L_{1,2},&\cdots,&L_{1,c} \\
L_{2,1},& L_{2,2},&\cdots,&L_{2,c}  \\
\cdots,&\cdots,&L_{i,j},&\cdots  \\
L_{v,1},&L_{v,2},&\cdots,&L_{v,c}
\end{matrix}\right\}$
\par}

% $\mathfrak{L}=\left \{ \begin{matrix}
% L_{1,1},&L_{1,2},&\cdots,&L_{1,c}\\
% L_{2,1},& L_{2,2},&\cdots,&L_{2,c}\\
% \cdots,&\cdots,&L_{i,j},&\cdots\\
% L_{v,1},&L_{v,2},&\cdots,&L_{v,c}
% \end{matrix}\right\}$
\noindent where $L_{i,j} = (ID_{RP_j}, t_{i,j}, [ID_{U_i}]ID_{RP_j})$, $1 \leq i \leq v$ and $1 \leq j \leq c$.
$\mathfrak{L}$ is an \emph{ordered} subset of $\mathbb{L}$ denoting the linked logins that are initiated by $v$ malicious users to the $c$ malicious RPs. Note that $\mathcal{U}_M=\{u_1,...,u_v\}$ is known to all colluding entities.

Consider any subset of $\mathbb{L}^{\Sigma}$:
   \{  Given $L'$, and ``the corresponding subset matrix'' of $\mathfrak{L}$ that is included in the subset of $\mathbb{L}^{\Sigma}$, a malicious $RP'$ and the $c$ (or $c-\delta$) colluding RPs cannot determine if $U' \in$ ``the corresponding subset'' of $\mathcal{U}_M$ according to Lemma 3.
     Consequently, they cannot determine if $U' \in$ \{``the corresponding subset'' of $\mathcal{U}^{\Sigma}$ $\setminus$ ``the corresponding subset'' of $\mathcal{U}_M$\}, combined with the deduction from Lemma 2. \}

Finally, even with $v$ malicious user,
    any $\mathcal{L}$ in $\mathbb{L}^{\Sigma} \setminus \mathfrak{L}$ is \emph{indistinguishable},
        because the above conclusion is about ``any subset of $\mathbb{L}^{\Sigma}$ $\setminus$ \{the subset by malicious user\}''.
 \hfill $\square$


%Given $L'$ and $\mathbb{L}$, a malicious $RP'$ and the other $c$ colluding RPs cannot determine if $U' \in \mathcal{U}$ according to Lemma 2.

%This indicates the additional information provided by colluding users about their own logins does not assist the adversaries in determining the relationship between the honest users' logins.
%an arbitrary login in $(\mathbb{L}-\mathfrak{L})$, which is not initiated by a malicious user, is initiated by the honest user $U'$ or by another honest user in the system.
%As a result, they still cannot link $L'$ to any of the logins initiated by $U'$.

% Therefore, we prove that \usso~prevents RP-based identity linkage, even when multiple malicious RPs and malicious users collude.
% 这句话在Theorem 7之前的那一段谈过了。


\vspace{2mm}
\noindent\textsc{Lemma 2.} { Given the logins to $RP$ by $w$ users denoted as $\mathbb{L} = \{{L_i}\}$, where $L_i = ([r]G, t_i, [u_i r]G)$ and $1 \leq i \leq w$, two colluding RPs, $RP$ and $RP'$, cannot decide whether a login to $RP'$ denoted as $L' = (ID_{RP'}, t', [ID_{U'}]ID_{RP'})$, is initiated by one of the $w$ users or not, i.e., $ID_{U'} \in \{u_1, u_2, \cdots, u_w\}$ or $ID_{U'} \in \mathbb{Z}_n$.}



\vspace{0.75mm}
\noindent{\textsc{Proof.} We define a game $\mathcal{G}_r$ between an adversary and a challenger,
    to describe this RP-based login linkage:
 the adversary receives $\mathbb{L} = \{{L_i}\}$ and $L' = (ID_{RP'}, t', [ID_{U'}]ID_{RP'})$ from the challenger, and outputs the result $s = 1$ if it decides $ID_{U'} \in \{u_1, u_2, \cdots, u_w\}$ or $s = 0$ if $ID_{U'}$ is randomly chosen in $\mathbb{Z}_n$.
The adversary succeeds in $\mathcal{G}_r$ with an advantage $\mathbf{Adv}_{A}$:
\begin{align*}
&{\rm Pr}_1={\rm Pr}\{\mathcal{G}_r(\mathbb{L}, L'| ID_{U'} \in \{u_1, u_2, \cdots, u_w\})=1\} \\
&{\rm Pr}_2={\rm Pr}\{\mathcal{G}_r(\mathbb{L}, L'|ID_{U'} \in \mathbb{Z}_n)=1\}\\
&{\mathbf{Adv}}_{A}=|{\rm Pr}_1-{\rm Pr}_2|
\end{align*}

\begin{figure}[tb]
  \centering
  \includegraphics[width=0.99\linewidth]{fig/dalgorithm-lemma2.pdf}
  \caption{The PPT algorithm $\mathcal{D}^*_r$ constructed based on the RP-based identity linkage game, to solve the ECDDH problem.}
  \label{fig:dalgorithm}
\end{figure}


Based on $\mathcal{G}_r$, we design a PPT algorithm $\mathcal{D}^*_r$ to solve the elliptic curve decisional Diffie-Hellman (ECDDH) problem:
 given $(G, [x]G$, $[y]G$, $[z]G)$, decide whether $z$ equals to $xy$ or is randomly chosen in $\mathbb{Z}_n$,
    where $G$ is a point on an elliptic curve $\mathbb{E}$ of order $n$, and $x$ and $y$ are integers randomly and independently chosen in $\mathbb{Z}_n$.
%The probability of solving the ECDDH problem using $\mathcal{D}^*_r$ is defined as $|{\rm Pr}^*_1 - {\rm Pr}^*_2|$, where:
Therefore, according to the ECDDH assumption,
    $|{\rm Pr}^*_1 - {\rm Pr}^*_2| = \epsilon_{r}(\lambda)$ becomes negligible when the security parameter $\lambda$ is sufficiently large,
        where
\begin{align*}
&{\rm Pr}^*_1 =  {\rm Pr}\{\mathcal{D}^*(G, [x]G, [y]G, [xy]G)=1\} \\
&{\rm Pr}^*_2 =  {\rm Pr}\{\mathcal{D}^*(G, [x]G, [y]G, [z]G)=1\}
\end{align*}


As shown in Figure \ref{fig:dalgorithm}, the input of $\mathcal{D}^*_r$ is in the form of ($G, Q_1=[x]G, Q_2=[y]G, Q_3=[z]G$).
Upon receiving the input, the challenger randomly chooses $u_i$, $r$, $t_i$, and $t'$ in $\mathbb{Z}_n$ for $1 \leq i \leq w$, and assigns $\mathbb{L} = \{{L_i}\} = \{([r]G, t_i, [u_ir]G)\}$.
Then, it randomly selects $d \in [1,w]$, replaces $[u_dr]G$ with $[r]Q_1=[xr]G$ in $L_d$; i.e., $\mathbb{L} = \{([r]G, t_d, [r]Q_1)\} \cup \{([r]G, t_i, [u_ir]G)\}$ ($1 \leq i \leq d-1$, $d+1 \leq i \leq w$). Finally, the challenger constructs $L' = (Q_2, t', Q_3)$. When the adversary receives $\mathbb{L}$ and $L'$, it outputs $s$.


According to the above construction of $\mathbb{L}$ and $L'$,
 $x$ is actually inserted into $\mathbb{L}$ as $u_d$, and $z/y$ is assigned to $ID_{U'}$.
Thus, if $z = xy$, then $z/y = x$ and $ID_{U'} \in \{u_1, u_2, \cdots, u_w\}$; otherwise, $ID_{U'} \in \mathbb{Z}_n$.
The advantage $|{\rm Pr}^*_1 - {\rm Pr}^*_2|$ of solving the ECDDH problem using $\mathcal{D}^*_r$ is deduced as:
\begin{align*}
&{\rm Pr}^*_1 =  {\rm Pr}\{\mathcal{D}^*_r(G, [x]G, [y]G, [xy]G)=1\} \\=&{\rm Pr}\{\mathcal{G}_r(\mathbb{L}, L'|ID_{U'} \in \{u_1, u_2, \cdots, u_w\})=1\} = {\rm Pr}_1 \\
&{\rm Pr}^*_2= {\rm Pr}\{\mathcal{D}^*_r(G, [x]G, [y]G, [z]G)=1\} \\=&  {\rm Pr}\{\mathcal{G}_r(\mathbb{L}, L'|ID_{U'} \in \mathbb{Z}_n)=1\} = {\rm Pr}_2\\
&|{\rm Pr}^*_1-{\rm Pr}^*_2|=|{\rm Pr}_1-{\rm Pr}_2|={\mathbf{Adv}}_{A}
\end{align*}

If the adversary has advantages in $\mathcal{G}_r$ to decide whether $L'$ is initiated by $U'$ chosen from \{${U_1}, {U_2}, \cdots, {U_w}$\} or randomly from the universal user set,
    then $|{\rm Pr}^*_1-{\rm Pr}^*_2|=|{\rm Pr}_1-{\rm Pr}_2|$ is non-negligible regardless of $\lambda$.
This violates the ECDDH assumption.
Thus, $RP$ and $RP'$ cannot decide whether $L'$ is initiated by a user in \{${U_1}, {U_2}, \cdots, {U_w}$\} or randomly from the universal user set.
\hfill $\square$


\oldc
\vspace{2mm}
\noindent\textsc{Lemma 3.} { Given
    the logins visiting $c$ RPs by $v$ colluding users denoted as
 $\mathfrak{L}=\left \{ \begin{matrix}
L_{1,1},&L_{1,2},&\cdots,&L_{1,c}\\
L_{2,1},& L_{2,2},&\cdots,&L_{2,c}\\
\cdots,&\cdots,&L_{i,j},&\cdots\\
L_{v,1},&L_{v,2},&\cdots,&L_{v,c}
\end{matrix}\right\}$, where
 $L_{i, j}=([r_j]G, t_{i,j}, [u_ir_j]G)$, $1 \le i \le v$ and $1 \le j \le c$,
the $c$ RPs and $RP'$ cannot collude to decide whether a login to $RP'$ denoted as $L' = (ID_{RP'}, t', [ID_{U'}]ID_{RP'})$, is initiated by one of the $v$ users or not, i.e., $ID_{U'} \in \{u_1, u_2, \cdots, u_v\}$ or $ID_{U'} \in \mathbb{Z}_n$.}


% \vspace{1mm}
% In every login, without knowing $u$ and $r$, an RP holds $ID_{RP}$ and $Acct$, receives $t$, calculates $PID_{RP}$, and verifies $PID_{RP}$ and $PID_U$ in the identity token. After filtering out the redundant information (i.e., $PID_{RP}= [t]{ID_{RP}}$ and $Acct = [t^{-1}]PID_{U}$), the RP actually receives $(ID_{RP}, t, Acct) = ([r]G, t, [ur]G)$. Therefore, in \usso~the prevention against the RP-based identity linkage is expressed as follows.

% \vspace{1mm}
% \noindent\textcolor{blue}{\textbf{Privacy against Colluding RPs.}~~Provided that $u$ and $r$ are kept unknown to RPs,
% based on the collected information of logins by $v$ users,
% $c$ colluding RPs cannot decide whether a login to another RP is initiated by one of these $v$ users or not,
%     where
%     the collected logins are denoted as $\mathfrak{L}=\left\{ \begin{matrix}
% L_{1,1}, & L_{1,2}, & \cdots, & L_{1,c}\\
% L_{2,1}, & L_{2,2}, & \cdots, & L_{2,c}\\
% \cdots, & \cdots, & \cdots, & \cdots\\
% L_{v,1}, & L_{v,2}, & \cdots, & L_{v,c}
% \end{matrix}\right\}$, $L_{i, j} = (ID_{RP_j}, t_{i, j}, [ID_{U_i}]{ID_{RP_j}}) = ([r_j]G, t_{i,j}, [u_ir_j]G)$,
%     and the login to $RP_{c+1}$ is $L'=(ID_{RP_{c+1}}, t', [ID_{U'}]ID_{RP_{c+1}}) = ([r_{c+1}]G, t', [u'r_{c+1}]G)$.}


\vspace{0.75mm}
\noindent{\textsc{Proof.} It is proved similarly to Lemma 2.
We define a game $\mathcal{G}_r$ between an adversary and a challenger to describe this login linkage:
 the adversary receives $\mathfrak{L}$  and $L'$ from the challenger,
  and outputs $s = 1$ if it decides $ID_{U'} \in \{u_1, u_2, \cdots, u_v\}$ or $s = 0$ if $ID_{U'}$ is randomly chosen in $\mathbb{Z}_n$.
The adversary succeeds in the game with $\mathbf{Adv}_{A}$:
\begin{align*}
&{\rm Pr}_1={\rm Pr}\{\mathcal{G}_r(\mathfrak{L}, L'|ID_{U'} \in \{u_1, u_2, \cdots, u_v\})=1\} \\
&{\rm Pr}_2={\rm Pr}\{\mathcal{G}_r(\mathfrak{L}, L'|ID_{U'} \in \mathbb{Z}_n)=1\}\\
&{\mathbf{Adv}}_{A}=|{\rm Pr}_1-{\rm Pr}_2|
\end{align*}

%We define the RP-based identity linkage game $\mathcal{G}_r$: after receiving $\mathfrak{L}$ and $L'$ from a challenger, the adversary outputs the result $s = 1$ if it decides $u' \in \{u_1, u_2, \cdots, u_v\}$ or $s = 0$ if $u'$ is randomly chosen in $\mathbb{Z}_n$. The adversary's advantage in $\mathcal{G}_r$ is defined as $\mathbf{Adv}_{A}$. %If the adversary is able to distinguish whether $u' \in \{u_1, u_2, \cdots, u_v\}$ or not, the adversary will have non-negligible advantages in $\mathcal{G}_r$ and ${\rm Adv}_A$ is non-negligible.

Based on $\mathcal{G}_r$, we design a PPT algorithm $\mathcal{D}^*_r$ %as shown in Figure \ref{fig:dalgorithm},
to solve the ECDDH problem: given $(G, [x]G$, $[y]G$, $[z]G)$, decide whether $z$ equals to $xy$ or is randomly chosen in $\mathbb{Z}_n$.
Thus, 
    $|{\rm Pr}^*_1 - {\rm Pr}^*_2| = \epsilon_{r}(\lambda)$ becomes negligible when $\lambda$ is sufficiently large,
        where
\begin{align*}
&{\rm Pr}^*_1 =  {\rm Pr}\{\mathcal{D}^*(G, [x]G, [y]G, [xy]G)=1\} \\
&{\rm Pr}^*_2 =  {\rm Pr}\{\mathcal{D}^*(G, [x]G, [y]G, [z]G)=1\}
\end{align*}

%\cite{GoldwasserK16}.
%That is, while there is the login to an RP, for colluded RPs, they cannot decide whether this login and any logins to other RPs are from the same user.
%
% Let $\mathbb{E}$ be an elliptic curve, and $G$ be a point on $\mathbb{E}$ of order $n$. For any PPT algorithm $\mathcal{D}$, the probability of distinguishing $([x]G$, $[y]G$, $[xy]G)$ and $([x]G$, $[y]G$, $[z]G)$
% is negligible, where $x$, $y$ and $z$ are integers randomly and independently chosen in $\mathbb{Z}_n$.


The input of $\mathcal{D}^*_r$ is in the form of ($G, Q_1=[x]G, Q_2=[y]G, Q_3=[z]G$).
Upon receiving an input, the challenger first chooses $\{u_1, u_2, \cdots, u_v\}$, $\{r_1, r_2, \cdots, r_c\}$, $\{t_{1, 1}, t_{1, 2}, \cdots, t_{v, c}\}$, and $t'$ randomly in $\mathbb{Z}_n$ and assigns $L_{i, j} = ([r_j]G, t_{i, j}, [u_ir_j]G)$.
   % for $1 \le i \le v$ and $1 \le j \le c$.
Then, it randomly chooses $d \in [1, v]$ and replaces $[u_d r_j]G$ with $[r_j]Q_1=[xr_j]G$ for $1\leq j \leq c$,
 to construct $\mathfrak{L}=\left \{ \begin{matrix}
L_{1,1},&L_{1,2},&\cdots,&L_{1,c}\\
L_{2,1},& L_{2,2},&\cdots,&L_{2,c}\\
\cdots,&\cdots,&L_{i,j},&\cdots\\
([r_{1}]G, t_{d, 1}, [r_{1}]Q_1),&\cdots,&\cdots,&([r_{c}]G, t_{d, c}, [r_{c}]Q_1)\\
\cdots,&\cdots,&\cdots,&\cdots\\
L_{v,1},&L_{v,2},&\cdots,&L_{v,c}
\end{matrix}\right\}$.
%
%$\mathfrak{L}$ and $L'$ as follows: it assigns $L_{i, j} = ([r_j]G, t_{i, j}, [u_ir_j]G)$, %$1\leq i \leq v$ and $1\leq j \leq c$, and then randomly chooses $d \in [1, v]$ to replace $[u_d r_j]G$ with $[r_j]Q_1=[xr_j]G$ for $1\leq j \leq c$. So,
%$L=$\{($[r_1]G$, $t_{1, 1}$, $[[u_1][r_1]G$), ($[r_2]G$, $t_{1, 2}$, $[u_1][r_2]G$), $\cdots$, ($[r_{\beta}]G$, $t_{\alpha, \beta}$, $[r_{\beta}]Q_1$), $\cdots$, ($[r_b]G$, $t_{a, b}$, $[u_a][r_b]G$)\}
%
Finally, it assigns $L' = (Q_2, t', Q_3) = ([y]G, t', [z/y][y]G)$. The challenger sends $\mathfrak{L}$ and $L'$ to the adversary, which outputs $s$.

According to the above construction of $\mathfrak{L}$ and $L'$,
    $x$ is actually inserted into $\mathfrak{L}$ as $u_d$
    and $z/y$ is assigned to $ID_{U'}$.
Thus, if $z = xy$, then $z/y=x$ and $ID_{U'} \in \{u_1, u_2, \cdots, u_v\}$;
    otherwise, $ID_{U'} \in \mathbb{Z}_n$.
The advantage $|{\rm Pr}^*_1 - {\rm Pr}^*_2|$ of solving the ECDDH problem using $\mathcal{D}^*_r$ is deduced as:
\begin{equation*}
|{\rm Pr}^*_1-{\rm Pr}^*_2|=|{\rm Pr}_1-{\rm Pr}_2|={\mathbf{Adv}}_{A}
\end{equation*}

The ECDDH assumption means in $\mathcal{G}_r$ the adversary does not have advantages, i.e., the colluding RPs cannot decide whether $L'$ is initiated by a user in \{${U_1}, {U_2}, \cdots, {U_v}$\} or randomly from the universal user set.
%    (indistinguishability of users to colluding RPs).
\hfill $\square$
