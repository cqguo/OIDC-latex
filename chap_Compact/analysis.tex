\section{The Analysis of Security and Privacy}
\label{sec:analysis}
In this section, we presents the analysis that UPPRESSO achieves the required properties of security and privacy.


\subsection{Security}
UPPRESSO satisfies with the four security requirements of identity tokens in SSO services,
     as listed in Section \ref{subsec:basicrequirements}.

%while the detailed process of proof is provided in the Appendix.
% ����汾���Ȳ��ܸ�¼��

\vspace{1mm}
\noindent\textbf{User Identification.}
The RP always derives an identical permanent account from different identity tokens binding $PID_U$ and $PID_{RP}$.
That is,
    in the user's any $i$-th and $i'$-th ($i \neq i'$) login instances to the RP,
 $\mathcal{F}_{Acct}(PID_{U}^i, PID_{RP}^i) = \mathcal{F}_{Acct}(PID_{U}^{i'}, PID_{RP}^{i'}) = [ID_U]ID_{RP}$


\vspace{1mm}
\noindent\textbf{RP Designation.}
An identity token binding $PID_U$ and $PID_{RP}$,
    designates the target RP, and only the target RP.
The RP calculates $PID_{RP}$ by itself with the trapdoor $t$ sent from the user,
    and checks $PID_{RP}$ in the identity token.
So the target RP will accept this token.
Meanwhile,
        the honest IdP guarantees, within its validity period, the $PID_{RP}$ will be dynamically registered only once.% $PID_{RP}$ will be bound in some identity token.

\vspace{1mm}
\noindent\textbf{Confidentiality.}
There is no event leaking the identity tokens to any malicious entity other than the authenticated user and the designated RP.
First of all, the communications among the IdP, RPs and users,
    are protected by TLS,
    and the \verb+postMessage+ HTML5 API ensures the dedicated channels between two scripts within the user agent,
    so that adversaries cannot eavesdrop the transmission of identity tokens.
Meanwhile, the honest IdP sends the identity token only to the authenticated user,
    and this user forwards it to the RP through $Enpt_{RP}$.
The binding of $Enpt_{RP}$ and $ID_{RP}$ is ensured by the RP certificate,
so only the designated target RP receives this identity token.
%The detailed process of proof is shown in Appendix.

\vspace{1mm}
\noindent\textbf{Integrity.}
The identity token binds $ID_U$ and $ID_{RP}$,
    implicitly or explicitly, and any breaking will result in some failed check or verification.
The integrity is ensured by the IdP's signatures:
 (\emph{a}) the identity token binding $PID_U$ and $PID_{RP}$, is signed by the IdP,
  and (\emph{b}) the relationship between $PID_{RP}$ and $t$ (or collision-free $H(t)$) is also bound by the IdP's signature in the $PID_{RP}$ registration result.
Thus,
    $ID_U$ and $ID_{RP}$ are actually bound by the IdP's signatures,
        due to the one-to-one mapping between (\emph{a}) the pair of $ID_U$ and $ID_{RP}$ and (\emph{b}) the triad of $PID_U$, $PID_{RP}$, and $t$.

%The detailed process of proof is shown in Appendix.

%The detailed process of proof is shown in Appendix.



\subsection{Privacy}
We show that UPPRESSO effectively prevents the attacks of IdP-based login tracing and RP-based identity linkage.

\vspace{1mm}
\noindent\textbf{IdP-based Login Tracing.}
The information accessible to the IdP and derived from the RP's identity,
    is only $PID_{RP}$, where $PID_{RP} = [t]ID_{RP}$ is calculated by the user.
Because  (\emph{a}) $t$ is a number randomly chosen from $(1,n)$ by the user and kept secret to the IdP
 and (\emph{b}) $ID_{RP} = [r]G$ and $G$ is the base point,
 the IdP has to view $PID_{RP}$ as randomly and independently chosen from $\mathbb{E}$.
So, the IdP cannot derive the RP's identity or link any pair of $PID_{RP}^i$ and $PID_{RP}^{i'}$,
    and the IdP-based identity linkage is impossible.

\vspace{1mm}
\noindent\textbf{RP-based Identity Linkage.}
We prove UPPRESSO prevents the RP-based identity linkage,
 based on the elliptic curve decision Diffie-Hellman (ECDDH) assumption \cite{GoldwasserK16} as below.
%We briefly introduce the ECDDH assumption.
Let $\mathbb{E}$ be an elliptic curve over a finite field $\mathbb{F}_q$,
    and $P$ be a point on $\mathbb{E}$ of order $n$.
For any probabilistic polynomial time (PPT) algorithm $\mathcal{D}$,
 $([x]P$, $[y]P$, $[xy]P)$ and $([x]P$, $[y]P$, $[z]P)$
are computationally indistinguishable,
 where $x$, $y$ and $z$ are integer numbers randomly and independently chosen from $(1,n)$.
Let  $Pr\{\}$ denote the probability and
 we define
\begin{align*}
Pr_1 = & Pr\{\mathcal{D}(P, [x]P, [y]P, [xy]P)=1\} \\
Pr_2 = & Pr\{\mathcal{D}(P, [x]P, [y]P, [z]P)=1\} \\
\epsilon(k) = & Pr_1 - Pr_2
\end{align*}
Then, $\epsilon(k)$ becomes negligible with the security parameter $k$.

%Let $q$ be a large prime and $\mathbb{G}$ denotes a cyclic group of order $n$ of an elliptic curve $E(\mathbb{F}_q)$.
%Assume that $n$ is also a large prime. Let $P$ be a generator point of $\mathbb{G}$.

%where $q$ and $n$ are large primitive number, and $P$ is the point of $\mathbb{G}$.
%For any probabilistic polynomial time (PPT) algorithm $D$, the distributions, \{$P$, $aP$, $bP$, $abP$\}$_{a,b \in \mathbb{Z}_n}$ and \{$P$, $aP$, $bP$, $cP$\}$_{a,b,c \in \mathbb{Z}_n}$, are computationally indistinguishable. There is a negligible $\sigma(k)$, where $k$ is the security parameter.

In the login flow,
    an RP holds $ID_{RP}$ and $Acct$, receives $t$, calculates $PID_{RP}$,
    and verifies two signed responses (i.e., $PID_{RP}$ and $H(t)$ in the $PID_{RP}$ registration result,
            and $PID_{RP}$ and $PID_U$ in the identity token).
After filtering out the redundant information (i.e., $PID_{RP}= [t]{ID_{RP}}$ and $Acct = [t^{-1}]PID_{U}$),
    the RP actually receives only $(ID_{RP}, t, PID_U)$ in each SSO login instance, where $PID_U = [ID_U][t]{ID_{RP}}$.

% ���̫���ˣ�����Ҫͼ
%\begin{figure*}
%  \centering
%  \includegraphics[width=0.82\linewidth]{fig/game1.pdf}
%  \caption{The Game.}
%  \label{fig:game}
%\end{figure*}


\begin{figure}[tb]
  \centering
  \includegraphics[width=1\linewidth]{fig/dalgorithm.pdf}
  \caption{The algorithm based on the RP-based identity linkage, to solve the ECDDH problem.}
  \label{fig:dalgorithm}
\end{figure}

In order to launch identity linkage,
    two RPs bring two triads received in SSO login instances, $(ID_{RP_j}$, $t_j$, $[ID_U][t_j]{ID_{RP_j}})$ and
    $(ID_{RP_{j'}}$, $t_{j'}$, $[ID_{U'}] [t_{j'}] {ID_{RP_{j'}}})$.
We describe the attack of RP-based identity linkage as the following game $\mathcal{G}$ between an adversary and a challenger:
    The adversary receives $(ID_{RP_j}$, $t_j$, $[ID_U][t_j]{ID_{RP_j}}, ID_{RP_{j'}}$, $t_{j'}$, $[ID_{U'}] [t_{j'}] {ID_{RP_{j'}}})$ from the challenger,
     and outputs the result $s$.
The result is 1, when the adversary guesses that $ID_U = ID_{U'}$;
     otherwise, the adversary thinks they are unequal and $s=0$.
%The game is shown as Figure \ref{fig:game}.

We define $Pr_c$ as the probability that
    the adversary outputs $s=1$ when $ID_U = ID_{U'}$ (i.e., a \emph{correct} identity linkage),
    and $Pr_{\bar{c}}$ as the probability that $s=1$ but $ID_U \neq ID_{U'}$ (i.e., an \emph{incorrect} result).
The successful RP-based identity linkage means
    the adversary has non-negligible advantages in $\mathcal{G}$.

We design a PPT algorithm $\mathcal{D}^*$ based on $\mathcal{G}$, shown in Figure \ref{fig:dalgorithm}.
The input of $\mathcal{D}^*$ appears in the form of $(Q_1, Q_2, Q_3, Q_4)$, and each $Q_i$ is a point on $\mathbb{E}$.
On receiving the input,
 the challenger of $\mathcal{G}$ randomly chooses $t_j$ and $t_{j'}$ in $(1,n)$,
   % and sets $ID_{RP_j}=Q_1$, $ID_{RP_{j'}}=Q_2$, $PID_{U,j}=[t_j]Q_3$, and $PID_{U',j'}= [t_{j'}] Q_4$.
    and sends $(Q_1, t_j, [t_j]Q_3, Q_2, t_{j'}, [t_{j'}] Q_4)$ to the adversary.
Finally,
    $\mathcal{D}^*$ directly outputs $s$ from the adversary as the result of $\mathcal{D}^*$.

Let ($P$, $[x]P$, $[y]P$, $[xy]P$) and  ($P$, $[x]P$, $[y]P$, $[z]P$) be two inputs of $\mathcal{D}^*$.
Thus, we obtain
\begin{equation}\label{eq:game-succed}
\begin{split}
&Pr\{\mathcal{D}^*(P,[x]P,[y]P,[xy]P)=1\}\\
=&Pr\{\mathcal{G}(P, t_j, [t_j][y]P, [x]P, t_{j'},[t_{j'}][xy]P)=1\}\\
=&Pr\{\mathcal{G}(P, t_j, [y][t_j]P, [x]P, t_{j'},[y][t_{j'}][x]P)=1\}=Pr_c
\end{split}
\end{equation}
\begin{equation}\label{eq:game-fail}
\begin{split}
&Pr\{\mathcal{D}^*(P,[x]P,[y]P,[z]P)=1\} \\
=&Pr\{\mathcal{G}(P, t_j, [t_j][y]P, [x]P, t_{j'},[t_{j'}][z/x] [x]P)=1\}\\
=&Pr\{\mathcal{G}(P, t_j, [y][t_j]P, [x]P, t_{j'},[z/x][t_{j'}][x]P)=1\}=Pr_{\bar{c}}
\end{split}
\end{equation}
Equation \ref{eq:game-succed} is equal to $Pr_c$
        because it represents the correct case of $ID_{U} = ID_{U'} = y$,
 while Equation \ref{eq:game-fail} is $Pr_{\bar{c}}$ for it represents the incorrect case of $ID_{U} =y$ but $ID_{U'} = z/x \bmod n$.


The adversary has non-negligible advantages in $\mathcal{G}$ means
    $Pr_c - Pr_{\bar{c}}>\sigma(k)$,
    and then $\mathcal{D}^*$ significantly distinguishes ($P$, $[x]P$, $[y]P$, $[xy]P$) from  ($P$, $[x]P$, $[y]P$, $[z]P$),
    which violates the ECDDH assumption.
So the adversary has no advantages in the game,
    and the RP-based identity linkage is computationally impossible.
