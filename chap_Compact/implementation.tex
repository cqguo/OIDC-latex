\section{Implementation and Evaluation}
\label{sec:implementation}

We implemented a prototype of \usso\footnote{The prototype is open-sourced at https://github.com/uppresso/.} and conducted experimental comparisons with two open-source SSO systems: (\emph{a}) MITREid Connect \cite{MITREid}, which is a PPID-enhanced OIDC system to prevent RP-based identity linkage, and (\emph{b}) SPRESSO \cite{SPRESSO}, which prevents IdP-based login tracing.


\subsection{Prototype Implementation}
\label{subsec:proto-imple}

The \usso\ prototype implemented identity transformations on the NIST P256 elliptic curve, with RSA-2048 and SHA-256 serving as the digital signature and hash algorithms, respectively. The IdP and RP scripts consist of approximately 160 and 140 lines of JavaScript code, respectively.  %to provide the functions in Steps 2.1, 2.3, and 4.3.
The cryptographic computations such as $Cert_{RP}$ verification and $PID_{RP}$ negotiation are performed using jsrsasign \cite{jsrsasign}, an open-source JavaScript cryptographic library.

The IdP was developed on MITREid Connect\cite{MITREid}, a Java implementation of OIDC, %certificated by the OpenID Foundation \cite{OIDF},
with minimal code modifications. Only three lines of code were added for calculating $PID_U$ and 20 lines were added to modify the method of forwarding identity tokens.
The calculations for $ID_{RP}$ and $PID_U$ were implemented using Java cryptographic libraries.

We developed a Java-based RP SDK with about 500 lines of code on the Spring Boot framework. It includes two functions for encapsulating the \usso\ protocol steps: one for requesting identity tokens and the other for deriving accounts. The cryptographic computations are executed using the Spring Security library.
An RP can easily integrate \usso\ by  adding less than 10 lines of Java code to invoke the necessary functions.
