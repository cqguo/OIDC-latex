\section{Discussion}
\label{sec:discussion}
We provide some discussions about UPRESSO here.

%����:Ŀǰֻ�ṩ��ʽģʽ,������ģʽ��Э��ļ��ݷ���
\noindent{\textbf{OIDC authorization code flow support.}} The privacy-preserving functions $\mathcal{F}_{ID_{U} \mapsto PID_{U}}$, $\mathcal{F}_{ID_{RP} \mapsto PID_{RP}}$ and $\mathcal{F}_{PID_{U} \mapsto Account}$ can be integrated into OIDC authorization code flow directly, therefore  RP-based identity linkage and IdP-based access tracing are still prevented during the construction and parsing of identity proof.
The only privacy leakage is introduced by the transmission, as RP servers obtain the identity proof directly from the IdP in this flow, which allows the IdP to obtain RP's network information (e.g., IP address).
UPRESSO could integrate anonymous networks (e.g., Tor)  to prevent this leakage.



%SPRESSO��ƽ̨:����Ŀǰ�ķ���ֻ��browserʵ��
\noindent{\textbf{Platform independent.}}
Firstly, a user could use any machine (the Chrome extension installed) to log in at RPs and need no extra processing (e.g., synchronization),
as the user's machine doesn't need to store any persistent data and only caches $Cert_{RP}$, $PID_{RP}$ and one-time endpoint during the processing of one login.
Secondly, the implementation of UPRESSO could  be further improved, then the user could  log in at RPs without the Chrome extension, whose JavaScript program is then fetched from the honest IdP. The processing is similar as SPRESSO. That is, 1) the RP's window opens a new iframe  visiting to the login interface at RP and being redirected to IdP; 2) this iframe downloads the JavaScript  program from IdP and performs the processing in Steps 2.1.3, 2.2.1, 2.4 and 5.2; 3) the opener handle of this new window is preserved (i.e., RP's window) and then postMessages can be adopted to exchange messages between the two windows for Steps 2.1.4, 2.1.6, 2.2.4, 2.3 and 5.3. The redirection in the new iframe is necessary to prevent the Referer header containing RP's URL from being sent to the IdP. However, it cannot be achieved by introduced the special attribute (rel="noreferrer") while opening the new iframe, as it results in the new window does not have a handle on opening window (opener) which is used for message exchanging between RP's window and IdP's window.

%1) RP's window opens a new window to download a document from RP; 2) the document uses JavaScript to navigate this window to IdP, while the noreferrer attribute is set to clear Referer header and prevent the browser from sending RP's URL to the IdP; 3) this window downloads the  JavaScript  program from IdP and performs the processing in Steps 2.1.3, 2.2.1, 2.4 and 5.2; 4) the opener handle of this new window is preserved (i.e., RP's window) and then postMessages can be adopted to exchange messages between the two windows for Steps 2.1.4, 2.1.6, 2.2.4, 2.3 and 5.3.

%the RP's web page initiates the loading of IdP's web page and sets the attribute (rel="noreferrer") to prevent the browser from sending Referer header with RP's URL to the IdP; 2) the IdP's web page contains the JavaScript program for user's processing in Steps 2.1.3, 2.2.1, 2.4 and 5.2; 3) postMessages is adopted
%; and 4) the correct RP sets subresource integrity (SRI) to prevent the IdP from providing incorrect JavaScript programs.
%A native web technology called subresource integrity (SRI)7 is currently under development at the W3C. SRI allows a document to create an iframe with an attribute integrity that takes a hash value. The browser now would guarantee that the document loaded into the iframe hashes to exactly the given value. So, essentially the creator of the iframe can enforce the iframe to be loaded with a aspecific document. This would enable SPRESSO to automatically check the integrity of FWDdoc without any extensions.


\noindent{\textbf{Scalability on $\mathbf{ID_{RP}}$ and $\mathbf{PID_{RP}}$.}} The adversary cannot exhaust $ID_{RP}$ and $PID_{RP}$.
For $ID_{RP}$, it is generated only in RP's initial registration, and existing DoS mitigation  could be adopted to mitigate malicious registration.
For $PID_{RP}$,  in practice, we only need to ensure all $PID_{RP}$s are different among the unexpired identity proof (the number denoted as $n$). The probability
that two or more  $PID_{RP}$s are same among the unexpired identity proof, is $1-\prod_{i=0}^{n-1}(1-i/q)$ which increases with $n$.
For an IdP with throughput $2*10^8$ req/s and valid period of identity proof set as 5 minutes, $n$ is less than $2^{36}$, then the probability is less than $2^{-183}$ for 256-bit $q$.

\noindent{\textbf{Malicious IdP mitigation.}} The IdP is assumed to assign unique $ID_{RP}$ in $Cert_{RP}$ for each RP and generate the correct $PID_U$ for each login. The malicious IdP may attempt to provide incorrect $ID_{RP}$ and $PID_U$, which could be prevented by integrating certificate transparency~\cite{rfc6962} and user's identifier check~\cite{SPRESSO}. With certificate transparency~\cite{rfc6962}, the monitors  checks the uniqueness of $r$ among all the certificates stored in the log server. To prevent the malicious IdP from injecting a incorrect $PID_U$, the correct user could provide a nickname to the correct RP for an extra check as in SPRESSO~\cite{SPRESSO}.



