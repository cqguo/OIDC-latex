\documentclass[letterpaper,onecolumn,10pt]{article}

\usepackage{amsmath}
\usepackage{bm}
\usepackage{color}
\usepackage{filecontents}
\usepackage{wasysym}
\usepackage[small]{titlesec}
\usepackage{lipsum,mwe,cuted}
\usepackage{float}%%%%�ṩ�������[H]ѡ�����ȡ������
\usepackage{caption}%%�ṩ\captionof����
\pagestyle{plain}
\usepackage{graphicx}
\usepackage{verbatim}
\usepackage{caption}
\usepackage{amsmath,amssymb,amsthm}
\usepackage{graphicx}
\usepackage{url}
\usepackage{multirow}
\usepackage{listings}
\usepackage{cite}
\usepackage{array}
\usepackage{enumerate}
\usepackage{booktabs}
\usepackage{amsthm}
\usepackage{color}
\usepackage{soul}
\usepackage{multicol}
\usepackage[algo2e]{algorithm2e}
\usepackage{algorithm}
\usepackage{algorithmic}
\usepackage{macros}
\usepackage{todonotes}
\usepackage[backref]{hyperref} 
\newtheorem{theorem}{Theorem}
\newtheorem{definition}{Definition}
\newtheorem{lemma}{Lemma}


\makeatletter
\newenvironment{breakablealgorithm}
  {% \begin{breakablealgorithm}
   \begin{center}
     \refstepcounter{algorithm}% New algorithm
     \hrule height.8pt depth0pt \kern2pt% \@fs@pre for \@fs@ruled
     \renewcommand{\caption}[2][\relax]{% Make a new \caption
     {\raggedright\textbf{\thealgorithm} ##2\par}%
       \ifx\relax##1\relax % #1 is \relax
         \addcontentsline{loa}{algorithm}{\protect\numberline{\thealgorithm}##2}%
       \else % #1 is not \relax
         \addcontentsline{loa}{algorithm}{\protect\numberline{\thealgorithm}##1}%
       \fi
       \kern2pt\hrule\kern2pt
     }
  }{% \end{breakablealgorithm}
     \kern2pt\hrule\relax% \@fs@post for \@fs@ruled
   \end{center}
  }
\makeatother






\hyphenation{op-tical net-works semi-conduc-tor}

\begin{document}



\renewcommand{\algorithmicrequire}{\textbf{Input:}}
\newcommand{\deflet}{\textbf{let}}
\newcommand{\mystate}[1]{\STATE \textbf{let} {{}#1}}
\newcommand{\mystop}[1]{\STATE \textbf{stop} \myss{\myangle{{{}#1}}, s'}}
\newcommand{\mystopp}[1]{\STATE \textbf{stop} \myss{\myangle{{{}#1}}}}
\newcommand{\myss}[1]{${{}#1}$}
\newcommand{\myangle}[1]{\langle {{}#1} \rangle}
\newcommand{\myif}[1]{\IF{\myss{{{}#1}}}}
\newcommand{\myelse}[1]{\ELSIF{\myss{{{}#1}}}}

\newcommand{\aaa}[1]{\STATE \textbf{if} #1 \textbf{then} \begin{ALC@g}}
\newcommand{\bbb}[1]{\end{ALC@g} \STATE \textbf{else if} #1 \textbf{then} \begin{ALC@g}}
\newcommand{\ccc}{\end{ALC@g} \STATE \textbf{else} \textbf{then} \begin{ALC@g}}
\newcommand{\ddd}{\end{ALC@g} \STATE \textbf{endif}}

\newcommand{\SWITCH}[1]{\STATE \textbf{switch} #1\ \textbf{do} \begin{ALC@g}}
\newcommand{\ENDSWITCH}{\end{ALC@g}\STATE \textbf{end switch}}
\newcommand{\CASE}[1]{\STATE \textbf{case} #1\textbf{:} \begin{ALC@g}}
\newcommand{\ENDCASE}{\end{ALC@g}}
\newcommand{\CASELINE}[1]{\STATE \textbf{case} #1\textbf{:} }
\newcommand{\DEFAULT}{\STATE \textbf{default:} \begin{ALC@g}}
\newcommand{\ENDDEFAULT}{\end{ALC@g}}
\newcommand{\DEFAULTLINE}[1]{\STATE \textbf{default:} }


\renewcommand{\thesection}{\Alph{section}} 


\section{Preparation}

%所有和SPRESSO的不同点放在同一个章节中一起说明。

Our formal security analysis of UPPRESSO is based on 
the general Dolev-Yao web model in SPRESSO. 
To facilitate the definition of UPPRESSO, however, 
we have some difference from SPRESSO. 
In particular, we remove some processes and 
add some function symbols for asymmetric encryption/decryption.

\subsection{Functions Symbols}

Since our model is using ECC(Elliptic Curve Cryptography) to encrypt/decrypt the data,
we add the following symbols to the signature $\Sigma$ for the terms and messages:

\begin{itemize}
  \item $\mathbb{E}$ is an elliptic curve over a finite field $\mathbb{F}_q$, $G$ is a base point(or generator) of $\mathbb{E}$ and the order of $G$ is a prime number n.
  \item $[t]P$ means using asymmetric key $t$ to encrypt the point $P=[p]G$ on the elliptic curve where $p$ is the actual plaintext.
  \item $[t^{-1}]C$ means using the reverse of $t$ to decrypt the point $C=[c]G=[tm]G$ on the elliptic curve where $c$ is the cipertext.  
  \item $\str{isValid}(P)$ checks whether $P$ is a valid point on the elliptic curve. That is to say whether $P=[m]G$ for the base point $G$ and some nonce $m$.
\end{itemize}

\subsection{DNS servers}

%如果是自己编写的脚本引入了DNS请求,那就需要在模型中考虑DNS服务器,UPPRESSO可以不需要考虑。

In SPRESSO, when receiving an e-mail address, 
RP needs to send DNS requests to DNS servers manually 
to fetch the information of the IdP server. 
Since there may be various DNS servers in SPRESSO, 
DNS server security issues need to be given special consideration.
As a result, DNS servers are added into the formal model of SPRESSO.

In UPPRESSO, however, we only have one centralized IdP server, and 
all RPs know the relevant information of the IdP in advance.
So all DNS requests are generated spontaneously by the browser, 
not introduced by our scripts.
Therefore, we remove DNS servers from the formal model of UPPRESSO.


\section{Formal Model of UPPRESSO}
\label{app:model-uppresso}
We here present the full details of our formal model of UPPRESSO. For our analysis regarding our authentication and privacy properties below, we will further restrict this generic model to suit the setting of respective analysis.\par
We model UPPRESSO as a web system. We call a web system $\uppressowebsystem=(\bidsystem, \scriptset, \mathsf{script}, E^0)$ an UPPRESSO web system if it is of the form described in what follows.

\subsection{Outline}\label{app:outlineuppressomodel}
The system $\bidsystem=\mathsf{Hon}\cup \mathsf{Web} \cup \mathsf{Net}$ consists of 
web attacker processes (in $\mathsf{Web}$), network attacker processes (in $\mathsf{Net}$), 
a finite set $\fAP{B}$ of web browsers, 
a finite set $\fAP{RP}$ of web servers for the relying parties, 
a finite set $\fAP{IDP}$ of web servers containing only one identity provider 
with $\mathsf{Hon} := \fAP{B} \cup \fAP{RP} \cup \fAP{IDP}$. 
More details on the processes in $\mathpzc{W}$ are provided below. 
%
Figure~\ref{fig:scripts-in-w} shows the set of scripts $\scriptset$ 
and their respectice string representations that are defined by the 
mapping $\mathsf{script}$. 
%
The set $E^0$ contains only the trigger events.

\begin{figure}[htb]
  \centering
  \begin{tabular}{|@{\hspace{1ex}}l@{\hspace{1ex}}|@{\hspace{1ex}}l@{\hspace{1ex}}|}\hline 
    \hfill $s \in \scriptset$\hfill  &\hfill $\mathsf{script}(s)$\hfill  \\\hline\hline
    $\Rasp$ & $\str{att\_script}$  \\\hline
    $\mi{script\_rp}$ & $\str{script\_rp}$  \\\hline
    $\mi{script\_idp}$ &  $\str{script\_idp}$  \\\hline
  \end{tabular}
  
  \caption{List of scripts in $\scriptset$ and their respective string
    representations.}
  \label{fig:scripts-in-w}
\end{figure}

This outlines $\uppressowebsystem$. We will define the DY processes in 
$\uppressowebsystem$ and their addresses, domain names, and secrets in more detail. 
The scripts are defined in detail in Appendix~\ref{app:uppresso-scripts}

\subsection{Addresses and Domain Names}
The set $\addresses$ contains for every web attacker in $\fAP{Web}$, 
every network attacker in $\fAP{Net}$, 
every relying party in $\fAP{RP}$, 
the only one identity provider in $\fAP{IDP}$, 
and every browser in $\fAP{B}$ a finite set of addresses each. 
By $\mapAddresstoAP$ we denote the corresponding
assignment from a process to its address. 
The set $\dns$ contains a finite set of domains for 
every relying party in $\fAP{RP}$, 
the only one identity provider in $\fAP{IDP}$, 
every web attacker in $\fAP{Web}$, and 
every network attacker in $\fAP{Net}$. 
Browsers (in $\fAP{B})$ do not have a domain.

By $\mapAddresstoAP$ and $\mapDomain$ we denote the assignments from
atomic processes to sets of $\addresses$ and $\dns$, respectively.

%需不需要为椭圆曲线的点单独设立一个集合?
\subsection{Keys and Secrets} 
The set $\nonces$ of nonces is partitioned into four sets, 
an infinite sequence $N$, 
an infinite set $K_\text{SSL}$, 
an infinite set $K_\text{sign}$, 
an infinite set $K_\text{id}$, 
an infinite set $K_\text{point}$, 
and a finite set $\RPSecrets$. 
We thus have
\begin{align*}
\def\hereMaxHeightPhantom{\vphantom{K_{\text{p}}^\bidsystem}}
\nonces = 
\underbrace{N\hereMaxHeightPhantom}_{\text{infinite sequence}} 
\dot\cup \underbrace{K_{\text{SSL}}\hereMaxHeightPhantom}_{\text{finite}} 
\dot\cup \underbrace{K_{\text{sign}}\hereMaxHeightPhantom}_{\text{finite}}
\dot\cup \underbrace{K_{\text{point}}\hereMaxHeightPhantom}_{\text{finite}}  
\dot\cup \underbrace{\RPSecrets\hereMaxHeightPhantom}_{\text{finite}}\ .
\end{align*}
The set $N$ contains the nonces that are available for each DY process
in $\bidsystem$ (it can be used to create a run of $\bidsystem$). 

The set $K_\text{SSL}$ contains the keys that will be used for SSL
encryption. Let $\mapTLSKey\colon \dns \to K_\text{SSL}$ be an injective
mapping that assigns a (different) private key to every domain.

The set $K_\text{sign}$ contains the keys that will be used by IdPs
for signing IAs. Let $\mapSignKey\colon \fAP{IdPs} \to K_\text{sign}$
be an injective mapping that assigns a (different) private key to every identity
provider.

The set $K_\text{point}$ contains all valid points on the curve. 
The set $K_\text{point}$ will be used to generate identities of $\fAP{B}$ and $\fAP{RP}$.

The set $\RPSecrets$ is the set of passwords (secrets) 
the browsers share with the identity providers. 

\subsection{Identities}\label{app:uppresso-identities}
There are many different types of identities in UPPRESSO, 
We denote the identity of the broswer at the IdP with $u$,
the identity of the relying party at the IdP with $r$ and 
the identity of the broswer at the relying party with $acct$.
The details are defined below.

%用户的id表示:<id, username, idp-domain, <acct1, rp1-domain>, <acct2, rp2-domain>, <acct3, rp3-domain>, ...>
%RP的id表示:,<id, rp-commonname, idp-domain>

%def1.用户的id表示:<id, username, idp-domain>
%def2.用户的account:<<acct1, rp1-domain>, <acct2, rp2-domain>, <acct3, rp3-domain>, ...>
%def3.RP的id表示,<id, rp-commonname, idp-domain>

\begin{definition}
  An \emph{identity} $u$ is a term of the form 
  $u=\an{\mi{id},\mi{username},\mi{domain_{idp}}}$ 
  with $\mi{id}\in N$, $\mi{username}\in \mathbb{S}$ and $\mi{domain_{idp}} \in \dns$.

  %Let $\IDs$ be the finite set of identities. By $\IDs^y$ we denote
  %the set $\{ \an{\mi{id},\mi{name},\mi{domain}} \in \IDs\,|\, \mi{domain}
  %\in \mapDomain(y) \}$.

  %We say that an ID is \emph{governed} by the DY process to which the
  %domain of the ID belongs. Formally, we define the mapping $\mapGovernor:
  %\IDs \to \bidsystem$, $\an{\mi{id},\mi{name},\mi{domain}} \mapsto
  %\mapDomain^{-1}(\mi{domain})$.
\end{definition}

\begin{definition}
  An \emph{identity} $r$ is a term of the form 
  $r=\an{\mi{id},\mi{commonname},\mi{domain_{idp}}}$ 
  with $\mi{id}\in K_{\text{point}}$, $\mi{commonname}\in \mathbb{S}$ and $\mi{domain_{idp}} \in \dns$.
\end{definition}

\begin{definition}
  An \emph{identity} $acct$ is a term of the form 
  $acct=\langle\an{\mi{acct}_1,\mi{domain_{rp_1}}},\an{\mi{acct}_2,\mi{domain_{rp_2}}}, \ldots\rangle \ .$ 
  with $\mi{acct}_i\in K_{\text{point}}$, $\mi{domain_{rp_i}}\in \dns$.
\end{definition}

Let $\IDs^u$ be the finite set of identities u.

By $\mapIDtoPLI:\IDs^u \to \RPSecrets$ we denote the bijective mapping
that assigns secrets to all identities. 

Let $\mapPLItoOwner: \RPSecrets \to \fAP{B}$ denote the mapping that
assigns to each secret a browser that \emph{owns} this secret. Now, we
define the mapping $\mapIDtoOwner: \IDs^u \to \fAP{B}$, $i \mapsto
\mapPLItoOwner(\mapIDtoPLI(i))$, which assigns to each identity the
browser that owns this identity (we say that the identity belongs to
the browser).

To be concise, we usually use $u$ and $r$ to
refer to $u.id$ and $r.id$ if we don't say they are identities or $u,r\in\IDs$

\subsection{Tags, Identity Tokens and Service Tokens}\label{app:identity-assertions}

\begin{definition}\label{def:tag}
  A \emph{tag} is a term of the form $\mi{PID_{rp}}=[t]ID_{rp}=[tr]G$ for a nonce 
  (here used as a asymmetric key) $t$.
\end{definition}
\begin{definition}
  An \emph{identity Tokens (IDToken)} is a term of the form 
  $\an{PID_{rp}, PID_u, ver}$ for a tag $PID_{rp}$, an encrypted identity 
  $PID_u=[u]PID_{rp}=[utr]G$ and a signature $ver=\sig{\an{PID_{rp},PID_u}}{k}$ 
  for a nonce $k$.
\end{definition}
%使用<n,i>来定义 service token;Acct or pidu?
\begin{definition}
  A \emph{service token} is a term of the form 
  $\myangle{\mi{nonce}, \mi{Acct}}$ with 
  $\mi{Acct} = [t^{-1}]PID_u=[t^{-1}][utr]G=[ur]G$ 
  for a nonce $t$.
\end{definition}

%是否足够描述semi-honest的状态?
%\subsection{Curiosity}
%同样,也会有人给idp发送命令,让他进入curiosity状态
%进入curiosity状态之后,就可以开始把历史上所有的信息,开始用来推导各种事情。
%或者IdP始终honest-but-curious的状态,在安全性证明时不考虑curious的状态。

\subsection{Corruption}\gs{DNS servers may not be corrupted?}
RPs can become corrupted: If they receive the message
$\corrupt$, they start collecting all incoming messages in their state
and (upon triggering) send out all messages that are derivable from
their state and collected input messages, just like the attacker
process. We say that an RP is \emph{honest} if the according
part of their state ($s.\str{corrupt}$) is $\bot$, and that they are
corrupted otherwise.

We are now ready to define the processes in $\websystem$ as well as
the scripts in $\scriptset$ in more detail. 

\subsection{Processes in $\bidsystem$ (Overview)}

We first provide an overview of the processes in $\bidsystem$. All
processes in $\websystem$ contain in their initial states all public
keys and the private keys of their respective domains (if any). We
define $I^p=\mapAddresstoAP(p)$ for all $p\in \mathsf{Hon} \cup \mathsf{Web}$.

%两个attacker是否足够描述corrupt的程度?
\subsubsection{Web Attackers.}  Each $\mi{wa} \in \mathsf{Web}$  is a
web attacker who uses only his own addresses for sending and listening. 

\subsubsection{Network Attackers.}  Each $\mi{na} \in \mathsf{Net}$  is a
network attacker who uses all addresses for sending and listening. 

\subsubsection{Browsers.} Each $b \in \fAP{B}$ is a web browser. 
The initial state contains all secrets owned by $b$, stored under the origin of the
respective IdP. See Appendix~\ref{app:browsers-uppresso} for details.

\subsubsection{Relying Parties.} 
A relying party $r \in \fAP{RP}$ is a web server. RP knows four distinct paths: 
$\mathtt{/script}$, where it serves $\str{script\_rp}$ to open a new window 
and facilitate the login flow.
$\mathtt{/loginSSO}$, where it only accepts GET requests and sends 
redirect response to redirect the browser to the IdP to download $\str{script\_IdP}$
$\mathtt{/startNegotiation}$, where it only accepts POST requests logically sent 
from $\str{script\_rp}$ using postMessge and checks whether the data $t\in K_\text{id}$.
If the request valid, it send back a certificate.
$\mathtt{/uploadToken}$ running in the browser. It checks the ID token and, 
if the data is deemed ``valid'', it issues a service token (again, for details, see below). 
Intuitively, a client having such a token can use the service of the RP 
(for a specific identity record along with the token). 
Just like IdPs, RPs can become corrupted.

\subsubsection{Identity Providers.} Each IdP is a web server, 
users can authenticate to the IdP with their credentials. 
IdP tracks the state of the users with sessions. 
Authenticated users can receive IDTokens from the IdP. 
%When receiving a special message ($\corrupt$) IdPs can become corrupted. 
%Similar to the definition of corruption for the browser,
%IdPs then start sending out all messages that are derivable from their state.

%\subsubsection{DNS.} Each $\mi{dns} \in \fAP{DNS}$ is a DNS server.
%Their state contains the allocation of domain names to IP addresses.

\subsection{TLS Key Mapping}\label{app:common-data-structures}
Before we define the atomic DY processes in more detail, we first
define the common data structure that holds the mapping of domain
names to public TLS keys: For an atomic DY process $p$ we define
\[\mi{tlskeys}^p = \an{\left\{\an{d, \mapTLSKey(d)} \mid d \in \mapDomain(p)\right\}}.\]
%ssl改成tls

\subsection{Web Attackers}\label{app:webattackers-uppresso}
Each $\mi{wa} \in \fAP{Web}$ is a web attacker. 
The initial state of each $\mi{wa}$ is 
$s_0^\mi{wa} = \an{\mi{attdoms}, \mi{tlskeys}, \mi{signkeys}}$, 
where $\mi{attdoms}$ is a sequence of all domains along with 
the corresponding private keys owned by $\mi{wa}$, 
$\mi{tlskeys}$ is a sequence of all domains and 
the corresponding public keys, and 
$\mi{signkeys}$ contains the public signing key for the IdP. 
%All other parties use the attacker as a DNS server.

\subsection{Network Attackers}\label{app:networkattackers-uppresso}
As mentioned, each network attacker $\mi{na}$ is modeled to 
be a network attacker. We allow it to listen to/spoof all 
available IP addresses, and hence, define 
$I^\mi{na} = \addresses$. 
The initial state is $s_0^\mi{na} = 
\an{\mi{attdoms}, \mi{tlskeys}, \mi{signkeys}}$, 
where $\mi{attdoms}$ is a sequence of all domains along with 
the corresponding private keys owned by the attacker 
$\mi{na}$, $\mi{tlskeys}$ is a sequence of all domains 
and the corresponding public keys, and 
$\mi{signkeys}$ contains the public signing key for the IdP. 

%不用
\subsection{Browsers}\label{app:browsers-uppresso} 
Each $b \in \fAP{B}$ is a web browser with 
$I^b := \mapAddresstoAP(b)$ being its addresses.

To define the inital state, first let $\mi{ID}^b := 
\mapIDtoOwner^{-1}(b)$ be the set of all IDs of $b$, 
%$\mi{ID}^{b,d} := \{i \mid \exists\, x,n:\ i = \an{id, n, d} \in \mi{ID}^b\}$ 
%be the set of IDs of $b$ for a domain $d$, and 
%$\mi{SecretDomains}^b := \{d \mid \mi{ID}^{b,d} \neq \emptyset \}$ 
%be the set of all domains that $b$ owns identities for.
Then, the initial state $s_0^b$ is defined as follows: the key mapping
maps every domain to its public (ssl) key, according to the mapping
$\mapTLSKey$; the DNS address is $\mapAddresstoAP(p)$ with $p \in \bidsystem$;
the list of secrets contains an entry $\an{\an{d,\https}, s}$ for each
$d \in \mi{SecretDomains}^b$ and $s = \mapIDtoPLI(i)$ for some $i \in
\mi{ID}^{b,d}$ ($s$ is the same for all $i$); $\mi{ids}$ is
$\an{\mi{ID}^b}$; $\mi{sts}$ is empty.

\subsection{Relying Parties} \label{app:relying-parties-uppresso}

A relying party $r \in \fAP{RP}$ is a web server modeled as an atomic
DY process $(I^r, Z^r, R^r, s^r_0)$ with the addresses $I^r :=
\mapAddresstoAP(r)$. Its initial state $s^r_0$ contains its domains,
the private keys associated with its domains.
The full state additionally contains the sets of service tokens and login 
session identifiers the RP has issued. RP only accepts HTTPS requests.

RP manages two kinds of sessions: The \emph{login sessions}, which are
only used during the login phase of a user, and the \emph{service
  sessions} (we call the session identifier of a service session a
\emph{service token}). Service sessions allow a user to use RP's
services. The ultimate goal of a login flow is to establish such a
service session.

In a typical flow with one client, $r$ will first receive an HTTP GET
request for the path $\str{/script}$. In this case, $r$ returns the script
$\str{script\_rp}$ (see below).

After the user loaded the script in his browser, $r$ will receive an 
HTTP GET request for the path $\str{/loginSSO}$ sent from the new window opened
by $\str{script\_rp}$. In this request, $r$ will send back a redirect response  
for downloading $\str{script\_IdP}$ from IdP.

When the IdP document in the browser generates a number $t$,
$r$ will receive the third request for the path $\str{/startNegotiate}$.
$r$ will verify $t$ and if valid, $r$ will create the corresponding 
login session with a $\mi{loginSessionToken}$ as the identifier. After that,
$r$ will use $t$ to generate $PID_{rp}$ and bind it with the login session.
After all these are down, $r$ send its certificate signed by the specific IdP that browser selected.

Finally, $r$ receives a last request in the login flow. This POST request 
contains the IDToken. To conclude the login, $r$ looks up the user's login session, 
compare the $\mi{IDToken}.\str{PID_{rp}}$ with the $\mi{PID_{rp}}$ in the login session, and checks 
whether $\mi{IDToken}.\str{PID_{ver}}$ is a correct signature. If successful, $r$ calculates the 
service token and returns it, which is also stored in the state of $r$.

If $r$ receives a corrupt message, it becomes corrupt and acts like
the attacker from then on.

We now provide the formal definition of $r$ as an atomic DY process
$(I^r, Z^r, R^r, s^r_0)$. As mentioned, we define $I^r =
\mapAddresstoAP(r)$. Next, we define the set $Z^r$ of states of
$r$ and the initial state $s^r_0$ of $r$.

%和tag是否一致?
\begin{definition}
  A \emph{login session record} is a term of the form 
  $\an{\mi{t}, \mi{PID_{rp}}}$ with 
  $\mi{t}, \mi{PID_{rp}}=[tr]G(t,r\in K_{\text{id}})$.
\end{definition}

\begin{sloppypar}
  \begin{definition}\label{def:relying-parties}
    A \emph{state $s\in Z^r$ of an RP $r$} is a term of the form
    $\langle\mi{keyMapping}$, 
    $\mi{tlskeys}$, 
    $\mi{loginSessions}$, 
    $\mi{serviceTokens}$, 
    $\mi{corrupt}$, 
    $\mi{IdPConfig}$, 
    $\mi{rp}\rangle$ where 
    $\mi{keyMapping} \in \dict{\mathbb{S}}{\nonces}$,
    $\mi{tlskeys}=\mi{tlskeys}^r$,
    $\mi{serviceTokens} \in \nonces$,
    $\mi{loginSessions} \in \dict{\nonces}{\terms}$ 
    is a dictionary of login session records,
    $\mi{corrupt} \in \terms$,
    $\mi{IdPConfig} \in \terms$ 
    is the configuration retrieved from IdP server,
    $\mi{rp} \in \IDs$ is the identity of the RP, 
    see details in Appendix~\ref{app:uppresso-identities}.

    The \emph{initial state $s^r_0$ of $r$} is a state of 
    $r$ with $s^r_0.\str{serviceTokens} = 
    s^r_0.\str{loginSessions} = \an{}$,
    $s^r_0.\str{corrupt} = \bot$, 
    $s^r_0.\str{keyMapping}$ 
    is the same as the keymapping for browsers above,
    $s^r_0.\str{IdPConfig} = \an{\mi{pubkey},\mi{scriptUrl},\mi{Cert_{rp}}}$ and
    $s^r_0.\str{rp} = \an{\mi{id},\mi{commonname},\mi{domain_{idp}}}$.
  \end{definition}
\end{sloppypar}

We now specify the relation $R^r$. We describe this relation by a non-deterministic algorithm. 

\captionof{algorithm}{\label{alg:rp} Relation of a Relying Party $R^r$}
\begin{algorithmic}[1]
  \REQUIRE \myss{\myangle{a, b, m}, s}
  \mystate{\myss{s':=s}}
  \myif{s'.\str{corrupt} \not\equiv \bot \vee m \equiv \corrupt}
    \mystate{\myss{s'.\str{corrupt} := \an{\an{a, f, m}, s'.\str{corrupt}}}}
    \mystate{\myss{m' := d_{V}(s')}\label{line:usage-of-signkey-corrupt-uppresso}}
    \mystate{\myss{a' := \addresses}}
    \mystop{a',a,m'}
  \ENDIF
  \mystate{\myss{m_{dec},k,k',\mi{inDomain}} \textbf{such that} \breakalgohook{0}
    \myss{\an{m_{\text{dec}}, k} \equiv \dec{m}{k'} \wedge \an{inDomain,k'} \in s'.\str{sslkeys}}\breakalgohook{0}
    \textbf{if possible; otherwise stop} \myss{\myangle{}, s'}}
  \mystate{\myss{n, method, path, parameters, headers, body} \textbf{such that} \breakalgohook{0}
    \myss{\myangle{\mathtt{HTTPReq},n,method,path,parameters,headers,body} \equiv m_{dec}}\breakalgohook{0}
    \textbf{if possible; otherwise stop} \myss{\myangle{}, s'}}
  \myif{path \equiv /script}
    \mystate{\myss{m':=\encs{\myangle{\mathtt{HTTPResp},n,200, \myangle{}, \str{script\_rp}}}{k}}}
    \mystop{b, a, m'}
  \myelse{path \equiv /loginSSO}
    \mystate{\myss{m':=\encs{\myangle{\mathtt{HTTPResp},n,302,\myangle{\myangle{\mathtt{Location}, s'.\str{IdPConfig}.\mi{scriptUrl}}}, \myangle{}}}{k}}}
    \mystop{b, a, m'}
  \myelse{path \equiv /startNegotiation}
    \mystate{\myss{\mi{loginSessionToken} := \nu_1}}
   %\mystate{\myss{cookie := headers[Cookie]}}
   %\mystate{\myss{session := s'.SessionList[cookie]}}
    \mystate{\myss{\mi{t} := body[t]}}\label{line:gen-t}
   %\mystate{\myss{t^{-1}:= \mathtt{Inverse}(t)}}
    \mystate{\myss{\mi{ID_{rp}} := [s'.\str{rp}.\mi{id}]G}}
    \mystate{\myss{\mi{PID_{rp}} := [\mi{t}]\mi{ID_{rp}}}}
    \mystate{\myss{\mi{state} := \str{expectToken}}}
    \mystate{\myss{s'.\str{loginSessions}[\mi{loginSessionToken}] := \an{\mi{t}, \mi{PID_{rp}}, \mi{state}}}}
   %\mystate{\myss{session[t] := t}}
   %\mystate{\myss{session[t^{-1}] := t^{-1}}}
   %\mystate{\myss{session[state] := expectToken}}
    \mystate{\myss{m' := \encs{\myangle{\mathtt{HTTPResp}, n, 200, \myangle{}, \myangle{\mathtt{Cert_{RP}}, s'.\str{IdPConfig}.\mi{Cert_{RP}}}}}{k}}}
    \mystop{b, a, m'}
  \myelse{path \equiv /uploadToken}
    \mystate{\myss{\mi{loginSessions} := s'.\str{loginSessions}[body[\str{loginSessionToken}]]}}
   %\mystate{\myss{cookie := headers[Cookie]}}
    \myif{\mi{loginSessions} \equiv \an{}}
      \mystop{}
    \ENDIF
   %\mystate{\myss{session := s'.SessionList[cookie]}}
   %\myif{session[state] \not\equiv expectToken}
    \myif{\mi{loginSessions}.\str{state} \not\equiv expectToken}
      \mystate{\myss{m' := \encs{\myangle{\mathtt{HTTPResp}, n, 200, \myangle{}, \mathtt{Fail}}}{k}}}
      \mystop{b, a, m'}
    \ENDIF
    \mystate{\myss{s'.\str{loginSessions} := s'.\str{loginSessions} - body[\mi{loginSessionToken}]}}
    \mystate{\myss{\mi{IDToken} := body[\str{IDToken}]}}
    \myif{\mi{IDToken}.\str{PID_{rp}} \not\equiv \mi{loginSessions}.\str{PID_{rp}}}
      \mystate{\myss{m' := \encs{\myangle{\mathtt{HTTPResp}, n, 200, \myangle{}, \mathtt{Fail}}}{k}}}
      \mystop{b, a, m'}
    \ENDIF
    \myif{\checksigThree{\mi{IDToken}.\str{ver}}{\an{\mi{IDToken}.\str{PID_{rp}}, \mi{IDToken}.\str{PID_{u}}}}{s'.\str{IdPConfig}.\mi{pubkey}} \equiv \bot}
      \mystate{\myss{m' := \encs{\myangle{\mathtt{HTTPResp}, n, 200, \myangle{}, \mathtt{Fail}}}{k}}}
      \mystop{b, a, m'}
    \ENDIF
   %\mystate{\myss{Time := \mathtt{CurrentTime}()}}
   %\mystate{\myss{PIDValidity := session[PIDValidity]}}
   %\mystate{\myss{Content := Token.Content}}
   %\myif{Time>Content.Validity}
     %\mystate{\myss{m' := \myangle{\mathtt{HTTPResp}, n, 200, \myangle{}, \mathtt{Fail}}}}
     %\mystop{b, a, m'}
   %\ENDIF
    \mystate{\myss{\mi{PID_u} := \mi{IDToken}.\str{PID_{u}}}}
    \mystate{\myss{\mi{Acct} := [\mi{loginSessions}.\str{t}]\mi{PID_u}}}\label{line:gen-acct}
   %\mystate{\myss{Acct := \mathtt{Multiply}(PID_U, t^{-1})}}
   %\myif{Acct \not\in \mathtt{ListOfUser}()}
     %\mystate{\myss{\mathtt{AddUser}(Acct)}}
   %\ENDIF
   %\mystate{\myss{session[user] := Acct}}
    \mystate{\myss{s'.\str{serviceTokens} := s'.\str{serviceTokens} + ^{\myangle{}}\mi{Acct}}}\label{line:add-service-token}
   %\mystate{\myss{s'.serviceTokens := s'.serviceTokens + ^{\myangle{}}\myangle{IDToken, Acct}}}
    \mystate{\myss{m' := \encs{\myangle{\mathtt{HTTPResp}, n, 200, \myangle{}, \mathtt{LoginSuccess}}}{k}}}
    \mystop{b, a, m'}
  \ENDIF
  \mystop{}
\end{algorithmic}\setlength{\parindent}{1em}

\subsection{Identity Providers} \label{app:idps}

The identity provider $\mathsf{IdP}$ is a web server 
modeled as an atomic process $(I, Z, R, s_0)$ with 
the addresses $I := \mapAddresstoAP(\mathsf{IdP})$. 
Its initial state $s_0$ contains a list of its 
domains and (private) TLS keys, 
a list of users and identites, and a private key 
for signing IDTokens. Besides this, 
the full state of $\mathsf{IdP}$ further contains a list 
of used nonces, and information about active sessions.

$\mathsf{IdP}$ react to four types of requests:

First, they provide the $\str{script\_idp}$, where a $t$ 
will be chosen and following requests to $\mathsf{IdP}$ 
will be sent. $\mathsf{IdP}$ will transfer the data
to RP by the communicating between two scripts $\str{script\_idp}$ 
and $\str{script\_rp}$ using $\tPostMessage$.

Second, they provide $\mi{IDToken}$ when receiving $\mi{PID_{rp}}$ and this 
$\mi{PID_{rp}}$ has already first. If not, IdPs will redirect to the login dialog.

After the user enter his username and password(secret) in the login dialog, a login
request will send to $\str{/authentication}$. IdPs will check the parameters and 
set the login session.

The last type of requests IdPs react to is authorize requests with $\mi{PID_{rp}}$ and attribute
scopes as parameters. After receving consent from browsers, IdPs will calculate 
$\mi{PID_{u}}$ and construct $\mi{IDToken}$.

\subsubsection{Formal description.} In the following, we 
will first define the (initial) state of IdP formally and 
afterwards present the definition of the relation $R$.

To define the initial state, we will need a term that 
represents the ``user database'' of the IdP. We will 
call this term $\mi{userset}$. This database defines, 
which secret is valid for which identity. It is encoded as 
a mapping of identities to secrets. For example, if the 
secret $\mi{secret}_1$ is valid for the identites
$\mi{id}_1$and the secret $\mi{secret}_2$ is valid for the 
identity $\mi{id}_2$, the $\mi{userset}^i$ looks as follows:
\begin{align*}
\mi{userset} = [\mi{id}_1.\str{username}{:}\myangle{\mi{id}_1, \mi{secret}_1}, 
  \mi{id}_2.\str{username}{:}\myangle{\mi{id}_2, \mi{secret}_2}]
\end{align*}

We define $\mi{userset}$ as $\mi{userset} = \an{\{\an{u.\str{username},
    \myangle{u, \mi{secret} = \mapIDtoPLI(u)}}\, |\, u \in \IDs^u\}}$.

\begin{definition}\label{def:initial-state-idp}
  A \emph{state $s\in Z$ of the IdP} is a term of the form
  $\langle\mi{tlskeys}$, $\mi{users}$, $\mi{signkey}$,
  $\mi{sessions}$, $\mi{corrupt}\rangle$ where 
  $\mi{tlskeys} = \mi{tlskeys} $, 
  $\mi{users} = \mi{userset}$, 
  $\mi{signkey} \in \nonces$ 
  (the key used by the IdP to sign IDTokens),
  $\mi{sessions}\in\dict{\nonces}{\terms}$, $\mi{corrupt} \in \terms$.

  An \emph{initial state $s_0$ of IdP} is a state of the form 
  $\an{\mi{tlskeys}, \mi{userset}, \mapSignKey(\mathsf{IdP}), \an{}, \bot}$.
\end{definition}

The relation $R$ that defines the behavior of the IdP is defined as follows:

%和代码基本保持一致
\captionof{algorithm}{\label{alg:idp} Relation of IdP $R$}
\begin{algorithmic}[1]
  \REQUIRE \myss{\myangle{a, b, m}, s}
  \mystate{\myss{s':=s}}
  \myif{s'.\str{corrupt} \not\equiv \bot \vee m \equiv \corrupt}
    \mystate{\myss{s'.\str{corrupt} := \an{\an{a, f, m}, s'.\str{corrupt}}}}
    \mystate{\myss{m' := d_{V}(s')}\label{line:usage-of-signkey-corrupt-uppresso}}
    \mystate{\myss{a' := \addresses}}
    \mystop{a', a, m'}
  \ENDIF
  \mystate{\myss{m_{dec},k,k',\mi{inDomain}} \textbf{such that} \breakalgohook{0}
    \myss{\an{m_{\text{dec}}, k} \equiv \dec{m}{k'} \wedge \an{inDomain,k'} \in s'.\str{sslkeys}}\breakalgohook{0}
    \textbf{if possible; otherwise stop} \myss{\myangle{}, s'}}
  \mystate{\myss{n, method, path, parameters, headers, body} \textbf{such that} \breakalgohook{0}
    \myss{\myangle{\mathtt{HTTPReq},n,method,path,parameters,headers,body} \equiv m_{dec}}\breakalgohook{0}
    \textbf{if possible; otherwise stop} \myss{\myangle{}, s'}}
  \myif{path \equiv /script}
    \mystate{\myss{m':=\encs{\myangle{\mathtt{HTTPResp},n,200, \myangle{}, \str{script\_idp}}}{k}}}
    \mystop{b, a, m'}
  \myelse{path \equiv /authentication}
    \mystate{\myss{\mi{username} := \mi{body}[\str{username}]}}
    \mystate{\myss{\mi{password} := \mi{body}[\str{password}]}}
    \myif{\mi{password} \not\equiv s'.\str{userset}[\mi{username}].\mi{secret}}
      \mystate{\myss{m':=\encs{\myangle{\mathtt{HTTPResp},n,200,\myangle{},\mathtt{LoginFailure}}}{k}}}
      \mystop{b, a, m'}
    \ENDIF
    \mystate{\myss{\mi{sessionid} := \nu_2}}
    \mystate{\myss{s'.\str{sessions}[\mi{sessionid}] := \mi{username}}}
    \mystate{\myss{\mi{setCookie} := \myangle{\cSetCookie, \myangle{\myangle{\str{sessionid}, \mi{sessionid}, \True, \True, \True}}}}}
    \mystate{\myss{m' :=\myangle{\mathtt{HTTPResp},n,200,\myangle{\mi{setCookie}},\mathtt{LoginSucess}}}}
    \mystop{b, a, m'}
  \myelse{path \equiv /reqToken}
    \mystate{\myss{\mi{cookie} := headers[\str{Cookie}]}}
    \myif{\mi{cookie}[\str{sessionid}] \equiv \myangle{}}
      \mystate{\myss{m' := \encs{\myangle{\mathtt{HTTPResp},n,200,\myangle{},\mathtt{Unauthenticated}}}{k}}}
      \mystop{b, a, m'}
    \ENDIF
    \mystate{\myss{\mi{sessionid} := \mi{cookie}[\str{sessionid}]}}
    \mystate{\myss{\mi{PID_{rp}} := \mi{parameters}[\str{PID_{rp}}]}}
    \myif{s'.\str{sessions}[\mi{sessionid}].\mi{IDToken}[\mi{PID_{rp}}] \equiv \myangle{}}
      \mystate{\myss{m' := \encs{\myangle{\mathtt{HTTPResp},n,200,\myangle{},\mathtt{Unauthorized}}}{k}}}
      \mystop{b, a, m'}
    \ENDIF
    \mystate{\myss{\mi{IDToken} := s'.\str{sessions}[\mi{sessionid}].\mi{IDToken}[\mi{PID_{rp}}]}}
    \mystate{\myss{m' := \encs{\myangle{\mathtt{HTTPResp},n,200,\myangle{}, \mi{IDToken}}}{k}}}
    \mystop{b, a, m'}
  \myelse{path \equiv /authorize}\label{line:uppresso-idp-set-pidu}
    \mystate{\myss{\mi{cookie} := headers[\str{Cookie}]}}
    \myif{\mi{cookie}[\str{sessionid}] \equiv \myangle{}}\label{line:uppresso-idp-check-login-state}
      \mystate{\myss{m' := \encs{\myangle{\mathtt{HTTPResp},n,200,\myangle{},\mathtt{Unauthenticated}}}{k}}}
      \mystop{b, a, m'}
    \ENDIF
    \mystate{\myss{\mi{sessionid} := \mi{cookie}[\str{sessionid}]}}
    \mystate{\myss{\mi{PID_{RP}} := \mi{parameters}[\str{PID_{RP}}]}}
    \myif{\mathtt{IsValid}(PID_{RP}) \equiv \bot}
      \mystate{\myss{m' := \encs{\myangle{\mathtt{HTTPResp}, n, 200, \myangle{}, \mathtt{Fail}}}{k}}}
      \mystop{b, a, m'}
    \ENDIF
    \myif{\mathtt{IsInScope}(uid, \mi{body}[\str{Attr}]) \equiv \bot}
      \mystate{\myss{m' := \encs{\myangle{\mathtt{HTTPResp}, n, 200, \myangle{}, \mathtt{Fail}}}{k}}}
      \mystop{b, a, m'}
    \ENDIF
    \mystate{\myss{u := s'.\str{sessions}[\mi{sessionid}].u}}
    \mystate{\myss{\mi{ID_u} := u.\str{id}}}
    \mystate{\myss{\mi{PID_u} := [\mi{ID_u}]\mi{PID_{rp}}}}
    %\mystate{\myss{Validity := \mathtt{CurrentTime} ()+ s'.Validity}}
    %\mystate{\myss{Content := \myangle{PID_{RP}, PID_U, s'.Issuer, Validity}}}
    \mystate{\myss{\mi{content} := \myangle{PID_{rp}, PID_u}}}
    \mystate{\myss{\mi{ver} := \sig{\mi{content}}{s'.\str{signkey}}}}\label{line:sign-token}
    \mystate{\myss{\mi{IDToken} := \myangle{\mi{content}, \mi{ver}}}}
    \mystate{\myss{s'.\str{sessions}[\mi{IDTokens}]:=s'.\str{sessions}[\mi{IDTokens}]+^{\myangle{}}\myangle{PID_{rp}, IDToken}}}
    \mystate{\myss{m':=\encs{\myangle{\mathtt{HTTPResp}, n, 200, \myangle{}, \mi{IDToken}}}{k}}}
    \mystop{b, a, m'}
  \ENDIF
  \mystop{}
\end{algorithmic}\setlength{\parindent}{1em}

\subsection{UPPRESSO Scripts}\label{app:uppresso-scripts}
As already mentioned in Appendix~\ref{app:outlineuppressomodel}, the set $\scriptset$ 
of the web system $\uppressowebsystem=(\bidsystem, \scriptset, \mathsf{script}, E^0)$ 
consists of the scripts $\Rasp$, $\mi{script\_rp}$, $\mi{script\_idp}$, and with their 
string representations being $\str{att\_script}$, $\str{script\_rp}$, $\str{script\_idp}$, 
and (defined by $\mathsf{script}$). 

In what follows, the scripts $\mi{script\_rp}$ and $\mi{script\_idp}$ are
defined formally.

\subsubsection{Relying Party Page (script\_rp).}\label{app:uppresso-script-rp}
As defined in SPRESSO, a script is a relation that takes a termas input and outputs 
a new term. The input term is provided by the browser. It contains the current 
internal state of the script (which we call \emph{scriptstate} in what follows) and
additional information containing all browser state information the
script has access to, such as the input the script has obtained so far
via \xhrs and \pms, information about windows, etc. The browser
expects the output term to contain, among other information, the new internal \emph{scriptstate}.

We first describe the structure of the internal scriptstate
of the script $\mi{script\_rp}$.

\begin{definition} \label{def:scriptstaterp} 
A \emph{scriptstate $s$ of $\mi{script\_rp}$} is a term of the form $\langle 
\mi{phase}$, 
%$\mi{loginSessionToken}$, 
$\mi{refXHR}\rangle$, 
where $phase \in \mathbb{S}$, 
%$\mi{loginSessionToken}$,
$\mi{refXHR}\in \nonces \cup \{\bot\}$. 

The \emph{initial scriptstate $\mi{initState_{rp}}$} of $\mi{script\_rp}$ is 
$\an{\str{start},
%\bot,
\bot}$.
\end{definition}

We now specify the relation $\mi{script\_rp}$ formally. We describe this relation
by a non-deterministic algorithm.

\captionof{algorithm}{\label{alg:uppresso-script-rp} Relation of $\mi{script\_rp}$}
\begin{algorithmic}[1]
\REQUIRE \myss{\langle\mi{tree},\mi{docnonce},\mi{scriptstate},\mi{scriptinputs},\mi{cookies},\mi{localStorage},\mi{sessionStorage},}
\breakalgohook{-1}\myss{\mi{ids},\mi{secret}\rangle}
\mystate{\myss{ s' := \mi{scriptstate}}}
\mystate{\myss{\mi{command} := \myangle{}}}
\mystate{\myss{\mi{origin} := \mathsf{GETORIGIN}(\mi{tree},\mi{docnonce})}}
\mystate{\myss{\mi{RPDomain} := \mi{origin}.\str{host}}}
\SWITCH{\myss{s'.\str{phase}}}
\CASE{\myss{\str{start}}}
  \mystate{\myss{\mi{url} := \an{\tUrl, \https, \mi{RPDomain}, \str{/loginSSO}, \myangle{}}}}
  \mystate{\myss{\mi{command} := \an{\tHref,\mi{url},\wBlank,\an{}}}}
  \mystate{\myss{s'.\str{phase} := \str{expectt}}}
\ENDCASE
\CASE{\myss{\str{expectt}}}
  \mystate{\myss{\mi{pattern} := \myangle{\tPostMessage, target, *, \myangle{\str{t}, *}}}}
  \mystate{\myss{\mi{input} := \textsf{CHOOSEINPUT}(\mi{scriptinputs}, \mi{pattern})}}
  \myif{\mi{input} \not\equiv \bot}
    \mystate{\myss{t := \pi_2(\pi_4(\mi{input}))}}\label{line:receive-t}
    %\mystate{\myss{\mi{url} := \myangle{\tUrl, \https, \mi{RPDomain}, \str{/startNegotiation}, \myangle{}}}}
    \mystate{\myss{\mi{body} := \myangle{\myangle{\str{t},t}}}}
    \mystate{\myss{\mi{command} := \langle\tXMLHTTPRequest,\textsf{URL}^{\mi{RPDomain}}_\str{/startNegotiation},\mPost,\mi{body},}\breakalgohook{0}\myss{s'.\str{refXHR}\rangle}}
    \mystate{\myss{s'.\str{phase} := \str{expectCert}}}
  \ENDIF
\ENDCASE
\CASE{\myss{\str{expectCert}}}
  \mystate{\myss{pattern := \myangle{\tXMLHTTPRequest,*,s'.\str{refXHR}}}}
  \mystate{\myss{\mi{input} := \textsf{CHOOSEINPUT}(\mi{scriptinputs}, \mi{pattern})}}
  \myif{\mi{input} \not\equiv \bot}
    \mystate{\myss{\mi{Cert_{rp}} := \pi_2(\mi{input}).\str{Cert_{rp}}}}
    \mystate{\myss{\mi{IdPWindowNonce} := \pi_1(\textsf{SUBWINDOWS}(\mi{tree},\mi{docnonce})).\str{nonce}}}
    \mystate{\myss{\mi{IdPOrigin} := \mathsf{GETORIGIN}(\mi{tree}, \mi{IdPWindowNonce})}}
    \mystate{\myss{\mi{command} := \langle\tPostMessage, \mi{IdPWindowNonce}, \myangle{\str{Cert}, \mi{Cert_{rp}}},}\breakalgohook{0}\myss{\mi{IdPOrigin}\rangle}}
    \mystate{\myss{s'.\str{phase} := \str{expectToken}}}
  \ENDIF
\ENDCASE
\CASE{\myss{\str{expectToken}}}
  \mystate{\myss{\mi{pattern} := \myangle{\tPostMessage, target, *, \myangle{\str{IDToken}, *}}}}
  \mystate{\myss{\mi{input} := \textsf{CHOOSEINPUT}(\mi{scriptinputs}, \mi{pattern})}}
  \myif{input \not\equiv \bot}
    \mystate{\myss{\mi{IDToken} := \pi_2(\pi_4(\mi{input}))}}
    %\mystate{\myss{\mi{url} := \myangle{\tUrl, \https, \mi{RPDomain}, \str{/uploadToken}, \myangle{}}}}
    \mystate{\myss{\mi{body} := \myangle{\myangle{\str{IDToken},\mi{IDToken}}}}}
    \mystate{\myss{\mi{command} := \langle\tXMLHTTPRequest,\textsf{URL}^{\mi{RPDomain}}_\str{/uploadToken},\mPost,\mi{body},}\breakalgohook{0}\myss{s'.\str{refXHR}\rangle}}
    \mystate{\myss{s'.\str{phase} := \str{expectLoginResult}}}
  \ENDIF
\ENDCASE
\CASE{\myss{\str{expectLoginResult}}}
  \mystate{\myss{pattern := \myangle{\tXMLHTTPRequest,*,s'.\str{refXHR}}}}
  \mystate{\myss{\mi{input} := \textsf{CHOOSEINPUT}(\mi{scriptinputs}, \mi{pattern})}}
  \myif{input \not\equiv \bot}
    \myif{\pi_2(input) \equiv \str{LoginSuccess}}
    \mystate{Load Homepage}
    \ENDIF
  \ENDIF
\ENDCASE
\ENDSWITCH\\
\mystopp{s',\mi{cookies},\mi{localStorage},\mi{sessionStorage},\mi{command}}
\end{algorithmic}\setlength{\parindent}{1em}

\subsubsection{Identity Provider Page (script\_idp).}\label{app:uppresso-script-Idp}

\begin{definition}\label{def:scriptstateidp}
  A \emph{scriptstate $s$ of $\mi{script\_idp}$} is a term of the form
  $\langle \mi{phase}$, $\mi{user}$, $\mi{parameters} \rangle$ with $\mi{phase} \in
  \mathbb{S}$, $\mi{user} \in \IDs \cup \{\an{}\} \in \gterms$ and $\mi{parameters} \in \dict{\mathbb{S}}{\terms}$,. The 
  \emph{initial scriptstate} of $\mi{script\_idp}$ is $\an{\str{start},*,\myangle}$.
\end{definition}

We now formally specify the relation of $\mi{script\_idp}$

\captionof{algorithm}{\label{alg:uppresso-script-idp} Relation of $\mi{script\_idp}$ }
\begin{algorithmic}[1]
\REQUIRE \myss{\langle\mi{tree},\mi{docnonce},\mi{scriptstate},\mi{scriptinputs},\mi{cookies},\mi{localStorage},\mi{sessionStorage},}
\breakalgohook{-1}\myss{\mi{ids},\mi{secret}\rangle}
\mystate{\myss{s' := scriptstate}}
\mystate{\myss{\mi{command} := \myangle{}}}
\mystate{\myss{\mi{target} := \textsf{OPENERWINDOW}(\mi{tree},\mi{docnonce})}}
\mystate{\myss{\mi{origin} := \mathsf{GETORIGIN}(\mi{tree},\mi{docnonce})}}
\mystate{\myss{\mi{IdPDomain} := \mi{origin}.\str{host}}}
\SWITCH{\myss{s'.\str{phase}}}
\CASE{\myss{start}}
  \mystate{\myss{t := \str{random}()}}
  \mystate{\myss{\mi{command} := \myangle{\tPostMessage, \mi{target}, \myangle{\str{t}, t}, \myangle{}}}}\label{line:send-t}
  \mystate{\myss{s'.\str{parameters}[t] := t}}
  \mystate{\myss{s'.\str{phase} := \str{expectCert}}}
\ENDCASE
\CASE{\myss{\str{expectCert}}}
  \mystate{\myss{\mi{pattern} := \myangle{\tPostMessage, target, *, \myangle{\str{Cert}, *}}}}
  \mystate{\myss{\mi{input} := \textsf{CHOOSEINPUT}(\mi{scriptinputs}, \mi{pattern})}}
  \myif{input \not\equiv \bot}
    \mystate{\myss{\mi{Cert_{rp}} := \pi_2(\pi_4(input))}}
    \myif{\checksigThree{\mi{Cert_{rp}}.\str{ver}}{\mi{Cert_{rp}}.\str{content}}{s'.\str{IdPConfig}.\mi{pubkey}} \equiv \True}
      \mystate{\myss{s'.\str{parameters}[\mi{cert}] := \mi{Cert_{rp}}}}
      \mystate{\myss{t := s'.\str{parameters}[t]}}
      \mystate{\myss{\mi{PID_{rp}} := [t]\mi{Cert_{rp}}.\str{content}[\mi{ID_{rp}}]}}\label{line:gen-pidrp}
      \mystate{\myss{s'.\str{parameters}[\mi{PID_{rp}}] := \mi{PID_{rp}}}}
      \mystate{\myss{\mi{body} := \myangle{\myangle{\str{PID_{rp}},\mi{PID_{rp}}}}}}
      \mystate{\myss{\mi{command} := \langle\tXMLHTTPRequest,\textsf{URL}^{\mi{IdPDomain}}_\str{/reqToken},\mPost,\mi{body},}\breakalgohook{0}\myss{s'.\str{refXHR}\rangle}}
      \mystate{\myss{s'.\str{phase} := \str{expectReqToken}}}
    \ENDIF
  \ENDIF
\ENDCASE
\CASE{\myss{\str{expectReqToken}}}
  \mystate{\myss{pattern := \myangle{\tXMLHTTPRequest,*,s'.\str{refXHR}}}}
  \mystate{\myss{\mi{input} := \textsf{CHOOSEINPUT}(\mi{scriptinputs}, \mi{pattern})}}
  \myif{input \not\equiv \bot}
    \myif{\pi_2(input) \equiv \str{Unanthenticated}}
      \mystate{\myss{s'.\str{user} \gets \mi{ids}}}
      \mystate{\myss{\mi{username} := s'.\str{user}.\mi{name}}}
      \mystate{\myss{\mi{password} := \textsf{secretOfID}(s'.\str{user})}}
      \mystate{\myss{\mi{body} := \myangle{\myangle{\str{username}, \mi{username}}, \myangle{\str{password}, \mi{password}}}}}
      \mystate{\myss{\mi{command} := \langle\tXMLHTTPRequest,\textsf{URL}^{\mi{IdPDomain}}_\str{/authentication},\mPost,\mi{body},}\breakalgohook{0}\myss{s'.\str{refXHR}\rangle}}
      \mystate{\myss{s'.\str{phase} := \str{expectLoginResult}}}
    \myelse{\pi_2(input) \equiv \str{Unauthorized}}
      \mystate{\myss{\mi{PID_{rp}} := s'.\str{parameters}[\mi{PID_{rp}}]}}
      \mystate{\myss{\mi{Attr} := \textsf{GETPARAMETERS}(\mi{tree}, \mi{docnonce})[\str{iaKey}]}}
      \mystate{\myss{\mi{body} := \myangle{\myangle{\str{PID_{rp}}, \mi{PID_{rp}}}, \myangle{\str{Attr}, \mi{Attr}}}}}
      \mystate{\myss{\mi{command} := \langle\tXMLHTTPRequest,\textsf{URL}^{\mi{IdPDomain}}_\str{/authorize},\mPost,\mi{body},}\breakalgohook{0}\myss{s'.\str{refXHR}\rangle}}
      \mystate{\myss{s'.\str{phase} := \str{expectToken}}}
    \myelse{}
      \mystate{\myss{IDToken := \pi_2(input)[\str{IDToken}]}}
      \mystate{\myss{RPOringin := \myangle{s'.\str{parameters}[\mi{cert}].\mi{Content}[\str{Enpt}], \mathtt{S}}}}
      \mystate{\myss{\mi{command} := \myangle{\tPostMessage, \mi{target}, \myangle{\str{IDToken},\mi{IDToken}}, RPOrigin}}}
      \mystate{\myss{s'.\str{phase} := \str{stop}}}
    \ENDIF
  \ENDIF
\ENDCASE
\CASE{\myss{\str{expectLoginResult}}}
  \mystate{\myss{pattern := \myangle{\tXMLHTTPRequest,*,s'.\str{refXHR}}}}
  \mystate{\myss{\mi{input} := \textsf{CHOOSEINPUT}(\mi{scriptinputs}, \mi{pattern})}}
  \myif{input \not\equiv \bot}
    \myif{\pi_2(input) \equiv \str{LoginSuccess}}
      \mystate{\myss{\mi{PID_{rp}} := s'.\str{parameters}[\mi{PID_{rp}}]}}
      \mystate{\myss{\mi{Attr} := \textsf{GETPARAMETERS}(\mi{tree}, \mi{docnonce})[\str{iaKey}]}}
      \mystate{\myss{\mi{body} := \myangle{\myangle{\str{PID_{rp}}, \mi{PID_{rp}}}, \myangle{\str{Attr}, \mi{Attr}}}}}\label{line:send-pidrp}
      \mystate{\myss{\mi{command} := \langle\tXMLHTTPRequest,\textsf{URL}^{\mi{IdPDomain}}_\str{/authorize},\mPost,\mi{body},}\breakalgohook{0}\myss{s'.\str{refXHR}\rangle}}
      \mystate{\myss{s'.\str{phase} := \str{expectToken}}}
    \ENDIF
  \ENDIF
\ENDCASE
\CASE{\myss{\str{expectToken}}}
  \mystate{\myss{pattern := \myangle{\tXMLHTTPRequest,*,s'.\str{refXHR}}}}
  \mystate{\myss{\mi{input} := \textsf{CHOOSEINPUT}(\mi{scriptinputs}, \mi{pattern})}}
  \myif{input \not\equiv \bot}
    \mystate{\myss{IDToken := \pi_2(input)[\str{IDToken}]}}
    \mystate{\myss{RPOringin := \myangle{s'.\str{parameters}[\mi{cert}].\mi{Content}[\str{Enpt}], \mathtt{S}}}}
    \mystate{\myss{\mi{command} := \myangle{\tPostMessage, \mi{target}, \myangle{\str{IDToken},\mi{IDToken}}, RPOrigin}}}\label{line:token-send}
    \mystate{\myss{s'.\str{phase} := \str{stop}}}
  \ENDIF
\ENDCASE
\ENDSWITCH
\mystopp{s',\mi{cookies},\mi{localStorage},\mi{sessionStorage},\mi{command}}
\end{algorithmic}\setlength{\parindent}{1em}


\section{Proof of Security}

To state the security properties for \uppresso, we first
define an \emph{\uppresso web system for authentication analysis}. This
web system is based on the \uppresso web system and only considers one
network attacker (which subsumes all web attackers and further network
attackers).

\begin{definition}
  Let $\uppressoauthwebsystem = (\bidsystem, \scriptset, \mathsf{script}, E^0)$
  an \uppresso web system. We call $\uppressoauthwebsystem$ an
  \emph{\uppresso web system for authentication analysis} iff
  $\bidsystem$ contains only one network attacker process
  $\fAP{attacker}$ and no other attacker processes (i.e.,
  $\mathsf{Net} = \{\fAP{attacker}\}$, $\mathsf{Web} = \emptyset$).
  %Further, $\bidsystem$ contains no DNS servers. DNS servers are
  %assumed to be dishonest, and hence, are subsumed by
  %$\fAP{attacker}$. In the initial state $s_0^b$ of each browser $b$
  %in $\bidsystem$, the DNS address is
  %$\mapAddresstoAP(\fAP{attacker})$. Also, in the initial state
  %$s_0^r$ of each relying party $r$, the DNS address is
  %$\mapAddresstoAP(\fAP{attacker})$.
\end{definition}

The security properties for \uppresso are formally defined 
as follows. First note that every $Acct$ recorded in RP was
calculated by RP as the result of an HTTPS $\mPost$ request 
$m$. We refer to $m$ as the 
\emph{request corresponding to $Acct$}. 

%讨论acct的唯一性
In the following definition, when we say a browser 
$b\in \fAP{B}$ owns $Acct$, we holds that for some RP 
$r\in \fAP{RP}$ that calculate it and 
an identity $u\in \fAP{ID}$ with $\mapIDtoOwner(u) = b$.
\[\mi{Acct}=\mi{acct}^u[\mapDomain(r)]\]

We now define the similar security properties as the definition 52 in SPRESSO. 

\begin{definition}\label{def:uppresso-security-property} 
  Let $\uppressoauthwebsystem$ be an \uppresso web system for authentication analysis. 
  We say that \emph{$\uppressoauthwebsystem$ is secure} if for every run $\rho$ of
  $\uppressoauthwebsystem$, every state $(S^j, E^j, N^j)$ in $\rho$,
  every $r\in \fAP{RP}$ that is honest in $S^j$, every RP service token of the form 
  $Acct$ recorded in $S^j(r).\str{serviceTokens}$, the following two conditions are
  satisfied:

  \textbf{(A)} If $Acct$ is derivable from the attackers knowledge
  in $S^j$ (i.e., $Acct \in d_{\emptyset}(S^j(\fAP{attacker}))$),
  then it follows that the browser $b$ owning $Acct$ is fully corrupted
  in $S^j$ (i.e., the value of $\mi{isCorrupted}$ is $\fullcorrupt$)
  or the only IdP is dishonest (in $S^j$).

  \textbf{(B)} If the request corresponding to $Acct$ was sent by
  some $b\in \fAP{B}$ which is honest in $S^j$, then $b$ owns $Acct$.
\end{definition}

%First note that the RP service token should be defined as $\langle IDToken$, $Acct \rangle$ 
%which is $\langle n$, $i \rangle$ in SPRESSO. That is,  

%let  $\mathcal{U\!W\!S}^{auth}$ be an UPPRESSO web system for authentication analysis. We say that $\mathcal{U\!W\!S}^{auth}$ is secure if for every run $\rho$ of $\mathcal{U\!W\!S}^{auth}$, every state ($S^j$, $E^j$, $N^j$) in $\rho$, every $r \in$ $\mathtt{RP}$ that is honest, every RP service token of the form $\langle IDToken$, $Acct \rangle$ recorded in $S^j$($r$).$\mathtt{serviceTokens}$, the following two conditions are satisfied:

%(A) If $\langle IDToken$, $Acct \rangle$ is derivable from the attackers knowledge in $S^j$ (i.e., $\langle IDToken$, $Acct \rangle \in d_{\emptyset}$($S^j$($\mathtt{attacker}$))), then it follows that the browser b owning $Acct$ is fully corrupted in $S^j$ (i.e., the value of $isCorrupted$ is $\mathtt{FULLCORRUPT}$) or $\mathtt{governor}$($Acct$) is not an honest IdP (in $S^j$).

%(B) If the request corresponding to $\langle IDToken$, $Acct \rangle$ was sent by some $b \in \mathtt{B}$ which is honest in $S^j$, then b owns the $ID_U$ which satisfies $Acct=[ID_U]S^j(r).ID_{RP}$.

To prove Theorem 1 in section 5.1, we are going to prove the following Lemmas.

\begin{lemma}\label{lemma:k-does-not-leak-from-honest-rp} 
  If in the processing step $s_i \rightarrow s_{i+1}$ of a run $\rho$
  of $\uppressoauthwebsystem$ an honest relying party $r$ (I) emits an HTTPS
  request of the form

  \[ m = \ehreqWithVariable{\mi{req}}{k}{\pub(k')} \]
%
  (where $\mi{req}$ is an HTTP request, $k$ is a nonce (symmetric
  key), and $k'$ is the private key of some other DY process $u$), and (II) in the
  initial state $s_0$ the private key $k'$ is only known to $u$, and
  (III) $u$ never leaks $k'$, then all of the following
  statements are true:
  \begin{enumerate}
  \item There is no state of $\uppressoauthwebsystem$ where any party except
    for $u$ knows $k'$, thus no one except for $u$ can
    decrypt $\mi{req}$.
    \label{prop:attacker-cannot-decrypt-spresso}
  \item If there is a processing step $s_j \rightarrow s_{j+1}$ where
    the RP $r$ leaks $k$ to $\bidsystem \setminus \{u, r\}$ there
    is a processing step $s_h \rightarrow s_{h+1}$ with $h < j$
    where $u$ leaks the symmetric key $k$ to $\bidsystem \setminus
    \{u,r\}$ or $r$ is corrupted in
    $s_j$. \label{prop:k-doesnt-leak-spresso}
  \item The value of the host header in $\mi{req}$ is the domain that
    is assigned the public key $\pub(k')$ in RP's keymapping
    $s_0.\str{keyMapping}$ (in its initial
    state). \label{prop:host-header-matches-spresso}
  \item If $r$ accepts a response (say, $m'$) to $m$ in a processing step $s_j
    \rightarrow s_{j+1}$ and $r$ is honest in $s_j$ and $u$ did not
    leak the symmetric key $k$ to $\bidsystem \setminus \{u,r\}$ prior
    to $s_j$, then $u$ created the HTTPS response $m'$ to the HTTPS
    request $m$, i.e., the nonce of the HTTP request $\mi{req}$ is not known to
    any atomic process $p$, except for the atomic DY processes $r$ and
    $u$.\label{prop:only-owner-answers-spresso}
  \end{enumerate}
\end{lemma}

%\begin{lemma}\label{lemma:wkcache-never-lies}
%  For every honest relying party $r \in \fAP{RP}$, every $s \in \rho$, every
%  $\an{\mi{host}, \mi{wkDoc}} \inPairing S(r).\str{wkCache}$ it holds
%  that $\mi{wkDoc}[\str{signkey}] \equiv
%  \pub(\mathsf{signkey}(\mapDomain^{-1}(\mi{host})))$ if
%  $\mapDomain^{-1}(\mi{host})$ is an honest IdP.
%\end{lemma}

\begin{lemma}\label{lemma:uppresso-request-exists}
  In a run $\rho$ of $\uppressoauthwebsystem$, for every 
  state $s_j \in\rho$, every RP $r \in \fAP{RP}$ that is 
  honest in $s_j$, every $\myangle{nonce, Acct} \inPairing 
  S^j(r).\str{serviceTokens}$, the following properties hold:

  \begin{enumerate}
  \item There exists exactly one $l' < j$ such that there exists a
    processing step in $\rho$ of the form
    \[ s_{l'} \xrightarrow[r \rightarrow \an{\an{a',f',m'}}]{e'
      \rightarrow r} s_{l'+1}\]
    with $e'$ being some events, $a'$ and $f'$
    being addresses and $m'$ being a service token response for $Acct$.

  \item There exists exactly one $l < j$ such that there exists a
    processing step in $\rho$ of the form 
    \[ s_{l} \xrightarrow[r \rightarrow e]{\an{a,f,m} \rightarrow r}
    s_{l+1} \] with $e$ being some events, $a$ and $f$ being
    addresses and $m$ being a service token request for $Acct$.

  \item The processing steps from (1) and (2) are the same, i.e., $l = l'$.

  \item The service token request for $Acct$, $m$ in (2), is an HTTPS message of the following form:
    \[ \mathsf{enc}_\mathsf{a}(\langle \hreq{ 
      nonce=n_\text{req}, 
      method=\mPost,
      xhost=d_r, 
      path=\str{/authorize}, 
      parameters=x, 
      headers=h,
      xbody=b}, k\rangle, \pub(\mapTLSKey(d_r))) \]
    for $d_r \in \mapDomain(r)$, some terms $x$, $h$, $n_\text{req}$, and a dictionary $b$ such that 
    \[ b[\str{IDToken}] \equiv \myangle{\mi{PID_{rp}, \mi{PID_u}, ver}} \]
    with 
    \[ \mi{PID_{rp}} \equiv [S^l(r).\str{loginSessions}[t]][S^l(r).\str{rp}.\mi{id}]G, \]
    \[ \mi{PID_{u}} \equiv [u]\mi{PID_{rp}}, \]
    \[ \mi{ver} \equiv \sig{\an{PID_{rp},PID_u}}{k_{sign}} \]
    for some nonces $u$, and $k_\text{sign}$.
  \item If the IdP $i$ is honest, we have that $k_\text{sign} = S^l(i).\str{signkey}$.
  \end{enumerate}
\end{lemma}

We define the Lemma~\ref{lemma:k-does-not-leak-from-honest-rp} 
and ~\ref{lemma:uppresso-request-exists}, which prove 
that the data transmitted through HTTPS is secure and the 
IdP's public key used for generating IDToken is secure. 
In UPPRESSO, only the single IdP is trusted, so that the 
public key is guaranteed to be always trusted. Therefore, 
we can also follow the proofs in SPRESSO.

\subsection{Proof of Property A}
Then we prove the Property $A$ is satisfied in UPPRESSO.
As stated above, the Property $A$ is defined as follows:
\begin{definition}\label{def:uppresso-security-property} 
  Let $\uppressoauthwebsystem$ be an \uppresso web system 
  for authentication analysis. We say that 
  \emph{$\uppressoauthwebsystem$ is secure 
  (with respect to Property A)} if for every run $\rho$ of 
  $\uppressoauthwebsystem$, every state $(S^j, E^j, N^j)$ in 
  $\rho$, every $r\in \fAP{RP}$ that is honest in $S^j$, 
  every RP service token of the form $\myangle{nonce, Acct}$ 
  recorded in $S^j(r).\str{serviceTokens}$ and derivable 
  from the attackers knowledge in $S^j$ (i.e., 
  $\myangle{nonce, Acct} \in 
  d_{\emptyset}(S^j(\fAP{attacker}))$), it follows that the 
  browser $b$ owning $Acct$ is fully corrupted in $S^j$ 
  (i.e., the value of $\mi{isCorrupted}$ is $\fullcorrupt$) 
  or the IdP is dishonest. 
\end{definition}

%\begin{definition}
%Let $\mathcal{U\!W\!S}^{auth}$  be an UPPRESSO web system for authentication analysis. We say that $\mathcal{U\!W\!S}^{auth}$  is secure (with respect to Property A) if for every run $rho$ of $\mathcal{U\!W\!S}^{auth}$ , every state ($S^j$, $E^j$, $N^j$) in $rho$, every $r \in \mathtt{RP}$ that is honest in
%$S^j$, every RP service token of the form $\langle IDToken$, $Acct \rangle$ recorded in $S^j$($r$).$\mathtt{serviceTokens}$ and derivable from the attackers knowledge in $S^j$ (i.e., $\langle IDToken$, $Acct \rangle \in d_{\emptyset}$($S^j$($\mathtt{attacker}$))), it follows that the browser b owning $Acct$ is fully corrupted in $S^j$ (i.e., the value of $isCorrupted$ is $\mathtt{FULLCORRUPT}$) or $\mathtt{governor}$($Acct$) is not an honest IdP (in $S^j$).
%\end{definition}


Same as the proof in SPRESSO, we want to show that every UPPRESSO web system is secure with regard to Property A and therefore assume that there exists an UPPRESSO web system that is not secure. We will lead this to a contradication and thereby show that all UPPRESSO web systems are secure (with regard to Property A).

%In detail, we assume: \emph{There is an UPPRESSO web system for authentication analysis $\mathcal{U\!W\!S}^{auth}$. We say that $\mathcal{U\!W\!S}^{auth}$  is secure (with respect to Property A) if for every run $rho$ of $\mathcal{U\!W\!S}^{auth}$ , every state ($S^j$, $E^j$, $N^j$) in $rho$, every $r \in \mathtt{RP}$ that is honest in
%$S^j$, every RP service token of the form $\langle IDToken$, $Acct \rangle$ recorded in $S^j$($r$).$\mathtt{serviceTokens}$ and derivable from the attackers knowledge in $S^j$ (i.e., $\langle IDToken$, $Acct \rangle \in d_{\emptyset}$($S^j$($\mathtt{attacker}$))), it follows that the browser b owning $Acct$ is not fully corrupted in $S^j$ and $\mathtt{governor}$($Acct$) is an honest IdP (in $S^j$).}

In detail, we assume: \emph{There exists an \uppresso web 
system $\uppressoauthwebsystem$, a run $\rho$ of 
$\uppressoauthwebsystem$, a state $s_j = (S^j, E^j, N^j)$ 
in $\rho$, a RP $r\in \fAP{RP}$ that is honest in $S^j$, 
an RP service token of the form $\myangle{nonce, Acct}$
recorded in $S^j(r).\str{serviceTokens}$ and derivable from 
the attackers knowledge in $S^j$ (i.e., $Acct \in
d_{\emptyset}(S^j(\fAP{attacker}))$), and the browser $b$ 
owning $i$ is not fully corrupted and IdP is honest 
(in $S^j$).}

We now proceed to proof that this is a contradiction. 
First, we can see that for $\an{n, Acct}$ and $s_j$, 
the conditions in Lemma~\ref{lemma:uppresso-request-exists} 
are fulfilled, i.e., a service token request $m$ and a 
service token response $m'$ to/from $r$ exist, and $m'$ is 
of form shown in Lemma~\ref{lemma:uppresso-request-exists} (4).
Let $I := \mapGovernor(Acct)$. 
We know that $I$ is an honest IdP.
As such, it never leaks its signing key (see Algorithm~\ref{alg:idp}). 
Therefore, the signed subterm $\mi{Content} := 
\myangle{PID_{rp}, PID_u}$, 
$\mi{ver} := \sig{\an{PID_{rp},PID_u}}{k_{sign}}$ and 
$\mi{IDToken} := \myangle{\mi{Content}, \mi{ver}}$ had to be 
created by the IdP $I$. An (honest) IdP creates signatures 
only in Line~\ref{line:sign-token} of Algorithm~\ref{alg:idp}.

\begin{lemma} %(Same as Lemma 4 in SPRESSO) 
  Under the assumption above, only the browser b can issue a request $req$
  (say, $m_{attr}$)that triggers the IdP I to create the signed term IDToken. The 
  request was sent by b over HTTPS using I's public HTTPS key.
\end{lemma}
\begin{proof}
  We have to consider two cases for the request $m_{attr}$:

  \textbf{(A).} First, if the user is not logged in with the identity $u$ at $I$
  (i.e., the browser $b$ has no session cookie that carries a nonce
  which is a session id at $I$ for which the identitiy $u$ is marked
  as being logged in, compare Line~\ref{line:uppresso-idp-check-login-state} of
  Algorithm~\ref{alg:idp}), then the request has to carry (in
  the request body) the password matching the identity $u$ ($\mapIDtoPLI(u)$) to the 
  path $\str{/authentication}$ to retrieve the session cookie. This secret is only 
  known to $b$ initially. Depending on the corruption status of $b$, we can now have two
  cases:
  \begin{enumerate}
  \item[a)] If $b$ is honest in $s_j$, it has not sent the secret to
    any party except over HTTPS to $I$ (as defined in the definition
    of browsers). 
  \item[b)] If $b$ is close-corrupted, it has not sent it to any other
    party while it was honest (case a). When becoming close-corrupted,
    it discarded the secret.
  \end{enumerate}

  I.e., the secret has been sent only to $I$ over HTTPS or to nobody
  at all. The IdP $I$ cannot send it to any other party. Therefore we
  know that only the browser $b$ can send the request $m_\text{attr}$
  in this case.

  \textbf{(B).} Second, if the user is logged in for the identity $i$
  at $I$, the browser provides a session id to $I$ that refers to a
  logged in session at $I$. This session id can only be retrieved from
  $I$ by logging in, i.e., case (A) applies, in particular, $b$ has to
  provide the proper secret, which only itself and $I$ know (see
  above). The session id is sent to $b$ in the form of a cookie, which
  is set to secure (i.e., it is only sent back to $I$ over HTTPS, and
  therefore not derivable by the attacker) and httpOnly (i.e., it is 
  not accessible by any scripts). The browser $b$ sends the cookie only 
  to $I$. The IdP $I$ never sends the session id to any other party than 
  $b$. The session id therefore only leaks to $b$ and $I$, and never to 
  the attacker. Hence, the browser $b$ is the only atomic DY process which 
  can send the request $m_\text{attr}$ in this case.

  We can see that in both cases, the request was sent by $b$ using
  HTTPS and $I$'s public key: If the browser would intend to sent the
  request without encryption, the request would not contain the
  password in case (A) or the cookie in case (B). The browser always
  uses the ``correct'' encryption key for any domain (as defined in
  $\uppressoauthwebsystem$).
%The proof is same as the Lemma 4's proof in SPRESSO.
%It can be proved that the $IDToken$ only contains the $PID_U:=[ID_U]PID_{RP}$ while $PID_U$ is provided by $b$, and $b$ owns the password of $ID_U$.
\end{proof}

\begin{lemma} %(Same as Lemma 5 in SPRESSO) 
  In the browser $b$, the request $m_{attr}$ was triggered by $\mi{script\_idp}$ 
  loaded from the origin $\myangle{d, S}$ for some $d \in \mathtt{dom}(I)$.
\end{lemma}
\begin{proof}
  First, $\an{d,\https}$ for some $d \in \mapDomain(I)$ is the only
  origin that has access to the secret $\mapIDtoPLI(u)$ for the
  identity $u$ (as defined in Appendix~\ref{app:browsers-uppresso}).

  With the general properties defined in~\cite{fett2015analyzing} and the
  definition of Identity Providers in Appendix~\ref{app:idps}, in
  particular their property that they only send out one script,
  $\mi{script\_idp}$, we can see that this is the only script that can
  trigger a request containing the secret.
%The proof follows the Lemma 5's proof in SPRESSO.
%It can be proved that only the IdP's script $script\_idp$ owns the password of $ID_U$ can request the $IDToken$ from $I$.
\end{proof}

\begin{lemma} \label{lemma:idp-to-script-idp} %(Same as Lemma 6 in SPRESSO)
  In the browser $b$, the script $\mi{script\_idp}$ receives the response to the 
  request $m_{attr}$ (and no other script), and at this point, the browser is still 
  honest.
\end{lemma}
\begin{proof}
  From the definition of browser corruption, we can see that the
  browser $b$ discards any information about pending requests in its
  state when it becomes close-corrupted, in particular any SSL keys.
  It can therefore not decrypt the response if it becomes
  close-corrupted before receiving the response.

  The rest follows from the general properties defined
  in~\cite{fett2015analyzing}.
%The proof follows Lemma 6's proof in SPRESSO.
%It is proved that only the closed-corrupted browser cannot receive the $IDToken$ responding to the $req$ started by the honest browser $b$.
\end{proof}

We now know that only the script $\mi{script\_idp}$ received the
response containing the IDToken. For the following lemmas, we will assume
that the browser $b$ is honest. In the other case (the browser is
close-corrupted), the IA $\mi{ia}$ and any information about pending
HTTPS requests (in particular, any decryption keys) would be discarded
from the browser's state (as seen in the proof for 
Lemma~\ref{lemma:idp-to-script-idp}). This would be a contradiction to
the assumption (which requires that the IDToken arrived at the RP).

%Lemma 7 in SPRESSO is not useful here because there is no FWD server in UPPRESSO.

\begin{lemma} %(Same as Lemma 8 in SPRESSO) 
  The script $\mi{script\_idp}$ forwards the IDToken only to the script 
  $\mi{script\_rp}$ loaded from the origin $\langle d_r, \https\rangle$.
\end{lemma}
\begin{proof}
  It is clear that, the IDToken held by the honest $script\_idp$ is only 
  sent to the origin $\langle Cert_{rp}.Enpt_{rp}, \https\rangle$, 
  while the $IDToken.PID_{rp} \equiv [t]Cert_{rp}.ID_{rp}$, and $t$ is the 
  one-time random number. The relation of $Cert_{rp}.ID_{rp}$ and 
  $Cert_{rp}.Enpt_{rp}$ is guaranteed by the signature $Cert_{rp}.ver$ generated 
  by IdP $I$. The process is shown at Line~\ref{line:token-send}
  Algorithm~\ref{alg:uppresso-script-idp}.
%The proof is same as proof of Lemma 8 in SPRESSO.
%It can be proved that, the $IDToken$ held by the honest $script\_idp$ is only sent to the origin $\langle Cert_{RP}.Enpt_{RP}, S \rangle$, while the $IDToken.PID_{RP} \equiv [t]Cert_{RP}.ID_{RP}$, and $t$ is the one-time random number.  The relation of $ID_RP$ and $Enpt$ is guaranteed by the signature generated by IdP $I$. The process is shown at Line 9, 16, 19, 21, 38, 39, 59, 60  in Algorithm~\ref{alg:script_idp}.
\end{proof}

\begin{lemma} %(Same as Lemma 9 in SPRESSO) 
  From the RP document, the IDToken is only sent to the RP r and over HTTPS
\end{lemma}
\begin{proof}
%The proof follows the proof of Lemma 9 in SPRESSO.
It is proved that $script\_rp$ of the origin $\langle Cert_{rp}.Enpt_{rp}, \https\rangle$ 
would only sent to the corresponding RP $r$, which is shown in Algorithm~\ref{alg:uppresso-script-rp}.
\end{proof}

The proofs show that the IDToken is only sent to the honest browser and target RP.
It cannot be known to the attacker or any corrupted party, as none of the listed 
parties leak it to any corrupted party or the attacker. 
Above proofs can be reduced to the {\color{blue}Confidentiality and Integrity 
Properties}, simply described as the {\color{blue} Theorem 3 and 4} in section 5.2.

%Now we need also guarantee that $\fAP{attacker} \in \mathsf{Net}$ has no other way
%to generate the IDToken. This can be simply proved by the {\color{blue} Theorem 1} 
%in section 5.2, as the {\color{blue} RP Designation Property}.

These proofs are enough for SPRESSO system to show its security, however, they are not enough for UPPRESSO.
So far, the proofs only guarantee that the $IDToken$ must be sent to the target RP.
In SPRESSO, as the $tag$ can be only decrypted to unique $Domain$, the target RP must be the honest RP (the target of an adversary).
However, in UPPRESSO, while an RP receives an $IDToken$, he may try to use this token to login another honest RP, as long as he can find the $t^{adversary}$ satisfied $IDToken.PID_{RP} \equiv [t^{adversary}]ID_{RP}^{honest}$.
Therefore, the following Lemma should be proved.

\begin{lemma}
The $t^{adversary}$ is not derivable from the attackers knowledge in $S^j$ (i.e., $\langle IDToken$, $Acct \rangle \in d_{\emptyset}$($S^j$($\mathtt{attacker}$))), which satisfies that $IDToken.PID_{RP} \equiv [t^{adversary}]ID_{RP}^{honest}$.
\end{lemma}
\begin{proof}
This Lemma can be proved by the {\color{blue} Theorem 1} in section 5.2, as the {\color{blue} RP Designation Property}.
\end{proof}

Therefore, there is a contradication to the assumption, where we assumed that 
$Acct \in d_{\emptyset}(S^j(\fAP{attacker}))$. 
This shows every $\mathcal{U\!W\!S}^{auth}$ is secure in the sense of Property A.

\subsection{Proof of Property B}
As stated above, Property B is defined as follows:
\begin{definition}\label{def:B}
  Let $\uppressoauthwebsystem$ be an \uppresso web system. We say that
  \emph{$\uppressoauthwebsystem$ is secure (with respect to Property B)} if
  for every run $\rho$ of $\uppressoauthwebsystem$, every state $(S^j, E^j, N^j)$
  in $\rho$, every $r\in \fAP{RP}$ that is honest in $S^j$, 
  every RP service token of the form $Acct$ recorded in
  $S^j(r).\str{serviceTokens}$, with the request corresponding to
  $Acct$ sent by some $b\in \fAP{B}$ which is honest in $S^j$, $b$ owns $Acct$.
%Let $\mathcal{U\!W\!S}^{auth}$  be an UPPRESSO web system for authentication analysis. We say that $\mathcal{U\!W\!S}^{auth}$  is secure (with respect to Property A) if for every run $rho$ of $\mathcal{U\!W\!S}^{auth}$ , every state ($S^j$, $E^j$, $N^j$) in $rho$, every $r \in \mathtt{RP}$ that is honest in
%$S^j$, every RP service token of the form $\langle IDToken$, $Acct \rangle$ recorded in $S^j$($r$).$\mathtt{serviceTokens}$, with the request corresponding to $\langle IDToken$, $Acct \rangle$ sent by some $b \in B$ which is honest in $S^j$, b owns Acct.
\end{definition}

First we call the request corresponding to $Acct$ (or service token request) $m$ and
its response $m'$, and we refer to the state of $\uppressoauthwebsystem$ in the run 
$\rho$ where $r$ processes $m$ by $s_l$. We are going to prove the $IDToken$ uploaded 
by honest $b$ can only be related with the $Acct$ owned by $b$.

%we follows the Lemma 10 and its proof in SPRESSO, which guarantees that the request corresponding to $\langle IDToken$, $Acct \rangle$ sent by honest $b$ is loaded from $script\_rp$. 
%Then we are going to prove the $IDToken$ uploaded by honest $b$ can only be related with the $Acct$ owned by $b$ (which is quite different from SPRESSO).

\begin{lemma}
For every $IDToken$ uploaded by honest $b$ during authentication, the honest $r \in RP$ can always derive the service token of the form $\langle IDToken$, $Acct \rangle$ recorded in $S^j$($r$).$\mathtt{serviceTokens}$, where b owns Acct. 
\end{lemma}
\begin{proof}
  Following the definiton of browser scripts, we know that $m$ was sent by $script\_rp$. 
  The RP accepts the user's identity at line~\ref{line:add-service-token} in Algorithm~\ref{alg:rp}.
  And the identity is generated at Line~\ref{line:gen-acct}, based on the $PID_u$ retrieved 
  from the IDToken and the trapdoor $t^{-1}$. The $t^{-1}$ is generated and set at 
  Line~\ref{line:gen-t} which is never changed. 
  The IDToken is issued at Line~\ref{line:sign-token} in Algorithm~\ref{alg:idp}.
  The IdP generates the $PID_u$ based on the $PID_{rp}$ and $ID_u$ related to $b \in \fAP{B}$.

  An attacker may allure the honest user to upload the $\text{IDToken} \in d_{\emptyset}$($S^j$($\mathtt{attacker}$)) 
  to honest $r\in\fAP{RP}$, so that there may be $Acct \in d_{\emptyset}$($S^j$($\mathtt{attacker}$)).
  However, while $b$ has already negotiated the $PID_{rp}$ with $r$, the opener of 
  the $script\_idp$ must be the $script\_rp$. As the $t$ generated at Line~\ref{line:gen-t}, 
  Algorithm~\ref{alg:uppresso-script-idp}, and $PID_{RP}$ generated at Line~\ref{line:gen-pidrp} 
  in Algorithm~\ref{alg:uppresso-script-idp}.
  The $t$ is only sent to $script\_rp$ at Line~\ref{line:send-t} in 
  Algorithm~\ref{alg:uppresso-script-idp}, 
  and the $script\_rp$ receives it at Line~\ref{line:receive-t} in 
  Algorithm~\ref{alg:uppresso-script-rp}.
  The $PID_{RP}$ is sent to the honest IdP at Lines~\ref{line:send-pidrp} in 
  Algorithm~\ref{alg:uppresso-script-idp},  which is used for generating the $IDToken$.

  For every IDToken sent by honest $b$ and honest $r$, there must be 
  $IDToken.PID_{rp} \equiv [t]Cert_{rp}.ID_{rp}$, 
  $IDToken.PID_u \equiv [ID_u]IDToken.PID_u$ and 
  $Acct \equiv [t^{-1}]IDToken.PID_u$. 
  According to the proof of {\color{blue}Theorem 2} in section 5.2, 
  the $Acct$ must be owned by honest $b$ ($Acct \equiv [ID_U]S^j(r).ID_{RP}$, 
  where $ID_U$ owned by $b$), which can be define as the  
  {\color{blue} User Identification Property}.
\end{proof}

With the above proofs, we now can guarantee that every $\uppressoauthwebsystem$ system satisfies the requirements in Definition~\ref{def:B}, therefore $\uppressoauthwebsystem$ must be secure of Property B.

\section{Proof of Privacy against IdP-based Login Tracing}

In our privacy analysis, we show that an identity provider in UPPRESSO cannot learn 
where its users log in. We formalize this property as an indistinguishability 
property: an identity provider (modeled as a web attacker) cannot distinguish 
between a user logging in at one relying party and the same user logging in at 
a different relying party.

We will here first describe the precise model that we use for privacy.
After that, we define an equivalence relation between configurations,
which we will then use in the proof of privacy.

\subsection{Formal Model of UPPRESSO for Privacy Analysis}

\begin{definition}[Challenge Browser]
  Let $\mi{dr}$ some domain and $b(\mi{dr})$ a DY process. We call $b(\mi{dr})$ a \emph{challenge browser} iff $b$
  is defined exactly the same as a browser with two exceptions: (1) the
  state contains one more property, namely $\mi{challenge}$, which  initially 
  contains the term $\top$. The broswer's algorithm is extended by the following 
  at its very beginning: It is checked if a message $m$ is addressed to the domain
  $\str{CHALLENGE}$ (which we call the challenger domain). If $m$ is
  addressed to this domain and no other message $m'$ was addressed to
  this domain before (i.e., $\mi{challenge} \not\equiv \bot$), then
  $m$ is changed to be addressed to the domain $\mi{dr}$ and $\mi{challenge}$ is 
  set to $\bot$ to recorded that a message was addressed to $\str{CHALLENGE}$.
\end{definition}

\begin{definition}[Deterministic DY Process]
  We call a DY process $p = (I^p,Z^p,R^p,s_0^p)$ \emph{deterministic} iff 
  the relation $R^p$ is a (partial) function.

  We call a script $R_\text{script}$ \emph{deterministic} iff the relation 
  $R_\text{script}$ is a (partial) function.
\end{definition}

\begin{definition}[\uppresso Web System for Privacy Analysis]\label{def:uppresso-ws-priv}
  Let $\uppressowebsystem = (\bidsystem, \scriptset, \mathsf{script}, E^0)$ 
  be an UPPRESSO web system with 
  $\bidsystem = \mathsf{Hon} \cup \mathsf{Web} \cup \mathsf{Net}$, 
  $\mathsf{Hon} = \fAP{B} \cup \fAP{RP} \cup \fAP{IDP} \cup \fAP{DNS}$.
  (as described in Appendix~\ref{app:outlineuppressomodel}).
  $\fAP{RP} = \{r_1,r_2\}$, $r_1$ and $r_2$ two (honest) relying parties,
  $\fAP{dns}$ an honest DNS server. 
  Let $\fAP{attacker} \in \mathsf{Web}$ be some web attacker.
  Let $\mi{dr}$ be a domain of $r_1$ or $r_2$ and $b(\mi{dr})$ a challenge browser. 
  Let $\mathsf{Hon}' := \{ b(\mi{dr}) \} \cup \fAP{RP} \cup \fAP{DNS}$, $\mathsf{Web}' := \mathsf{Web}$, 
  and $\mathsf{Net}' := \emptyset$ (i.e., there is no network attacker).
  Let $\bidsystem' := \mathsf{Hon}' \cup \mathsf{Web}' \cup \mathsf{Net}'$.  
  Let $\scriptset' := \scriptset \setminus \{\str{script\_idp}\}$ and $\mathsf{script}'$ be accordingly.
  We call $\uppressoprivwebsystem(\mi{dr}) = (\bidsystem', \scriptset', \mathsf{script}', E^0, \fAP{attacker})$ 
  an \emph{\uppresso web system for privacy analysis} 
  iff the domain $\mi{dr}_1$ the only domain assigned to $r_1$, and
  $\mi{dr}_2$ the only domain assigned to $r_2$. The browser
  $b(\mi{dr})$ owns exactly one identity and this identity
  is governed by some attacker.  All honest parties (in
  $\mathsf{Hon}$) are not corruptible, i.e., they ignore any
  $\str{CORRUPT}$ message. Identity providers are assumed to be
  dishonest, and hence, are subsumed by the web attackers (which
  govern all identities). %In the initial state $s_0^b$ of the (only)
  %browser in $\bidsystem'$ and in the initial states $s_0^{r_1}$,
  %$s_0^{r_2}$ of both relying parties, the DNS address is
  %$\mapAddresstoAP(\fAP{dns})$. Further, $\mi{wkCache}$ in the initial
  %states $s_0^{r_1}$, $s_0^{r_2}$ is equal and contains a public key
  %for each domain registered in the DNS server (i.e., 
  the relying
  parties already know some public key to verify \uppresso identity
  assertions from all domains known in the system\gs{$\nf$ genauer
  formulieren} and they do not have to fetch them from IdP.
\end{definition}

As all parties in an \uppresso web system for privacy analysis are either web 
attackers, browsers, or deterministic processes and all scripting processes are 
either the attacker script or deterministic, it is easy to see that in \uppresso 
web systems for privacy analysis with configuration $(S,E,N)$ a command $\zeta$ 
induces at most one processing step. We further note that, under a given infinite 
sequence of nonces $N^0$, all schedules $\sigma$ induce at most one run 
$\rho = ((S^0,E^0,N^0),\dots,(S^i,E^i,N^i),\dots,(S^{|\sigma|},E^{|\sigma|},N^{|\sigma|}))$ 
as all of its commands induce at most one processing step for the $i$-th configuration.

We will now define our privacy property for \uppresso:

\begin{definition}[IdP-Privacy]\label{def:idp-privacy}
  Let 
  \begin{align*}
    \uppressoprivwebsystem_1 := \uppressoprivwebsystem(\mi{dr}_1) =
    (\bidsystem_1, \scriptset, \mathsf{script}, E^0, \fAP{attacker}_1)&\text{ and}\\
    \uppressoprivwebsystem_2 := \uppressoprivwebsystem(\mi{dr}_2) =
    (\bidsystem_2, \scriptset, \mathsf{script}, E^0, \fAP{attacker}_2)&
  \end{align*}
  be \uppresso web systems for privacy analysis.  Further, we require
  $\fAP{attacker}_1 = \fAP{attacker}_2 =: \fAP{attacker}$ and for $b_1
  := b(\mi{dr}_1)$, $b_2 := b(\mi{dr}_2)$ we require $S(b_1) = S(b_2)$
  and $\bidsystem_1 \setminus \{b_1\} = \bidsystem_2 \setminus
  \{b_2\}$ (i.e., the web systems are the same up to the parameter of
  the challenge browsers).  We say that $\uppressoprivwebsystem$ is
  \emph{IdP-private} iff $\uppressoprivwebsystem_1$ and
  $\uppressoprivwebsystem_2$ are indistinguishable.
\end{definition}

\subsection{Definition of Equivalent Configurations}\label{app:defin-equiv-stat}

Let $\uppressoprivwebsystem_1 = (\bidsystem_1, \scriptset, \mathsf{script}, E^0, \fAP{attacker})$ 
and $\uppressoprivwebsystem_2 = (\bidsystem_2, \scriptset, \mathsf{script}, E^0, \fAP{attacker})$ 
be \uppresso web systems for privacy analysis. Let $(S_1,E_1,N_1)$ 
be a configuration of $\uppressoprivwebsystem_1$ and $(S_2,E_2,N_2)$ 
be a configuration of $\uppressoprivwebsystem_2$.

\begin{definition}[Proto-Tags]
  We call a term of the form $[t]R$ with the variable
  $R$ as a placeholder for an $ID_{rp}$, and $t$ some nonces a
  \emph{proto-tag}.
\end{definition}

\begin{definition}[Term Equivalence up to Proto-Tags]
  Let $\theta = \{a_1, \ldots, a_l \}$ be a finite set of proto-tags.
  Let $t$ and $t'$ be terms. We call $t_1$ and $t_2$
  \emph{term-equivalent under a set of proto-tags $\theta$} iff there
  exists a term $\tau \in \terms(\{x_1,\dots,x_l\})$ such that
  $t_1 = (\tau [ a_1 / x_1 , \dots , a_l / x_l ])[ ID_{\mi{dr}_1} / R ]$ and
  $t_2 = (\tau [ a_1 / x_1 , \dots , a_l / x_l ])[ ID_{\mi{dr}_2} / R ]$.
  We write $t_1 \prototagequiv{\theta} t_2$.

  We say that two finite sets of terms $D$ and $D'$ are
  \emph{term-equivalent under a set of proto-tags $\theta$} iff
  $|D| = |D'|$ and, given a lexicographic ordering of the elements in
  $D$ of the form $(d_1,\dots,d_{|D|})$ and the elements in $D'$ of
  the form $(d'_1,\dots,d_{|D'|})$, we have that for all
  $i \in \{1,\dots,|D|\}$: $d_i \prototagequiv{\theta} d'_i$. We then
  write $D \prototagequiv{\theta} D'$.
\end{definition}

\begin{definition}[Equivalence of HTTP Requests]
  Let $m_1$ and $m_2$ be (potentially encrypted) HTTP requests and
  $\theta = \{a_1, \ldots, a_l \}$ be a finite set of proto-tags. 
  We call $m_1$ and $m_2$ \emph{$\delta$-equivalent under a set of proto-tags $\theta$} 
  iff $m_1 \prototagequiv{\theta} m_2$ or all subterms are equal with the following exceptions:
  \begin{enumerate}
  \item the Host value and the Origin/Referer headers in both requests
    are the same except that the domain $\mi{dr}_1$ in $m_1$ can be
    replaced by $\mi{dr}_2$ in $m_2$,
  \item the HTTP body $g_1$ of $m_1$ and the HTTP body $g_2$ of $m_2$
    are (I) term-equivalent under $\theta$, (II) for $j\in \{1,2\}$ if
    $g_j[\str{IDToken}] \sim \myangle{PID_{dr_j}, [*]PID_{dr_j}, 
    \sig{\myangle{PID_{dr_j}, [*]PID_{dr_j}}}{*}}$
    and the origin (HTTP header) of HTTP message in $m_j$ is
    $\an{\mi{dr}_j,\https}$ then the receiver of this message is
    $r_j$, and (III) if $g_1$ contains a dictionary key
    $\str{loginSessionToken}$ then there exists an $l' \in L$ such
    that $g_1[\str{loginSessionToken}] \equiv l'$, and
  \item if $m_1$ is an encrypted HTTP request then and only then $m_2$
    is an encrypted HTTP request and the keys used to encrypt the
    requests have to be the correct keys for $\mi{dr}_1$ and
    $\mi{dr}_2$ respectively.
  \end{enumerate}
  We write $m_1 \httptagequiv{\theta} m_2$.
\end{definition}

\begin{definition}[Extracting Entries from Login Sessions]
  Let $t_1$, $t_2$ be dictionaries over $\nonces$ and $\terms$,
  $\theta$ be a finite set of proto-tags, and $d$ a domain. We call
  $t_1$ and $t_2$ \emph{$\eta$-equivalent} iff $t_2$ can be
  constructed from $t_1$ as follows: For every proto-tag
  $a \in \theta$, we remove the entry identified by the dictionary key
  $i$ for which it holds that $\proj{4}{t_1[i]} \equiv a[ID_r/ R]$, if
  any. We denote the set of removed entries by $D$. We write
  $\logsessminus{t_1}{t_2}{\theta}{r}{D}$.
\end{definition}

\begin{definition}
  Let $a$ be a proto-tag, $S_1$ and $S_2$ be states of \uppresso web
  systems for privacy analysis, and $l$ a nonce. We call $l$ a login
  session token for the proto-tag $a$, written
  $l \in \mathsf{loginSessionTokens}(a,S_1,S_2)$ iff for any
  $i \in \{1,2\}$ and any $j \in \{1,2\}$ we have that
  $\proj{4}{S_i(r_j).\str{loginSessions}[l]} = a[ID_{dr_j}/ R]$.
\end{definition}

\begin{definition}[Equivalence of States]\label{def:eq-of-states}
  Let $\theta$ be a set of proto-tags and $H$ be a set of nonces. 
  Let $T:=\{t\mid [t]R\in \theta\}$. 
  We call $S_1$ and $S_2$ \emph{$\gamma$-equivalent under $(\theta, H)$} 
  iff the following conditions are met:
  \begin{enumerate}
  \item\label{eqs:dns} $S_1(\fAP{dns}) = S_2(\fAP{dns})$, and
  \item\label{eqs:r1} $S_1(\fAP{r_1})$ equals $S_2(\fAP{r_1})$ except
    for the subterms $\str{loginSessions}$ and $\str{serviceTokens}$, and
  \item\label{eqs:r2} $S_1(\fAP{r_2})$ equals $S_2(\fAP{r_2})$ except
    for the subterms $\str{loginSessions}$ and $\str{serviceTokens}$, and
  \item\label{eqs:logsess} for two sets of terms $D$ and $D'$:
    $\logsessminus{S_1(\fAP{r_1}).\str{loginSessions}}{S_2(\fAP{r_1}).\str{loginSessions}}{\theta}{\mi{dr}_1}{D}$,
    $\logsessminus{S_2(\fAP{r_2}).\str{loginSessions}}{S_1(\fAP{r_2}).\str{loginSessions}}{\theta}{\mi{dr}_2}{D'}$,
    and $D \prototagequiv{\theta} D'$, and
  \item\label{eqs:att-not-t} $\forall t \in T$:
    $t \not\in d_\emptyset(\bigcup_{i \in \{1,2\},\ A\, \in\, \mathsf{Web}\, \cup \,
    \mathsf{Net}\, 
    %\cup\, \{\mathsf{dns}, \mathsf{fwd}
    \}}S_i(A))$
  \item\label{eqs:att} for each attacker $A$:
    $S_1(A) \prototagequiv{\theta} S_2(A)$, and
  \item\label{eqs:att-not-l} for all $a\in\theta$ and all attackers $A$ we have that
    $\nexists\ l \in \mathsf{loginSessionTokens}(a,S_1,S_2)$ such that
    $l$ is a subterm of $S_1(A)$ or $S_2(A)$.
  \item\label{eqs:b} $S_1(b_1)$ equals $S_2(b_2)$ except for for the
    subterms $\str{challenge}$, $\str{windows}$
    %, $\str{pendingDNS}$, $\str{pendingRequests}$ 
    and we have that
    \begin{enumerate}
    \item \label{eqs:b:challenge}
      $S_1(b_1).\str{challenge} = \mi{dr}_1 \wedge
      S_2(b_2).\str{challenge} = \mi{dr}_2$
      or $S_1(b_1).\str{challenge} = S_2(b_2).\str{challenge} = \bot$,
      and
    \item $S_1(b_1).\str{windows}$ equals $S_2(b_2).\str{windows}$ with
      the exception of the subterms $\str{location}$, $\str{referrer}$,
      $\str{scriptstate}$, and $\str{scriptinputs}$ of some document terms
      pointed to by $\mathsf{Docs}^+(S_1(b_1)) = \mathsf{Docs}^+(S_2(b_2)) =: J$. 
      For all $j \in J$ we have that: \label{eqs:b:w}
      \begin{enumerate}
      \item there is no $t \in T$ such that
        \begin{align*}
          t \in d_{\nonces \setminus \{t\}}(\{&S_1(b_1).j.\str{location}
          ,  S_2(b_2).j.\str{location},\\ & S_1(b_1).j.\str{referrer} , 
          S_2(b_2).j.\str{referrer}\})
        \end{align*}
      \item\label{eqs:b:w:script_rp} if
        $S_1(b_1).j.\str{origin} \in \{\an{\mi{dr}_1, \https},\an{\mi{dr}_2, \https}\}$
        then $S_1(b_1).j.\str{script} \equiv \str{script\_rp}$ and \
        \begin{enumerate}
        \item $S_1(b_1).j.\str{location}$ and $S_2(b_2).j.\str{location}$
          are term-equivalent under $\theta$ except for the host part,
          which is either equal or $\mi{dr}_1$ in $b_1$ and $\mi{dr}_2$ in
          $b_2$, and
        \item $S_1(b_1).j.\str{referrer}$ and $S_2(b_2).j.\str{referrer}$
          are term-equivalent under $\theta$ except for the host part,
          which is either equal or $\mi{dr}_1$ in $b_1$ and $\mi{dr}_2$ in
          $b_2$, and
        \item
          $S_1(b_1).j.\str{scriptstate} \prototagequiv{\theta}
          S_2(b_2).j.\str{scriptstate}$ and if $\exists\, l \in L$ such that $l$ is a subterm of $S_1(b_1).j.\str{scriptstate}$, then $S_1(b_1).j.\str{location}.\str{host} \equiv \mi{dr}_1$ and $S_2(b_2).j.\str{location}.\str{host} \equiv \mi{dr}_2$, and
        \item if $\exists\, l \in L$ such that $l$ is a subterm of
          $S_1(b_1).j.\str{scriptinputs}$, then
          $S_1(b_1).j.\str{location}.\str{host} \equiv \mi{dr}_1$ and
          $S_2(b_2).j.\str{location}.\str{host} \equiv \mi{dr}_2$, and
        \end{enumerate}
      \item\label{eqs:b:w:att_script} if
        $S_1(b_1).j.\str{origin} \not\in
        \{\an{\mi{dr}_1,\https},\an{\mi{dr}_2,\https}\}$
        then $S_1(b_1).j.\str{script} \equiv \str{script\_idp}$ and \
        \begin{enumerate}
        \item
          $S_1(b_1).j.\str{location} \prototagequiv{\theta}
          S_2(b_2).j.\str{location}$, and
        \item
          $S_1(b_1).j.\str{referrer} \prototagequiv{\theta}
          S_2(b_2).j.\str{referrer}$, and
        \item
          $S_1(b_1).j.\str{scriptstate} \prototagequiv{\theta}
          S_2(b_2).j.\str{scriptstate}$, and
        \item
          $S_1(b_1).j.\str{scriptinputs} \prototagequiv{\theta}
          S_2(b_2).j.\str{scriptinputs}$, and
        \item 
          $\forall t \in T$: $t$ is not contained in any subterm of 
          $S_1(b_1).j.\str{scriptstate}$ except for 
          $S_1(b_1).j.\str{scriptstate}.\mi{parameters}[\str{t}]$, and
        \item $\nexists\, l \in L$ such that $l$ is a subterm of
          $S_1(b_1).j.\str{scriptstate}$ or of
          $S_1(b_1).j.\str{scriptinputs}$, and
        \end{enumerate}
      \end{enumerate}
    \item\label{eqs:b:misc} for
      $x \in \{\str{cookies},\str{localStorage},\str{sessionStorage},\str{sts}\}$
      we have that $S_1(b_1).x \prototagequiv{\theta} S_2(b_2).x$. For the
      domains $\mi{dr}_1$ and $\mi{dr}_2$ there are no entries in the
      subterms $x$.
    \end{enumerate}
  \end{enumerate}
\end{definition}

\begin{definition}[Equivalence of Events]
  \label{def:Events}
  Same as Definition 80 in SPRESSO except that the forth condition in Definition 80 in SPRESSO is not applicable.
\end{definition}

\begin{definition}[Equivalence of Configurations]
  We call $(S_1,E_1,N_1)$ and $(S_2,E_2,N_2)$
  \emph{$\alpha$-equivalent} iff there exists a set of proto-tags
  $\theta$ and a set of nonces $H$ such that $S_1$ and $S_2$ are
  $\gamma$-equivalent under $(\theta,H)$, $E_1$ and $E_2$ are
  $\beta$-equivalent under $(\theta,L,H)$ for
  $L := \bigcup_{a\in\theta} \mathsf{loginSessionTokens}(a,S_1,S_2)$,
  and $N_1 = N_2$.
\end{definition}

\subsection{Privacy Proof}

\begin{theorem} \label{theorem:A}Every UPPRESSO web system for privacy analysis is IdP-private.
\end{theorem}

Let $\mathcal{U\!W\!S}^{priv}$ be UPPRESSO web system for privacy analysis.\par
To prove Theorem \ref{theorem:A}, we have to show that the UPPRESSO web systems $\mathcal{U\!W\!S}^{priv}_1$ and $\mathcal{U\!W\!S}^{priv}_2$ 
are indistinguishable. To show the indistinguishability of $\mathcal{U\!W\!S}^{priv}_1$ and $\mathcal{U\!W\!S}^{priv}_2$, 
we show that they are indistinguishable under all schedules $\sigma$.
For this , we first note that for all $\sigma$, there is only one run induced by each $\sigma$(as our web system, when scheduled, is deterministic).
We now proceed to show that for all schedules $\sigma=(\zeta _1, \zeta_2,\dots)$, iff $\sigma$ induces a run $\sigma(\mathcal{U\!W\!S}^{priv}_1)$ there exists a run $\sigma(\mathcal{U\!W\!S}^{priv}_2)$ such that $\sigma(\mathcal{U\!W\!S}^{priv}_1)\approx\sigma(\mathcal{U\!W\!S}^{priv}_1)$\par
We now show that if two configurations are $\alpha$-equivalent, then the view of the attacker is statically equivalent.

\begin{lemma}
  (Same as Lemma 12 in SPRESSO) Let $(S_1,E_1,N_1)$ and $(S_2,E_2,N_2)$ be two $\alpha$-equivalent configurations. Then $S_1(attacker)\approx S_2(attacker)$.
\end{lemma}

\begin{lemma}
  (Same as Lemma 13 in SPRESSO) The initial configurations $(S_1^0,E^0,N^0)$ of $\mathcal{U\!W\!S}^{priv}_1$ and $(S_2^0,E^0,N^0)$ of $\mathcal{U\!W\!S}^{priv}_2$ are $\alpha$-equivalent.
\end{lemma}
\begin{proof}
  Let $\theta=H=L=\emptyset$.Obviously, both latter conditions are true. For all parties $p\in\mathcal{W}_1\setminus\{b_1\}$, it is clear that $S^0_1(p)=S^0_2(p)$.
  Also the states $S^0_1(b_1)=S^0_2(b_2)$ are equal. Therefore, all conditions of Definition \ref{def:C} are fulfilled. Hence, the  initial configurations are $\alpha$-equivalent.
\end{proof}

\begin{lemma}
  (Same as Lemma 14 in SPRESSO) Let $(S_1,E_1,N_1)$ and $(S_2,E_2,N_2)$ be two $\alpha$-equivalent configurations of $\mathcal{U\!W\!S}^{priv}_1$ and $\mathcal{U\!W\!S}^{priv}_2$, respectively.
  Let $\zeta\!=\!\left \langle ci,cp,\tau_{process},cmd_{switch},cmd_{window},\tau_{script},url\right \rangle $ be a web system command. Then, $\zeta$ induces a processing step in either both configurations or in none. In the former case, let $(S_1\prime,E_1\prime,N_1\prime)$ and $(S_2\prime,E_2\prime,N_2\prime)$ be configurations induced by $\zeta$ such that
  \begin{equation}
    \begin{aligned}
      (S_1,E_1,N_1)\xrightarrow{\zeta}(S_1\prime,E_1\prime,N_1\prime)\text{and}(S_2,E_2,N_2)\xrightarrow{\zeta}(S_2\prime,E_2\prime,N_2\prime)
    \end{aligned}
  \end{equation}
  Then $(S_1\prime,E_1\prime,N_1\prime)$ and $(S_2\prime,E_2\prime,N_2\prime)$ are $\alpha$-equivalent.
\end{lemma}
\begin{proof}
  Let $\theta$ be a set of proto-tags and $H$ be a set of nonces for which $\alpha$-equivalence holds and let $L:=\bigcup_{a\in\theta}\text{loginSessionTokens}(a,S_1,S_2)$,$K:=\{k|\exists n:enc_s(\left \langle y,n\right \rangle,k)\in\theta\}$\par
  To induce a processing step, the ci-th message from $E_1$ or $E_2$, respectively, is selected.Following Definition \ref{def:Events}, we denote these messages by $e_i^{(1)}$ or $e_i^{(2)}$, respectively. We now differentiate between the receivers of the messages by denoting the induced processing steps by
  \begin{equation}
    \begin{aligned}
      (S_1,E_1,N_1)\xrightarrow[p_1\rightarrow E_{out}^{(1)}]{\left \langle a_1,f_1,m_1\right \rangle\rightarrow p_1}(S_1\prime,E_1\prime,N_1\prime)\\
      (S_2,E_2,N_2)\xrightarrow[p_2\rightarrow E_{out}^{(2)}]{\left \langle a_2,f_2,m_2\right \rangle\rightarrow p_2}(S_2\prime,E_2\prime,N_2\prime)
    \end{aligned}
  \end{equation}
  \underline{Case $p_1=dns$:}
  In this case, only Cases 1a, 1b and 1c of Definition 80 can apply. Hence, $p_2=dns$.\par
  (*):As both events are static except for IP addresses, the HTTP nonce, and the HTTPS key, there is no k contained in the input messages(except potentially in tags, from where it cannot be extracted), and the output messages are sent to $f_1$ or $f_2$, respectively, they can not cantian any $l\in L$ or $k\in K$. Hence, Condition 2 of Definition 80 holds true.\par
  We note that (*) so-called Condition 2 applies analogously in cases 1a, 1b and 1c.
  In the case 1a, it is easy to see that $E_{out}^{(1)}\rightleftharpoons_\theta E_{out}^{(2)}$.In the case 1c, it is easy to that the DNS server only outputs empty events in both processing steps. In the case 1b, $E_{out}^{(1)}$ and $E_{out}^{(2)}$ are such that Case 1d of Definition 80 applies.\par
  Therefore, $E_1\prime$ and $E_2\prime$ are $\beta$-equivalent under $(\theta,H,L)$ in all three cases. As there are no changes to any state in all cases, we have that $S_1\prime$ and $S_2\prime$ are $\gamma$-equivalent under $(\theta,H)$. No new nonces are chosen, hence $N_1\prime=N_1=N_2=N_2\prime$.\\
  \underline{Case $p_1=r_1$:}
  In this case, we only distinct several cases of HTTP(S) requests that can happen. The others are ignored the same as SPRESSO.\par
  There are four possible types of HTTP requests that are accepted by $r_1$ in Algorithm \ref{alg:rp}:
  \begin{itemize}
    \item path=/script(get the rp-script), Line 3;
    \item path=/loginSSO(start a login), Line 6;
    \item path=/startNegotiation(derive a $PID_rp$), Line 9;
    \item path=/uploadToken(verify ID token, calculate Acct), Line 18.
  \end{itemize}
  \par From the cases in Definition \ref{def:Events}, only two can possibly apply here:Case 1a and Case 1e. For both cases, we will now analyze each of the HTTP requests listed above separately.\\
  Definition \ref{def:Events},Case 1a:$e_i^{(1)}\rightleftharpoons e_i^{(2)}$. This case implies $p_2=r_1=p_1$. As we see below, for the output events $E_{out}^{(1)}$ and $E_{out}^{(2)}$ (if any) only Case 1a of Definition \ref{def:Events} applies. This implies the nonce of both the incoming HTTP requests and HTTP responses cannot be in $H$.
  \begin{itemize}
    \item path=/script In this case, the same output event is produced whose message is 
    \begin{equation}
      \begin{aligned}
        \left\langle HTTPResp,n,200,\left\langle\right\rangle,RPScript\right\rangle
      \end{aligned}
    \end{equation}
    We can note that Condition 5 of Definition \ref{def:Events} holds true and, also, (*) applies.The remaining conditions are trivially fulfilled and $E_1\prime$ and $E_2\prime$ are $\beta$-equivalent under $(\theta,H,L)$.As there are no changes to any state, we have that $S_1\prime$ and $S_2\prime$ are $\gamma$-equivalent under $(\theta,H)$. No new nonces are chosen, hence $N_1\prime=N_1=N_2=N_2\prime$.
    \item path=/loginSSO In this case, the reason for equivalence holding is similar to the case above since the same output event is produced.
    \item path=/startNegotiation(derive a $PID_rp$), Line 9;
    \item path=/uploadToken(verify ID token, calculate Acct), Line 18.
  \end{itemize}
\end{proof}

\section{Proof of Privacy against RP-based Identity Linkage}

\subsection{Formal Model of UPPRESSO for Privacy Analysis}

\begin{definition}[Challenge IdP]
  Let $\mi{dr}$ some domain and 
  $idp(\myangle{\mi{dr}_1, \mi{dr}_2, u})$ a DY process. 
  We call it a \emph{challenge IdP} iff $b$
  is defined exactly the same as a identity server with two 
  exceptions: (1) the state contains one more property, 
  namely $\mi{challenge}$, which initially contains the term 
  $\top$. 
  (2) The IdP's algorithm is modified by the following at 
  line~\ref{line:uppresso-idp-set-pidu} in 
  algorithm~\ref{alg:idp}: It is checked if the login 
  request $m$ is addressed to the domain $\mi{dr}_1$ 
  If $m$ is addressed to this domain, then the $PID_u$ is 
  generated using the given $u$.
  It is also checked if the login request $m$ is addressed 
  to the domain $\mi{dr}_2$ and no other message $m'$ was 
  addressed to this domain before (i.e., $\mi{challenge} 
  \not\equiv \bot$), then the $PID_u$ is generated using 
  the given $u$ and $\mi{challenge}$ is set to $\bot$ to 
  recorded that a message was addressed to $\mi{dr}_2$.
\end{definition}

\begin{definition}[\uppresso Web System for Privacy Analysis]\label{def:uppresso-ws-priv}
  Let $\uppressowebsystem = (\bidsystem, \scriptset, 
  \mathsf{script}, E^0)$ be an UPPRESSO web system with 
  $\bidsystem = \mathsf{Hon} \cup \mathsf{Web} \cup 
  \mathsf{Net}$, 
  $\mathsf{Hon} = \fAP{B} \cup \fAP{RP} \cup \fAP{IDP}$.
  (as described in Appendix~\ref{app:outlineuppressomodel}).
  $\fAP{B} = \{b_1,b_2\}$, $b_1$ is honest and $b_2$ is 
  malicious and they both own some identities. 
  $\fAP{RP} = \{r_1,r_2\}$, $r_1$ and $r_2$ two (malicious) 
  relying parties, 
  Let $\fAP{attacker} = \{b_2\} \cup \{r_1,r_2\}$ be some 
  web attacker.
  Let $\mi{dr}_1$ be the domain of $r_1$, $\mi{dr}_2$ be the 
  domain of $r_2$ and $u_{idp}$ be an identity owned only by
  IdP, then $idp_c = idp(\myangle{\mi{dr}_1, \mi{dr}_2, 
  u_{idp}})$ is a challenge IdP. 
  Let $\mathsf{Hon}' := \fAP{B} \cup \{idp_c\}$, 
  $\mathsf{Web}' := \mathsf{Web}$, 
  and $\mathsf{Net}' := \emptyset$ 
  (i.e., there is no network attacker).
  Let $\bidsystem' := \mathsf{Hon}' \cup \mathsf{Web}' \cup 
  \mathsf{Net}'$.  
  Let $\scriptset' := \scriptset \setminus 
  \{\str{script\_rp}\}$ and $\mathsf{script}'$ be accordingly.
  We call $\uppressoprivwebsystem(\mi{dr}_1, \mi{dr}_2, 
  u_{idp}) = (\bidsystem', \scriptset', \mathsf{script}', 
  E^0, \fAP{attacker})$ an \emph{\uppresso web system for 
  privacy analysis} 
  iff the domain $\mi{dr}_1$ the only domain assigned to 
  $r_1$, and $\mi{dr}_2$ the only domain assigned to $r_2$. 
  All honest parties (in $\mathsf{Hon}$) are not corruptible, 
  i.e., they ignore any $\str{CORRUPT}$ message. Relying 
  Parties and some browsers are assumed to be dishonest, 
  and hence, are subsumed by the web attackers.
\end{definition}

\bibliographystyle{abbrv}
\bibliography{../ref}
\end{document}
