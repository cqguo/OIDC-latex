%usenix 2023ģ��
\documentclass[letterpaper,twocolumn,10pt]{article}
\usepackage{usenix-2020-09}



\usepackage[small]{titlesec}
% to be able to draw some self-contained figs
%\usepackage{tikz}
\usepackage{amsmath}
\usepackage{bm}

% inlined bib file
\usepackage{filecontents}

\usepackage{wasysym}


\usepackage{lipsum,cuted}
\usepackage{float}%%%%�ṩ�������[H]ѡ�����ȡ������
\usepackage{caption}%%�ṩ\captionof����


\pagestyle{plain}


\usepackage{graphicx}
\usepackage{verbatim}
\usepackage{caption}
%

%\usepackage{algpseudocode}
\usepackage{amsmath,amssymb,amsthm}

%\usepackage{graphicx}
%\usepackage{geometry}
\usepackage{subfigure}
\usepackage{url}
\usepackage{multirow}
\usepackage{listings}
\usepackage{cite}
\usepackage{array}
\usepackage{enumerate}
\usepackage{booktabs}
\usepackage{color}
\usepackage{xcolor}
\usepackage{soul}
\usepackage{multicol}
%\usepackage{algcompatible}
%\usepackage[compatible]{algpseudocode}

\newcommand\usso{\textsf{UPPRESSO}}
\newcommand\dy{\textsf{D-Y}}

\definecolor{Blue}{RGB}{0,0,255}
\newcommand{\newc}{\color{Blue}}
\newcommand{\oldc}{\color{black}}

% correct bad hyphenation here
\hyphenation{target-Origin js-rsa-sign op-tical net-works semi-conduc-tor}

\begin{document}
\date{}
\pagenumbering{arabic}
\title{\Large \bf UPPRESSO: Untraceable and Unlinkable Privacy-PREserving\\Single Sign-On Services}

% Chengqian Guo, Jingqiang Lin, Quanwei Cai, Wei Wang, Fengjun Li, Qiongxiao Wang, Jiwu Jing, Bin Zhao

%\author{
%{\rm Chengqian Guo$^{\sharp\nabla}$, \ Jingqiang Lin$^{\ddag}$, \ Quanwei Cai$^{\P}$, \ Wei Wang$^{\nabla}$, \ Fengjun Li$^{\S}$,}\\
%{\rm Qiongxiao Wang$^{\nabla}$, \ Jiwu Jing$^{\diamondsuit}$, \ Bin Zhao$^{\triangle}$}\\
%$\sharp$ Shenyang Aircraft Design \& Research Institute, China\\
%$\ddag$ School of Cyber Security, University of Science and Technology of China\\
%$\P$ Beijing Zitiao Network Technology Co., Ltd, China\\
%$\S$  Department of Electrical Engineering \& Computer Science, the University of Kansas, USA\\
%${\diamondsuit}$ School of Computer Science \& Technology, University of Chinese Academy of Sciences\\
%$\nabla$ Institute of Information Engineering, CAS\ \ \ \ \ \ \ \ \ \ \ \ \ \ \ \ \ \ \ \ \
%$\triangle$ JD.com Silicon Valley R\&D Center, USA}


\maketitle
\begin{abstract}
Single sign-on (SSO) allows a user to maintain only the credential at an identity provider (IdP) to login to multiple relying parties (RPs).
However, SSO introduces privacy threats,
    as ({\em a}) the curious IdP could potentially track a user's visits to all RPs,
    and ({\em b}) colluding RPs could learn a user's online profile by linking identities across these RPs.
This paper presents a privacy-preserving SSO system, called \emph{UPPRESSO}, to protect a user's online profile against a curious IdP and colluding RPs.
We propose an identity-transformation approach.
 In each SSO login instance,
    an \emph{ephemeral pseudo-identity} of an RP (denoted as $PID_{RP}$),
        is negotiated between a user and the RP.
$PID_{RP}$ is sent to the IdP and designated in the identity token, so that the IdP is not aware of the visited RP.
The IdP then uses $PID_{RP}$ to transform the user's permanent identity $ID_U$, into an ephemeral user pseudo-identity $PID_U$, in the token.
On receiving the identity token, %%%% the RP just waits for it.
    the RP transforms $PID_U$ into a \emph{permanent account} of the user, denoted as $Acct$.
Given a user, the account at each RP is unique and different from $ID_U$,
%%%% for different users, the accounts may be the same.
so colluding RPs cannot link a user's identities across RPs.
We have built a prototype system of UPPRESSO on top of MITREid Connect, an open-source SSO system.
The extensive evaluations show that UPPRESSO fulfills the security and privacy requirements of SSO
        %%%% I find that other big-4 conference papers just write "SSO", not SSOs.
 while introducing reasonable overheads.
\end{abstract}

%\begin{IEEEkeywords}
%Single sign-on, security, privacy. %, trace, linkage
%\end{IEEEkeywords}














\section{Introduction}
\label{sec:intro}

%SSO���ص�
%SSO����״
Single sign-on (SSO) systems, such as OpenID Connect~\cite{OpenIDConnect}, OAuth~\cite{rfc6749} and SAML~\cite{SAML}, have been widely deployed as the identity management and authentication infrastructure in the Internet.
SSO enables a website, called the \emph{relying party} (RP), to delegate its user authentication to a trusted third party called the \emph{identity provider} (IdP).
Thus, a user visits multiple RPs with only a single explicit authentication attempt at the IdP.
With the help of SSO, a user no longer needs to remember multiple credentials for different RPs;
 instead, she maintains only the credential for the IdP, which will generate \emph{identity proofs} for her visits to these RPs.
%        Moreover, SSO shifts the burden of user authentication from RPs to IdPs and reduces security risks and costs at RPs.
As a result, SSO has been widely integrated with many application services.
For example,
    we find that 80\% of the Alexa Top-100 websites~\cite{Alexa} support SSO,
 and the analysis on the Alexa Top-1M websites~\cite{GhasemisharifRC18} identifies 6.30\% with the SSO support.
Meanwhile, many email and social network providers (such as Google, Facebook, Twitter, etc.)
    are serving the IdP roles in the Internet. %social identity providers to support social login.


%SSOϵͳ�İ�ȫ���⣬��Ҫ����identity proof�������ԣ������ԣ�����
%�����ԣ�ʹ�ù�����Ϊ�ܱ�������Ϣ��Ϊidentity proof
%�����ԣ���֤��IdP���͸���Ӧ��RP�����Ҵ��������ͨ����user agent�����ܵ�������
%���ԣ�identity proofһ��Ҫ���Ӧ��RPʵ�ְ�
%SPRESSO���impersonation��identity injection�������� Authentication is the most fundamental security property of an SSO system. That is, i) an adversary should not be able to log in to an RP, and hence, use the service of the RP, as an honest user, and ii) an adversary should not be able to log in the browser of an honest user under an adversary��s identity (identity injection).
%��������������飺1.secure authentication �����ܷ�סimpersonation��identity injection�� Ȼ���������кܶ๥��ʹ���������޷��õ���֤������������Ҫ�������������ԭ��1)RP������δ�������Ա��������ݣ�2)δ�󶨣�3)й¶��
%binding ԭ������ ChenPCTKT14 CCS14 Friendcaster(Friendcaster��һ��Facebook�ĵ�����Ӧ�ã��ȹٷ���Ĺ��ܸ���ȫ��) blindly accepting an access token received from a user's device then using this token to exchange for the user's Facebook ID. A malicious application could obtain a legitimate access token from a user, then use this access token to log into Friendcaster as the user. cause they thought Facebook's access tokens were bound to relying parties and checked with every API call: \From Facebook's perspective, the API calls wouldn't appear to be originating from Friendcaster, but the attacker's own app."
%������ԭ������ IEEE S&P2012~\cite{WangCW12}����Google��������we found that the RPs of Google ID SSO often assume that message fields they require Google to sign would always be signed, which turns out to be a serious misunderstanding (Section 4.1). These problems make us believe that a complete answer to our question can only be found by analyzing SSO schemes on real websites.

% Move to section III
%The primary goal of SSO services is to implement secure user authentication~\cite{SPRESSO}, i.e., to ensure that an honest user can always log in to an honest RP under the correct account. To achieve this, an identity proof generated by the IdP should explicitly specify the authenticated user (i.e., \emph{user identification}) and the RP to which the user attempts to log in (i.e., \emph{receiver designation}). The identify proof should be received only by that RP and user but not by any other entities (i.e., \emph{confidentiality}) and verified by the receiver (i.e., \emph{integrity}). However, various attacks exploit vulnerabilities in different SSO systems to break at least one of these requirements~\cite{ChenPCTKT14, FettKS16,WangCW12,ZhouE14,WangZLG16,YangLLZH16,SomorovskyMSKJ12,MohsenS16}, where the adversaries attempt to either impersonate the victim user at an honest RP or %manipulate the victim user's browser to log in to honest RPs under the adversary's identity. %(i.e., identity injection). For example, Friendcaster was found to accept any received identity proof~\cite{ExplicatingSDK,ChenPCTKT14} (i.e., a violation of receiver designation) so that a malicious RP could log in to Friendcaster as the victim user by replaying the identity proof received from the user to Friendcaster~\cite{MohsenS16}. \cite{WangCW12} reported that some RPs of Google ID SSO accepted user attributes that were not tied to the identity proof (i.e., a potential violation of integrity). As a result, a malicious user could insert arbitrary attributes (e.g., an email address) into the identity proof to impersonate another user at the RP.


%SSO �����µ���˽����
%IdP֪���û���¼�ĸ�RP
%RP֮����Ժ�ı֪��ͬһ���û���¼��ЩRP
The adoption of SSO also raises several privacy concerns regarding online user tracking and profiling~\cite{maler2008venn,NIST2017draft}.
User privacy leaks in all existing SSO protocols and implementations.
Taking a widely used SSO protocol, OpenID Connect (OIDC), as an example, we explain its login process and the privacy leakage risk. As shown in Fig.~\ref{fig:OpenID}, upon receiving a user login request (Step 1), the RP constructs an authentication request with its identity and redirects it to the IdP (Step 2).
After authenticating the user, the IdP  generates an identify proof containing the identities of the user and the RP in Step 4, which is forwarded to the RP by the user in Step 5. Finally, the RP verifies the identity proof and allows the user to log in. In such authentication sessions, any curious IdP or multiple collusive RPs can easily break users' privacy as follows.
\begin{itemize}
\item {\em IdP-based login tracing}. The IdP knows the identities of the RP and user in a single authentication session in Step 2 and Step 3, respectively. As a result, a curious IdP could easily discover all the RPs that the victim user attempts to visit and profile her online activities.
\item {\em RP-based identity linkage}. The RP learns user's identity from the identify proof. When the IdP generates identity proofs for a user, if a same user identifier is used in identity proofs generated for different RPs, % or the user's identifiers can be derived from each other,
which is the case of several widely deployed SSO systems~\cite{BrowserID,SPRESSO}, malicious RPs can collude to not only link the user's login activities at different RPs for online tracking but also correlate her attributes across multiple RPs~\cite{maler2008venn}.
\end{itemize}


%google and facebook�ĸ�������
%1. service provider����DNS��֪��������ˣ����Դ����ܶ����⡣���ǻ��Dz�ͬ�ģ�DNS��profile��Ҫ��������two behavior vectors��similarities����IdP�ж�����Ҫ����ΪIdP�ܹ�����two behavior vectors �Ƿ�������ͬ�ڵ㡣

Large IdPs, especially social IdPs such as Google and Facebook, are known to be interested in collecting users' online behavioral information for various purposes (e.g., Screenwise Meter~\cite{googlenews}, Onavo~\cite{Onavo}). By simply serving the IdP role, these companies can easily collect a large amount of data to reconstruct users' online traces.
On the other hand, in today's Internet, many service providers host a variety of web services and therefore take an advantaged position to link a user's multiple logins at different RPs. Through an internal information integration, rich information can be obtained from the SSO data for user profiling. Meanwhile, privacy-preserving record linkage~\cite{agrawal2003information} and private set intersection~\cite{de2010practical} technologies allow multiple organizations (e.g., RPs) to share and link the data without violating their clients' privacy, which has paved the path for cross-organizational RP-based identity linkage.

While the privacy problems in SSO have been widely recognized~\cite{maler2008venn,NIST2017draft}, only a few solutions were proposed to protect user privacy~\cite{persona,SPRESSO}. Among them, Pairwise Pseudonymous Identifier (PPID)~\cite{OpenIDConnect, SAMLIdentifier} is a most straightforward and commonly adopted solution to defend against RP-based identity linkage. It requires the IdP to create different identifiers for the user when she logs in to different RPs, so that the pairwise pseudo-identifiers of the same user cannot be used to link user logins at different RPs no matter they collude or not. As a recommended practice by NIST~\cite{NIST2017draft}, PPID has been specified in many widely adopted SSO standards including OIDC~\cite{OpenIDConnect} and SAML~\cite{SAMLIdentifier}. However, PPID-based approaches cannot prevent the IdP-based login tracing attacks, as the IdP still knows which RP the user visits.

To the best of our knowledge, only two schemes (i.e., BrowserID~\cite{BrowserID} and SPRESSO~\cite{SPRESSO}) have been proposed so far to defend against IdP-based login tracing.
In BrowserID (and its prototype system known as Mozilla Persona~\cite{persona} and Firefox Accounts~\cite{FirefoxAccount}), IdP generates a user certificate to bind the user's unique identifier (i.e., email address) to a public key. With the corresponding private key, that user can bind the receiving RP's identifier to the identity proof %(including the user certificate),
and send it to that RP. In this way, the IdP does not need to know the RP's identity when generating identify proofs.
SPRESSO requires the RP to create a pseudo-identifier at each login for the IdP to generate the identity proof and therefore hides RP's real identity from the IdP. The RP employs a third-party entity (named {\em forwarder}), which works as a proxy to relay the identity proof from the IdP to the corresponding RP. In both schemes, the RPs' identifiers are protected from the IdP, however, the IdP has to know the unique identifier (e.g., email address) of the user and includes it in the identity proofs so that the RP can recognize a same user from her multiple logins. As a result, both schemes are vulnerable to RP-based linkage.

As discussed above, none of the existing SSO systems can defend against both IdP-based login tracing and RP-based identity linkage attacks at the same time. Before presenting our solution, we first formally analyze the privacy problems and solutions in SSO. Let us denote user's and RP's identifiers as $ID_U$ and $ID_{RP}$, respectively, where $ID_U$ is known to the user and IdP and $ID_{RP}$ is known to the user and RP. To protect user's privacy against RP-based identity linkage, $ID_U$ should not be explicitly included in the identity proof. Instead, a privacy-preserving pseudo-identifier $PID_{U}$ should be used (as in the PPID-based approaches~\cite{OpenIDConnect, SAMLIdentifier}), which can be generated by a one-way identifier transformation function $\mathcal{F}_{ID_{U} \mapsto PID_{U}}$ at the IdP. Similarly, to prevent IdP-based login tracing, $ID_{RP}$ should not be explicitly included in the identity proof but replaced by a pseudo-identifier $PID_{RP}$ (as in SPRESSO~\cite{SPRESSO} and BrowserID~\cite{BrowserID}), which is generated by another one-way function $\mathcal{F}_{ID_{RP} \mapsto PID_{RP}}$ at the RP. However, if both $PID_{U}$ and $PID_{RP}$ are used in identity proofs to replace $ID_U$ and $ID_{RP}$ at the same time, assuming they can be securely exchanged between IdP and RP through the user in an SSO session, the RP will authenticate and authorize the user under the account $PID_U$, which will be different in the user's multiple logins at a same RP. As a result, the RP has no clue to know the real account of the user but treats her as a one-time user every time when she logs in. Obviously, this violates the security of SSO services.

In this paper, we propose an Unlinkable Privacy-PREserving Single Sign-On (UPPRESSO) system to provide comprehensive protection against IdP-based login tracing and RP-based identity linkage. The key idea is to design a new one-way identifier transformation function $\mathcal{F}_{PID_{U} \mapsto Account}$, which maps {\em all} $PID_U$s of a user to a unique $Account$ at the RP, where $Account$ is created when the user first registers at the RP. Since $PID_U$ and $PID_{RP}$ are separately generated by IdP and RP respectively at every authentication session, we have to design new $\mathcal{F}_{ID_{U} \mapsto PID_{U}}$ and $\mathcal{F}_{ID_{RP} \mapsto PID_{RP}}$ functions, so that three transformation functions work cooperatively to ensure: (i) when a user logs in to an RP multiple times, the RP can always map $PID_U$s to an unique $Account$ without knowing the user's identity; when a user logs in to multiple RPs, (ii) a curious IdP cannot link multiple $PID_{RP}$s to a particular RP or associate them together, and collusive RPs (iii) cannot link $PID_U$s to a particular user or associate them together, (iv) nor link $Account$s of a same user. To achieve these goals, we design three one-way transformation functions based on the discrete logarithm problem. First, we design a one-way trapdoor function $\mathcal{F}_{ID_{RP} \mapsto PID_{RP}}(ID_{RP}, t)$ for an RP to generate a random $PID_{RP}$ based on a randomly chosen trapdoor $t$, and a one-way function  $\mathcal{F}_{ID_{U} \mapsto PID_{U}}(ID_U, PID_{RP})$ for the IdP to generate $PID_U$ based on $PID_{RP}$. With $ID_{RP}$, $PID_{RP}$, $PID_U$ and the trapdoor $t$, the RP can invoke $\mathcal{F}_{PID_{U} \mapsto Account}(PID_U, PID_{RP}, t)$ to derive the unique $Account$. We summarize our contributions as follows.
%���ǵĹ���
%�����
%�����ܷ����ģ�ͽ��з���
%ʵ��ԭ��ϵͳ
%The main contributions of UPPRESSO are as follows:
\begin{itemize}
\item We formally analyze the privacy problems in SSO as an identifier-transformation problem, and propose the first comprehensive solution to hide users' login traces from both curious IdPs and malicious, collusive RPs. To the best of our knowledge, UPPRESSO is the first SSO system that provides secure SSO services against IdP-based login tracing and RP-based identity linkage.
\item We have implemented a prototype of UPPRESSO based on an open-source implementation of OIDC, which requires only small modifications to support three new transformation functions for privacy protection. Unlike  BrowserID and SPRESSO, UPPRESSO does not require ``non-trivial re-designs'' of SSO implementations, which makes it compatible with existing SSO systems.
\item We systematically analyze  the security of UPPRESSO and show that it achieves the same security level as existing SSO systems.
\item We compare the performance of UPPRESSO prototype with the state-of-the-art SSO systems (i.e., OIDC and SPRESSO), and demonstrate its efficiency.
\end{itemize}

%���½ṹ
The rest of the paper is organized as follows. We first introduce the background and preliminaries in Section~\ref{sec:background}. Then, we describe the identity-transformation based approach, the threat model, and our UPPRESSO design in Sections~\ref{sec:challenge}, \ref{sec:assumptionandthreatmodel} and \ref{sec:UPPRESSO}, followed by a systematical security analysis in Section~\ref{sec:analysis}. We provide the implementation specifics and experiment evaluation in Section~\ref{sec:implementation}, discuss the related works in Section~\ref{sec:related}, and conclude our work in Section~\ref{sec:conclusion}.

\section{Background and Related Works}
\label{sec:background}
%UPPRESSO is designed to be compatible with OpenID Connect (OIDC) and provide privacy protections based on the discrete logarithm problem. Next,

This section introduces OIDC \cite{OpenIDConnect},
 to describe typical SSO login flows.
Then, we discuss existing privacy-preserving SSO solutions and other related works.

\subsection{OpenID Connect and SSO Features}
\label{subsec:OIDC}
OIDC is one of the most popular SSO protocols for web applications. %As other SSO protocols \cite{SAMLIdentifier}, OIDC
%It involves three entities, i.e., {\em users}, the {\em identity provider (IdP)}, and {\em relying parties (RPs)}.
Users and RPs initially register at the IdP with their identities %($ID_U$, $ID_{RP}$ and $PID_U$ in some schemes) %(or $PID_{RP}$ in some schemes)
and other necessary information such as user credentials (i.e., passwords or public keys)
 and RP endpoints (i.e., the URLs to receive identity tokens).
% below can be removed
%The IdP should maintain these attributes securely.

%\vspace{1mm}
%\noindent\textbf{Implicit Login Flow.}
OIDC supports three types of login flows: implicit flow, authorization code flow, and hybrid flow (i.e., a mix-up of the previous two).
%In the implicit flow, an {\em id token} is generated as the identity token, which contains a user identifier, an RP identifier, the issuer (i.e., IdP), the validity period, and other requested attributes. The IdP signs the id token using its private key to ensure integrity, and sends it to the RP through the user.
%In the authorization code flow, the IdP binds an authorization code with the RP, and sends it to the RP through the user; then, the RP establishes an HTTPS connection to the IdP and uses the authorization code with the RP's credential to obtain the user's identifier and other attributes.
%UPPRESSO is compatible with all three flows.
These flows work with different steps to request and receive identity tokens,
    but with the common security requirements of identity tokens.
We introduce the implicit flow and present our design and implementation based on this flow,
    and Section \ref{sec:discussion} presents the extensions to support the authorization code flow.

\begin{figure}[htb]
  \centering
  \includegraphics[width=0.88\linewidth]{fig/OIDC1.pdf}
  \caption{The implicit SSO login flow of OIDC.}
  \label{fig:OpenID}
%  \vspace{-5mm}
\end{figure}

As shown in Figure \ref{fig:OpenID}, a user firstly initiates a login request to an RP.
Then, the RP constructs an identity-token request with its own identity
% the endpoint to receive the identity token % endpointӦ��Ԥ��ע�ᡢ�̶��ģ����ﲻ��ǿ�����
 and the scope of requested user attributes.
This identity-token request is redirected to the IdP.
After authenticating the user,
    the IdP issues an identity token
        which is forwarded by the user to the RP endpoint.
The token contains a user identity (or pseudo-identity),
    the RP identity, a validity period, the requested user attributes, etc.
Finally, the RP verifies the received identity token and
 allows the user to login as the (pseudo-)identity enclosed in this token.

Before issuing the identity token,
    the IdP obtains the user's authorization to enclose the requested user attributes,
    which are maintained at the IdP by the user. The IdP is also a web service.
%The identity token is usually signed by the IdP,
%    and transmitted through secure channels such as TLS/HTTPS.
The user's operations including redirection, authorization, and forwarding,
    are implemented in a software called \emph{user agent} (i.e., a browser for web applications).

%extracts user's identifier and returns the authentication result to the user (Step 7).

In addition to the basic login flows enabling
    an RP to verify that the user has been authenticated by the IdP,
    the following features are supported by widely-used popular SSO solutions.

\noindent \textbf{User Identity at an RP.}
The identity tokens facilitate the target RP to identify each user as a unique account at this RP,
    and this account links the user's multiple login instances to this RP
        for customized information services.
On the contrary, in anonymous SSO systems \cite{WangWS13,HanCSTW18,HanCSTWW20}
        the RP only verifies whether he is a legitimate user authenticated by the IdP
            and receives no information to identify or distinguish the users.

\noindent \textbf{User Authentication Only to the IdP.}
A ``pure'' SSO protocol  \cite{OpenIDConnect,rfc6749,SAML} does not include authentication steps:
    the authentication between a user and the IdP is conducted independently.
It eliminates the authentication step between users and an RP,
        and an RP only verifies tokens issued by the IdP.
This feature
    enables the IdP to authenticate users by any appropriate means (e.g., password,
one-time password, multi-factor authentication, and smart card).
So a user \emph{only} needs to hold the credential to authenticate himself to the IdP,
    and then the user burden is mitigated.
Thus, if a user's computer or device was compromised or lost,
    the user only renews the credential at the IdP.
On the other hand, if a user proves some non-ephemeral secret to the RP and this secret is valid across multiple login instances
    (i.e., some authentication steps are actually involved),
                the user has to notify each RP if this secret was leaked  or lost during its validity period. %(or even the user logins from another computer).

\noindent \textbf{IdP-Confirmed Selective Attribute Provision.}
The IdP usually provides also user attributes in the tokens \cite{OpenIDConnect,rfc6749,SAML},
    in addition to user (pseudo-)identities.
These attributes are maintained by users at the IdP;
    for example,
        when Facebook provides social networking services,
         it also runs as an IdP with identity tokens enclosing user identities and various attributes.
Before enclosing attributes in an identity token for any RP,
    the IdP needs to obtain the user's authorization;
    or the provided attributes are selected by the user.
Thus,
    no distinctive attributes such as such as telephone number, Email, etc.,
        are enclosed in the tokens of privacy-preserving SSO systems.

%\vspace{1mm}
%\noindent\textbf{RP Dynamic Registration.}
%In addition to manual registrations,
%    OIDC also supports dynamic registrations
%    for an RP to register by online means \cite{DynamicRegistration}.
%The (unregistered) RP sends a registration request
%        with endpoints to receive identity tokens (and other information),
%        to the IdP.
%After a successful registration,
% the IdP assigns a unique RP identity in the response.
%

%UPPRESSO leverages this function and slightly modifies the dynamic registration process to implement the {\em $PID_{RP}$ registration} process (see details in Section \ref{sec:UPPRESSO}.C), which allows an RP to generate different privacy-preserving RP identifiers and register them with the IdP.





\begin{table*}[htb]
\footnotesize
    \caption{Privacy-Preserving Solutions of SSO and Federated Identity Management.}
    \centering
    \begin{tabular}{|c|c|c|c|c|c|c|}
  \hline
  \multirow{3}*{\textbf{~~Solution~~}} &
  \multicolumn{3}{c|}{\textbf{SSO Feature} - supported $\CIRCLE$, unsupported $\Circle$, or partially $\LEFTcircle$} & \multicolumn{3}{c|}{\textbf{Privacy Threat} - prevented $\CIRCLE$ or not $\Circle$} \\ \cline{2-7}
  & User Identity & User Authentication & IdP-Confirmed Selective  & IdP-based & RP-based & Collusive Attack \\
  & at an RP & Only to the IdP &  Attribute Provision & Login Tracing & Identity Linkage & by the IdP and RPs \\\hline\hline
  OIDC with PPID & $\CIRCLE$ & $\CIRCLE$ & $\CIRCLE$ & $\Circle$ & $\CIRCLE$ & - \\ \hline
  BrowserID & $\CIRCLE$ & $\LEFTcircle$$^1$ & $\Circle$ & $\CIRCLE$ & $\Circle$ & - \\ \hline
  PRIMA & $\CIRCLE$ & $\Circle$ & $\CIRCLE$ & $\CIRCLE$ & $\Circle$ & - \\ \hline
  SPRESSO & $\CIRCLE$ & $\CIRCLE$ & $\Circle$$^2$ & $\CIRCLE$ & $\Circle$ & - \\ \hline
  PseudoID & $\CIRCLE$ & $\Circle$ & $\Circle$$^3$ & $\CIRCLE$ & $\CIRCLE$ & $\CIRCLE$ \\ \hline
  EL PASSO & $\CIRCLE$ & $\Circle$ & $\CIRCLE$ & $\CIRCLE$ & $\CIRCLE$ & $\CIRCLE$ \\ \hline
  UnlimitID & $\CIRCLE$ & $\Circle$ & $\CIRCLE$ & $\CIRCLE$ & $\CIRCLE$ & $\CIRCLE$ \\ \hline
  Opaak & $\CIRCLE$$^4$ & $\Circle$ & $\Circle$ & $\CIRCLE$ & $\CIRCLE$ & $\CIRCLE$ \\ \hline
  Hyperledger Idemix & $\Circle$$^5$ & $\Circle$ & $\CIRCLE$ & $\CIRCLE$ & $\CIRCLE$ & $\CIRCLE$ \\ \hline
  U-Prove & $\CIRCLE$ & $\Circle$ & $\LEFTcircle$$^6$ & $\CIRCLE$ & $\CIRCLE$ & $\CIRCLE$ \\ \hline
  UPPRESSO & $\CIRCLE$ & $\CIRCLE$ & $\CIRCLE$ & $\CIRCLE$ & $\CIRCLE$ & $\Circle$ \\ \hline
\end{tabular}
    \label{tbl:comparison-protocol}
\flushleft
{\footnotesize
1. A user generates an \emph{ephemeral} private key to sign every ``subsidiary'' token,
 which is verified by the RP.\\
2. SPRESSO can be extended to provide user attributes in then tokens, while the prototype does not support it.\\
3. Blindly-signed user attributes can be selectively provided using zero-knowledge proofs,
    not implemented in the preliminary version of PseudoID.\\
4. Two options: anonymous and non-anonymous.\\
5. However, in its original version \cite{idemix}, user identity is non-anonymous at an RP.\\
6. IdP-confirmed attribute and unconfirmed attributes are both enclosed.}
\end{table*}





\subsection{Privacy-Preserving Solutions of SSO and Federated Identity Management}
\label{subsec-solutions}
Pairwise pseudonymous identifiers (PPIDs) are recommended by NIST \cite{NIST2017draft}
 and specified in several SSO protocols \cite{OpenIDConnect, SAMLIdentifier} to protect user privacy against curious RPs.
When issuing an identity token,
        the IdP encloses a user pseudo-identity (but not the user identity at the IdP) in the token.
Given a user,
    the IdP assigns a unique PPID based on the target RP,
    so that collusive RPs cannot link the user's different PPIDs across these RPs.
PPID-based approaches cannot prevent the IdP-based login tracing,
 since in the generation of identity tokens the IdP needs to know which RP the user is visiting.

Some solutions prevent the IdP-based login tracing,
    but vulnerable to the threat of RP-based identity linkage.
In BrowserID \cite{BrowserID} (and its prototypes known as Firefox Accounts \cite{FirefoxAccount} and Mozilla Persona \cite{persona}),
 the IdP %(called the primary identity authority in BrowserID)
  issues a special ``token'' to bind the user identity to an \emph{ephemeral} public key,
 so that the user utilizes the corresponding private key to sign a ``subsidiary'' identity token
    to bind his identity with the target RP's identity and then sends both tokens to the RP.
%When a user logins to different RPs, the RPs still extract the identical user identities
%    from different pairs of tokens and link these login instances.
In PRIMA the IdP signs an authentication credential to bind a verification key with a set of user attributes \cite{prima}, where the verification key is considered as his unique identity.
And then the user provides selective IdP-confirmed attributes to any RP using his signing key \cite{Oblivion}, cooperatively with the IdP.
In SPRESSO \cite{SPRESSO} an RP creates a one-time but verifiable pseudo-identity for itself in each login instance.
Then, the IdP generates an identity token binding this RP pseudo-identity and the user identity.
In these schemes,
    collusive RPs could link a user based on his unique identity in the tokens (or credentials).

PseudoID \cite{PseudoID} introduces an independent token service in addition to the IdP,
    to  \emph{blindly} sign an access token binding a pseudonym and a user secret.
Then, the IdP will assert this access token,
    which allows the user to login to the RP using his secret.
Two kinds of privacy threats are prevented, because (\emph{a}) the RP's identity is not enclosed in the access token,
    and (\emph{b}) the user by himself encloses different pseudonyms when visiting RPs.
Moreover, collusive privacy attacks by the IdP and RPs are also prevented,
    because even the IdP and the token service cannot link two blindly-signed access tokens.

% the user must use the secret to login to RP, because no RP identity is enclosed in the token.
    

%However, the user has to permanently keep the secret preimage for each account in an RP.

A user of EL PASSO \cite{ELPASSO}
    keeps a secret on his device.
After authenticating the user,
    the IdP signs an anonymous credential \cite{anon-credential} binding the secret,
         also kept on the user's device.
When attempting to login to any RP,
    the user proves that he is the owner of this credential to the RP without exposing the secret,
        and selectively discloses some attributes in the credential.
Although such a credential is proved to multiple RPs,
        user-maintained pseudonyms and anonymous credentials prevent the IdP, even when collusive with some RPs, from linking the login instances.
UnlimitID \cite{UnlimitID} presents similar designs,
    also based on anonymous credentials \cite{anon-credential},
        and prevents collusive attacks by the IdP and RPs.
NEXTLEAP \cite{nextleap} adopts UnlimitID for secure messaging.

%    but a user has to by himself manage pseudonyms for different RPs.
%For example,
%    the RP domain (e.g., \verb+www.RP.com+) is used as a factor to locally generate the user's account (or pseudonym) at this RP.

+++++

Anonymous credentials \cite{anon-credential-2001} can be utilized in flexible ways.
Opaak \cite{opaak} keeps anonymous credentials in mobile phones as ,
    which are .
Hyperledger Idemix \cite{hyperledge-idemix}.



U-Prove \cite{uprov}

%Moreover,
%    a user has to change its account at each RP if its long-term secret was lost,
%        because the account is calculated based on this secret \cite{ELPASSO,UnlimitID}.
% ��һ�㣬�Ͳ���Ϊ�����ˡ���Ϊaccount��RP ID��أ��������ͻ������������������������RP Cert���Ͳ���仯��

%On the contrary,
%    in the commonly-used SSO systems \cite{OpenIDConnect,rfc6749,SAML} and privacy-preserving schemes
%    such as BrowserID \cite{BrowserID}, SPRESSO \cite{SPRESSO} and also UPPRESSO,
%        a user only renews his credential at the IdP if it was compromised,
%    because (\emph{a}) authentication happens only between the user and the IdP
%    and (\emph{b}) an RP verifies only tokens generated by the IdP.





Table \ref{tbl:comparison-protocol} summaries these privacy-preserving solutions of SSO and federated identity management.
A ``pure'' SSO protocol allows the user to authenticate himself a \emph{single} time by ,
    but need no any permanent secret when attempting to login to an RP.
Some identity federation solutions \cite{ELPASSO,UnlimitID,opaak} protect user privacy against collusive attacks by the IdP and RPs,
    while a user has to hold long-term secret in addition to the credentials (or tokens).
On the other hand,
    identity federation enables identity to be accepted by any one more than the issuer (where registration),
    with or without more secret.
Some privacy-preserving SSO schemes and UPPRESSO try to keep the user convenience lightweight,
In EL PASSO, UnlimitID and PseudoID \cite{ELPASSO,UnlimitID,PseudoID}
    there are some authentication steps between the user and RPs to verify the user secrets.
Therefore, in these systems
    a user needs to notify each RP if a credential was lost or compromised,
for the user is authenticated by the RP with his credentials
            to prove that he owns the long-term secret.



It brings some burdens to users,
    while PPIDs are maintained by the IdP and in UPPRESSO the accounts are determined automatically.


\begin{comment}

\subsection{Discrete Logarithm Problem}
\label{sec:dlp}


Based on the discrete logarithm problem, UPPRESSO designs the identifier-transformation functions. %$\mathcal{F}_{ID_{RP} \mapsto PID_{RP}}$ and $\mathcal{F}_{ID_{U} \mapsto PID_{U}}$ to generate privacy-preserving user identifier (e.g. $PID_U$) and RP identifier (e.g. $PID_{RP}$), respectively.
Here, we briefly review the discrete logarithm problem.
%A number $g$ ($0<g<p$) is called a primitive root modular a prime $p$, if for ${\forall}y$ ($0<y<p$), there is a  number $x$ ($0\le x <p-1$) satisfying $y=g^x \pmod p$.
For the finite field $GF(p)$ where $p$ is a large prime, a number $g$ is called a generator of order $q$, if it constructs a cyclic  group of $q$ elements by calculating $y=g^x \ mod\ p$.
And $x$ is called the discrete logarithm of $y$ modulo $p$. Given a large prime $p$, a generator $g$ and a number $y$, it is computationally infeasible to solve the discrete logarithm (i.e., $x$) of $y$ \cite{WXWM}, which is called the discrete logarithm problem.
The hardness of solving discrete logarithms is utilized to design several secure cryptographic primitives, including Diffie-Hellman key exchange and the digital signature algorithm (DSA).

%In the process of $F_{PID_{RP}}$ and $F_{PID_U}$, we needs to calculate the primitive root for a  large prime $p$ as follows \cite{Shoup,Wang}. First, we retrieve a primitive root $g_m$  modulo $p$ from all the integers by finding the first integer passing  the primitive root checking.  A lemma is propose to simply the checking, that if $p=2q+1$ ($q$ is a prime),  an integer $\mu \in (1, p-1)$ is a primitive root if and only if $\mu^2\neq 1 \ mod \ p$ and $\mu^q\neq 1 \ mod \ p$. Then, based on $g_m$, we can calculate a new primitive root $g = g_{m}^{t} mod \ p$, where $t$ is an integer coprime to $p-1$.
\end{comment}

\subsection{Related Works}%各个方向全都加入,例如安全分析
\label{sec:related}

%Such tokens (or credentials) authorize a user to conduct operations
%        in privacy-preserving ways.
%
%    tokens (or credentials) authorize a user to conduct operations
%        in privacy-preserving ways.
%Privacy-enhancing technologies have been applied in various scenarios,
%  but not adopted to comprehensively transform the five (pseudo-)identities in SSO services.
\noindent\textbf{Anonymous Token.}
\textcolor{blue}{PrivacyPass and TrustToken allow a user to receive tokens \cite{privacypass,trusttoken}, each of which is denoted as ($T, T^{k}$), where $k$ is the token server's private key.
 These tokens are used to access resources anonymously.
To unlink token signing and redemption,
    a user generates a random number $e$ for each token, blinds $T$ into $T^{e}$,
        and sends it to return ($T^e, T^{ek}$) from the token server.
The user then utilizes $e$ to obtain $T^k$ from $T^{ek}$, and then only ($T, T^{k}$) is redeemed to access resources.
This cryptographic skill \cite{oprf-proved} is adopted in UPPRESSO similarly:
    a user transforms $ID_{RP}$ to $PID_{RP} = [t]ID_{RP}$ by a random number $t$,
 and $PID_{RP}$ is transformed again by an IdP to $[tu]ID_{RP}$.
The visited RP calculates $Acct = [u]ID_{RP}$ from $[tu]ID_{RP}$ by using $t$
 (see Table \ref{tbl:notations-protocol} for detailed descriptions of these notations).}

\textcolor{blue}{UPPRESSO differs from PrivacyPass and TrustToken as below.
Firstly,
    PrivacyPass and TrustToken work as anonymous SSO to some extent, where one consistent private key serves all users,
but UPPRESSO identifies each user at an RP.
Secondly,
   the above cryptographic skill \cite{oprf-proved} is differently utilized.
UPPRESSO integrates it to transform identities in SSO:
scalar $u$ is known by the IdP and a user as his user identity, and
random number $t$ is shared by the user and the RP. Meanwhile,
exponent $k$ is held only by the PrivacyPass/TrustToken server as a key,
 and random number $e$ is only known to a user.
Lastly and most importantly,
more privacy requirements are satisfied in UPPRESSO.
The unlinkability between token signing and redemption \cite{privacypass,trusttoken}, or
($T^e, T^{ek}$) and  ($T, T^k$),
roughly corresponds to only the IdP-untraceability in UPPRESSO:
 an IdP cannot link any pair among $[t_i]ID_{RP}$ and $ID_{RP}$,
 $i = 1, 2, \cdots$. % which fundamentally depends on the ECDLP assumption.
UPPRESSO also supports the unlinkability across RPs:
given multiple users, e.g., identified as $u$ and $u'$,
    ($ID_{RP}, t, [u]ID_{RP}$) and ($ID_{RP'}, t', [u']ID_{RP'}$) are indistinguishable to
    colluding RPs.
This property of the adopted cryptographic skill is not considered
    in either anonymous tokens \cite{privacypass,trusttoken} or oblivious pseudorandom functions (OPRFs) \cite{oprf-proved}.}

\noindent\textbf{Anonymous SSO.}
Such schemes allow authenticated users to access a service protected by an IdP,
    without revealing their identities.
Anonymous SSO was proposed for GSM communications \cite{ElmuftiWRR08},
    and formalized \cite{WangWS13}.
Privacy-preserving primitives, such as group signature, zero-knowledge proof, Chebyshev Chaotic Maps and proxy re-verification,
     were adopted to design anonymous SSO \cite{WangWS13,HanCSTW18,Lee18,HanCSTWW20}.
Anonymous SSO schemes work for some special applications,
    but are unapplicable in most systems that require user identification for customized services.

\noindent\textbf{Privacy-Preserving Credential.}
ZKlaims \cite{zklaim} allows users to prove statements on the credentials issued by a server
    using zero-knowledge proofs,
        but the credential contents are not revealed.
Crypto-Book \cite{crypto-book} coordinates servers to generate a ring-signature private key,
 and a user picks up his key through a list of Email addresses. % (i.e., an anonymity set).
 Then the key pair works as an untraceable pseudonym. %to sign messages.
Two-party threshold schemes are implemented with a central server,
    to protect a user's private keys \cite{mRSA,ss-rsa}:
    to sign/decrypt a message, a user needs a token from the server.
    Tandem \cite{tandem} decouples the obtaining and using of such tokens,
for the privacy of key usage.


%\vspace{0.5mm}
\noindent\textbf{Formal Analysis on SSO Protocols.}
%The SSO standards (e.g., SAML, OAuth, and OIDC) have been formally analyzed.
Fett et al. \cite{FettKS16, FettKS17} formally analyzed OAuth 2.0 and OIDC using a Dolev-Yao style model \cite{FettKS14},
    and presented the attacks of 307 redirection and IdP mix-up.
        %When the IdP misuses an HTTP 307 status code for redirection, the sensitive information (e.g., credentials) entered at the IdP
         %   will  be leaked to the RP through  the user's browser.
        %The IdP mix-up attack confuses the RP about which IdP is used and the victim RP sends the token to a malicious IdP,
        % which breaks the confidentiality of identity tokens.
%According to these formal proofs \cite{FettKS16, FettKS17},
%    OAuth 2.0 and OIDC are secure except these two attacks.
%UPPRESSO could be integrated into OIDC, which simplifies its security analysis.
SAML-based SSO was analyzed \cite{ArmandoCCCT08},
    and it is found that RP identities were not correctly bound in the identity tokens of a variant designed by Google.



%\vspace{0.5mm}
\noindent\textbf{SSO Implementation Vulnerabilities.}
Vulnerabilities were found in SSO implementations for web applications,
    resulting in effective attacks %of impersonation and identity injection
     by breaking confidentiality \cite{WangCW12,ccsSunB12,ArmandoCCCPS13,DiscoveringJCS,dimvaLiM16}, integrity \cite{WangCW12,SomorovskyMSKJ12,WangZLG16,MainkaMS16, MainkaMSW17,dimvaLiM16} or RP designation \cite{WangZLG16,MainkaMS16,MainkaMSW17,YangLCZ18,dimvaLiM16} of identity tokens.
%In the SSO services of Google and Facebook, %from the view of browser-relayed traffics
%    logic flaws of the IdPs and RPs were detected \cite{WangCW12}.  % to break the confidentiality and integrity of identity tokens.
Integrity of identity tokens was violated in SSO systems  %\cite{SomorovskyMSKJ12,WangCW12,WangZLG16,MainkaMS16, MainkaMSW17}
due to software flaws such as
 defective verification by RPs \cite{WangCW12,WangZLG16,MainkaMSW17}, XML signature wrapping \cite{SomorovskyMSKJ12}, and IdP spoofing \cite{MainkaMS16,MainkaMSW17}.
RP designation is broken
    for incorrect binding by an IdP \cite{YangLCZ18,WangZLG16} or insufficient verification by RPs \cite{MainkaMS16,MainkaMSW17,YangLCZ18}.

Automatic tools such as SSOScan \cite{ZhouE14}, OAuthTester \cite{YangLLZH16} and S3KVetter \cite{YangLCZ18},
detect the violations of confidentiality, integrity, or RP designation of SSO identity tokens.
Wang et al. \cite{ExplicatingSDK} detect the vulnerable applications
    built with authentication/authorization SDKs,
     due to the implicit assumptions of these SDKs.
Navas et al. \cite{NavasB19} discussed the possible attack patterns against OIDC services.

In a mobile system,
browsers, IdP Apps,
    or IdP-provided SDKs %(e.g., an encapsulated WebView)
         are responsible for forwarding identity tokens, %from the IdP App to RP Apps.
but none of them ensures an identity token is sent to the designated RP only \cite{ChenPCTKT14,WangZLLYLG15}.
%    because a WebView or the system browser cannot authenticate the RP Apps and the IdP App may be repackaged.
%SSO protocols are modified for mobile Apps, but the modifications are not well understood by developers \cite{ChenPCTKT14,YangLS17}.
Vulnerabilities were found in Android Apps,
    to break confidentiality \cite{ChenPCTKT14,WangZLLYLG15,YangLS17,ShiWL19}, integrity \cite{ChenPCTKT14,YangLS17}, and RP designation \cite{ChenPCTKT14,ShiWL19} of identity tokens.
A flaw was found in Google Apps \cite{ArmandoCCCPS13}, allowing a malicious RP to hijack a user's authentication attempt and inject a payload to steal the cookie or identity token belonging to another RP.

If a user is compromised,
    attackers will login to RPs on behalf of him.
Single sign-off helps the victim user
 to revoke all his tokens accepted and logout from the RPs  \cite{GhasemisharifRC18}.
FedCM \cite{FedCM} attempts to disable iframe and third-party cookies in SSO, which might be exploited to track users.
%UPRRSSO protects privacy in SSO through ID transformations and our prototype does NOT use either iframe or third-party cookies.

%以下内容被拆分到上面三点内容了,但是注释内容有对论文的详细表述

%MohsenS16分析了在移动端使用webview实现SSO会导致恶意RP可以向WebView中插入JS代码获取用户的token (confidentiality),同时提供了对WebView的保护防止恶意RP获得token
%WangZLLYLG15 使用自动化工具对Android应用进行分析,总结了Android应用面临的安全问题:1 Vulnerability I (V1): Improper User-agent(使用WebView实现SSO), 2 Vulnerabili ty II (V2): Lack of Authentication(使用app间消息传递实现SSO), 3 Vulnerability III (V3): Inadequate transmission protection (网络传输缺少保护), 4 Vulnerabili ty IV (V4): Insecure secret Management (上面的 client-side logic), 5 Vulnerability V (V5): Problematical server-side validation (RP server与IdP server之间的消息传递没有受到保护,用户的信息可能被泄露,用户信息可能被篡改), 6 Vulnerabili ty VI (V6): Wrong authentication token (使用公开信息作为identity token,破坏confidentiality )
%YangLS17 使用工具分析Android 应用OAuth实现的问题:1 Untrusted Identity Token (使用server-to-server transmission保护,使用签名保护 integrity), 2 Heavy Client-Side Logic(客户端逻辑)
%ShiWL19 使用model-based的自动化工具,分析了Android应用OAuth实现的问题,包括:Access Token Replacement(替换token,破坏binding), access token Disclosure (confidentiality), code Disclosure (confidentiality),App Secret Disclosure (客户端逻辑,只曝露secret不会破坏安全性),Augmented Token Replacement (提供token与RP的绑定,但是可以被绕过, the attacker can extract the associated user information of victims from the IdP directly with either the stolen (i.e., network attacker) or obtained (i.e., malicious RP attacker) token, e.g., replaying Step 7 in Fig. 1. Consequently, the attacker can inject both the token and its corresponding user information in his own session),Profile Vulnerability(缺少用户明确授权,泄露用户的隐私数据)
% In 2016 Mohsen et al. \cite{MohsenS16} proposed the security of SSO systems implemented through WebView, one of the most important Android components, also facing the threaten of untrusted identity token transmission.
% Moreover, in 2016 Wang et al. \cite{WangZLG16} analysed the design and implementation of SSO systems for multiple platforms with the automatic testing. In 2015 Wnag et al. \cite{WangZLLYLG15}, in 2017 Yang et al. \cite{YangLS17} and in 2019 Shi et al. \cite{ShiWL19} issued the new vulnerabilities in mobile SSO systems and conducted security assessments for the top Android applications and and achieve the statistical result of these applications.

%YeBWD15在facebook的webview实现中,通过第三方应用获得facebook的cookie,使攻击者能够以受害者的身份登录facebook应用。we build a dummy app using the Facebook Login and we authorize the app with public profile permission. Then we used adb tool kit with root privilege to access to storage of the mobile phone. We successfully locate the cookies c_user and xs as well as the credential access_token. The cookies are stored in an sqlite database in mobile phone’s storage at path /data/data/<Apps package name> /databases/webviewCookiesChromium.db and the access_token is stored in an xml file at path /data/data/<App’s package name> /com.facebook.AuthorizationClient. WebViewAuthHandler.TOKEN\_STORE\_KEY.xml.

\section{The Identity-Transformation Approach}
\label{sec:challenge}

We discuss the security requirements of privacy-preserving SSO and present the identity-transformation approach.


\subsection{Security Requirements for SSO Services}
\label{subsec:basicrequirements}

Non-anonymous SSO Services \cite{OpenIDConnect,rfc6749,SAML,SAMLIdentifier,NIST2017draft} are designed to allow a \emph{legitimate} user to login to an \emph{honest} RP with her account at this RP, %correlating multiple login instances,
by presenting \emph{identity tokens} issued by a \emph{trusted} IdP.
To achieve this goal, the trusted IdP issues an identity token that specifies the RP being accessed (i.e., \emph{RP designation}) and the authenticated user's identity or pseudo-identity (i.e., \emph{user identification}).
An honest RP verifies the RP's (pseudo-)identity in the identity token before accepting it,
 and authorizes the token holder to login as the specified user. This prevents malicious RPs from replaying received identity tokens to gain unauthorized access to other honest RPs as some victim user.
\emph{Confidentiality} and \emph{integrity} of identity tokens are also necessary to prevent eavesdropping and tampering. Identity tokens are forwarded to the target RPs by the authenticated user and should not be revealed to any other parties. %Otherwise, an eavesdropper who possesses the token could successfully login to the designated RP. Maintaining integrity is also critical to prevent adversaries from tampering with a token. So,
So they are usually signed by the trusted IdP and transmitted over HTTPS \cite{OpenIDConnect,rfc6749,SAML}.

The requirements for secure SSO services, i.e., the RP designation, user identification, integrity, and confidentiality of identity tokens, have been extensively studied in the literature \cite{ArmandoCCCT08, FettKS16, FettKS17}.
Any vulnerabilities that undermine these properties result in various attacks \cite{SomorovskyMSKJ12, WangCW12, ArmandoCCCPS13, ZhouE14, WangZLLYLG15, WangZLG16, YangLLZH16, MainkaMS16, MainkaMSW17, YangLCZ18, YangLS17, ShiWL19, ChenPCTKT14, ccsSunB12, DiscoveringJCS, dimvaLiM16, CaoSBKVC14, TowardsShehabM14}.
We prove that an SSO system satisfying these four properties is secure in
Section \ref{analysis-security}.

%\subsection{The Identity Dilemma of Privacy-Preserving SSO}
%\label{subsec:challenges}
\begin{table}
\footnotesize
    \caption{The (pseudo-)identities in privacy-preserving SSO}
    \centering
%    \begin{tabular}{|l|l|l|}
    \begin{tabular}{|p{1.0cm}|p{5.1cm}|p{1.13cm}|} \hline
    {\textbf{Notation}} & {\textbf{Description}} & {\textbf{Lifecycle}} \\ \hline
    {$ID_U$} & {The user's unique identity at the IdP.} & {Permanent} \\ \hline
    {$ID_{RP_j}$} & {The $j$-th RP's unique identity at the IdP.} & {Permanent} \\ \hline
    {$PID_{U,j}^i$} & {The user's pseudo-identity in her $i$-th login instance to the $j$-th RP.} & {Ephemeral} \\ \hline
    {$PID_{RP_j}^i$} & {The $j$-th RP's pseudo-identity in the user's $i$-th login instance to this RP.} & {Ephemeral} \\ \hline
    {$Acct_j$} & {The user's identity (or account) at the $j$-th RP.} & {Permanent} \\ \hline
    \end{tabular}
    \label{tbl:notations-dilemma}
\end{table}

\subsection{Identity Transformation}
\label{subsec:solutions}

We aim to develop a privacy-preserving SSO system that ensures security properties while preventing both IdP-based login tracing and RP-based identity linkage.
%We explicitly distinguish a user's identity (or \emph{account}) at an RP,
%     from the user's \emph{identity} at an IdP and the user's \emph{pseudo-identities} enclosed in identity tokens.
In \usso, these requirements are satisfied through \emph{transformed identities} in the identity tokens. Table \ref{tbl:notations-dilemma} lists the notations used in this paper.
The subscript $j$ and/or the superscript $i$ may be ignored if it does not cause ambiguity.

\newc
In an SSO login flow, a user initiates the process by negotiating an \emph{ephemeral} pseudo-identity $PID_{RP}$  with the target RP and sending an identity-token request that encloses $PID_{RP}$ to the IdP.
After successfully authenticating the user as $ID_U$, the IdP calculates an \emph{ephemeral} $PID_U$ based on $ID_U$ and $PID_{RP}$ and issues an identity token that binds $PID_U$ and $PID_{RP}$. Upon receiving a token with a matching $PID_{RP}$, the RP calculates the user's \emph{permanent} $Acct$ and authorizes the token holder to log in.
%
%Given a user,
%    (\emph{a}) an identity token contains only pseudo-identities, i.e., $PID_{U,j}^i$ and $PID_{RP_j}^i$,
%        which are independent of each other for different RPs and in multiple login instances, respectively,
%    and (\emph{b}) these \emph{ephemeral} pseudo-identities enable the target RP to derive a \emph{permanent} account, i.e., $Acct_j$.
The relationships among the permanent and ephemeral (pseudo-)identities are depicted in Figure \ref{fig:IDCorrelation}, where the \emph{red} and \emph{green} blocks represent \emph{permanent} and \emph{ephemeral} (pseudo-)identities, respectively, and the labeled arrows denote the transformations of (pseudo-)identities.
%It describes the {\em identity dilemma} of privacy-preserving SSO as below:

%An identity token binds the (pseudo-)identities of an authenticated user and an RP.
%Since an IdP authenticates users and then always knows the user's identity (i.e., $ID_U$),
%    to prevent the IdP-based login tracing,
%    we should not reveal the target RP's permanent identity (i.e., $ID_{RP}$) to the IdP.

To ensure RP designation, $PID_{RP}$ should be \emph{uniquely} associated with the target RP.
For user identification, the \emph{ephemeral} $PID_{U}^i$ in each login instance should enable the RP to derive the user's \emph{permanent} account  (i.e., $Acct$) at this RP.
To prevent IdP-based login tracing, it is essential to ensure that the IdP does not obtain any information about $ID_{RP}$ from any $PID_{RP}^i$.
Thus, in a user's multiple login instances to the same RP, he should generate independent $PID_{RP}^i$\footnote{The IdP should not be able to link multiple login instances to a given RP, while the RP's identity is unknown to the IdP.} % the IdP-based login tracing is still effective, to correlate a user's multiple login instances.
and also independent $PID_U^i$\footnote{If $PID_U^i$ is not completely independent of each other, it implies that there is a possibility for the IdP to link multiple login instances to a given RP.}.
Finally, to prevent RP-based identity linkage,
% the IdP does not enclose $ID_U$ in identity tokens.
%a user pseudo-identity (i.e., $PID_U$) is bound instead:
%$PID_U$ is bound in identity tokens:
the RP should not obtain any information about $ID_U$ from any $PID_{U,j}$, which implies that $PID_{U,j}$ for different RPs should also be independent of each other.
Therefore, we propose identity transformations as follows:
\vspace{-\topsep}\begin{itemize}
\setlength{\topsep}{0pt}
\setlength{\partopsep}{0pt}
\setlength{\itemsep}{0pt}
\setlength{\parsep}{0pt}
\setlength{\parskip}{0pt}
\item
$\mathcal{F}_{PID_{RP}}(ID_{RP}) = PID_{RP}$, calculated by the user and the RP.
From the IdP's view,
$\mathcal{F}_{PID_{RP}}()$ is a one-way function and $PID_{RP}$
is \emph{indistinguishable} from random variables.
\item
$\mathcal{F}_{PID_U}(ID_U, PID_{RP}) = PID_{U}$, calculated by the IdP.
From the target RP's view,
    $\mathcal{F}_{PID_U}()$ is a one-way function and $PID_{U}$ is \emph{indistinguishable} from random variables.
\item
$\mathcal{F}_{Acct}(PID_{U}, PID_{RP}) = Acct$, calculated by the RP.
Given $ID_U$ and $ID_{RP}$, $Acct$ is %\emph{permanent} and
\emph{unique} to other accounts at this RP.
That is, in a user's two different login instances to the RP,
 $\mathcal{F}_{Acct}(PID_{U}^i, PID_{RP}^i) = \mathcal{F}_{Acct}(PID_{U}^{i'}, PID_{RP}^{i'})$.
\end{itemize}

\oldc

\begin{figure}[bt]
  \centering
  \includegraphics[width=0.99\linewidth]{fig/IDCorrelation.pdf}
  \caption{Identity transformations in privacy-preserving SSO}
  \label{fig:IDCorrelation}
\end{figure}

%We pose the \emph{identity dilemma} (or challenge) of SSO identity tokens
%    to satisfy the requirements of both security and privacy:
%
%\noindent\emph{Given an authenticated user and an unknown RP (i.e., permanent $ID_U$ and ephemeral $PID_{RP}$),
%    an IdP is expected to generate an ephemeral pseudo-identity (i.e., $PID_{U}$)
%     which will be correlated with the user's permanent identity at this RP (i.e., $Acct$),
%     while knowing nothing about the RP's identity or the user's account at this RP (i.e., $ID_{RP}$ or $Acct$).}

%Existing privacy-preserving SSO solutions (i.e., SPRESSO \cite{SPRESSO}, BrowserID \cite{BrowserID} and PPID \cite{NIST2017draft})
%  do not explicitly and comprehensively consider all five (pseudo-)identities in the SSO login flow,
%    and $ID_U$ or $ID_{RP}$ is still enclosed in identity tokens.
%So only one type of privacy threat is prevented.




\section{Threat Model and Assumption}
\label{sec:assumptionandthreatmodel}
UPPRESS follows the same service mode as traditional SSO systems (e.g., SAML and OIDC), 
    and it consists of an IdP, a number of RPs and users.
The IdP provides user authentication services for all RPs.
In this section,
    we describe the threat model and assumptions of these components in UPRESSO.

\subsection{Threat Model}
The IdP is assumed to be curious-but-honest,
 while some users and some RPs could be completely controlled by adversaries. % or even collude with each other.
Malicious users and RPs behave arbitrarily and may collude with each other,
     to break the guarantees of security and privacy for other correct users.
%While, the IdP will follow the protocol correctly, and is only curious about the user's privacy.
%The details are as follows.

\vspace{1mm}\noindent \textbf{Curious-but-honest IdP.}
The IdP strictly follows its specification, but is curious about the user privacy, especially the login activities at different RPs.
The IdP is well-protected and will never leak any sensitive information.
For example, the private key for generating the identity proof and RP certificate (used in Section~\ref{implementations}) will never be leaked,
 therefore the adversary fails to impersonate as the IdP to forge an identity proof or RP certificate.
The honest IdP processes the requests of RP registration and identity proof correctly,
and never colludes with others (e.g., malicious RPs and users).
For example, IdP ensures the uniqueness of $ID_{RP}$ and $PID_{RP}$, and generates the correct RP certificate, $PID_U$ and identity proof.

However, the curious IdP may attempt to break the user's privacy without violating the protocol.
For example, the curious IdP may store and analyze the received messages, and perform the timing attacks, attempting to conduct visit tracing or identity linkage.


%User's goal: ��IdP����identity proof��ʹIdP��Ϊ�Լ�����һ��victim
%��ʽ��1���Ѿ�ӵ����Ч��identity proof��ϣ����IdPЭ�̳���ͬ��PID_RP��2��ͨ���۸Ļ���α��identity proof��ʵ�ֹ���
\vspace{1mm}\noindent \textbf{Malicious users.}
The adversary could control a set of users, for example through stealing the users' credentials~\cite{WangZWYH16, SunCL12} or registering  at the IdP and RPs directly.
These malicious users
 aim to break the security of the SSO system.
That is, they  attempt to impersonate  an uncontrolled user at the victim RP, and  make  a victim user log in at the correct RP under a controlled identity.
To achieve this, they could behave arbitrarily~\cite{WangCW12, SomorovskyMSKJ12}.
For example, the malicious users may forge the identity proof, modify the forwarding messages (requests of identity proof, identity proof,  RP registration request and result, and etc.), and provide incorrect values for negotiating $PID_{RP}$ (detailed in Section~\ref{implementations}).

%RP's goal:1)���Ŀǰ��¼�û�������RP���õ�identity proof��2��collusive RP �����û�
\vspace{1mm}\noindent \textbf{Malicious RPs.}
The adversary could control a set of RPs, by registering an RP at the IdP or exploiting various vulneraries to attack RPs.
These malicious RPs aim to break the security and privacy of the correct users, and could behave arbitrarily.
For example, to break the security, the malicious RPs need to obtain an identity proof valid for other RP, and attempt to achieve this by behaving as follows:
 impersonating other RP at the user by providing the incorrect RP certificate,
 using incorrect values during the negotiation of  $PID_{RP}$ to make  the generated $PID_{RP}$ be same as the one for other RP,
 or constructing an incorrect request to trigger the IdP issuing an identity proof binding with other RP.
Moreover, the malicious RPs may attempt to perform the RP-based identity linkage and break the user's privacy.
To achieve this, the RPs could  behave arbitrarily and collude with each other.
For example, the RPs may attempt to derive the $ID_U$ from $PID_U$ by providing incorrect values to the IdP,
 and the colluded RPs may attempt to link the user's multiple logins, by providing correlated values (e.g., $PID_{RP}$) to the IdP.

\vspace{1mm}\noindent \textbf{Collusive users and RPs.} %In particular,
The malicious users and RPs may collude and behave arbitrarily, attempting to break the security of UPRESSO.
For example,
the adversary may first act as a malicious RP, and make an incorrect identity proof generated for the visiting user,
  then act a malicious user, and use this identity proof to impersonate this victim user at another RP.
The adversary could also first act as a user to login a correct RP and obtain an identity proof,
 then act a malicious RP to perform the identity injection attack, by injecting this identity proof to the session between the victim user and the correct RP with other web attacks (e.g., CSRF).


\subsection{Assumption}
In UPRESSO,
we assume that the user agent deployed at the honest user is correctly implemented,
and will transmit the messages to the correct destination.
The TLS is also correctly implemented at the user agent, IdP and RP, which ensures the confidentiality and integrity of the network traffic between correct entities.

We also assume a secure random number generator is adopted in UPRESSO to provide the unpredictable random numbers;
and the adopted cryptographic algorithms, including the RSA and SHA-256, are secure and implemented correctly.
Therefore,  no one without private key can forge the signature, and the adversary fails to infer the private key during the computation.
Moreover, we also assume the security of the discrete logarithm problem is ensured.

The collusive RPs may attempt to link a user  based on the identifying attributes, such as the telephone number and credit number.
Here, we assume that the users refuse to provide these attributes to the RPs, and the correct RPs never collect these attributes as required by privacy laws (e.g., GDPR~\cite{wachter2017counterfactual}).
Moreover, the global network traffic analysis may be adopted to correlate the user's logins at different RPs.
  However, UPRESSO may integrate existing defenses to prevent this attack.


\section{Design of UPRESSO}
\label{sec:UPRESSO}
In this section, we provide designs of UPRESSO, a secure and privacy-preserving SSO system.
First, we present how to achieve the trapdoor user identification and transformed receiver designation. Then, we describe the detailed protocol for providing the SSO service. Finally, we discuss the compatibility of UPRESSO with OIDC.

\subsection{Features}
\label{subsec:overview}
The three functions $F_{PID_{RP}}$, $F_{PID_U}$ and $F_{Account}$  are essential for the trapdoor user identification and  transformed receiver designation.
In UPRESSO, these functions are constructed based on discrete logarithm cryptography with the public parameters $p$, $q$ and $N$,
 where  $p$ is a large prime defines the finite field $GF(p)$, $N$ is the length of $q$, and $q$ ($2^{N-1} < q < 2^N$) is a prime divisor of ($p-1$).  
%the prime number $q$  is the order of a multiplicative subgroup of $GF(p)$, which is generated with the generator $g$ by $\{g\ mod\ p, g^2\ mod\ p, ..., g^{q-1}\ mod\ p, 1=g^q\ mod\ p\}$.

In UPRESSO, IdP assigns a  random number as  $ID_U$ ($0 < ID_U <q $)  at the user's registration, and generates a generator of order $q$ as $ID_{RP}$ at the RP's initial registration. The $ID_{RP}$ must  be different from the ones used ever before.

For each login, the RP chooses a random number $n_{RP}$ ($1 < n_{RP}<q $), the user chooses a random number $n_{U}$ ($1 < n_{U}<q $). Then, the RP and user cooperatively  generate $PID_{RP}$ using the function $F_{PID_{RP}}$ as Equation~\ref{equ:PIDRP}. The function $F_{PID_{RP}}$ satisfies the requirements described in Section~\ref{subsec:challenges}. That is, the function $F_{PID_{RP}}$ is invoked to generate $PID_{RP}$ for each login, while IdP  fails to derive $ID_{RP}$ from $PID_{RP}$ and cannot find the relation among ${PID_{RP}}$s for a same RP, which is ensured by the discrete logarithm cryptography.
Moreover, $n_{U}$ and $n_{RP}$  serves as the nonce which ensures that the $PID_{RP}$ (also identity proof) is exactly constructed for this login, and
the cooperation between the user and RP prevents the malicious user and RP from controlling the  $PID_{RP}$. For example, the malicious user fails to make a correct RP accept a $PID_{RP}$ used in another login, while the collusive RPs fail to use a same or correlated $PID_{RP}$s for different logins.


 \begin{equation}
    PID_{RP} = {ID_{RP}}^{n_{u}* n_{RP}} mod \ p
   \label{equ:PIDRP}
   \end{equation}

For the user $ID_U$ to login at an RP with a privacy-preserving identifer $PID_{RP}$, IdP calculates the user's privacy-preserving identifer $PID_U$ using the function  $F_{PID_{U}}$ as Equation~\ref{equ:PIDU}. The function $F_{PID_{U}}$ satisfies the requirements described in Section~\ref{subsec:challenges}.
Combining Equation~\ref{equ:PIDRP} and~\ref{equ:PIDU}, we get that  $PID_U$ equals to ${ID_{RP}}^{n_U*n_{RP}*ID_U}\ mod \ p$.
The discrete logarithm cryptography ensures that the RPs fail to derive $ID_U$ from $PID_U$,
nor link a user's $PID_U$s at different RPs who have different $ID_{RP}$. %discrete logarithm of $ID_{RP}$ modulo $ID_{RP}^'$


\begin{equation}
 PID_U = {PID_{RP}}^{ID_U} \ mod \ p
 \label{equ:PIDU}
\end{equation}

Finally, the RP derives $Account$ for the user with the function $F_{Account}$ as Equation~\ref{equ:Account}. Here, the value $(n_U*n_{RP})^{-1} mod \ q$ is the trapdoor $t$. As $q$ is a prime number, $1< n_U < q$ and $1< n_{RP} < q$, therefore $q$ is coprime to $n_U*n_{RP}$, and the $t$ that satisfies $t*(n_U*n_{RP}) = 1\ mod \ q$ always exists. The function $F_{Account}$ satisfies the requirements described in Section~\ref{subsec:challenges}.  As shown in Equation~\ref{equ:AccountNotChanged},  for a user's multiple logins at an RP, $F_{Account}$ outputs an unchanged $Acount$  which equals to ${{ID_{RP}}^{ID_U}} mod \ p$.
Same as the analysis of $PID_U$, the collusive RPs fail to derive $ID_U$ from $Account$ nor link a user's $Account$s at different RPs. 

 \begin{equation}
   Account = {PID_U}^{(n_U*n_{RP})^{-1} mod \ q} mod \ p
   \label{equ:Account}
   \end{equation}

The \textbf{trapdoor user identification} is supported with these three functions.
For a user's multiple logins, each PR obtains the different $PID_U$s and the corresponding $t$s , then derives the unchanged $Account$  as shown in Equation~\ref{equ:AccountNotChanged}.
The function $F_{PID_{RP}}$ prevents the curious IdP from linking the $PID_{RP}$s of different logins at an RP, and therefore avoids  the  IdP-based access tracing.
The functions $F_{PID_{U}}$ and $F_{Account}$ prevents the collusive RPs from linking a user's $PID_U$s and $Account$s at different RPs, and therefore avoids the RP-based identity linkage.
 \begin{multline}\label{equ:AccountNotChanged}
   Account =  {PID_{U}}^{t} mod \ p  \\
   = {({PID_{RP}}^{ID_U})}^{{(n_U*n_{RP})^{-1} mod \ q}} mod \ p \\
   = {ID_{RP}} ^ {ID_U * n_U * n_{RP} *t\ mod\ q} = {ID_{RP}}^{ID_U} mod \ p
 \end{multline}

The \textbf{transformed receiver designation} is also supported with the efficient functions $F_{PID_{RP}}$ and $F_{PID_U}$, together with  a user-centric verification.
The $F_{PID_{RP}}$ ensures that the user and RP cooperatively generate a fresh $PID_{RP}$  for a user's login,
 while $F_{PID_U}$ ensures that the IdP generates the exact $PID_U$ for the $ID_U$ who logins at $PID_{RP}$.
The IdP will bind $PID_{U}$ with $PID_{RP}$ in the identity proof, which designates this identity proof to $PID_{RP}$.
In the user-centric verification,  the user checks that $PID_{RP}$ is global unique and exactly generated for the RP $ID_{RP}$,
 and then sends the identity proof  only  to this RP. Therefore, the $PID_{RP}$ is designated to $ID_{RP}$.
Finally, the transformed receiver designation is provided through the two-step designations.


\begin{table}[tb]
    \caption{The notations used in UPRESSO.}
    \centering
    \begin{tabular}{|c|c|c|}
    \hline
    {Notation} & {Definition} & {Attribute} \\
    \hline
    {$p$} & {A large prime.} & {Long-term} \\
    \hline
    {$q$} & {A large prime.} & {Long-term} \\
    \hline
    {$N$} & {Length of $q$. } & {Long-term} \\
   % \hline
    %{$SK_{ID}$, $PK$} & {The private/public key to sign/verify identity proof.} & {System-unique} \\
    \hline
    {$ID_U$} & {User's unique identifier.} & {Long-term} \\
    \hline
    {$PID_U$} & {User's privacy-preserving identifier.} & {One-time}\\
    \hline
    {$Account$} & {User's identifier at an RP.} & {Long-term} \\
    \hline
    {$ID_{RP}$} & {RP's original identifier.} & {Long-term} \\
    \hline
    {$PID_{RP}$} & {RP's privacy-preserving identifier.} & {One-time} \\
    \hline
    {$n_U$} & {User-generated random nonce for $PID_{RP}$.} & {One-time} \\
    \hline
    {$n_{RP}$} & {RP-generated random nonce for $PID_{RP}$.} & {One-time} \\
    \hline
    {$Y_{RP}$} & {Public value for $n_{RP}$, $(ID_{RP})^{n_{RP}} \ mod\ p$.} & {One-time} \\
    \hline
    {$t$} & {A trapdoor, $t=(n_U*n_{RP})^{-1} mod \ q$.} & {One-time} \\
    \hline
    {$Cert_{RP}$} & {An RP certificate. } & {Long-term} \\
    \hline
    {$SK$, $PK$} & {The private/public key of IdP. } & {Long-term} \\
     \hline
    \end{tabular}
    \label{tbl:notations}
\end{table}

\subsection{The implementations of UPRESSO}
\label{implementations}
UPRESSO contains four sub-protocols, i.e., system initialization, RP initial registration, user registration and SSO login.
The system initialization is invoked by the IdP to initialize the SSO system and only needs to be invoked once for each SSO system.
The RP initial registration is invoked by each RP to obtain the necessary parameters (a unique identifier $ID_{RP}$ and an RP certificate $Cert_{RP}$) from the IdP and only needs to be invoked once for each RP.
The user registration is only invoked once by each user to create a unique user identifier $ID_U$ and the corresponding credential.
While, the SSO login is invoked once a user wants to log in an RP, and therefore will be invoked frequently.
The process for user registration is the same as the one in the typical SSO systems,
therefore, we focus on the processes in  system initialization, RP initial registration and SSO login.
For clarity, we list the used notations  in Table~\ref{tbl:notations}.

\vspace{1mm}\noindent \textbf{System initialization.} The IdP chooses $N$, generates a large prime $p$ and a large prime $q$ as  the parameters for the discrete logarithm cryptography, and generates one asymmetric key pair ($SK$ denotes the private key and $PK$ is the public key) for the generation of the identity proof and $Cert_{RP}$.
The IdP keeps $SK$ secretly, and provides $p$, $q$, $N$ with $PK$ as the public parameters.
The values of $p$, $q$ and $N$ remain the same during the full lifecycle of an SSO system.
While, the asymmetric key pair ($SK$, $PK$) will be updated when necessary. For example, when $SK$ is leaked, IdP must update ($SK$,$PK$).

\vspace{1mm}\noindent\textbf{RP initial registration.}
The RP initial registration is invoked only once by an RP, to apply $ID_{RP}$ and $Cert_{RP}$ from IdP.
The detailed processes are as follows:
\begin{enumerate}
\item RP sends a request $Req_{Cert_{RP}}$ to the IdP. The $Req_{Cert_{RP}}$ contains the RP's distinguished name $Name_{RP}$ (e.g., DNS name) and the endpoint for receiving the identity proof.
\item IdP generates $ID_{RP}$ by choosing a  $q$-order generator that has never been used,  generates the signature $Sig_{SK}$ of [$ID_{RP}, Name_{RP}$] with $SK$, and returns [$ID_{RP}, Name_{RP}, Sig_{SK}$] as $Cert_{RP}$.
\item The RP  verifies $Cert_{RP}$ using $PK$,  and stores $ID_{RP}$ with the valid $Cert_{RP}$ for the further use.
\end{enumerate}

\vspace{1mm}\noindent\textbf{SSO login.}
Once a user attempts to log in at an RP, the SSO login is invoked. We use the OIDC implicit protocol flow as an example, to demonstrate  how to integrate the three functions $F_{PID_U}$, $F_{PID_{RP}}$ and $F_{Account}$ into the typical SSO systems.
As shown in Figure~\ref{fig:UPRESSO}, the SSO login sub-protocol contains four phases, RP identifier transforming, RP identifier refreshing, $PID_U$ generation and $Account$ calculation.
In the RP identifier transforming, the user and RP negotiate $PID_{RP}$ based on Diffie-Hellman key exchange~\cite{DiffieH76}, where $PID_{RP}$ is calculated as in Equation~\ref{equ:PIDRP}.
In the RP identifier refreshing, the user registerers the unique  $PID_{RP}$ at IdP.
In the $PID_U$ generation, IdP calculates $PID_U$ with $ID_U$ and $PID_{RP}$ as in Equation~\ref{equ:PIDU}.
And in the $Account$ calculation, the RP derives the unchanged $Account$ as in Equation~\ref{equ:Account}.

\subsection{Overall protocol flow overview}
\label{sebsec:loginprocess}
In UPRESSO, the SSO login sub-protocol provides the secure SSO service and prevents both the Idp-based access tracing and RP-based identity linkage.
The protocol, shown in Figure~\ref{fig:process},  prevents the curious IdP from obtaining the RP's identifying information during the interchanges,
  and avoids the adversary to break the security and user's privacy.
Here we introduce the detailed processes for each step in Figure~\ref{fig:process}.

\begin{figure*}
  \centering
  \includegraphics[width=0.85\linewidth]{fig/process.pdf}
  \caption{Process for each user login.}
  \label{fig:process}
\end{figure*}

\vspace{1mm}\noindent\textbf{RP identifier transforming.}
In this phase, the user and RP cooperative to generate $PID_{RP}$ as follows:
\begin{itemize}
  \item The user sends a login request to trigger the negotiation of $PID_{RP}$ (Step 1).
  \item The RP chooses a random $n_{RP}$ ($1 < n_{RP} <q$), calculates $Y_{RP}={ID_{RP}}^{n_{RP}} mod \ p$ (Step 2.1.1); and sends $Cert_{RP}$ with $Y_{RP}$ to the user (Step 2.1.2).
  \item The user checks the $Cert_{RP}$, extracts $ID_{RP}$ from the valid $Cert_{RP}$, chooses a random $n_U$ ($1 < n_U <q$) to calculate $PID_{RP}={Y_{RP}}^{n_{u}} mod \ p$ (Step 2.1.3); and sends $n_U$ with $PID_{RP}$ to the RP (Step 2.1.4).
  \item The RP calculates $PID_{RP}$ with $n_U$ and $Y_{RP}$, checks its consistency with the received one, derives the trapdoor $t={(n_U*n_{RP})}^{-1} \ mod \ q$ (Step 2.1.5); and sends the calculated $PID_{RP}$ to the user (Step 2.1.6).
  \item The user checks the consistency of the received $PID_{RP}$ with the stored one.
\end{itemize}
During the process, the user will halt the login, if  the $Cert_{RP}$ is invalid or the received $PID_{RP}$ is different from the stored one. The RP also halts the process if the $PID_{RP}$ sent by the user is inconsistent with the calculated one.

\vspace{1mm}\noindent\textbf{RP identifier refreshing.}
The user registers $PID_{RP}$ at the IdP as follows.
\begin{itemize}
  \item The user generates an one-time endpoint to hide the RP's endpoint from IdP (Step 2.2.1), and sends the registering request [$Reg$, $PID_{RP}$, one-time endpoint] to the IdP (Step 2.2.2).
  \item The IdP checks $PID_{RP}$, and constructs the response [$RegRes$, $RegMes$, $Sig_{Reg}$] (Step 2.2.3). The $RegRes$ is registration result, and is set as $OK$ only when $PID_{RP}$ is never used before and is of order $q$ module $p$. The $RegMes$ is the same as the dynamic registration response, and contains $PID_{RP}$, the issuing time and valid time. The $Sig_{Reg}$ is the signature for $RegRes$ and $RegMes$ generated by the IdP with $SK$.
  \item The user accepts $RegRes$ directly due to the secure connection with IdP, and forwards the registration result to the RP (Step 2.2.4).
  \item The IdP checks $Sig_{SK}$ and $RegMes$, and accepts $RegRes$ only when $Sig_{Reg}$ is valid, $PID_{RP}$ is the same as the negotiated one, and $RegMes$ is not expired.
\end{itemize}
If $RegRes$ is $OK$, the RP identifier refreshing completes. Otherwise, the user and RP will renegotiate the $PID_{RP}$.

\vspace{1mm}\noindent\textbf{$\mathbf{PID_U}$ generation.}
In this phase, the RP continues the process of the user's login and obtains the $PID_U$ generated by the IdP. The processes are as follows.
\begin{itemize}
  \item The RP uses $PID_{RP}$ and the endpoint to construct an identity proof request, which is the same as the one in  OIDC. (Step 2.3).
  \item The user checks the consistency of the received $PID_{RP}$  with the negotiated one (Step 2.4); replaces the endpoint with the one-time endpoint generated in Step 2.2.1, and sends the modified identity proof request to the IdP (Step 2.5).
  \item The IdP authenticates the user if she hasn't been authenticated (Step 3); checks whether $PID_{RP}$ and the one-time endpoint have been registered,
   calculates $PID_U$ using Equation~\ref{equ:PIDU},  constructs the identity proof [$PID_{RP}$, $PID_U$, $ValTime$, $Attr$,$Sig_{IdProof}$] where $ValTime$ is the valid period, $Attr$ contains the  attributes that the user agrees to provide to the RP, $Sig_{IdProof}$ is the signature of the identity proof generated by IdP with $SK$ (Step 4). Then, the IdP sends the identity proof with the one-time endpoint to the user (Step 5.1).
  \item The user finds  the  endpoint corresponding to the one-time endpoint (Step 5.2),
   and forwards the identity proof to the RP through this endpoint (Step 5.3).
\end{itemize}
The user halts the process if the $PID_{RP}$ in the identity proof request is inconsistent with  the negotiated one.
The IdP rejects the identity proof request, if the $PID_{RP}$ and the one-time endpoint have not been registered.


\vspace{1mm}\noindent\textbf{$\mathbf{Account}$ calculation.}
Finally, RP derives the user's  $Account$ and completes the user's login as follows. The RP performs the checks on the identity proof, including the valid time, correctness of $Sig_{IdProof}$, and   the consistency between $PID_{RP}$ and the  negotiated one. If all the checks pass, the RP extracts $PID_U$, and calculates $Accout$ according to Equation~\ref{equ:Account} (Step 6); and sends the $Success$ as the login result to the user (Step 7). If any check fails, the RP returns the $Fail$ to the user.


\subsection{Compatibility with OIDC}
\label{subsec:compatible}
UPRESSO could be integrated in the traditional SSO systems, to  prevent the IdP-based access tracing and RP-based identity linkage.
The integration doesn't degrade the security and only requires minimal modification.
Here, we use the implicit protocol flow of OIDC as an example to demonstrate the compatibility of UPRESSO with the traditional SSO systems.
The further analysis, such as integration with the authorization code flow of OIDC,  is provided in Section~\ref{sec:discussion}.


\vspace{1mm}\noindent \textbf{Consistency with OIDC.}
As shown in Figure~\ref{fig:UPRESSO}, the architecture of UPRESSO is the same as the one in OIDC. UPRESSO does not introduce any new entity, but only integrates the three function $F_{PID_U}$, $F_{PID_{RP}}$ and $F_{Account}$ into the processes at the IdP, RP, and user.

The formats of the  identity proof and corresponding request, and the verification of the identity proof,  are almost same in OIDC and UPRESSO.
The only difference is that $ID_{RP}$ and endpoint are replaced with the privacy-preserving versions, i.e., $PID_{RP}$ and one-time endpoint, in UPRESSO.
As $PID_{RP}$ is also unique and corresponds exactly to $ID_{RP}$, and one-time endpoint corresponds to the RP's endpoint correctly,
 the binding, integrity and confidentiality of identity proof will also be ensured in UPRESSO, and there is no degradation on the security of OIDC.

\vspace{1mm}\noindent \textbf{Minimal modification to OIDC.}
UPRESSO only requires small modification on OIDC to integrate $F_{PID_U}$, $F_{PID_{RP}}$ and $F_{Account}$.
For $F_{PID_U}$ and $F_{Account}$ , we directly use them to replace original functions for $PPID$ at the IdP and the $Account$ at the RP.
For $F_{PID_{RP}}$, we inject a negotiation process and a dynamic registration for each SSO login,
 where the negotiation process between the user and RP generates a $PID_{RP}$,
  while the dynamic registration is used to check the uniqueness of $PID_{RP}$.
In UPRESSO, the dynamic registration is slightly modified as follows: an RP identifer ($PID_{RP}$)  is added in the request, and a signature ($Sig_{Res}$)  is included in the response for its verification at the RP.






\section{Formal Analysis of UPPRESSO}
\label{sec:analysis}
\begin{figure*}
  \centering
  \includegraphics[width=0.82\linewidth]{fig/game1.pdf}
  \vspace{-6mm}
  \caption{The Game.}
  \label{fig:game}
\vspace{-7mm}
\end{figure*}
In this section, we propose the security properties of privacy-preserving SSO schemes and then give the proofs that UPPRESSO follows the security properties.

++++


\subsection{The Web Model}
\label{subsec:webmodel}
The Dolev-Yao model abstracts the entities in a system, such as browsers and web servers, as {\em atomic processes}, which communicate with each other through the {\em events}. \cite{SPRESSO} also defines {\em scripting processes} to model client-side scripting such as JavaScript, so a web system consists of a set of atomic and scripting processes. The state of a system, called a {\em configuration}, consists of the current states of all atomic processes and all the events that can be accepted by these processes. We list the definitions of these notations as below \cite{SPRESSO}.

%\vspace{1mm}
\noindent{\em Messages}  are the basic data carriers among web nodes, such as HTTP requests and responses.

\noindent{\em Events} are the basic communication elements in the model. An event contains the addresses of sender and receiver and a message.
%An event is of the form $\langle a, f, m \rangle$, where $a$ and $f$ represent the addresses of the sender and receiver respectively, and $m$ is the message to be transmitted.

\noindent{\em Atomic Processes.} An {\em atomic Dolev-Yao (DY) process} is a tuple $p=$ $(I^p, Z^p, R^p,s_0^p )$, where $I^p$ is the set of addresses that the process listens to, $Z^p$ is the set of states (i.e., terms) that describes the process, $s_0^p$ is an initial state, and $R^p$ is the mapping from an input state $s \in Z^p$ and an event $e$ to a new state $s'$ and an event $e'$. %It is worth noting that for one process in a state, only a finite set of events can be accepted by the process as the state and event are defined as the input of $R^p$.
Each atomic process also contains a set of nonces that it may use.

\noindent{\em Scripting Processes} represent client-side scripts loaded by the browser to provide server-defined functions to the browser. However, a scripting process must rely on an atomic process, such as the browser, and provide the relation $R$ called by this atomic process.


\noindent {\bf Web system.} We can represent the web infrastructure as a web system of the form ($\mathcal{W}$, $\mathcal{S}$, $\mathtt{script}$, $E^0$), where $\mathcal{W}$ is the set of atomic processes containing both honest and malicious processes, $\mathcal{S}$ is the set of scripting processes including honest and malicious scripts, $\mathtt{script}$ is the set of concrete script codes related to specific scripting processes in $\mathcal{S}$, and $E^0$ is the set of events acceptable to the processes in $\mathcal{W}$.

\noindent A {\em configuration} of this web system is a tuple ($S, E, N$), where $S$ is the current states of all processes in $\mathcal{W}$, $E$ is the set of events that the processes accept, and $N$ is a global sequence of nonces that have not been used by the processes yet.

\noindent A {\em run step} is the system migrating from configurations ($S, E, N$) to ($S', E', N'$) by processing an event $e \in E$.

\noindent\textbf{The Formal Model of UPPRESSO.}
Accordingly, we model UPPRESSO as a web system, which is defined as $\mathcal{UWS} = (\mathcal{W}, \mathcal{S}, \mathtt{script}, E^0)$. $\mathcal{W}$ is a finite set of atomic processes in UPPRESSO, which contains an IdP server process, a finite set of web servers for the honest RPs, a finite set of honest browsers, and a finite set of attacker processes. Here, we consider all the RP processes and browser processes are honest, and model an RP or a browser controlled by an adversary as an atomic attacker process.\
$\mathcal{S}$ is a finite set of scripting processes, which contains {\sf script\_rp}, {\sf script\_idp} and {\sf script\_attacker}, where {\sf script\_rp} and {\sf script\_idp} are honest scripts downloaded by an RP process and the IdP process, and {\sf script\_attacker} denotes a script downloaded by an attacker process that exists in all browser processes.
The details about each process are described in Appendix.


\subsection{Proof of Security}
Let's review the security properties in Section~\ref{subsec:basicrequirements}, including \textbf{user identification}, \textbf{RP designation}, \textbf{confidentiality} and \textbf{integrity}. In this section, we will give the brief illustration about how UPPRESSO guarantees the security properties, while the detailed process of proof is provided in the Appendix.

\noindent\textbf{User identification.}
The RP can always derive the constant and correct account from the changing $PID_U$. That is, the $PID_{RP}$ equal to $N_UID_{RP}$ (verified by RP), and $PID_U$ equals to $ID_UPID_{RP}$. The $PID_{RP}$ and $PID_U$ are protected by IdP generated signature. Thus, an RP can always derive the constant account $A$ equalled with $ID_UID_{RP}$.

\noindent\textbf{RP designation.}
For an honest RP, it only accepts the identity token carrying the $PID_{RP}$ equalled with $N_UID_{RP}$, while the $N_U$ is guaranteed by IdP issued registration result. Thus, the identity proof would only be accepted by the owner of $ID_{RP}$, as it is impossible to make another RP accept it by providing a malicious $N_U$. The detailed process of proof is shown in Appendix.

\noindent\textbf{confidentiality.}
Based on the formal model of UPPRESSO (detailed model is shown in Appendix), we can find that, there are no events carrying honest user's identity proofs, cookies and passwords sent to other malicious entities. Thus, the confidentiality of UPPRESSO system is guaranteed. The detailed process of proof is shown in Appendix.

\noindent\textbf{Integrity.}
The IdP-issued proofs include the $Cert$, $RegistrationResult$ and $Token$. We can easily find that the IdP does not send the private key to any processes so that the attackers cannot obtain the private key. The detailed process of proof is shown in Appendix.



\subsection{Proof of Privacy}
\begin{figure*}
  \centering
  \includegraphics[width=0.65\linewidth]{fig/dalgorithm.pdf}
  \caption{The distinguishing algorithm.}
  \vspace{-5mm}
  \label{fig:dalgorithm}
\end{figure*}
In this section, we will give the privacy proof and show that UPPRESSO is secure against both IdP-based login tracing and RP-based identity linkage attacks.

\noindent\textbf{IdP-based login tracing.}
As shown in figure~\ref{fig:process}, the only information that is related to the RP's identity and is accessible to the IdP is $PID_{RP}$, which is converted from $ID_{RP}$ using a random $N_U$. Since $N_U$ is randomly chosen from $\mathbb{Z}_n$ by the user and the IdP ha no control of the process, the IdP should treat $PID_{RP}$ as being randomly chosen from $\mathbb{G}$. So, the IdP cannot recognize the RP nor derive its real identity. Therefore, IdP-based identity linkage becomes impossible in UPPRESSO.

Next, we will prove that UPPRESSO prevents RP-based identity linkage based on the Decisional Diffie-Hellman (DDH) assumption \cite{GoldwasserK16}. Here, we briefly introduce the DDH assumption:
%\noindent\textbf{The DDH Assumption.}
Let $q$ be a large prime and $\mathbb{G}$ denotes a cyclic group of order $n$ of an elliptic curve $E(\mathbb{F}_q)$.
Assume that $n$ is also a large prime. Let $P$ be a generator point of $\mathbb{G}$. The DDH assumption for $\mathbb{G}$ states that for any probabilistic polynomial time (PPT) algorithm $D$, the two probability distributions \{$aP$, $bP$, $abP$\} and \{$aP$, $bP$, $cP$\}, where $a$, $b$, and $c$ are randomly and independently chosen from $\mathbb{Z}_n$, are computationally indistinguishable in the sense that there is a negligible $\sigma(n)$ with the security parameter $n$ such that:
%where $q$ and $n$ are large primitive number, and $P$ is the point of $\mathbb{G}$.
%For any probabilistic polynomial time (PPT) algorithm $D$, the distributions, \{$P$, $aP$, $bP$, $abP$\}$_{a,b \in \mathbb{Z}_n}$ and \{$P$, $aP$, $bP$, $cP$\}$_{a,b,c \in \mathbb{Z}_n}$, are computationally indistinguishable. There is a negligible $\sigma(k)$, where $k$ is the security parameter.
\vspace{-\topsep}
\begin{multline*}
Pr[D(P, aP, bP, abP)=1]-Pr[D(P, aP, bP, cP)=1]=\sigma(n)
\end{multline*}
\vspace{-\topsep}

\vspace{-2mm}
\noindent\textbf{RP-based identity linkage.}
At the very beginning, let us see the data exposed to RP during the authentication in the UPPRESSO system.
We can find that, RP holds $ID_{RP}$,  generates $PID_{RP}$, $T$, $Account$, and receives $N_U$, registration result (containing $PID_{RP}$, hash($N_U$) and $Endpoint_U$), and identity proof (including $PID_{RP}$ and $PID_U$). Then we delete the irrelevant data, $Endpoint_U$, as it is generated randomly and does not participates and further calculation. And the repetitive data should also be deleted, for example, $PID_{RP}$ is generated as $PID_{RP}=N_U \cdot {ID_{RP}}$ so that $PID_{RP}$ can be omitted. Finally, the effective date collected by the RP during the authentication flow is $\langle ID_{RP}, N_U, PID_U\rangle$, while $PID_U$ equals to $ID_U \cdot{(N_U \cdot {ID_{RP})}}$.

%We assume that collusive can correlate $PID_U$s at different RPs and guess if they belong to the same user. Thus,
There is the data set, $\langle ID_{RP1}$, $N_{U1}$, $ID_U \cdot{(N_{U1} \cdot {ID_{RP1})}}$, $ID_{RP2}$, $N_{U2}$, $ID_U' \cdot{(N_{U2} \cdot {ID_{RP2})}}\rangle$, that RP-based identity linkage attack can be considered as RP guesses whether $ID_U$ equals $ID_U'$.
Here, we can define the game, the adversary owns the same ability as the RP. The adversary receive the input $\langle ID_{RP1}$, $N_{U1}$, $ID_U \cdot{(N_{U1} \cdot {ID_{RP1})}}$, $ID_{RP2}$, $N_{U2}$, $ID_U' \cdot{(N_{U2} \cdot {ID_{RP2})}}\rangle$ from the challenger, and returns the result $b$.
The $b$ is set 1, while adversary guess that $ID_U$ equals $ID_U'$, otherwise 0. The game is shown as Figure~\ref{fig:game}.
Therefore, whether the RP-based identity linkage attack is available is equivalent to whether adversary has the advantage on the guessing game.
We define $Pr_1$ is the probability, while the adversary returns $b=1$ as $ID_U$ equals to $ID_U'$. And $Pr_2$ is the probability, while the adversary returns $b=1$ as $ID_U$ does not equal to $ID_U'$.
Adversary has the advantage on the guessing game means that
\vspace{-\topsep}
\begin{equation}
Pr_1-Pr_2>\sigma(n)
\end{equation}
While an adversary has the advantage on the guessing game, based on the challenger and adversary, we can build a PPT distinguishing algorithm $D$ that breaks DDH assumption. The algorithm is shown as Figure~\ref{fig:dalgorithm}. That is, the input of $D$ in the form ($X$,$Y$,$Z$,$N$), while $X$,$Y$,$Z$,$N$ are points on the elliptic curve. The challenger receives the input and set $ID_{RP1}=X$, $ID_{RP2}=Y$, $PID_{U1}=N_{U1} \cdot{Z}$, and $PID_{U2}=N_{U2} \cdot{N}$.  At the end, $D$ returns the $b$ as the result.
Now, we let ($P$, $aP$, $bP$, $abP$) and  ($P$, $aP$, $bP$, $cP$) be the input. Thus there is
\vspace{-\topsep}
\begin{multline*}
\ \ \ \ \ \ \ \ \ \ \ \ \ \ \ \ \ Pr[D(P,aP,bP,abP)=1]=\\ Pr[A(P, N_{U1}, b \cdot{N_{U1}P}, aP, N_{U2},b\cdot{N_{U2} aP})=1]=Pr_1\\
Pr[D(P,aP,bP,cP)=1]=\ \ \ \ \ \ \ \ \ \
\\ Pr[A(P, N_{U1}, b \cdot{N_{U1}P}, aP, N_{U2},c/a \cdot{N_{U2}aP})=1]=Pr_2
\end{multline*}
in the first equation $ID_{U} $ and $ ID_{U}'$ all equal to $b$, while in the second equation $ID_{U}$ does not equal to $ID_{U}'$, so that
\begin{equation}
Pr_1-Pr_2=\sigma(n)
\end{equation}
There is a contradiction between equation (4) and equation (5). Therefore, an adversary cannot have the advantage on the guessing game. Thus, the RP-identity linkage attack is not available.



\section{Implementation and Evaluation}
\label{sec:implementation}
We implemented the UPPRESSO prototype,\footnote{The prototype is open-sourced at https://github.com/uppresso/.}
 and experimentally compared it
 with two open-source SSO systems:
  (\emph{a}) MITREid Connect \cite{MITREid}
    which supports the PPID-enhanced OIDC protocol to prevent the RP-based identity linkage,
     and (\emph{b}) SPRESSO \cite{SPRESSO} preventing the IdP-based login tracing.

\subsection{Prototype Implementation}
\label{subsec:proto-imple}
The identity transformations are defined on
        the NIST P256 elliptic curve.
RSA-2048 and SHA-256 are used as the signature algorithm and the hash function, respectively.

An IdP is built on top of MITREid Connect \cite{MITREid},
    an open-source OIDC Java implementation, %certificated by the OpenID Foundation \cite{OIDF},
    and only a few modifications are needed.
We add only 3 lines of Java code to calculate $PID_U$,
    and 20 lines to modify the way to send identity tokens.
The calculations of $ID_{RP}$ and $PID_U$ are implemented based on Java cryptographic libraries.

The IdP and RP scripts include about 160 and 140 lines of JavaScript code, respectively.
%to provide the functions in Steps 2.1, 2.3, and 4.3.
The cryptographic computations such as $Cert_{RP}$ verification and $PID_{RP}$ negotiation, are finished based on jsrsasign \cite{jsrsasign}, an efficient JavaScript cryptographic library.
%This chrome extension requires permissions  \emph{chrome.tabs} and \emph{chrome.windows} to obtain the RP's URL from the browser's tab,  and \emph{chrome.webRequest} to intercept, block, modify requests to the IdP or RP \cite{chromeExtension}.


We finished a Java RP SDK.
This SDK provides two functions to encapsulate the protocol steps:
 one to request identity tokens,
    and the other to derive accounts. %from identity tokens.
It is implemented based on the Spring Boot framework  with about 500 lines of Java code,
 and cryptographic computations are done based on the Spring Security library.
An RP invokes these functions for the integration,
    by less than 10 lines of Java code.

\subsection{Performance Evaluation}
\label{sec:evaluation}
%We have compared the processing time of each user login in UPPRESSO, with the original OIDC implementation (MITREid Connect) and SPRESSO which only hides the user's accessed RPs from IdP.
\noindent\textbf{Environment.} The evaluation was performed on 3 machines,
one (3.4GHz CPU, 8GB RAM, 500GB SSD, Windows 10) as IdP,
one (3.1GHz CPU, 8GB RAM, 128GB SSD, Windows 10) as an RP,
and the last one (2.9GHz CPU, 8GB RAM, 128GB SSD, Windows 10) as a user.
The user agent is Chrome v75.0.3770.100.
And the machines are connected by an isolated 1Gbps network.

%RP���������ض�������SDK����Լ��Ҫ230��JAVA����
%OIDC��MITREid����SDK��Ҫ��Լ20�е�JAVA���룬��Ҫ��������һ��HTML�ļ���������Լ20��JavaScript���룩
%UPPRESSO��SDK��Ҫ��Լ1100�д��룬����Ҫ���Ӷ����HTML�ļ�
%OIDC��UPPRESSO��SDKֻ��ҪRP�ṩ��������ӿڣ���ַ����Ȼ���ڶ�Ӧ������ӿ������ö�Ӧ��API��ÿ���ӿڶ�Ӧһ��API���ֱ�����ΪtokenRequestGenerate��userAccountAchieve���������Ĵ������̾���SDK���
%SPRESSO���ڽṹ��OIDC��ȫ��ͬ������ʹ����SPRESSO�ṩ��RP�Ŀ�Դ����
%For better evaluation, we build one RP for both UPPRESSO and MITREid Connect which is also implemented  based on Spring Boot framework, as well as the identity proof transmission from user to RP in MITREid Connect is implemented by JavaScript running in RP's web page. The IdP in MITREid Connect is achieved from github~\cite{MITREid}. However, the SPRESSO system is downloaded from~\cite{spressome} containing IdP, RP and FWD.

%A DELL OptiPlex 9020 PC (Intel Core i7-4770 CPU, 3.4GHz, 500GB SSD and 8GB RAM) with Window 10 prox64 works as the IdP. A ThinkCentre M9350z-D109 PC (Intel Core i7-4770s CPU, 3.1GHz, 128GB SSD and 8GB RAM) with  Window 10 prox64 servers as RP. The user adopts Chrome v75.0.3770.100 as the user agent on the Acer VN7-591G-51SS Laptop (Intel Core i5-4210H CPU, 2.9GHz, 128GB SSD and 8GB RAM) with  Windows 10 prox64. For SPRESSO, the extra trusted entity FWD is deployed on the same machine as IdP.
%û����Ϊ������ͬһ��������ʹ�ÿ����䳤��monitorָϵͳ�ļ�����
%The monitor demonstrates that the calculation and network processing of the IdP does not become a bottleneck (the load of CPU and network is in the moderate level).

\noindent\textbf{Setting.}
We compare UPPRESSO with MITREid Connect~\cite{MITREid} and SPRESSO~\cite{SPRESSO},
where MITREid Connect provides open-source Java implementations~\cite{MITREid} of IdP and RP's SDK and SPRESSO provides the JavaScript implementations based on node.js for all entities~\cite{SPRESSO}.
We implemented a Java RP based on Spring Boot framework for UPPRESSO and MITREid Connect, by integrating the corresponding SDK respectively.
The RPs in all three schemes provide the same function, i.e.,   extracting the user's account from the identity proof.
We have measured the time for a user's login at an RP and calculated the average values of $1,000$ measurements.


We divide a login instance into 3 phases according to the life-cycle of the identity proof:
\textbf{\em Identity proof requesting} (Steps 1.1-4.3 in Figure~\ref{fig:process}), the RP (and user) constructing and transmitting the request to IdP;
\textbf{\em Identity proof generation} (Steps 4.4-4.6 in Figure~\ref{fig:process}), the IdP generating identity proof (no user authentication);
and \textbf{\em Identity proof acceptance} (Steps 4.5-5.2 in Figure~\ref{fig:process}), the RP server receives, verifies and parse the identity proof relayed from the IdP.
% and \textbf{Identity proof verification} (Steps 5.1 and 5.2 in Figure~\ref{fig:process}), the RP verifying and parsing the identity proof.

\noindent\textbf{Results.}
The evaluation results are provided in Figure~\ref{fig:evaluation}.
The overall processing times are  113 ms, 308 ms, and 310 ms for MITREid Connect, SPRESSO, and UPPRESSO, respectively. The details are as follows.
The significant overhead  in UPPRESSO is opening the new window and downloading the script from IdP, which needs about 104 ms. This overhead could be reduced by implicitly conducting this procedure when the user visits the RP website.


\begin{figure}
  \centering
  \includegraphics[width=0.9\linewidth]{fig/evaluation2.pdf}
  \vspace{-6mm}
  \caption{The Evaluation.}
  \label{fig:evaluation}
\vspace{-7mm}
\end{figure}

In the requesting, UPPRESSO requires 271 ms in total.
The significant overhead  in UPPRESSO occurs when opening the new window and downloading the script from IdP, which needs about 104 ms. This overhead could be reduced by implicitly conducting this procedure when the user visits the RP website.
SPRESSO needs 19 ms for the RP to obtain IdP's public key and encrypt its domain,
and MITREid Connect only needs 10 ms.



%�����ܼ��ٶ��٣�
In the generation, UPPRESSO needs in total 34 ms, including computing $PID_U$, compared to MITREid Connect which only needs 32 ms.
SPRESSO requires 71 ms, as it implements the IdP based on node.js and therefore can only adopt a JavaScript cryptographic library, while others adopt a more efficient Java library.
%As the processings in SPRESSO and MITREid Connect are the same, the processing time in SPRESSO may be reduced to 32 ms.
%And, then the overall time in SPRESSO will be 269 ms, still larger than 254 ms in UPPRESSO.

%transmission & extraction
In the identity proof acceptance, UPPRESSO only needs  about 6 ms. % where the scripts relay the identity proof to the RP server, and RP server verifies and parses the proof.
MITREid Connect requires the IdP to send the identity proof to the RP's web page which then sends the proof to the RP server through a JavaScript function, and needs 71 ms.
SPRESSO needs the longest time (210 ms) due to the complicated processing at the user's browser.
  %which needs the browser to obtain identity proofs from the IdP, download the JavaScript program from a trusted entity (forwarder), execute the program to decrypt RP's endpoint, send identity proofs to this endpoint (an RP's web page) who finally transmits the proof to RP server.
%In the evaluation, the forwarder and IdP are deployed in one machine, which doesn't introduce performance degradation based on the observation. % as  FWD and IdP work sequently for one login.

%SPRESSO needs a trusted entity named FWD for transmitting the identity proof. We deployed FWD and IdP on the same machine to reduce transmitting delay between them, while the computation never becomes the bottleneck according to the observation.


%In the verification, UPPRESSO needs an extra calculation for $Account$, which then requires  58 ms,
% compared to 14 ms in MITREid Connect and 17 ms in SPRESSO.
\section{Discussions}
\label{sec:discussion}

%Some related issues are discussed in this section.


%\vspace{1mm}
\noindent{\textbf{Applicability of Identity Transformations.}}
Three identity-transformation functions, i.e., $\mathcal{F}_{PID_{RP}}()$, $\mathcal{F}_{PID_U}()$, and $\mathcal{F}_{Acct}()$,
    are applicable to various SSO scenarios
        (e.g., web application, mobile App, and native software),
    because these functions follow the common model of SSO protocols
    and do not depend on any special implementation or runtime environment.
Although the prototype system runs for web applications,
    it is feasible to apply the identity-transformation functions to other SSO scenarios
        to protect user privacy.
%Moreover,
%    because the authentication steps are independent of the UPPRESSO protocol,
%the IdP can choose any appropriate mechanism to authenticate users.


%In this paper, the functions are implemented based on the elliptic curve algorithm and the mechanism provided by OIDC system.
%However, the transformation functions can also be implemented based on other mechanisms according to the protocol and environment, for example, the TPM may be used to take responsibility to transform identities.

\vspace{1mm}\noindent{\textbf{Scalability.}} Adversaries cannot exhaust $ID_{RP}$ or $PID_{RP}$.
$ID_{RP}$ is generated uniquely only in every RP's initial registration,
    and the capacity is $n$ (i.e., the order of $G$). For example, for the NIST P256 elliptic curve, $n$ is approximately $2^{256}$.
As for $PID_{RP}$,  we only need to ensure $PID_{RP}$ is unique among \emph{unexpired} identity tokens,
    the number of which is denoted as $\sigma$.
The probability that at least two $PID_{RP}$s are identical among these $\sigma$ ones,
    is $1-\prod_{i=0}^{\sigma-1}(1-i/n)$.
For example,
    when the IdP serves $10^{8}$ requests per second and the validity period of identity tokens is 10 minutes,
     $\sigma$ is less than $2^{36}$ and the probability is less than $2^{-183}$ for the NIST P256 elliptic curve.
This probability is negligible.

The capacity of accounts at any RP is also $n$.
$\mathbb{E}$ is a finite cyclic group,
    so $ID_{RP} = [r]G$ is also a generator and the order of $ID_{RP}$ is $n$.
Therefore, given an RP,
    a unique account is inherently assigned to every user,
        because $Acct =  [ID_U]ID_{RP} = [u]ID_{RP}$ ($1 < u < n$).

%%DoS
%%Security against ID exhaustion DoS
%%1.�û���������ģ������º���Ƶģ� 2. IdP�� ������OIDC����һ������ǩ���� ����һ��ģ�ݣ���Դ���� --> DoS attack
%\vspace{1mm}\noindent{\textbf{Security against DoS attack.}}
%The adversary may attempt to perform DoS attack on the IdP and RP. For example, the adversary may act as a user to invoke the $PID_{RP}$ registration and identity token generation at the IdP, which requires the IdP to perform two signature generations and one modular exponentiation.
%However, as the user has already been authenticated at the IdP, the IdP could identify the malicious users based on audit, in addition to the existing DoS mitigation schemes.
%%The adversary may act as a user requesting to log into an RP, and make the RP perform two modular exponentiations.
%%The RP could previously calculated a set of $Y_{RP}$s to mitigate this attack.


%��Ȩ��
%����:Ŀǰֻ�ṩ��ʽģʽ,������ģʽ��Э��ļ��ݷ���
\vspace{1mm}\noindent{\textbf{Compatibility with the Authorization Code Flow.}}
In the authorization code flow of OIDC \cite{OpenIDConnect},
    the IdP does not directly issue the identity token;
        instead,
            an access token is forwarded to the RP,
            and then the RP uses this access token to ask for identity tokens from the IdP.
The identity-transformation functions $\mathcal{F}_{PID_{U}}$, $\mathcal{F}_{PID_{RP}}$ and $\mathcal{F}_{Acct}$
    can be integrated into the authorization code flow to generate and verify two types of tokens.
That is,
    $PID_{RP}$ but not $ID_{RP}$ is enclosed in access tokens by the IdP,
        and the RP accepts an access token with matching $PID_{RP}$ only.
Then,
    the RP uses the access token to request the identity token with $PID_U$,
 so the privacy threats are still prevented in authorization code flow.

However,
    as the RP receives the identity token directly from the IdP in this flow,
            it allows the IdP to obtain the RP's network information (e.g., IP address).
To prevent this leakage,
    the RP needs to integrate anonymous networks (e.g., Tor) to ask for identity tokens  in the authorization code flow.

%ƽ̨�޹�
%Similar as SPRESSO, we can integrated for***
%SPRESSO��ƽ̨:����Ŀǰ�ķ���ֻ��browserʵ��
%�������ڵ�ʵ��ʹ��һ���û�����Ҫ�ڻ����ϴ��κζ�����ֻ��װ�Ǹ���������ʵ�ֿ��Խ�һ���ƹ㵽�����������ð�װ***
%\vspace{1mm}\noindent{\textbf{Platform independent.}}
%Our current implementation only requires the user to install a Chrome extension and doesn't need to store any persistent data at the user's machine.
%Moreover, the implementation could  be further extended to remove the Chrome extension, whose JavaScript program is then fetched from the honest IdP. The processing is similar as SPRESSO. That is, 1) the RP's window (window A)  opens a new iframe (window B) to visit the RP's web page, while the RP's web page redirects window B to the IdP; 2) window B downloads the JavaScript  program from the IdP and performs the processing in Steps 1.3, 1.5, 2.1, 3.2 and 3.5; 3) then postMessages are adopted to exchange messages between window A and B for Steps 1.2, 1.3, 1.4, 2.3, 3.1 and 3.5.
%The opener handle of window B is preserved (i.e., window A) for the postMessage, as window A  opens window B with a web page from the RP;
 %and window B is redirected to the IdP with \emph{noreferrer} attribute set, to prevent the browser from sending RP's URL in the Referrer header to the IdP.



%malicious IdP
%���⣬collusive RPs-->PSI PSIʵ�ֵ�ǰ���������ƻ��� PSIʹ��RP�ڲ��ַ��û���˽��ǰ���£�����idenity linkage��
%RP designation --> RP specification / binding
%The primary goal of SSO services is to implement secure user authentication [6], ---> 6��Ӽ����ο�����
%User privacy leaks in all existing SSO protocols and implementations. -->�Ӽ����ο�����






%1) RP's window opens a new window to download a document from RP; 2) the document uses JavaScript to navigate this window to IdP, while the noreferrer attribute is set to clear Referer header and prevent the browser from sending RP's URL to the IdP; 3) this window downloads the  JavaScript  program from IdP and performs the processing in Steps 2.1.3, 2.2.1, 2.4 and 5.2; 4) the opener handle of this new window is preserved (i.e., RP's window) and then postMessages can be adopted to exchange messages between the two windows for Steps 2.1.4, 2.1.6, 2.2.4, 2.3 and 5.3.

%the RP's web page initiates the loading of IdP's web page and sets the attribute (rel="noreferrer") to prevent the browser from sending Referer header with RP's URL to the IdP; 2) the IdP's web page contains the JavaScript program for user's processing in Steps 2.1.3, 2.2.1, 2.4 and 5.2; 3) postMessages is adopted
%; and 4) the correct RP sets subresource integrity (SRI) to prevent the IdP from providing incorrect JavaScript programs.
%A native web technology called subresource integrity (SRI)7 is currently under development at the W3C. SRI allows a document to create an iframe with an attribute integrity that takes a hash value. The browser now would guarantee that the document loaded into the iframe hashes to exactly the given value. So, essentially the creator of the iframe can enforce the iframe to be loaded with a aspecific document. This would enable SPRESSO to automatically check the integrity of FWDdoc without any extensions.



%While USPRESS is designed for semi-honest IdP, it could be extended to prevent


%\noindent{\textbf{Malicious IdP mitigation.}} The IdP is assumed to assign a unique $ID_{RP}$ in $Cert_{RP}$ for each RP and generate the correct $PID_U$ for each login. The malicious IdP may attempt to provide the incorrect $ID_{RP}$ and $PID_U$, which could be prevented by integrating certificate transparency \cite{rfc6962} and user's identifier check \cite{SPRESSO}. With certificate transparency \cite{rfc6962}, the monitors  check the uniqueness of $ID_{RP}$ among all the certificates stored in the log server. To prevent the malicious IdP from injecting any incorrect $PID_U$, the user could provide a nickname to the RP for an extra check as in SPRESSO \cite{SPRESSO}.

\vspace{1mm}
\noindent{\textbf{Collusive Attack by the IdP and RPs.}}
When the IdP is still kept curious-but-honest but shares messages in the login flow (i.e.,
$ID_U$, $PID_{RP}$, and $PID_U$)
        with some collusive RPs,
UPPRESSO still provides secure SSO services,
    provided that the signed identity tokens are sent to the authenticated users only;
however,
    the collusive adversaries are able to trace the users' login activities to these RPs.
Anyway, even in this case, a user's login activities at the other RPs not collusive with the IdP,
        are still protected from the IdP and these collusive RPs,
        because a triad of $t$, $PID_U$ and $PID_{RP}$ is ephemeral and independent of each other.

\vspace{1mm}\noindent{\textbf{Implementation with Browser Extensions.}}
To improve the portability of user agents,
    the user functions of UPPRESSO are implemented by web scripts in the prototype.
However,
    when these functions are implemented with browser extensions,
    it results in better performance.
In this case,
    a user downloads and installs the browser extension,
    before he visits the RPs.
%After some experiments while the IdP and RPs in the prototype system are unmodified,
Experiments show that at least 102 ms will be saved for each login instance (i.e., about 208 ms in total),
    compared with the version implemented with portable web scripts.

\vspace{1mm}\noindent{\textbf{Restriction of the RP Script's Origin.}}
When the IdP script forwards identity tokens to the RP script within the browser,
    the receiver is restricted by the \verb+postMessage+ targetOringin mechanism \cite{postm-targeto},
        to ensure it will forward the tokens to $Enpt_{RP}$,
        which is bound in the RP certificate.
A targetOringin is specified as
    a domain (e.g., \verb+www.RP.com+) and the parts of protocol and port,
        and it requires the RP script's origin accurately matches the targetOringin.
This implies that only one RP is assumed to run on a domain.
If two independent RPs run on one domain but with different endpoints to receive tokens
        (e.g., \verb+https://www.RP.com/RP1/uploadToken+ and \verb+https://www.RP.com/RP2/uploadToken+),
         they cannot be distinguished actually.
Anyway, these RPs are distinguished by the implementation with browser extensions.
%
%We do not consider two hostile RPs on a domain.
%Similar attacks exist in HTTPS, because an HTTPS server certificate
%only binding the domain.
%So, since they share the domain and the private key of HTTPS server certificate,
%    RP1 can always be the MitM attackers of RP2.
%RP1 exploits the vulnerability of HTTPS server certificate is fine to attack,
%    by the shared private key.

\vspace{1mm}\noindent{\textbf{Alternative Way to Bind $ID_{RP}$ and $Enpt_{RP}$.}}
In the prototype, we bind $ID_{RP}$ and $Enpt_{RP}$ by RP certificates.
This binding may be finished in another way:
    $ID_{RP}$ is deterministically computed based on the RP's domain,
        when an RP's domain is unambiguous and unique.
For example,
    $H(DN_{RP})$ \cite{xxxx}. xxx xxx xxx.
Here, $H()$ is a collision-free hash function, and $DN_{RP}$ is an RP's domain.
Then, RP certificates are unnecessary.
    This elimination of RP certificates improves the downloading of IdP scripts,
        and on average xxx ms are saved.

However,
$Acct = [ID_U]ID_{RP}$ changes inevitably,
     once an RP runs on an updated domain.
When such an event happens,
    the RP needs special operations by each user to migrate an account
            to the updated RP system.
Note that this migration of accounts requires extra cooperations explicitly by each user;
    otherwise,
        two collusive RPs are actually able to link a user's accounts across these RPs.


%\vspace{1mm}
%\noindent{\textbf{Active Identity Linkage by Malicious RPs.}}
%When a user logins to multiple RPs \emph{concurrently} on one device as different accounts,
%        malicious RPs might actively correlate the accounts by online means.
%For instance,
%     when the user has logined to $RP_j$ with $Acct_j$,
%         malicious $RP_j$ might try to redirect $Acct_j$ to another $RP_{j'}$ through a hidden iframe in the browser.
%As long as he has logined to $RP_{j'}$,
%     two accounts would be correlated by $RP_{j'}$.
%This active attack appears not only in SSO systems, but also exists in all web applications.
%Anyway,
%    this attack can be prevented through various methods,
% such as checking hidden iframes in the script
%  and detecting the redirection by browser extensions.
%

\section{Conclusion}
\label{sec:conclusion}
We propose UPPRESSO, an untraceable and unlinkable privacy-preserving single sign-on system,
 which protects a user's login activities at different RPs against both curious IdP and collusive RPs.
To the best of our knowledge,
 UPPRESSO is the first approach that defends against both the privacy threats of IdP-based login tracing and RP-based identity linkage at the same time.
To achieve these goals, we convert the privacy dilemma in SSO services into an identity-transformation challenge
 and design three transformation functions based on elliptic curve cryptography,
 where $\mathcal{F}_{ID_{RP} \mapsto PID_{RP}}$ prevents curious IdP from knowing RP's identity,
 $\mathcal{F}_{ID_{U} \mapsto PID_{U}}$ prevents collusive RPs from linking a user based on her identifier,
 and $\mathcal{F}_{PID_{U} \mapsto Account}$ allows RP to derive an identical account for a user in her multiple logins.
The three functions could be integrated with existing SSO protocols,
    such as OIDC,
    to enhance the protection of user privacy,
    without breaking any security guarantees of existing SSO designs.
Moreover, the evaluation of the prototype of UPPRESSO demonstrates
 that it supports an efficient SSO service, where a single login takes only 310 ms on average.





% An example of a floating figure using the graphicx package.
% Note that \label must occur AFTER (or within) \caption.
% For figures, \caption should occur after the \includegraphics.
% Note that IEEEtran v1.7 and later has special internal code that
% is designed to preserve the operation of \label within \caption
% even when the captionsoff option is in effect. However, because
% of issues like this, it may be the safest practice to put all your
% \label just after \caption rather than within \caption{}.
%
% Reminder: the "draftcls" or "draftclsnofoot", not "draft", class
% option should be used if it is desired that the figures are to be
% displayed while in draft mode.
%
%\begin{figure}[!t]
%\centering
%\includegraphics[width=2.5in]{myfigure}
% where an .eps filename suffix will be assumed under latex,
% and a .pdf suffix will be assumed for pdflatex; or what has been declared
% via \DeclareGraphicsExtensions.
%\caption{Simulation Results}
%\label{fig_sim}
%\end{figure}

% Note that IEEE typically puts floats only at the top, even when this
% results in a large percentage of a column being occupied by floats.


% An example of a double column floating figure using two subfigures.
% (The subfig.sty package must be loaded for this to work.)
% The subfigure \label commands are set within each subfloat command, the
% \label for the overall figure must come after \caption.
% \hfil must be used as a separator to get equal spacing.
% The subfigure.sty package works much the same way, except \subfigure is
% used instead of \subfloat.
%
%\begin{figure*}[!t]
%\centerline{\subfloat[Case I]\includegraphics[width=2.5in]{subfigcase1}%
%\label{fig_first_case}}
%\hfil
%\subfloat[Case II]{\includegraphics[width=2.5in]{subfigcase2}%
%\label{fig_second_case}}}
%\caption{Simulation results}
%\label{fig_sim}
%\end{figure*}
%
% Note that often IEEE papers with subfigures do not employ subfigure
% captions (using the optional argument to \subfloat), but instead will
% reference/describe all of them (a), (b), etc., within the main caption.


% An example of a floating table. Note that, for IEEE style tables, the
% \caption command should come BEFORE the table. Table text will default to
% \footnotesize as IEEE normally uses this smaller font for tables.
% The \label must come after \caption as always.
%
%\begin{table}[!t]
%% increase table row spacing, adjust to taste
%\renewcommand{\arraystretch}{1.3}
% if using array.sty, it might be a good idea to tweak the value of
% \extrarowheight as needed to properly center the text within the cells
%\caption{An Example of a Table}
%\label{table_example}
%\centering
%% Some packages, such as MDW tools, offer better commands for making tables
%% than the plain LaTeX2e tabular which is used here.
%\begin{tabular}{|c||c|}
%\hline
%One & Two\\
%\hline
%Three & Four\\
%\hline
%\end{tabular}
%\end{table}


% Note that IEEE does not put floats in the very first column - or typically
% anywhere on the first page for that matter. Also, in-text middle ("here")
% positioning is not used. Most IEEE journals/conferences use top floats
% exclusively. Note that, LaTeX2e, unlike IEEE journals/conferences, places
% footnotes above bottom floats. This can be corrected via the \fnbelowfloat
% command of the stfloats package.




% conference papers do not normally have an appendix


% use section* for acknowledgement
%\section*{Acknowledgment}

% trigger a \newpage just before the given reference
% number - used to balance the columns on the last page
% adjust value as needed - may need to be readjusted if
% the document is modified later
%\IEEEtriggeratref{8}
% The "triggered" command can be changed if desired:
%\IEEEtriggercmd{\enlargethispage{-5in}}

% references section

% can use a bibliography generated by BibTeX as a .bbl file
% BibTeX documentation can be easily obtained at:
% http://www.ctan.org/tex-archive/biblio/bibtex/contrib/doc/
% The IEEEtran BibTeX style support page is at:
% http://www.michaelshell.org/tex/ieeetran/bibtex/
%\bibliographystyle{IEEEtranS}
% argument is your BibTeX string definitions and bibliography database(s)
%\bibliography{IEEEabrv,../bib/paper}
%
% <OR> manually copy in the resultant .bbl file
% set second argument of \begin to the number of references
% (used to reserve space for the reference number labels box)
%\begin{thebibliography}{1}

%\end{thebibliography}
\bibliographystyle{plain}
\bibliography{ref}



%%%%%%%%%%%%%%%%%%%%%%%%%%%%%%%%%%%%%%%%%%%%%%%%%%%%%%%%%%%%%%%%%%%%%%%%%%%%%%%%%%%%%%%%%

%\appendix

\renewcommand{\algorithmicrequire}{\textbf{Input:}}
\newcommand{\deflet}{\textbf{let}}
\newcommand{\mystate}[1]{\STATE \textbf{let} {{}#1}}
\newcommand{\mystop}[1]{\STATE \textbf{stop} \myss{\myangle{{{}#1}}, s'}}
\newcommand{\myss}[1]{${{}#1}$}
\newcommand{\myangle}[1]{\langle {{}#1} \rangle}
\newcommand{\myif}[1]{\IF{\myss{{{}#1}}}}
\newcommand{\myelse}[1]{\ELSIF{\myss{{{}#1}}}}
\newcommand{\SWITCH}[1]{\STATE \textbf{switch} #1\ \textbf{do} \begin{ALC@g}}
\newcommand{\ENDSWITCH}{\end{ALC@g}\STATE \textbf{end switch}}
\newcommand{\CASE}[1]{\STATE \textbf{case} #1\textbf{:} \begin{ALC@g}}
\newcommand{\ENDCASE}{\end{ALC@g}}
\newcommand{\CASELINE}[1]{\STATE \textbf{case} #1\textbf{:} }
\newcommand{\DEFAULT}{\STATE \textbf{default:} \begin{ALC@g}}
\newcommand{\ENDDEFAULT}{\end{ALC@g}}
\newcommand{\DEFAULTLINE}[1]{\STATE \textbf{default:} }




\section{The Security Proofs of UPPRESSO}
\label{ape:model}
We formally analyze  the security properties of UPPRESSO based on the Dolev-Yao style web model as follows.
%which has been widely used in the formal analysis of SSO protocols such as OAuth 2.0~\cite{FettKS16} and OIDC~\cite{FettKS17}.
% ��仰˵�˺ܶ�Σ���appendix�У���ֻ���Ż����ļ�����Ϣ��
We first describe the expressive Dolev-Yao style model of the web infrastructure.
Then,
    we formulate the UPPRESSO system using this model,
        including all entities (i.e., browsers, servers, and scripts) and the transmitted data.
%    present the model of UPPRESSO system, containing normal data and each entities (including browser, servers, and scripts).
%the servers, scripts and the communications among them.
%introduction about how to  indicate the entities (e.g., servers) and communications (e.g., HTTP request) of the web system with Dolev-Yao style web model.
%Then, we describe the model of frequent used data format (e.g., HTTP package) and entity  (e.g., browser).
%Following, we present the modelling UPPRESSO system, containing the servers, scripts and the communications among them.
Finally, we give the security proofs of UPPRESSO, that capture the tracks of all essential data, such as identity token, from its birth to being consumed, to guarantee integrity and confidentiality.



\subsection{The Dolev-Yao Style Web Model}
\label{subsec:webmodel}
Based on Dolev-Yao style models,
 the web infrastructure is modelled as the form of $(\mathcal{W}$, $\mathcal{S}$, $E^0)$.
These notations are explained as below.
\begin{itemize}
\item $\mathcal{W}$ is the set of \emph{atomic processes}.
    An atomic process represents an independent entity in the web system, such as browser and web server.
\item $\mathcal{S}$ is the set of \emph{script processes}. Besides the atomic processes in the web system, there are also entities processing data and communicating with other entities, but unable to run independently without other atomic processes. These entities are denoted as script processes, i.e., JavaScript code with a browser.
\item $E^0$ is the set of self-triggering \emph{events} initially acceptable to the processes in $\mathcal{W}$. An event is the basic communication elements in the model, representing atomic process A sending a message to atomic process B.
\end{itemize}



\vspace{1mm}
\noindent\textbf{Atomic process}. The atomic process is the modelling web nodes, in the form ($I^p$, $Z^p$, $R^p$, $z_0^p$). The definitions of these notations is described as below.
\begin{itemize}
\item $I^p$ is the set of addresses that the process listens to.
\item  $Z^p$ is the set of {\em states} that describes the process. %The state is the formal \textbf{term}.
\item $R^p$ is the {\em relation} between the input, state $s \in Z^p$ and an event $e$,
 to the output, the new state $s' \in Z^p$ and another event $e'$.
It means that, at one time point, while the process is on state $s$ and receive the event $e$, it would transition to state $s'$ and send the event $e'$ to other processes.
\item $z_0^p$ is the initial state.
\end{itemize}

\vspace{1mm}
\noindent\textbf{Term}. Term is the basic element in Dolev-Yao style model, to describe the modelled system.
It contains constants such as ASCII strings and nonce, sequence symbols such as n-ary sequences $\langle \rangle$, $\langle . \rangle$, $\langle . ,. \rangle$, and function symbols that model cryptographic primitives such as $\mathtt{sign}$, $\mathtt{verify}$ and $\mathtt{hash}$.
For example, an HTTP request is expressed as a term containing a type (e.g., $\mathtt{HTTPReq}$), a nonce, a method (e.g., $\mathtt{GET}$ or $\mathtt{POST}$), a domain, a path, URL parameters, request headers and a message body. % over the $\Sigma$ in the sequence symbol format.
So,
an HTTP GET request for the domain {\sf exa.com/path?para=1} with empty header and body can be described as $\langle\mathtt{HTTPReq},n,\mathtt{GET},exa.com,/path,\langle \langle para, 1\rangle \rangle ,\langle \rangle,\langle \rangle \rangle$.

\vspace{1mm}
\noindent\textbf{Operations over terms}. The following operations are defined.
\begin{itemize}
\item \textbf{Equational theory}. Equational theory
 uses the symbol $\equiv$ to represent the congruence relation on terms, and $\not\equiv$ for non-congruence relation.
 For example, while there are the data $Data$, its signature $Sig$, and the corresponding public key $PK$, the relation can be described as $\mathtt{SigVerify}(Data, Sig, PK)\equiv \mathtt{TRUE}$.
\item \textbf{Patten Matching}.  We define the term with the variable $*$ as the pattern, such as \myss{\myangle{a, b, *}}.
The pattern matches any term which only replaces the $*$ with other terms. For instance,  \myss{\myangle{a, b, *}} matches \myss{\myangle{a, b, c}}.
\item \textbf{Retrieve attributes from formatted term}. Formatted term is the data in the specific format, for instance, the HTTP request is the formatted data in the form \myss{\langle\mathtt{HTTPReq}}, \myss{nonce}, \myss{method}, \myss{host}, \myss{path}, \myss{parameters}, \myss{headers}, \myss{body\rangle}.  We assume there is an HTTP request \myss{r :=}  \myss{\langle\mathtt{HTTPReq}},  \myss{n},  \myss{\mathtt{GET}},  \myss{example.com},  \myss{/path},  \myss{\myangle{}},  \myss{\myangle{}},  \myss{\myangle{}\rangle}, here we define the operation on the $r$. That is, the elements in $r$ can be accessed in the form \myss{r.name}, such that \myss{r.method \equiv \mathtt{GET}},  \myss{r.path \equiv /path} and \myss{r.body \equiv \myangle{}}.
\item \textbf{Retrieve attributes from dictionary term}. Dictionary term is the data in the form \myss{\myangle{\myangle{type, value}, \myangle{type, value}, \dotsc}}, for instance the \myss{body} in HTTP request is dictionary data. We assume there is a \myss{body := \myangle{\myangle{username, alice}, \myangle{password, 123}}}, here we define the operation on the $body$. That is, we can access the elements in \myss{body} in the form \myss{body[name]}, such that \myss{body[username] \equiv alice} and \myss{body[password] \equiv 123}. We can also add the new attributes to the dictionary, for example after we set \myss{body[age] := 18}, the \myss{body} are changed into\myss{ \myangle{\myangle{username, alice}, \myangle{password, 123}, \myangle{age, 18}}}.
\end{itemize}

\vspace{1mm}
\noindent\textbf{State}. State is the term consisted of basic terms, to sketch the atomic process at the point of time. For example, the state of a login server can be simplistically described as the term $\langle SessionList,UserList \rangle$,
    while the $SessionList$ maintains the visitors' cookies with the corresponding login state,
    and the $UserList$ maintains all user identities (either logined or unlogined) and their corresponding credential verifiers.


\vspace{1mm}
\noindent\textbf{Relation}. The relation represents the model of procedure, showing how the entity deal with the received message. For example, while the login server receive the login request from a visitor (the request is denoted as the event $e$), the input of the relation is $e$ and current state $\langle SessionList,UserList \rangle$.   The server verifies the visitor's identity with the $UserList$ and add the login result into $SessionList$ (transitioning to $SessionList'$). Thus the output is the login response $e'$ and the new state $\langle SessionList',UserList \rangle$.


\vspace{1mm}
\noindent\textbf{Script process}. A script process is the formulation of browser scripts as a relation $R$,
    representing the server-defined functions,
    while the input and output are handled by the browser.
A script process is addressed by the \verb+Origin+ property, as the the protocol, hostname and port number of a URL. That is, while a script sends the messages to another script process and set the \verb+Origin+ value as the IdP domain,
 only the script downloaded from IdP server receives this message.


\vspace{1mm}
\noindent\textbf{Event}. The event  the formal term $\langle a, b, m \rangle$, where $a$ and $b$ represent the addresses of the sender and receiver respectively, and $m$ is the message transmitted.





\subsection{The Formulation of UPPRESSO}
In this section, we introduce the model of honest UPPRESSO system, in the form of $(\mathcal{W}$, $\mathcal{S}$, $E^0)$.
$\mathcal{W}$ is the set of atomic processes, including IdP server processes, RP server processes and browsers.
$\mathcal{S}$ is the set of script processes, containing of IdP script process and RP script process.
$E^0$  is the set of events acceptable to the processes in $\mathcal{W}$, which is to be illustrated in the individual processes part.


We only focus on the model of servers, browsers and the plain message transmissions among them in this paper, and neglect the HTTPS requests and other necessary Internet facilities (e.g., DNS server) deployed in the UPPRESSO system.
Because we assume that the internet architecture is well  built and HTTPS is well implemented, so that an adversary cannot conduct the attacks targeting these elements.

%This section is organized as follows. At first, we introduce the model of normal basic data (representing data in the form of terms), such as HTTP package, which is to be used to construct more complex models. Then we give the detailed description about the models IdP and RP servers in UPPRESSO system. At the end, the models of browser and scripts are also introduced.


\subsubsection{The Data Transmitted}
At the very beginning, we provide the modelling normal data, such HTTP messages, used to construct UPPRESSO model.

\vspace{1mm}
\noindent\textbf{HTTP Messages}.
An HTTP request message is the term of the form
\begin{multline*}
\ \ \ \ \ \ \ \ \langle \mathtt{HTTPReq}, nonce, method, host, path, \\
parameters, headers, body\rangle \ \ \ \ \ \ \ \
\end{multline*}
An HTTP response message is the term of the form
\begin{equation*}
    \myangle{\mathtt{HTTPResp}, nonce, status, headers, body}
\end{equation*}
The details are defined as follows:
\begin{itemize}
\setlength\itemsep{-2pt}
 \item \myss{\mathtt{HTTPReq}} and \myss{\mathtt{HTTPResp}} denote the types of messages.
 \item \myss{nonce} is a random number that maps the response to the corresponding request.
 \item \myss{method} is one of the HTTP methods, such as \myss{\mathtt{GET}} and \myss{\mathtt{POST}}.
 \item \myss{host} is the constant string domain of visited server.
 \item \myss{path} is the constant string representing the concrete resource of the server.
 \item \myss{parameters} contains the parameters carried by the url as the form \myss{\myangle{\myangle{name, value}, \myangle{name, value}, \dotsc}}, for example, the \myss{parameters} in the url \myss{http://www.example.com?type=confirm}  is \myss{\myangle{\myangle{type, confirm}}}.
 \item \myss{headers} is the header content of each HTTP messages as the form \myss{\myangle{\myangle{type, value}, \myangle{type, value}, \dotsc}}, such as \myss{\langle\myangle{Referer, http://www.example.com},} \myss{\myangle{Cookies, c}\rangle}.
 \item \myss{body} is the body content carried by HTTP \myss{\mathtt{POST}} request or HTTP response in the form \myss{\myangle{\myangle{type, value}, \myangle{type, value}, \dotsc}}.
  \item \myss{status} is the HTTP status code defined by HTTP standard, such as 200, 302 and 404.
\end{itemize}

\vspace{1mm}\noindent\textbf{URL}.
URL is a term \myss{\myangle{\mathtt{URL}, protocol,host,path,parameters}}, where \myss{\mathtt{URL}} is the type, \myss{protocol} is chosen in \myss{\{\mathtt{S}}, \myss{\mathtt{P}\}} as \myss{\mathtt{S}} stands for HTTPS and \myss{\mathtt{P}} stands for HTTP. The \myss{host, path}, and \myss{parameters} are the same as in HTTP messages.

\vspace{1mm}\noindent\textbf{Origin}.
An Origin is a term \myss{\myangle{host, protocol}} that can represent the server that a script is downloaded from, where \myss{host} and \myss{protocol} are the same as in URL.

\vspace{1mm}\noindent\textbf{POSTMESSAGE}.
PostMessage is used in the browser for transmitting messages between scripts. The postMessage package is defined as the form \myss{\myangle{\mathtt{POSTMESSAGE}, target, Content, Origin}}, where \myss{\mathtt{POSTMESSAGE}} is the type, \myss{target} is the constant nonce which stands for the receiver, \myss{Content} is the message transmitted and \myss{Origin} restricts the receiver's origin.

\vspace{1mm}\noindent\textbf{XMLHTTPREQUEST}.
XMLHTTPRequest is the HTTP message transmitted  by scripts in the browser. That is, the XMLHTTPRequest is converted from the HTTP message by the browser. The XMLHTTPRequest package is defined as the term in the form \myss{\myangle{\mathtt{XMLHTTPREQUEST}, URL, methods, Body, nonce}} can be converted into HTTP request message by the browser, and \myss{\myangle{\mathtt{XMLHTTPREQUEST}, Body, nonce}}  is converted from HTTP response message.

\subsubsection{The Servers of IdP and RP}
An atomic process is a tuple $p=$ $(I^p, Z^p, R^p,z_0^p )$, containing the addresses, states, and relations.
Here, we will focus on the state form and relation $R$. They can describe that what kind of event can be accepted by the process in each state, and the content of new output events and states.


\vspace{1mm}\noindent\textbf{IdP Server Process}.
The state of IdP server process is described as a term in the form \myss{\myangle{Issuer, SK, SessionList, users, RPs, Validity, Tokens}}.
That is,  the state of an IdP server  at the point in time can be sketched using these attributes.
Other data stored at IdP but not used during SSO authentication are not mentioned here.
\begin{itemize}
\setlength\itemsep{-2pt}
\item \myss{Issuer} is the identifier of IdP.
%\item \myss{p} is the large prime mentioned before.
\item \myss{SK} is the private key used by IdP to generate signatures.
\item \myss{SessionList} is the term in the form of \myss{\myangle{\myangle{cookie, session}}}, the cookie uniquely identifies the session storing the  browser uploaded messages.
\item \myss{UserList} is the set of user's information, including  \myss{username, credential, ID_U}  and other attributes.
\item \myss{RPs} is the set of RP information which consists of ID of RP \myss{(PID_{RP})}, \myss{Endpoints} (i.e., the set of RP's validity endpoints) and \myss{Validity}.
\item \myss{Validity} is the validity for IdP generated signatures.
\item \myss{Tokens} is the set of IdP generated Identity tokens.
\end{itemize}
To make the description clearer, we define the \myss{functions} to express the relations.
\begin{itemize}
\setlength\itemsep{-2pt}
\item \myss{\mathtt{CredentialVerify(credential)}} is used to authenticate the user.
\item \myss{\mathtt{UIDOfUser(credential)}} is used to search the user's \myss{ID_U}.
\item \myss{\mathtt{ListOfPID()}} is the set of IDs of registered RP.
\item \myss{\mathtt{EndpointsOfRP(r)}} is the set of endpoints registered by the RP with ID \myss{r}.
%\item \myss{\mathtt{ModPow(a, b, c)}} is the result of \myss{a^b \mod c}.
\item \myss{\mathtt{Multiply(P, a)}} is the result of \myss{aP}, where $P$ is the point on elliptic curve and $a$ is the integer.
\item \myss{\mathtt{CurrentTime()}} is the system current time.
\end{itemize}

The relation of IdP process $R^i$ is shown as Relation~\ref{alg1} in Appendix~\ref{ape:alg}.





\vspace{1mm}\noindent\textbf{RP server process}.
The state of RP server process is a term in the form \myss{\myangle{ID_{RP}, Enpt, IdP, Cert, SessionList, UserList}}. %Other attributes are not mentioned here.

\begin{itemize}
\setlength\itemsep{-2pt}
\item \myss{ID_{RP}} and \myss{Enpt} are RP's registered information at IdP.
\item \myss{Cert} is the IdP signed RP information containing \myss{ID_{RP}, Enpt} and other attributes.
\item \myss{IdP} is the term of the form \myss{\langle ScriptUrl}, %\myss{q},
    \myss{PK \rangle}, where \myss{ScriptUrl} is the site to download IdP script, %\myss{q} is the large prime defined before,
     and \myss{PK} is the public key used to verify the IdP signed messages.
\item \myss{SessionList} is same as it in IdP process.
\item \myss{UserList} is the set of users registered at this RP, each user is uniquely identified by the \myss{Acct}.
\end{itemize}

The new \myss{functions} are defined as follows:
\begin{itemize}
\setlength\itemsep{-2pt}
 \item \myss{\mathtt{Inverse}(a)}  calculates the trapdoor in UPPRESSO system.
  \item \myss{\mathtt{Random}()} generates a fresh random number.
  \item \myss{\mathtt{AddUser}(Acct)} add the new user with \myss{Acct} into RP's user list.
\end{itemize}
The relation of RP process $R^r$ is shown as Relation~\ref{alg2} in Appendix~\ref{ape:alg}.


\subsubsection{Browser}
%In UPPRESSO, we assume that the browsers are honest and well developed, so that, we neglect  the detailed process of browser, such as constructing and parsing the HTTP package.
We analyze how the browsers interact with other parties in UPPRESSO system.
Moreover, in UPPRESSO system, the message transmitted through a browser is constructing and parsing by the script, so that we only focus on how the script process runs in the browser.

In the browser, a window is the basic unit that shows the content to user.
In a window, the document %(Document Object Model, DOM)
 represents the whole page contained in this window.
The script is one of node in the document.
We introduce the windows and documents of the browser model, which provides inputs and parses the outputs of the script process.
The model of opened windows, documents and downloaded scripts are parts of the state of a browser.
%For a better reading experience, in this section, we only introduce the model of window and document, and the models of IdP and RP scripts are provided in next section



\vspace{1mm}
\noindent\textbf{Window}. A window \myss{w} is a term of the form \myss{w = \myangle{ID_w, documents, opener}}, representing the  the concrete browser window in the system. The \myss{ID_w} is the window reference to identify each windows.
The \myss{documents} is the set of documents (defined below) including the current document and cached documents (for example, the documents can be viewed via the ``forward" and ``back" buttons in the browser).
The \myss{opener} represents the window from which this window is created,
    for instance, while a user clicks the href in document \myss{d} and it creates a new window \myss{w},
    there is \myss{w.opener \equiv d.ID_d}.

\vspace{1mm}
\noindent\textbf{Document}. A document \myss{d} is a term of the form
\begin{multline*}
  \ \ \ \langle ID_d, location, referrer, script, scriptstate, \\
  scriptinputs, subwindows\rangle \ \ \
\end{multline*}
where document is the page content in the window.  The \myss{ID_d} locates the document. \myss{Location} is the URL where the document is loaded. \myss{Referrer} is same as the Referer header defined in HTTP standard. The \myss{script} is the script process downloaded from each servers. \myss{scriptstate} is define by the script, different in each scripts. The \myss{scriptinputs} is the message transmitted into the script process. The \myss{subwindows} is the set of \myss{ID_w} of document's created windows.
 %\myss{active} represents whether this document is active or not.



\subsubsection{The IdP Script and the RP Script}
The script process is the dependent process relying on the browser, which can be considered as a relation \myss{R} mapping a message input and a message output. And finally the browser will conduct the command in the output message. Here we give the description of the form of input and output.
\begin{itemize}
\setlength\itemsep{-2pt}
\item \textbf{Script Message Input. } The input is the term in the form
\begin{multline*}
\langle tree, docID, scriptstate, stateinputs,cookies,\\
localStorage, sessionStorage, ids, secret \rangle
\end{multline*}
\item \textbf{Script Message Output. }The output is the term in the form
\begin{multline*}
\ \ \ \ \ \langle scriptstate, cookies, localStorage, \\
sessionStorage, command \rangle \ \ \ \ \
\end{multline*}
\end{itemize}
The \myss{tree} is the structure of the opened windows and documents, which are visible to this script. \myss{DocID} is the document $ID_d$, representing a document. The  \myss{Scriptstate} is a term of the form defined by each script. \myss{Scriptinputs} is the message transmitted to script. However, the \myss{scriptinputs} is defined as standardized forms, for example, postMessage is one of the forms of \myss{scriptinputs}. \myss{Cookies} is the set of cookies that belong to the document's origin. \myss{LocalStorage} is the storage space for browser and \myss{sessionStorage} is the space for each HTTP sessions.  \myss{Ids} is the set of user IDs while \myss{secret} is the password to corresponding user ID. The \myss{command} is the operation which is to be conducted by the browser. Here we only introduce the form of commands used in UPPRESSO system. We have defined the postMessage and XMLHTTPRequest (for HTTP request) message which are the \myss{commands}. Moreover, a term in the form \myss{\myangle{\mathtt{IFRAME}, URL, WindowID}} asks the browser to create this document's subwindow and it visits the server with the URL.

\vspace{1mm}
\noindent\textbf{IdP script process}
The state of IdP script process \myss{scriptstate} is a term in the form \myss{\myangle{IdPDomain, Parameters, phase, refXHR}}, where
\begin{itemize}
\setlength\itemsep{-2pt}
\item \myss{IdPDomain} is the IdP's host.
\item \myss{Parameters} is used to store the parameters received from other processes.
%\item \myss{p} is the large prime defined before.
\item \myss{phase} is used to label the procedure point in the login.
\item \myss{refXHR} is the nonce to map HTTP request and response.
\end{itemize}
The new \myss{functions} are defined as follows.
\begin{itemize}
\setlength\itemsep{-2pt}
 \item \myss{\mathtt{PARENTWINDOW}(tree,docID)}. The first parameter is the input relation tree defined before, and the second parameter is the $ID_d$ of a document. The output returned by the function is the current window's opener's $ID_w$ (null if it doesn't exist nor it is invisible to this document).
  \item \myss{\mathtt{CHOOSEINPUT}(inputs,pattern)}. The first parameter is a set of messages, and the second parameter is a pattern. The result returned by the function is the message in \myss{inputs} matching the \myss{pattern}.
  \item \myss{\mathtt{RandomUrl}()} returns a newly generated host string.
    \item \myss{\mathtt{CredentialofUser}(user,secret)} creates the user's credential for authentication.
\end{itemize}
The relation of IdP script process $script\_idp$ is shown in Appendix~\ref{ape:alg} Relation~\ref{alg3}.





\vspace{1mm}
\noindent\textbf{RP script process}
The state of RP script process \myss{scriptstate} is a term in the form \myss{\langle IdPDomain}, \myss{RPDomain}, \myss{Parameters}, \myss{phase}, \myss{refXHR\rangle}. The \myss{RPDomain} is the host string of the corresponding RP server, and other terms are defined in the same way as in IdP script process.

Here, we define the function \myss{\mathtt{SUBWINDOW}(tree, docID)}, which takes the \myss{tree} defined above and the current document's \myss{ID_d} as the input. And it selects the \myss{ID_w} of the first window opened by this document as the output. However, if there is no opened windows, it returns  null.

The relation of RP script process $script\_rp$ is shown in Appendix~\ref{ape:alg} Relation~\ref{alg4}.

\subsection{Proofs of Security}
The security definition of UPPRESSO is that the system must ensure only the legitimate user logins to an honest RP under his unique account.
We consider the visits to RP's resource paths are controlled by the visitors' cookies,
 so that the attacker could break the security only when he owns the cookie bound to the honest user.
However, according to the same origin policy, a cookie is never leaked to any adversary, so the UPPRESSO system is vulnerable only when a adversary's cookie is bound with the honest user's identity.
Based on the model of RP server process, at Line 64-83, we find that only when the RP accepts a visitor's identity token,
    it will bind this visitor with the identity derived from this token.
Moreover, the RP must also make sure that, an honest user would always login to the server with her constant identity.
Therefore, we propose the definition about the secure UPPRESSO system.

\begin{definition}
%Let $\mathcal{UWS}$ be a UPPRESSO web system, $\mathcal{UWS}$ is secure \textbf{iff} for any honest RP $r$ $\in $ $\mathcal{W}$, it would not derive an honest user's identity from the token while the token belongs to an adversary.
An SSO system is secure {\em iff} for any honest RP, it would not derive an honest user's identity from the token while the token belongs to an adversary, or derive an incorrect identity from the token sent by an honest user.
\end{definition}

That is, the requirements of secure SSO system can be described as follows.
\begin{itemize}
\item An adversary cannot achieve an honest user's token issued for an honest RP ({\em confidentiality}).
\item An honest RP would not accept the token issued for another RP ({\em RP designation}).
\item Attacker cannot forge or modify the IdP-issued proofs ({\em integrity}).
\item An honest user must always login to the honest RP on her own identity ({\em user identification}).
\end{itemize}

We assume that all the network messages are transmitted using HTTPS, postMessage messages are protected by the browser, and the browsers are  honest, so web attackers can never break the security of UPPRESSO.
Here, we are going to prove that, in the UPRESSO system, the following lemmas are always workable.



\begin{lemma}
An adversary cannot achieve an honest user's token issued for an honest RP ({\em confidentiality}).
\label{lemma:confidentiality}
\end{lemma}
\begin{proof}
Here we only need to prove that attackers cannot receive \myss{Token} from other honest processes.
\begin{itemize}
\setlength\itemsep{-2pt}
\item Attacker cannot obtain the \myss{Token} from RP server.  We check all the messages sent by the RP server at Line 4, 7, 19, 25, 31, 36, 45, 55, 61, 66, 74, 84 in Relation~\ref{alg2}. It is easy to prove that the RP server does not send any \myss{Token} to other processes.
\item Attacker cannot obtain the \myss{Token} from RP script. The  messages sent by RP script can be classified into two classes. 1) The messages at Line 18, 36, 56 in Relation~\ref{alg4}  are sent to the RPDomain which is set at Line 4, so that attackers cannot receive these messages. 2) The messages at Line 26,  46 only carry the contents received from RP server, and we have proved that RP server does not send any \myss{Token}. Therefore, attackers cannot receive the \myss{Token} from RP script.
\item Attacker cannot obtain the \myss{Token} from IdP server.  Considering the messages at Line 4, 11, 15, 22, 25, 35, 43, 50, 66 in Relation~\ref{alg1}, we find that only the message at Line 66 carries the \myss{Token}. This \myss{Token} is generated at Line 64, following the trace where the \myss{Content} at Line 62, the \myss{PID_U} at Line 60, the \myss{ID_U} at Line 59, the \myss{session} at Line 47, and finally the \myss{cookie} at Line 46. That is, the receiver of \myss{Token}  must be the owner of the \myss{cookie} in which session that saves the parameter \myss{ID_U} . The \myss{ID_U} is set at Line 14 after verifying the password and never modified. As we assume that passwords cannot be known to attackers,  attackers cannot obtain the \myss{Token} from IdP server.
\item Attacker cannot obtain the \myss{Token} from IdP script. As the proof provided above, only IdP sends the \myss{Token} with the message at Line 66 in Relation~\ref{alg1}, the IdP script can only receive the \myss{Token} at  Line 99 in Relation~\ref{alg3}. Here we are going to prove that the token issued for user $u$'s login to RP $r$ (denoted as \myss{t(u,r)}) can only be  sent to the corresponding RP server through IdP script. The receiver of \myss{t(u,r)} is restricted by the \myss{RPOrigin} at Line 100, which is set at Line 55. The host in the \myss{RPOrigin} is verified using the one included in \myss{Cert} at Line 51. If the \myss{Cert} belong to \myss{r}, the attacker cannot obtain the \myss{t(u,r)}. Now we give the proof that the \myss{Cert} belongs to \myss{r}. Firstly we define the negotiated \myss{PID_{RP}} in \myss{t(u,r)} as \myss{p}. That is the \myss{PID_{RP}} at Line 69 in Relation~\ref{alg2}  must equal to \myss{p} and the \myss{PID_{RP}} is verified at Line 44 with the \myss{RegistrationToken}. This verification cannot be bypassed due to the state check at Line 60. At the same validity period, the IdP script needs to send the registration request with same \myss{p}  and receive the successful registration result. As the IdP checks the uniqueness of \myss{PID_{RP}} at  Line 31 in Relation~\ref{alg1}. The \myss{r} and IdP script must share the same \myss{RegistrationToken}. As the \myss{RegistrationToken} contains the \myss{\mathtt{Hash}(t)}, the IdP script and \myss{r} must share the same \myss{ID_{RP}}. Therefore, the  \myss{Cert} saved as the IdP scriptstate parameter must belong to \myss{r}.
\end{itemize}
Therefore, attackers cannot  learn users' valid identity tokens.
\end{proof}


\begin{lemma}
An honest RP would not accept the token issued for another RP ({\em RP designation}).
\label{lemma:RPdesignation}
\end{lemma}
\begin{proof}
We can find that an RP server finally accept the $PID_U$ in an token at Line 77 in Relation~\ref{alg2}, after the verification at Line 73. The $PID_{RP}$ retrieved at Line 69, must be verified at Line 44.
The content verified at Line 43 is received from IdP server, generated at Line 40, Relation~\ref{alg1} and protected by the signature generated at Line 41.
According to the verification at 31, the content containing $PID_{RP}$ is unique during the valid time.
Moreover, according to Line 42, 44 (the $t$ used at Line 42 is set  at  Line 15, used for generating $PID_{RP}$ at Line 13), the $PID_{RP}$ must be resigned for this RP, as no other RP cannot generate the same $PID_{RP}$ with this $t$.
As it has been proved at Lemma~\ref{lemma:confidentiality}, an malicious RP cannot receive the valid token issued for another RP, therefore, the adversary would not receive the token including an honest RP's $PID_{RP}$ from honest user.
\end{proof}


\begin{lemma}
Attacker cannot forge or modify the IdP-issued proofs ({\em integrity}).
\label{lemma:integrity}
\end{lemma}
\begin{proof}
The IdP-issued proofs include the \myss{Cert} used in \myss{script\_idp}, the \myss{RegistrationResult} and \myss{Token} used in \myss{P^r} . We can easily find that the IdP does not send the private key to any processes so that the attackers cannot obtain the private key. Then we only need to prove that all the proofs are well verified.
\begin{itemize}
\setlength\itemsep{-2pt}
\item \myss{Cert} is used at Line 21, 52 in Relation~\ref{alg3}. At Line 21, the \myss{Cert} has already been verified at Line 16. At Line 52, the \myss{Cert} is picked from the state parameters, and the cert parameter is set at Line 19.  At Line 19, the \myss{Cert} has already been verified at Line 16.
At Line 16 the \myss{Cert} is verified with the public key in the scriptstate, where the key is considered initially honest and the key is not modified at Relation~\ref{alg3}. Therefore, \myss{Cert} cannot be forged or modified.
\item \myss{RegistrationResult} is used in Relation~\ref{alg2} from Line 35 to 55, which is verified at Line 30. The public key is initially set in the RP and never modified. Therefore, \myss{RegistrationResult} cannot be forged or modified.
\item \myss{Token} is used in Relation~\ref{alg2} from Line 69 to 84 after Line 65 where it is verified.  As proved before, the public key is honestly set and never modified. Therefore, \myss{Token} cannot be forged or modified.
\end{itemize}
Therefore, this lemma is proved.
\end{proof}


\begin{lemma}
An honest user must always login to the honest RP on her own identity ({\em user identification}).
\label{lemma:userIdentification}
\end{lemma}
\begin{proof}
We can find that, the RP accepts the user's identity at Line 83, Relation~\ref{alg2}. And the identity is generated at Line 79, based on the $PID_U$  retrieved from the token and the trapdoor $t^{-1}$.
The $t^{-1}$ is generated at Line 14, set at Line 17, and never changed, as the multiplicative inverse of $t$.
According to Lemma~\ref{lemma:integrity}, only the IdP can generate an token, so that the token must be issued at Line 64, Relation~\ref{alg1}.
IdP generates the $PID_U$ based on the $PID_{RP}$ and user's $ID_U$.
According to Lemma~\ref{lemma:RPdesignation}, while the $PID_{RP}$ is accepted by an RP, the $PID_{RP}$ must be the generated between the RP and user.
According to Lemma~\ref{lemma:confidentiality}, the $ID_U$ must belongs to the honest user.
Therefore, the user's identity must equal with the constant $ [ID_U]ID_{RP}$.

Moreover, an adversary may lead the honest user to upload the adversary's token to an RP.
While the honest user has already negotiated the $PID_{RP}$ with the RP.
The opener of the IdP script must be the RP script.
As the $t$ generated at Line 7, Relation~\ref{alg3}, and $PID_{RP}$ generated at Line 21, Relation~\ref{alg3} and Line 13, Relation~\ref{alg2}.
The $t$ is only sent to RP script at Line 8, Relation~\ref{alg3}, and RP server at Line 18, Relation~\ref{alg4}.
The $PID_{RP}$ is only sent to IdP server at Line 28 and 90, Relation~\ref{alg3}. Moreover, the registration result including the $PID_{RP}$ is sent to IdP script at Line  35, Relation~\ref{alg1}, RP script at Line 40, Relation~\ref{alg3}, RP server at Line 36, Relation~\ref{alg4}. An adversary can never know the $PID_{RP}$ negotiated between the honest user and RP.
While the honest user has not negotiated the $PID_{RP}$ with the RP yet, the \verb+session+[$PID_{RP}$] at Line 69, Relation~\ref{alg2} and other similar attributes must be empty, as the $Cookie$ used at Line 58 belongs to the honest user and the negotiation between Line 10-20 does not conducted by the user.
Therefore, an adversary cannot lead the honest RP to accept the malicious token from the honest user.
\end{proof}

In conclusion, four lemmas prove that the UPPRESSO system satisfies the requirements of the secure SSO system.
Therefore, the following theorem is proved.
\begin{theorem}
UPPRESSO is the secure SSO system.
\end{theorem}






\onecolumn

\section{Relations}
\label{ape:alg}
Relation takes an event and a state as input and returns a new state and a sequence of events.
In a web system,
     it represents how the entities, such as the server, deal with the received messages, and send message to other entities.
This section provides the detailed relations of the processes (including atomic and script processes) in UPPRESSO.


\subsection{IdP process}
The IdP process only accepts the events, downloading scripts, accessing the login status of a cookie owner, authenticating with password, registering the $PID_{RP}$, and requiring the identity token.
The detailed procedure of dealing with these events is shown as follows.

\begin{breakablealgorithm}
  \caption{\textbf{IdP\_Process\_Relation}}
  \label{alg1}
  \begin{algorithmic}[1]
  \REQUIRE \myss{\myangle{a, b, m}, s}
  \mystate{\myss{s':=s}}
  \mystate{\myss{n, method, path, parameters, headers, body} \textbf{such that}}\\
  \ \ \myss{\myangle{\mathtt{HTTPReq},n,method,path,parameters,headers,body} \equiv m}\\
  \ \ \textbf{if} \myss{possible}; \textbf{otherwise} stop \myss{\myangle{}, s'}
  \myif{path \equiv /script}
  \mystate{\myss{m':=\myangle{\mathtt{HTTPResp},n,200, \myangle{}, \mathtt{IdPScript}}}}
  \mystop{b, a, m'}
  \myelse{path \equiv /login}
  \mystate{\myss{cookie := headers[Cookie]}}
  \mystate{\myss{session := s'.SessionList[cookie]}}
  %\mystate{\myss{username:=body[username]}}
  \mystate{\myss{credential:=body[credential]}}
  \myif{CredentialVerify(credential)}
  \mystate{\myss{m' :=\myangle{\mathtt{HTTPResp},n,200,\myangle{},\mathtt{LoginFailure}}}}
  \mystop{b, a,m'}
  \ENDIF
  \mystate{\myss{session[uid] := \mathtt{UIDOfUser}(credential)}}
  \mystate{\myss{m' :=\myangle{\mathtt{HTTPResp},n,200,\myangle{},\mathtt{LoginSucess}}}}
  \mystop{b, a,m'}
  \myelse{path \equiv /loginInfo}
  \mystate{\myss{cookie := headers[Cookie]}}
  \mystate{\myss{session := s'.SessionList[cookie]}}
  \mystate{\myss{uid := session[uid]}}
  \myif{uid \not\equiv \mathtt{null}}
  \mystate{\myss{m' := \myangle{\mathtt{HTTPResp},n,200,\myangle{},\mathtt{Logged}}}}
  \mystop{b, a,m'}
  \ENDIF
  \mystate{\myss{m' := \myangle{\mathtt{HTTPResp},n,200,\myangle{},\mathtt{Unlogged}}}}
  \mystop{b, a,m'}
  \myelse{path \equiv /dynamicRegistration}
  \mystate{\myss{PID_{RP} := body[PID_{RP}]}}
 \mystate{\myss{Enpt := body[Enpt]}}
  \mystate{\myss{Nonce := body[Nonce]}}
  \myif{PID_{RP} \in \mathtt{ListOfPID()}}
  \mystate{\myss{Content :=\myangle{Fail, PID_{RP}, Nonce}}}
  \mystate{\myss{Sig := \mathtt{SigSign}(Content, s'.SK)}}
  \mystate{\myss{RegistrationResult := \myangle{Content, Sig}}}
  \mystate{\myss{m' := \myangle{\mathtt{HTTPResp}, n, 200, \myangle{}, RegistrationResult}}}
  \mystop{b, a,m'}
  \ENDIF
  \mystate{\myss{Validity := \mathtt{CurrentTime} ()+ s'.Validity}}
 \mystate{\myss{s'.RPs := s'.RPs +  ^{\myangle{}} \myangle{PID_{RP}, Enpt, Validity}}}
  \mystate{\myss{Content := \myangle{OK, PID_{RP}, Nonce, Validity}}}
  \mystate{\myss{Sig := \mathtt{Sig}(Content, s'.SK)}}
  \mystate{\myss{RegistrationResult := \myangle{Content, Sig}}}
  \mystate{\myss{m' := \myangle{\mathtt{HTTPResp}, n, 200, \myangle{}, RegistrationResult}}}
  \mystop{b, a,m'}
  \myelse{path \equiv /authorize}
  \mystate{\myss{cookie := headers[Cookie]}}
  \mystate{\myss{session := s'.SessionList[cookie]}}
  \mystate{\myss{username := session[username]}}
  \myif{username \equiv \mathtt{null}}
  \mystate{\myss{m' := \myangle{\mathtt{HTTPResp}, n, 200, \myangle{}, \mathtt{Fail}}}}
  \mystop{b, a,m'}
  \ENDIF
  \mystate{\myss{PID_{RP} := parameters[PID_{RP}]}}
  \mystate{\myss{Enpt := parameters[Enpt]}}
  \myif{PID_{RP} \notin \mathtt{ListOfPID}() \lor Enpt \notin \mathtt{EndpointsOfRP}(PID_{RP})}
  \mystate{\myss{m' := \myangle{\mathtt{HTTPResp}, n, 200, \myangle{}, \mathtt{Fail}}}}
  \mystop{b, a,m'}
  \ENDIF
  \mystate{\myss{ID_U := session[uid]}}
  \mystate{\myss{PID_U := \mathtt{Multiply}(PID_{RP}, ID_U)}}
  \mystate{\myss{Validity := \mathtt{CurrentTime} ()+ s'.Validity}}
  \mystate{\myss{Content := \myangle{PID_{RP}, PID_U, s'.Issuer, Validity}}}
  \mystate{\myss{Sig := \mathtt{SigSign}(Content, s'.SK)}}
  \mystate{\myss{Token := \myangle{Content, Sig}}}
  \mystate{\myss{s'.Tokens := s'.Tokens + ^{\myangle{}}Token}}
  \mystate{\myss{m' := \myangle{\mathtt{HTTPResp}, n, 200, \myangle{}, \myangle{Token, Token}}}}
  \mystop{b, a, m'}
  \ENDIF
  \mystop{}
  \end{algorithmic}
\end{breakablealgorithm}


\subsection{RP process}
The RP process only accepts the events, downloading scripts, redirecting to IdP server, negotiating the $PID_{RP}$, uploading the $PID_{RP}$ registration result and identity token.
The detailed procedure of dealing with these events is shown as follows.
\begin{breakablealgorithm}
  \caption{\textbf{RP\_Process\_Relation}}
  \label{alg2}
  \begin{algorithmic}[1]
  \REQUIRE \myss{\myangle{a, b, m}, s}
  \mystate{\myss{s':=s}}
  \mystate{\myss{n, method, path, parameters, headers, body} \textbf{such that}}\\
  \ \ \myss{\myangle{\mathtt{HTTPReq},n,method,path,parameters,headers,body} \equiv m}\\
  \ \ \textbf{if} \myss{possible}; \textbf{otherwise} stop \myss{\myangle{}, s'}
  \myif{path \equiv /script}
\mystate{\myss{m':=\myangle{\mathtt{HTTPResp},n,200, \myangle{}, \mathtt{RPScript}}}}
  \mystop{b, a, m'}
  \myelse{path \equiv /login}
  \mystate{\myss{m'  := \myangle{\mathtt{HTTPResp},n,302,\myangle{\myangle{Location, s'.IdP.ScriptUrl}}, \myangle{}}}}
  \mystop{b, a, m'}
  \myelse{path \equiv /startNegotiation}
  \mystate{\myss{cookie := headers[Cookie]}}
  \mystate{\myss{session := s'.SessionList[cookie]}}
  \mystate{\myss{t := parameters[t]}}
  \mystate{\myss{PID_{RP} := \mathtt{Multiply}(s'.ID_{RP}, t)}}
  \mystate{\myss{t^{-1}:= \mathtt{Inverse}(t)}}
  \mystate{\myss{session[t] := t}}
  \mystate{\myss{session[PID_{RP}] := PID_{RP}}}
  \mystate{\myss{session[t^{-1}] := t^{-1}}}
  \mystate{\myss{session[state] := expectRegistration}}
  \mystate{\myss{m' := \myangle{\mathtt{HTTPResp}, n, 200, \myangle{}, \myangle{Cert, s'.Cert}}}}
 \mystop{b, a, m'}
 \myelse{path \equiv /registrationResult}
 \mystate{\myss{cookie := headers[Cookie]}}
  \mystate{\myss{session := s'.SessionList[cookie]}}
  \myif{session[state] \not\equiv expectRegistration}
  \mystate{\myss{m' := \myangle{\mathtt{HTTPResp}, n, 200, \myangle{}, \mathtt{Fail}}}}
  \mystop{b, a, m'}
  \ENDIF
  \mystate{\myss{RegistrationResult := body[RegistrationResult]}}
  \mystate{\myss{Content:=RegistrationResult.Content}}
  \myif{\mathtt{SigVerify}(Content, RegistrationResult.Sig, s'.IdP.PK) \equiv \mathtt{FALSE}}
  \mystate{\myss{m' := \myangle{\mathtt{HTTPResp}, n, 200, \myangle{}, \mathtt{Fail}}}}
  \mystate{\myss{session := \mathtt{null}}}
  \mystop{b, a, m'}
  \ENDIF
  \myif{Content.Result \not\equiv OK}
  \mystate{\myss{m' := \myangle{\mathtt{HTTPResp}, n, 200, \myangle{}, \mathtt{Fail}}}}
  \mystate{\myss{session := \mathtt{null}}}
  \mystop{b, a, m'}
  \ENDIF
  \mystate{\myss{PID_{RP} := session[PID_{RP}]}}
  \mystate{\myss{t := session[t]}}
  \mystate{\myss{Nonce := \mathtt{Hash}( t)}}
  \mystate{\myss{Time := \mathtt{CurrentTime}()}}
  \myif{PID_{RP} \not\equiv Content.PID_{RP} \lor Nonce \not\equiv Content.Nonce \lor Time > Content.Validity}
  \mystate{\myss{m' := \myangle{\mathtt{HTTPResp}, n, 200, \myangle{}, \mathtt{Fail}}}}
  \mystate{\myss{session := \mathtt{null}}}
  \mystop{b, a, m'}
  \ENDIF
  \mystate{\myss{session[PIDValidity] := Content.Validity}}
  \mystate{\myss{Enpt \equiv s'.Enpt}}
  \mystate{\myss{session[state] := expectToken}}
  \mystate{\myss{Nonce' := \mathtt{Random}()}}
  \mystate{\myss{session[Nonce] := Nonce'}}
  \mystate{\myss{Body := \myangle{PID_{RP}, Enpt, Nonce'}}}
  \mystate{\myss{m' := \myangle{\mathtt{HTTPResp}, n, 200, \myangle{}, Body}}}
  \mystop{b, a, m'}
  \myelse{path \equiv /uploadToken}
 \mystate{\myss{cookie := headers[Cookie]}}
  \mystate{\myss{session := s'.SessionList[cookie]}}
  \myif{session[state] \not\equiv expectToken}
  \mystate{\myss{m' := \myangle{\mathtt{HTTPResp}, n, 200, \myangle{}, \mathtt{Fail}}}}
  \mystop{b, a, m'}
  \ENDIF
  \mystate{\myss{Token := body[Token]}}
  \myif{\mathtt{checksig}(Token.Content, Token.Sig, s'.IdP.PK) \equiv \mathtt{FALSE}}
  \mystate{\myss{m' := \myangle{\mathtt{HTTPResp}, n, 200, \myangle{}, \mathtt{Fail}}}}
  \mystop{b, a, m'}
  \ENDIF
  \mystate{\myss{PID_{RP} := session[PID_{RP}]}}
  \mystate{\myss{Time := \mathtt{CurrentTime}()}}
  \mystate{\myss{PIDValidity := session[PIDValidity]}}
  \mystate{\myss{Content := Token.Content}}
  \myif{PID_{RP} \not\equiv Content.PID_{RP} \lor Time>Content.Validity \lor Time>PIDValidity}
  \mystate{\myss{m' := \myangle{\mathtt{HTTPResp}, n, 200, \myangle{}, \mathtt{Fail}}}}
  \mystop{b, a, m'}
  \ENDIF
  \mystate{\myss{PID_U := Content.PID_U}}
  \mystate{\myss{t^{-1} := session[t^{-1}]}}
  \mystate{\myss{Acct := \mathtt{Multiply}(PID_U, t^{-1})}}
  \myif{Acct \not\in \mathtt{ListOfUser}()}
  \mystate{\myss{\mathtt{AddUser}(Acct)}}
  \ENDIF
  \mystate{\myss{session[user] := Acct}}
  \mystate{\myss{m' := \myangle{\mathtt{HTTPResp}, n, 200, \myangle{}, \mathtt{LoginSuccess}}}}
  \mystop{b, a, m'}
  \ENDIF
  \mystop{}
  \end{algorithmic}
\end{breakablealgorithm}


\subsection{IdP script process}
The IdP script process accepts the events,
(a) self-triggering events  for starting $PID_{RP}$ negotiation;
(b) the postMessage from other scripts for sending the $Cert$, and request of identity token;
(c) the HTTP response for transmitting registration result and identity token.
The detailed procedure of dealing with these events is shown as follows.

\begin{breakablealgorithm}
  \caption{\textbf{IdP\_Script\_Relation}}
  \label{alg3}
  \begin{algorithmic}[1]
  \REQUIRE \myss{\myangle{tree, docID, scriptstate, scriptinputs, cookies, localStorage, sessionStorage, ids, secret}}
  \mystate{\myss{ s' := scriptstate}}
  \mystate{\myss{command := \myangle{}}}
  \mystate{\myss{target := \mathtt{PARENTWINDOW}(tree,docID)}}
  \mystate{\myss{IdPDomain := s'.IdPDomain}}
  \SWITCH{\myss{s'.phsae}}
    \CASE{\myss{start}}
      \mystate{\myss{t := \mathtt{Random}()}}
      \mystate{\myss{command := \myangle{\mathtt{POSTMESSAGE}, target, \myangle{\myangle{t, t}}, \mathtt{null}}}}
      \mystate{\myss{s'.Parameters[t] := t}}
      \mystate{\myss{s'.phase := expectCert}}
    \ENDCASE
    \CASE{\myss{expectCert}}
      \mystate{\myss{pattern := \myangle{\mathtt{POSTMESSAGE}, *, Content, *}}}
      \mystate{\myss{input := \mathtt{CHOOSEINPUT}(scriptinputs,pattern)}}
      \myif{input \not\equiv \mathtt{null}}
      \mystate{\myss{Cert := input.Content[Cert]}}
      \myif{\mathtt{checksig}(Cert.Content, Cert.Sig, s'.PubKey) \equiv \mathtt{null}}
      \mystate{\myss{\textbf{stop}\ \myangle{}}}
      \ENDIF
       \mystate{\myss{s'.Parameters[Cert] := Cert}}
      \mystate{\myss{t := s'.Parameters[t]}}
      \mystate{\myss{PID_{RP} := \mathtt{Multiply}(Cert.Content.ID_{RP}, t)}}
      \mystate{\myss{s'.Parameters[PID_{RP}] := PID_{RP}}}
      \mystate{\myss{Enpt := \mathtt{RandomUrl}()}}
      \mystate{\myss{s'.Parameters[Enpt] := Enpt}}
      \mystate{\myss{Nonce := \mathtt{Hash}(t)}}
      \mystate{\myss{Url := \myangle{\mathtt{URL}, \mathtt{S}, IdPDomain, /dynamicRegistration,\myangle{} }}}
      \mystate{\myss{s'.refXHR :=  \mathtt{Random}()}}
      \mystate{\myss{command : = \langle\mathtt{XMLHTTPREQUEST}, Url, \mathtt{POST},} \\\ \ \ \ \myss{ \myangle{\myangle{PID_{RP}, PID_{RP}}, \myangle{Nonce, Nonce}, \myangle{Enpt, Enpt}}, s'.refXHR\rangle}}
      \mystate{\myss{s'.phase := expectRegistrationResult}}
       \ENDIF
      \ENDCASE
      \CASE{expectRegistrationResult}
      \mystate{\myss{pattern := \myangle{\mathtt{XMLHTTPREQUEST},Body,s'.refXHR}}}
      \mystate{\myss{input := \mathtt{CHOOSEINPUT}(scriptinputs,pattern) }}
      \myif{input \not\equiv \mathtt{null} \land input.Content[RegistrationResult].type \equiv OK}
      \mystate{\myss{RegistrationResult := input.Body[RegistrationResult]}}
      \myif{ RegistrationResult.Content.Result \not\equiv OK}
      \mystate{\myss{s'.phase := stop}}
      \mystate{\myss{\textbf{stop}\ \myangle{}}}
      \ENDIF
      \mystate{\myss{command := \myangle{\mathtt{POSTMESSAGE}, target, \myangle{\myangle{RegistrationResult, RegistrationResult}}, \mathtt{null}}}}
      \mystate{\myss{s'.phase := expectProofRquest}}
      \ENDIF
      \ENDCASE
      \CASE{expectProofRquest}
      \mystate{\myss{pattern := \myangle{\mathtt{POSTMESSAGE}, *, Content, *}}}
      \mystate{\myss{input := \mathtt{CHOOSEINPUT}(scriptinputs,pattern)}}
      \myif{input \not\equiv \mathtt{null}}
       \mystate{\myss{PID_{RP} := input.Content[PID_{RP}]}}
       \mystate{\myss{Enpt_{RP} := input.Content[Enpt]}}
       \mystate{\myss{s'.Parameters[Nonce] := input.Content[Nonce]}}
       \mystate{\myss{Cert := s'.Parameters[Cert]}}
      \myif{Enpt_{RP} \not\equiv Cert.Content.Enpt \lor PID_{RP} \not\equiv s'.Parameters[PID_{RP}]}
      \mystate{\myss{s'.phase := stop}}
      \mystate{\myss{\textbf{stop}\ \myangle{}}}
      \ENDIF
       \mystate{\myss{s'.Parameters[Enpt_{RP}] := Enpt_{RP}}}
      \mystate{\myss{Url := \myangle{\mathtt{URL}, \mathtt{S}, IdPDomain, /loginInfo, \myangle{}}}}
      \mystate{\myss{s'.refXHR :=  \mathtt{Random}()}}
      \mystate{\myss{command : = \myangle{\mathtt{XMLHTTPREQUEST}, Url, \mathtt{GET},\myangle{}, s'.refXHR}}}
      \mystate{\myss{s'.phase := expectLoginState}}
      \ENDIF
      \ENDCASE
      \CASE{expectLoginState}
      \mystate{\myss{pattern := \myangle{\mathtt{XMLHTTPREQUEST},Body,s'.refXHR}}}
      \mystate{\myss{input := \mathtt{CHOOSEINPUT}(scriptinputs,pattern) }}
      \myif{input \not\equiv \mathtt{null}}
      \myif{input.Body \equiv \mathtt{Logged}}
      \mystate{\myss{user \in ids}}
      \mystate{\myss{Url := \myangle{\mathtt{URL}, \mathtt{S}, IdPDomain, /login, \myangle{}}}}
     \ mystate{\myss{s'.refXHR :=  \mathtt{Random}()}}
      \mystate{\myss{command : = \myangle{\mathtt{XMLHTTPREQUEST}, Url, \mathtt{POST}, \myangle{credential, \mathtt{CredentialofUser}(user,secret)}}, s'.refXHR}}
      \mystate{\myss{s'.phase := expectLoginResult}}
      \myelse{input.Body \equiv \mathtt{Unlogged}}
      \mystate{\myss{PID_{RP} := s'.Parameters[PID_{RP}]}}
      \mystate{\myss{Enpt := s'.Parameters[Enpt]}}
      \mystate{\myss{Nonce := s'.Parameters[Nonce]}}
      \mystate{\myss{Url := \langle \mathtt{URL}, \mathtt{S}, IdPDomain, /authorize,}\\\ \ \ \  \myss{\myangle{\myangle{PID_{RP}, PID_{RP}}, \myangle{Enpt, Enpt}, \myangle{Nonce, Nonce}} \rangle}}
      \mystate{\myss{s'.refXHR :=  \mathtt{Random}()}}
      \mystate{\myss{command : = \myangle{\mathtt{XMLHTTPREQUEST}, Url, \mathtt{GET},\myangle{}, s'.refXHR}}}
      \mystate{\myss{s'.phase := expectToken}}
      \ENDIF
      \ENDIF
      \ENDCASE
      \CASE{expectLoginResult}
      \mystate{\myss{pattern := \myangle{\mathtt{XMLHTTPREQUEST},Body,s'.refXHR}}}
      \mystate{\myss{input := \mathtt{CHOOSEINPUT}(scriptinputs,pattern) }}
      \myif{input \not\equiv \mathtt{null}}
      \myif{input.Body \not\equiv \mathtt{LoginSuccess}}
      \mystate{\myss{\textbf{stop}\ \myangle{}}}
      \ENDIF
      \mystate{\myss{PID_{RP} := s'.Parameters[PID_{RP}]}}
      \mystate{\myss{Enpt := s'.Parameters[Enpt]}}
      \mystate{\myss{Nonce := s'.Parameters[Nonce]}}
      \mystate{\myss{Url := \langle \mathtt{URL}, \mathtt{S}, IdPDomain, /authorize,}\\\ \ \ \  \myss{\myangle{\myangle{PID_{RP}, PID_{RP}}, \myangle{Enpt, Enpt}, \myangle{Nonce, Nonce}} \rangle}}
      \mystate{\myss{s'.refXHR :=  \mathtt{Random}()}}
      \mystate{\myss{command : = \myangle{\mathtt{XMLHTTPREQUEST}, Url, \mathtt{GET},\myangle{}, s'.refXHR}}}
      \mystate{\myss{s'.phase := expectToken}}
      \ENDIF
      \ENDCASE
      \CASE{expectToken}
      \mystate{\myss{pattern := \myangle{\mathtt{XMLHTTPREQUEST},Body,s'.refXHR}}}
      \mystate{\myss{input := \mathtt{CHOOSEINPUT}(scriptinputs,pattern) }}
      \myif{input \not\equiv \mathtt{null}}
      \mystate{\myss{Token := input.Body[Token]}}
      \mystate{\myss{RPOringin := \myangle{s'.Parameters[Enpt_{RP}], \mathtt{S}}}}
      \mystate{\myss{command := \myangle{\mathtt{POSTMESSAGE},target,\myangle{Token,Token},RPOrigin}}}
      \mystate{\myss{s .phase := stop}}
     \ENDIF
    \ENDCASE
  \ENDSWITCH
\mystate{\myss{\textbf{stop}\ \myangle{s',cookies,localStorage,sessionStorage,command}}}
    \end{algorithmic}
\end{breakablealgorithm}


\subsection{RP script process}
The RP script process accepts the events,
(a) self-triggering events  for opening the new window;
(b) the postMessage from other scripts for $PID_{RP}$ negotiation, posting registration result and identity token;
(c) the HTTP response for downloading the RP $Cert$, retrieving identity token request and confirming login result.
The detailed procedure of dealing with these events is shown as follows.

\begin{breakablealgorithm}
  \caption{\textbf{RP\_Script\_Relation}}
  \label{alg4}
  \begin{algorithmic}[1]
\REQUIRE \myss{\myangle{tree, docID, scriptstate, scriptinputs, cookies, localStorage, sessionStorage, ids, secret}}
\mystate{\myss{ s' := scriptstate}}
  \mystate{\myss{command := \myangle{}}}
  \mystate{\myss{IdPWindow := \mathtt{SUBWINDOW}(tree,docnonce).ID_w}}
  \mystate{\myss{RPDomain := s'.RPDomain}}
  \mystate{\myss{IdPOringin := \myangle{s'.IdPDomian, \mathtt{S}}}}
  \SWITCH{\myss{s'.phase}}
    \CASE{\myss{start}}
    \mystate{\myss{Url := \myangle{\mathtt{URL}, \mathtt{S}, RPDomain, /login, \myangle{}}}}
    \mystate{\myss{command := \myangle{\mathtt{IFRAME}, Url, \_SELF}}}
    \mystate{\myss{s'.phase := expectt}}
    \ENDCASE
    \CASE{\myss{expectt}}
    \mystate{\myss{pattern := \myangle{\mathtt{POSTMESSAGE}, *, Content, *}}}
      \mystate{\myss{input := \mathtt{CHOOSEINPUT}(scriptinputs,pattern)}}
      \myif{input \not\equiv \mathtt{null}}
      \mystate{\myss{t := input.Content[t]}}
      \mystate{\myss{Url := \myangle{\mathtt{URL}, \mathtt{S}, RPDomain, /startNegotiation, \myangle{}}}}
      \mystate{\myss{s'.refXHR :=  \mathtt{Random}()}}
      \mystate{\myss{command : = \myangle{\mathtt{XMLHTTPREQUEST}, Url, \mathtt{POST},\myangle{\myangle{t, t}}, s'.refXHR}}}
      \mystate{\myss{s'.phase := expectCert}}
      \ENDIF
      \ENDCASE
      \CASE{\myss{expectCert}}
      \mystate{\myss{pattern := \myangle{\mathtt{XMLHTTPREQUEST},Body,s'.refXHR}}}
      \mystate{\myss{input := \mathtt{CHOOSEINPUT}(scriptinputs,pattern) }}
      \myif{input \not\equiv \mathtt{null}}
      \mystate{\myss{Cert := input.Content[Cert]}}
      \mystate{\myss{command := \myangle{\mathtt{POSTMESSAGE}, IdPWindow, \myangle{\myangle{Cert, Cert}}, IdPOringin}}}
      \mystate{\myss{s'.phase := expectRegistrationResult}}
      \ENDIF
      \ENDCASE
      \CASE{\myss{expectRegistrationResult}}
      \mystate{\myss{pattern := \myangle{\mathtt{POSTMESSAGE}, *, Content, *}}}
      \mystate{\myss{input := \mathtt{CHOOSEINPUT}(scriptinputs,pattern)}}
      \myif{input \not\equiv \mathtt{null}}
      \mystate{\myss{RegistrationResult := input.Content[RegistrationResult]}}
      \mystate{\myss{Url := \myangle{\mathtt{URL}, \mathtt{S}, RPDomain, /registrationResult, \myangle{}}}}
      \mystate{\myss{s'.refXHR :=  \mathtt{Random}()}}
      \mystate{\myss{command : = \myangle{\mathtt{XMLHTTPREQUEST}, Url, \mathtt{POST},\myangle{\myangle{RegistrationResult, RegistrationResult}}, s'.refXHR}}}
      \mystate{\myss{s'.phase := expectTokenRequest}}
      \ENDIF
      \ENDCASE
      \CASE{\myss{expectTokenRequest}}
      \mystate{\myss{pattern := \myangle{\mathtt{XMLHTTPREQUEST},Body,s'.refXHR}}}
      \mystate{\myss{input := \mathtt{CHOOSEINPUT}(scriptinputs,pattern) }}
      \myif{input \not\equiv \mathtt{null}}
      \mystate{\myss{PID_{RP} := input.Content.Body[PID_{RP}]}}
      \mystate{\myss{Enpt := input.Content.Body[Enpt]}}
      \mystate{\myss{Nonce := input.Content.Body[Nonce]}}
      \mystate{\myss{command := \langle\mathtt{POSTMESSAGE}, IdPWindow}, \\\ \ \ \  \myss{\myangle{\myangle{PID_{RP}, PID_{RP}}, \myangle{Enpt, Enpt}, \myangle{Nonce, Nonce}},  IdPOringin\rangle}}
      \mystate{\myss{s'.phase := expectToken}}
      \ENDIF
      \ENDCASE
      \CASE{\myss{expectToken}}
      \mystate{\myss{pattern := \myangle{\mathtt{POSTMESSAGE}, *, Content, *}}}
      \mystate{\myss{input := \mathtt{CHOOSEINPUT}(scriptinputs,pattern)}}
      \myif{input \not\equiv \mathtt{null}}
      \mystate{\myss{Token := input.Content[Token]}}
      \mystate{\myss{Url := \myangle{\mathtt{URL}, \mathtt{S}, RPDomain, /uploadToken, \myangle{}}}}
      \mystate{\myss{s'.refXHR :=  \mathtt{Random}()}}
      \mystate{\myss{command : = \myangle{\mathtt{XMLHTTPREQUEST}, Url, \mathtt{POST},\myangle{\myangle{Token, Token}}, s'.refXHR}}}
      \mystate{\myss{s'.phase := expectLoginResult}}
      \ENDIF
      \ENDCASE
      \CASE{\myss{expectLoginResult}}
      \mystate{\myss{pattern := \myangle{\mathtt{XMLHTTPREQUEST},Body,s'.refXHR}}}
      \mystate{\myss{input := \mathtt{CHOOSEINPUT}(scriptinputs,pattern) }}
      \myif{input \not\equiv \mathtt{null}}
      \myif{input.Body \equiv \mathtt{LoginSuccess}}
      \mystate{Load Homepage}
      \ENDIF
      \ENDIF
    \ENDCASE
    \ENDSWITCH
\end{algorithmic}
\end{breakablealgorithm}


\begin{comment}
\section{Privacy Analysis}
\label{sec:privacy}


In this section, we will give the privacy proof and show that UPPRESSO is secure against both IdP-based login tracing and RP-based identity linkage attacks.

\noindent\textbf{IdP-based login tracing.}
As shown in figure~\ref{fig:process}, the only information that is related to the RP's identity and is accessible to the IdP is $PID_{RP}$, which is converted from $ID_{RP}$ using a random $t$. Since $t$ is randomly chosen from $\mathbb{Z}_n$ by the user and the IdP ha no control of the process, the IdP should treat $PID_{RP}$ as being randomly chosen from $\mathbb{G}$. So, the IdP cannot recognize the RP nor derive its real identity. Therefore, IdP-based identity linkage becomes impossible in UPPRESSO.

Next, we will prove that UPPRESSO prevents RP-based identity linkage based on the Decisional Diffie-Hellman (DDH) assumption~\cite{GoldwasserK16}. Here, we briefly introduce the DDH assumption:
%\noindent\textbf{The DDH Assumption.}
Let $q$ be a large prime and $\mathbb{G}$ denotes a cyclic group of order $n$ of an elliptic curve $E(\mathbb{F}_q)$.
Assume that $n$ is also a large prime. Let $P$ be a generator point of $\mathbb{G}$. The DDH assumption for $\mathbb{G}$ states that for any probabilistic polynomial time (PPT) algorithm $D$, the two probability distributions \{$aP$, $bP$, $abP$\} and \{$aP$, $bP$, $cP$\}, where $a$, $b$, and $c$ are randomly and independently chosen from $\mathbb{Z}_n$, are computationally indistinguishable in the sense that there is a negligible $\sigma(n)$ with the security parameter $n$ such that:
%where $q$ and $n$ are large primitive number, and $P$ is the point of $\mathbb{G}$.
%For any probabilistic polynomial time (PPT) algorithm $D$, the distributions, \{$P$, $aP$, $bP$, $abP$\}$_{a,b \in \mathbb{Z}_n}$ and \{$P$, $aP$, $bP$, $cP$\}$_{a,b,c \in \mathbb{Z}_n}$, are computationally indistinguishable. There is a negligible $\sigma(k)$, where $k$ is the security parameter.
\vspace{-\topsep}
\begin{multline*}
Pr[D(P, aP, bP, abP)=1]-Pr[D(P, aP, bP, cP)=1]=\sigma(n)
\end{multline*}
\vspace{-\topsep}

\vspace{-2mm}
\noindent\textbf{RP-based identity linkage.}
Collusive RPs can act arbitrarily to correlate $PID_U$s at different RPs and guess if they belong to the same user. Therefore, we consider the collusive RPs are playing a guessing Game. In this Game, the IdP and the users act as the challenger, while the collusive RPs act as the adversary (denoted as $A$ in Figure~\ref{fig:game}). RP-based identity linkage is impossible in UPPRESSO {\em if and only if the adversary has no advantage over the challenger in the guessing game.}


\vspace{1mm}
\begin{strip}
\centering\includegraphics[width=\textwidth, height=0.35\textheight]{fig/game.pdf}
\captionof{figure}{Interactions between the challenger and the adversary in three Games.}
\label{fig:game}
\vspace{-5mm}
\end{strip}


Next, we model the guessing Game to depict the interactions between the challenger and the adversary. First, we describe the challenger's actions in the Game as follows.
\begin{itemize}
\vspace{-\topsep}
\item[-] {\em Initialization:} In initialization, the challenger generates $ID_{RP}$s and $ID_U$s for multiple RPs and users using the initialization algorithm $Setup(G,n)$, where $G$ and $n$ are defined in table~\ref{tbl:notations}.
\vspace{-\topsep}
\item[-] {\em Random number generation:} When the challenger receives the $ID_{RP}$s from the adversary, it generates a random $t \in \mathbb{Z}_n$ for each $ID_{RP}$ using the algorithm $Random(n)$. %$t$ will be used to generate the corresponding $PID_{RP}$.
\vspace{-\topsep}
\item[-] {\em $PID_U$ generation:} When the challenger receives an $PID_{RP}$, it first verifies if the $PID_{RP}$ is generated for $ID_{RP}$ with the corresponding $t$, using the algorithm $Verify(ID_{RP},PID_{RP},t)$. Then, it generates the $PID_U$ with the algorithm $F_{PID_U}(ID_U,PID_{RP})$ and sends it to the adversary.
%\vspace{-\topsep}
\end{itemize}

To prove the privacy of UPPRESSO against RP-based identity linkage, we define three Games, as shown in figure~\ref{fig:game}. First, based on the above description, we model the $ID_U$-guessing game following the UPPRESSO design as $\mathtt{Game 0}$ : (1) First, the adversary receives two $ID_{RP}$s (i.e., $ID_{RP1}$ and $ID_{RP2}$) and three $t$s (i.e., $n_1$, $n_2$, and $n_3$). It then generates three $PID_{RP}$s accordingly. From the challenger's view, three $PID_{RP}$s are related to three $t$s, respectively. (2) Then, the challenger generates $PID_U$s for different $PID_{RP}$s, using two $ID_U$s (i.e., $u_1$ and $u_2$). %generated in the initialization phase.
In particular, the challenger generates $PID_{U1}$ from $ID_{U1}$ directly, and selects a random number $b \in \{0, 1\}$ to generate $PID_{U2}=ID_{Ub} \cdot PID_{RP2}$, and $PID_{U3}=ID_{U(1-b)} \cdot PID_{RP3}$. (3) Finally, the adversary sends its guess $b'$ to the challenger. If $b'=b$, the adversary wins the game.

We define the event [$b'=b$] in $\mathtt{Game 0}$ as $\Gamma$. If the adversary has no advantage in guessing $b$ correctly, which indicates the $PID_U$s are generated from the same $ID_U$, the probability $Pr[\Gamma]$ should be 1/2. Therefore, we conclude that in $\mathtt{Game 0}$, UPPRESSO is secure against RP-based identity linkage if and only if $Pr[\Gamma]=1/2$.


Next, we build the ideal model of the guessing game, denoted as $\mathtt{Game 1}$. In this model, the probability that the adversary correctly guessing $b$ is 1/2. This time, the challenger randomly selects $z$ and $r$ and uses them to generate $PID_{U2}$ and $PID_{U3}$, respectively. Since the adversary does not know $z$ and $r$, he does not know $b$ neither. Similarly, we define the event [$b'=b$] in $\mathtt{Game 1}$ as $\Gamma_1$, and $Pr[\Gamma_1]$ should be 1/2.

According to the DDH assumption, we need to prove that $|Pr[\Gamma_1]-Pr[\Gamma]|=\sigma(n)$, where $\sigma(n)$ is negligible. So, we build another model of Game, denoted as $\mathtt{Game 2}$, by setting the values of the parameters defined in $\mathtt{Game 0}$ to be: $ID_{RP2}=xID_{RP1}$, $u_1=y$ and $u_2=r$, where $r$ is a random number. Again, we define the event [$b'=b$] in  $\mathtt{Game 2}$ as $\Gamma_2$, and $Pr[\Gamma_2]=Pr[\Gamma]$ should be true.

After defining the three games, we prove that $|Pr[\Gamma_1]-Pr[\Gamma_2]|=\sigma(n)$ as follows. In each game, the adversary uses the algorithm $A_2$ to derive $b'$ from the collected data, i.e., $\{ID_{RP1},ID_{RP2},n_1,n_2,n_3,PID_{U1},PID_{U2},PID_{U3}\}$. Now, let us replace the parameters in $\mathtt{Game 1}$ and $\mathtt{Game 2}$ with the exact values:
\vspace{-\topsep}
\begin{equation*}
\begin{aligned}
    b'_{game1} \gets A_2(P,xP,n_1,n_2,n_3,yn_1P,zP,rP) \\
    b'_{game2}\gets A_2(P,xP,n_1,n_2,n_3,yn_1P,xyn_2P,rn_3P) \\
    or \; b'_{game2}\gets A_2(P,xP,n_1,n_2,n_3,yn_1P,rn_2P,xyn_3P)
\end{aligned}
\end{equation*}
\vspace{-\topsep}

Since $n_1$, $n_2$, $n_3$ are randomly chosen by the challenger, which are not related to $ID_U$, the adversary can easily remove them and obtain $b'_{game1}\gets A_2(P,xP,yP,zP,rP)$ in $\mathtt{Game 1}$ and $b'_{game2}\gets A_2(P,xP,yP,xyP,xrP)$ in $\mathtt{Game 2}$.

As $r$ is randomly chosen and unknown to the adversary, $rP$ and $xrP$ are also random points. We can re-write $b'_{game1}$ and $b'_{game2}$ as $b'_{game1}\gets A_2(P,xP,yP,zP,r_1P)$ and $b'_{game2}\gets A_2(P,xP,yP,xyP,r_2P)$, which means there is no non-negligible difference between the success probability in $\mathtt{Game 1}$ and $\mathtt{Game 2}$, according to the DDH assumption. Otherwise, we should be able to build a PPT distinguishing algorithm that breaks DDH assumption about the adversary.

Such distinguishing algorithm $D$ is shown in figure~\ref{fig:dalgorithm}.
The inputs of the algorithm is $\{P,X,Y,Z\}$. To the adversary, it is $\mathtt{Game 1}$ if the input is in the form $\{P,xP,yP,zP\}_{x,y,z \in \mathbb{Z}_n}$, and it is $\mathtt{Game 1}$ if the input is $\{P,xP,yP,xyP\}_{x,y \in \mathbb{Z}_n}$. As a result,
\vspace{-\topsep}
\begin{multline*}
   \ \ \ \ \ \ \ \ \ \ \ \ \ \ \ \ \  Pr[D(P,xP,yP,zP)=1]=Pr[{\Gamma_1}]\\
   Pr[D(P,xP,yP,xyP)=1]=Pr[{\Gamma_2}]\ \ \ \ \ \ \ \ \ \ \ \ \ \ \ \ \ \
\end{multline*}

\vspace{-\topsep}
Therefore, $|Pr[\Gamma_1]-Pr[\Gamma_2]|=\sigma(n)$, where $\sigma(n)$ is negligible, and $n$ is the security parameter. It means the adversary has no advantage in guessing $b$ in $\mathtt{Game 0}$. Therefore, he cannot distinguish if two $PID_U$s at two different RPs belong to the same user or not. This proves that UPPRESSO is resistant to RP-based identity linkage attacks.

\end{comment}


\begin{comment}
\begin{figure}[t]
  \centering
  \includegraphics[width=1\linewidth]{fig/game0.pdf}
  \vspace{-5mm}
  \caption{Game 0.}
  \label{fig:game0}
  \vspace{-5mm}
\end{figure}


\begin{figure}[t]
  \centering
  \includegraphics[width=1\linewidth]{fig/game1.pdf}
  \vspace{-5mm}
  \caption{Game 1.}
  \label{fig:game1}
    \vspace{-5mm}
\end{figure}

\begin{figure}[t]
  \centering
  \includegraphics[width=1\linewidth]{fig/game2.pdf}
  \vspace{-5mm}
  \caption{Game 2.}
  \label{fig:game2}
  \vspace{-5mm}
\end{figure}

\end{comment}

%\end{appendices}



% that's all folks
\end{document}
